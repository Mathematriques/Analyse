\documentclass[a4paper, 11pt, french]{book}
\synctex=1

%------- Packages ------- 
\usepackage[french]{babel}
\usepackage[T1]{fontenc}
\usepackage[utf8]{inputenc}
\usepackage{icomma} % pour les virgules françaises
\usepackage[inner=3cm, outer=2cm, top=2cm, bottom=2cm]{geometry}
\usepackage{amsmath, amssymb} % subset, boxed..., nabla...
\usepackage{cancel} % pour simplifier des termes
\usepackage{mathrsfs} % cursives
\usepackage{dsfont} % indicatrice
\usepackage{xcolor} % pour colorier du texte
\usepackage{graphicx} % les graphiques
\usepackage{bussproofs} % démonstration formelle
\usepackage{tikz}
%\usepackage[backend=biber, style=alphabetic, maxnames=15, sorting=ynt]{biblatex}
%\addbibresource{sample.bib} % la bibilographie
%\setlength\parindent{0pt}

%------- Commande en Anglais -------
\newenvironment{centre}{\center}{\endcenter}
\newenvironment{itemise}{\itemize}{\enditemize}
\let\centring=\centering
\let\colourlet=\colorlet
\let\definecolour=\definecolor
\let\colour=\color
\let\textcolour=\textcolor
\colourlet{grey}{gray}

%------ Paragraphes particuliers -------
\usepackage{amsthm}

\theoremstyle{plain} % plain = boldface title, italicised body
\newtheorem{theoreme}{Théorème}
\newtheorem{proposition}{Proposition}
\newtheorem{exercice}{Exercice}

\theoremstyle{definition} % definition = boldface title, Roman body
\newtheorem{definition}{Définition}
\newtheorem{remarque}{Remarque}

\theoremstyle{remark} % remark = italicized title, Roman body
\newtheorem{exemple}{Exemple}
\newtheorem{correction}{Correction}
\newtheorem*{demonstration}{Démonstration}
\newtheorem*{lemme}{Lemme}

%------- Paragraphes particuliers ------- 
\usepackage{titlesec}
%\setlength{\parindent}{0pt} % pas d'alinéa
%\setlength{\parskip}{0.5em} % distance entre les paragraphes \par
%\titlespacing{\paragraph}{0em}{1em}{0.5em}[] % espace à gauche, en haut, à droite
\titleformat{\chapter}[display]{\normalfont\bfseries}{}{-3cm}{\Huge\thechapter.\quad}
\titleformat{\section}[display]{\normalfont\bfseries}{}{-1cm}{\Large\thesection\quad}

% Raccourcis
\renewcommand{\setminus}{\backslash}
\newcommand{\1}{\mathds{1}}
\newcommand{\id}{\mathrm{id}}
\newcommand\vide{\varnothing}
\newcommand{\cont}{\mathcal{C}}
\newcommand{\infegal}{\leqslant}
\newcommand{\supegal}{\geqslant}
\newcommand{\inv}[1]{#1^{-1}}
\newcommand\et{\text{ et }}
\newcommand\ou{\text{ ou }}
\newcommand{\N}{\mathbb{N}}
\newcommand{\Z}{\mathbb{Z}}
\newcommand{\Q}{\mathbb{Q}}
\newcommand{\R}{\mathbb{R}}
\newcommand{\C}{\mathbb{C}}

%----- Pour l'instant pas utile -----
\newcommand\transp{\, {}^t}
\newcommand\diag{\mathrm{diag}}
\newcommand\rang{\mathrm{rg}}
\newcommand\image{\mathrm{Im}}
\newcommand\noyau{\mathrm{Ker}}
\newcommand\Vect{\mathrm{Vect}}
\newcommand\crochet[1]{\langle #1\rangle}
\newcommand\classe[1]{\mathcal{C}^#1}
\newcommand\intervalle[1]{[\![#1]\!]}
\newcommand\diff{\mathrm{d}}

\title{Analyse\\Mathématiques sophistiquées expliquées par un simple}
\author{Paul \textsc{Trillat}}
\date{\today}

\begin{document}
\maketitle
\tableofcontents

\chapter*{Introduction}
\begin{center}
	\begin{tikzpicture}[xscale=3, yscale=2]
		\node (G) at (1, 1) {$(G, \star)$};
		\node (M) at (2, 1) {$(\Omega, \mu)$};
		\begin{scope}[local bounding box=partieI]
			\node (T) at (0, 1) {$(E, \mathscr{T})$};
			\node (EU) at (0, 0) {structure uniforme};
		\end{scope}
		\draw (partieI.south west) rectangle (partieI.north east);
		\node[left] at (partieI.west) {Partie I};
		\begin{scope}[local bounding box=partieII]
			\node (GT) at (1, 0) {$(G, \star, \mathscr{T})$};
			\node (EVT) at (1, -1) {$(E, +, \cdot, \mathscr{T})$};
		\end{scope}
		\draw (partieII.south west) rectangle (partieII.north east);
		\node[right] at (partieII.east) {Partie II};
		\begin{scope}[local bounding box=partieIII]
			\node (ELC) at (0, -2) {loc. convexe};
			\node (EM) at (0, -3) {$(E, d)$};
			\node (EN) at (0, -4) {$(E, || \dots||)$};
		\end{scope}
		\node[left] at (partieIII.west) {Partie III};
		\draw (partieIII.south west) rectangle (partieIII.north east);
		\draw[black, -stealth] (T)--(EU);
		\draw[black, -stealth] (T)--(GT);
		\draw[black, -stealth] (EVT)--(ELC);
		\draw[black, -stealth] (EU)--(ELC);
		\draw[black, -stealth] (G)--(GT);
		\draw[black, -stealth] (M)--(GT);
		\draw[black, -stealth] (GT)--(EVT);
		\draw[black, -stealth] (ELC)--(EM);
		\draw[black, -stealth] (EM)--(EN);
	\end{tikzpicture}
\end{center}

\chapter{Ensembles, en grosse balle}

\begin{definition}
	Une relation binaire $\preccurlyeq$ est une relation d'ordre sur $E$ si elle est réflexive, antisymétrique et transitive.
	On dit alors que $(E, \preccurlyeq)$ est ordonné.
	Cela veut dire:
	\begin{itemise}
		\item $\forall x\in E, x\preccurlyeq x$
		\item si $x\preccurlyeq y$ et $y\preccurlyeq x$ alors $x=y$
		\item si $x\preccurlyeq z$ et $y\preccurlyeq z$ alors $x\preccurlyeq z$
	\end{itemise}
\end{definition}


\begin{definition}
	On dit qu'un partie $X$ d'un ensemble $(E, \preccurlyeq)$ ordonné est totalement ordonnée si et seulement si $\forall x, y\in X, (x\preccurlyeq y)\ou(y\preccurlyeq x)$.
\end{definition}


\begin{definition}
	On dit que $x^+\in E$ est un majorant d'une partie $X$ d'un ensemble $(E, \preccurlyeq)$ ordonné si et seulement si $\forall x\in X, x\preccurlyeq x^+$.
\end{definition}

\begin{definition}
	On dit que $x^*\in X$ est un élément maximal d'une partie $X$ d'un ensemble $(E, \preccurlyeq)$ ordonné si et seulement si $\forall x\in X, (x^*\preccurlyeq x)\Rightarrow(x^*=x)$.
\end{definition}


\begin{definition}
	On dit qu'un ensemble ordonné est inductif si toute partie totalement ordonnée admet un majorant.
\end{definition}

\begin{theoreme}[Lemme de Zorn]
	Tout ensemble ordonné, non vide et inductif admet un élément maximal.
\end{theoreme}

\chapter{Groupes, en grosse balle}

\section{Définitions et premières propriétés}

\begin{definition}
	On appelle $(G, \star)$ un groupe si et seulement si
	\begin{itemise}
		\item $\star$ est une opération binaire interne $\forall x, y\in G, x\star y\in G$.
		\item $\star$ admet un neutre $e$, $\forall x\in G, x\star e=e\star x=x$.
		\item Il existe toujours un symétrique par $\star$: $\forall x\in G, \exists y\in G, x\star y=y\star x=e$.
		\item $G$ est associative pour $\star$: $\forall x, y, z\in G, (x\star y)\star z=x\star(y\star z)$.
	\end{itemise}
\end{definition}

\begin{proposition}
	On peut alléger les axiomes en supposant que $e$ est juste un neutre à gauche et que le symétrique existe à gauche.
\end{proposition}

\begin{demonstration}
	Soit $e$ un neutre à gauche de $G$, $x$ dans $G$, $y$ un symétrique à gauche de $x$, $z$ un symétrique à gauche de $y$.
	$x\star y=e\star x\star y=z\star y\star x\star y=z\star e\star y=z\star y=e$.
	Ainsi un symétrique à gauche est aussi un symétrique à droite.
	$x\star e=x\star y\star x=e\star x=x$.
	Ainsi un neutre à gauche est aussi neutre à droite.
\end{demonstration}

\begin{proposition}
	Dans un groupe, il n'existe qu'un neutre et qu'un symétrique.
	Pour un élément $x$ de $G$, on notera $ x^{-1}$ ou $-x$ ce symétrique.
	De plus $\inv{(x^{-1})}=x$
\end{proposition}

\begin{demonstration}
	Soit $e$ et $e'$ deux neutres, alors $e\star e'=e'\star e=e=e'$.
	Soit $x$ un élément de $G$, $y$ et $y'$ deux symétriques, alors $y'=e\star y'=y\star x\star y'=y\star e=y$.
	De plus $(x^{-1})^{-1}=e\star(x^{-1})^{-1}=(x\star x^{-1})\star(x^{-1})^{-1}=x\star(x^{-1}\star(x^{-1})^{-1})=x\star e=x$.
\end{demonstration}

\begin{definition}
	On appelle sous-groupe d'un groupe $G$ une partie $H$ de $G$ sur laquelle la loi de $G$ défini une structure de groupe.
\end{definition}

\begin{proposition}
	$H$ est un sous groupe de $G$ si et seulement si $H\neq\vide\et\forall x, y\in H, x\star\inv{y}\in H$.
\end{proposition}

\begin{demonstration}
	Par double implication.
	\begin{itemise}
		\item[$\Rightarrow$] Par définition $e\in H\neq\vide$, et il suffit de composer l'existence d'un inverse avec le produit.
		\item[$\Leftarrow$] Puisque $H$ est non vide $\exists x\in H$, donc $x\star x^{-1}=e\in H$.
		Ainsi $\forall y\in H, e\star\inv{y}=\inv{y}\in H$.
		C'est donc une opération interne car $x\star y=x\star\inv{(\inv{y})}\in H$, admettant $e$ pour neutre car $x\star e=x\in H$.
	\end{itemise}
\end{demonstration}

\begin{proposition}
	L'intersection d'une famille de sous-groupe $(H_i)_{i\in I}$ est un sous-groupe.
\end{proposition}

\begin{demonstration}
	$\forall i\in I, e\in H_i$ donc $e\in\bigcap_{i\in I}H_i\neq\vide$.
	Soient $x, y\in\bigcap_{i\in I}H_i$ alors $\forall i\in I, x\star\inv{y}\in H_i$, donc $x\star y\in\bigcap_{i\in I}H_i$.
\end{demonstration}

\begin{definition}
	On appelle $\sigma_\text{groupe}(\mathfrak{B})=\bigcap_{\substack{H\text{ sous-groupe}\\\mathfrak{B}\subset H}}H$, le sous-groupe engendré par $\mathfrak{B}$.
\end{definition}

\begin{proposition}
	$\sigma_\text{groupe}(\mathfrak{B})=\{x_1^{\epsilon_1}\star\dots\star x_n^{\epsilon_n} | n\in\N, x_i\in\mathfrak{B}, \epsilon_i=\pm1\}$
\end{proposition}

\begin{demonstration}
	\colour{red}{À démontrer}
\end{demonstration}

\begin{definition}
	Un groupe est dit de type fini lorsqu'il admet une partie génératrice finie.
\end{definition}

\begin{proposition}
	Tout sous-groupe de $(\Z, +)$ est de la forme $a\Z$, pour un unique $a\supegal0$.
\end{proposition}

\begin{demonstration}
	Par double implication.
	\begin{itemise}
		\item[$\Leftarrow$] $a\Z$ est toujours un sous-groupe de $\Z$.
		\item[$\Rightarrow$] Si $G=\{0\}=0\Z$ le résultat est vrai.
		Soit $G$ un sous-groupe de $\Z$ avec un élément $x$ non nul.
		Quite à prendre $-x\in G$ on peut supposer $x>0$
		D'après l'algorithme d'Euclide, $\exists q\in\Z, \exists r\in[\![0, a-1]\!], x = qa + r$.
		Mais puisque $G$ est un groupe $x-qa=r\in G$.
		Si $r$ était non nul, ce serait un élément de $G$ strictement positif et strictement plus petit que $a$, ce qui est impossible.
		Ainsi $r=0$, et donc $x\in a\Z$.
	\end{itemise}
\end{demonstration}

\begin{proposition}
	Tout sous-groupe de $(\R, +)$ est de la forme $a\Z$, pour un unique $a\supegal0$, ou dense.
\end{proposition}

\begin{demonstration}
	Soit $G$ un sous-groupe de $\R$.
	Si $G = \{ 0 \}$ alors $G = 0\Z$ donc le résultat est vérifié.
	Sinon il existe alors un $x_0\in G$ non nul et quitte à prendre $-x_0$ on peut supposer que $x_0$ est strictement positif.
	Ainsi l’ensemble $\{x\in G \ | \ x > 0\}$ est non vide et minoré par 0, il admet donc une borne inférieure $a=\inf\{x\in G \ | \ x > 0\}$.
	\begin{itemise}
		\item Si $a = 0$ les éléments de $\{x\in G \ | \ x > 0\}$ sont moralement aussi petit qu'on veut.
		On veut donc montrer que $G$ est dense dans $\R$.
		Soit $x\in\R$ et $\epsilon > 0 = a$.
		Puisque $\epsilon$ n'est pas un minorant il existe $x_\epsilon\in G$ tel que $0 < x_\epsilon < \epsilon$.
		Puisque $x_\epsilon$ est non nul il est possible de couper $\R$ en bandes de tailles $x_\epsilon$ par $\bigcup_{n\in\Z}[nx_\epsilon, (n+1)x_\epsilon[$.
		Ainsi $x$ est dans l'une de ces bandes c'est-à-dire qu'il existe $n\in\Z$ tel que $nx_\epsilon\infegal x < (n + 1)x_\epsilon=nx_\epsilon + x_\epsilon < nx_\epsilon + \epsilon$.
		Donc $0\infegal x - nx_\epsilon < \epsilon$ et puisque $nx_\epsilon = x_\epsilon + \cdots + x_\epsilon\in G$ alors $G$ est dense dans $\R$.
		\item Si $a>0$ cela veut dire que $G$ ne peut pas décrire des nombres aussi petits que l'on veut.
		En fait on va montrer que le plus petit pas que l'on peut faire dans $G$ est exactement $a$.
		Puis que $G$ étant un groupe ses autres éléments seront donc des multiples de $a$.
		\begin{itemise}
			\item Montrons que $a\in G$ par l'absurde.
			Puisque $2a$ n'est pas un minorant il existe $x\in G$ tel que $a \infegal x < 2a$.
			Mais $a\notin G$ donc l'inégalité est stricte montrant que $x$ n'est pas un minorant.
			De même il existe $y\in G$ tel que $a < y < x < 2a$ ce qui montre $0 < x - y < a$.
			Puisque $G$ est un groupe $0 < x - y\in G$ ce qui est absurde car $a$ est la borne inférieure.
			Ainsi $a\in G$ et donc $a\Z\subset G$.
			\item Montrons que $G\subset a\Z$.
			Soit $x\in G$ en coupant $\R$ en bandes de taille $a$ il existe $n\in\Z$ tel que $na\infegal x<(n+1)a$ donc $0\infegal x - na < a$.
			Or $x - na\in G$ car $G$ est un groupe et $n$ est entier.
			Par défintion le seul élément de $G$ plus petit que $a$ est $0$ ce qui montre que $x=na\in a\Z$.
		\end{itemise}
	\end{itemise}
\end{demonstration}

\begin{definition}
	Soit $x$ un élément d'un groupe $G$ et $H$ un sous-groupe.
	On appelle classe à droite de $x$ l'ensemble $xH = \{xh\ |\ h\in H\}$.
	On définit de même la classe à gauche.
\end{definition}

\begin{proposition}
	Soit $H$ un sous-groupe de $G$.
	La relation $\sim_H$ définie par $x\sim_H y\iff xH = yH$  sur $G$ est une relation d'équivalence.
\end{proposition}

\begin{demonstration}
	Vérifions tout les axiomes d'une relation d'équivalence.
	\begin{itemise}
		\item $x\sim_H x$ par réflexivité de $xH = xH$.
		\item de $x\sim_H y$ on a $y\sim_H x$ par symétrie de $xH = yH$.
		\item de $x\sim_H y\et y\sim_H z$ on a $x\sim_H z$ par transitivité de $xH = yH\et yH\sim_H zH$.
	\end{itemise}
\end{demonstration}

\begin{definition}
	Un sous-groupe est disingué si ses classes à gauches sont égales à ses classes à droites.
	On note alors $H \trianglelefteq G$.
\end{definition}

\begin{proposition}
	Un sous-groupe $H$ est disingué si et seulement s'il est compatible avec la relation d'équivalence $\sim_H$
	(id est de $x, y, z, t\in G$ tels que $x\sim_H y$ et $z\sim_H t$ alors $xz\sim_H yt$).
\end{proposition}

\begin{demonstration}
	Par double implication.
	\begin{itemise}
		\item[$\Rightarrow$] Supposons $H$ disingué.
		Soient $x, y, z, t\in H$ tels que $x\sim_H y$ et $z\sim_H t$.
		En utilisant l'associativité $(xz)H = x(zH) = x(tH) = x(Ht) = (xH)t = (yH)t = (yt)H$.
		\item[$\Leftarrow$] Supposons $H$ compatible avec $\sim_H$ et prenons $x\in G$.
		Si $y\in xH$ est dans une classe à droite alors $y\sim_H x$.
		Par réfléxivité $x^{-1}\sim_H x^{-1}$ donc par compatibilité $yx^{-1}\sim_H xx^{-1}=e$.
		Ainsi il existe $h\in H$ tel que $ = eh = h$ donc $y=hx$ c'est à dire $y\in Hx$.
		De même pour l'inclusion réciproque.
	\end{itemise}
\end{demonstration}

\begin{definition}
	Le quotient d'un groupe $G$ par $H\trianglelefteq G$ noté $G/H$ est l'ensemble des classes de $H$.
	C'est un groupe pour la loi $xH, yH\longmapsto (xH)\star(yH)=(xy)H$.
\end{definition}

\begin{proposition}
	Théorème de factorisation.
	\colour{red}{À compléter !}
\end{proposition}

\part{Topologie}

\chapter{Espaces topologiques généraux}

\section{Équivalences}

Ayant été étudiés par de nombreuses personnes, il existe plusieurs façon équivalentes de définir un espace topologique.
On donne dans la suite les définitions les plus classiques.
Ces définitions étant équivalentes, on se permettra dans la suite de définir des topologies à partir d'une de ces constructions, voir d'un mélange de plusieurs d'entre elles.

\begin{definition}
	On appelle topologie d'ouverts sur $E$ une collection $\mathscr{O}\subset\mathfrak{P}(E)$ telle que:
	\begin{itemise}
		\item Toute réunion d'ensembles de $\mathscr{O}$ est un ensemble de $\mathscr{O}$.
		\item Toute intersection finie d'ensembles de $\mathscr{O}$ est un ensemble de $\mathscr{O}$.
	\end{itemise}
	Les éléments de $\mathscr{O}$ sont appelés les ouverts de $E$.
\end{definition}

\begin{definition}
	On appelle topologie de fermés sur $E$ une collection $\mathscr{F}\subset\mathfrak{P}(E)$ telle que:
	\begin{itemise}
		\item Toute réunion finie d'ensembles de $\mathscr{F}$ est un ensemble de $\mathscr{F}$.
		\item Toute intersection d'ensembles de $\mathscr{F}$ est un ensemble de $\mathscr{F}$.
	\end{itemise}
	Les éléments de $\mathscr{F}$ sont appelés les fermés de $E$.

\end{definition}

\begin{proposition}
	L'ensemble des topologies d'ouverts est en bijection avec celui des fermés.
\end{proposition}

\begin{demonstration}
	L'application $\Psi:X\longmapsto\complement X$ est une bijection car c'est une involution.
	Soit $\mathscr{O}$ une topologie d'ouverts et montrons que $\mathscr{F}=\Psi(\mathscr{O})$ est une topologie de fermés.
	\begin{itemise}
		\item Soit $I$ fini et $(F_i)_{i\in I}\in\mathscr{F}^I$ alors $\bigcup_{i\in I}F_i=\bigcup_{i\in I}\complement O_i=\complement\bigcap_{i\in I}O_i\in\Psi(\mathscr{O})=\mathscr{F}$.
		\item Soit $I$ quelconque et $(F_i)_{i\in I}\in\mathscr{F}^I$ alors $\bigcap_{i\in I}F_i=\bigcap_{i\in I}\complement O_i=\complement\bigcup_{i\in I}O_i\in\Psi(\mathscr{O})=\mathscr{F}$.
	\end{itemise}
	Inversement on  montre de même que si $\mathscr{F}$ une topologie de fermés alors $\mathscr{O}=\Psi(\mathscr{F})$ est une topologie d'ouverts.
\end{demonstration}

\begin{definition}
	On appelle voisinagination sur $E$ une fonction $\mathfrak{V}:E\rightarrow\mathfrak{P}(\mathfrak{P}(E))$ telle que:
	\begin{itemise}
		\item Toute partie de $E$ qui contient un ensemble de $\mathfrak{V}(x)$ appartient à $\mathfrak{V}(x)$.
		\item Toute intersection finie d'ensembles de $\mathfrak{V}(x)$ appartient à $\mathfrak{V}(x)$.
		\item Tout ensemble de $\mathfrak{V}(x)$ contient $x$.
		\item Si $V$ appartient à $\mathfrak{V}(x)$ il existe un ensemble $W$ de $\mathfrak{V}(x)$ tel que $\forall y\in W, V\in\mathfrak{V}(y)$.
	\end{itemise}
	Les images $\mathfrak{V}(x)$ sont appelées voisinages de $x$.
\end{definition}

\begin{proposition}
	Les topologies d'ouverts sont en bijection avec les voisinaginations.
\end{proposition}

\begin{demonstration}
	Soit $\Phi:\mathscr{T}\mapsto(x\mapsto\{V\subset E\ |\ \exists O\in\mathscr{T}, x\in O\subset V\})$. \\
	Pour une voisinagination $\mathfrak{V}$ montrons que $\mathscr{T}=\{O\subset E\ |\ \forall x\in O, O\in\mathfrak{V}(x)\}$ est une topologie.
	\begin{itemise}
		\item Soient $O_i$ des éléments de $\mathscr{T}$ et $x\in\bigcup_iO_i$.
		Alors $x$ appartient à un certain $O_j\in\mathscr{T}$ donc $O_j\in\mathfrak{V}(x)$.
		Or $\bigcup_iO_i$ inclut $O_j$ donc par définition des voisinages $\bigcup_iO_i\in\mathfrak{V}(x)$ ainsi $\bigcup_iO_i\in\mathscr{T}$.
		\item Soient $O_i$ une collection finie d'éléments de $\mathscr{T}$ et $x\in\bigcap_{i=1}^nO_i$.
		Alors $x$ appartient à chaque $O_i\in\mathscr{T}$ donc $O_i\in\mathfrak{V}(x)$.
		Or une intersection finie de voisinages de $x$ reste un voisinage de $x$ donc $\bigcap_{i=1}^nO_i\in\mathfrak{V}(x)$ ainsi $\bigcap_{i=1}^nO_i\in\mathscr{T}$.
	\end{itemise}
	Montrons par double inclusion que l'image par $\Phi$ de $\mathscr{T}$ est $\mathfrak{V}$ c'est à dire $\Phi(\mathscr{T})(x)=\mathfrak{V}(x)$.
	\begin{itemise}
		\item Soit $V\in\Phi(\mathscr{T})(x)$ alors il existe $O\in\mathscr{T}$ tel que $x\in O\subset V$.
		Étant dans $\mathscr{T}$ on a $\forall y\in O, O\in\mathfrak{V}(y)$ en particulier $O\in\mathfrak{V}(x)$.
		Puisque $V$ inclut $O$ alors $V\in\mathfrak{V}(x)$.
		\item Soit $V\in\mathfrak{V}(x)$ et posons (essayez de deviner, c'est magnifique) $U=\{y\in E\ |\ V\in\mathfrak{V}(y)\}$.
		\begin{itemise}
			\item Puisque $V\in\mathfrak{V}(x)$ alors $x\in U$.
			\item De $V\in\mathfrak{V}(y)$ on tire $y\in V$ montrant que si $y\in U$ alors $y\in V$ c'est à dire $U\subset V$.
			\item Si $y\in U$ alors $V\in\mathfrak{V}(y)$ qui est un voisinage donc il existe un $W\in\mathfrak{V}(y)$ tel que pour tout $z\in W$, $V$ appartienne à $\mathfrak{V}(y)$, c'est à dire $z\in U$.
			Ainsi $\mathfrak{V}(y)\ni W\subset U$ ce qui implique $O\in\mathfrak{V}(y)$.
			On a montré $\forall y\in U, U\in\mathfrak{V}(y)$ c'est à dire $U\in\mathscr{T}$.
		\end{itemise}
		Ainsi $U$ est un ouvert de $\mathscr{T}$ tel que $x\in U\subset V$ c'est à dire $V\in\Phi(\mathscr{T})(x)$.
	\end{itemise}
	Montrons que $\mathscr{T}$ est l'unique préimage par $\Phi$ de $\mathfrak{V}$.
	Soit donc une autre topologie $\mathscr{T}'$ elle aussi préimage, c'est à dire vérifiant $\Phi(\mathscr{T}')=\mathfrak{V}$ et montrons que $\mathscr{T}=\mathscr{T}'$ par double inclusion.
	\begin{itemise}
		\item Soit $O'\in\mathscr{T}'$ et $x\in O'$ alors $O'\in\mathfrak{V}(x)$ c'est à dire $\forall x\in O', O'\in\mathfrak{V}(x)$ donc $O'\in\mathscr{T}$.
		\item Soit $O\in\mathscr{T}$ et $x\in O$ alors $O\in\mathfrak{V}(x)$ donc il existe un ouvert $O_x'$ de $\mathscr{T}'$ tel que $x\in O_x'\subset O$.
		Ainsi $O=\bigcup_{x\in O}\{x\}\subset\bigcup_{x\in O}O_x'\subset\bigcup_{x\in O}O=O$ c'est à dire $O=\bigcup_{x\in O}O_x'$ qui est un ouvert de $\mathscr{T}'$ car réunion d'ouverts, soit $O\in\mathscr{T}'$.
	\end{itemise}
	On a donc montré que pour toute voisinagination $\mathfrak{V}$ il existait une unique topologie $\mathscr{T}$ telle que $\Phi(\mathscr{T})=\mathfrak{V}$ c'est à dire que $\Phi$ est bijective.
\end{demonstration}

\begin{definition}
	On appelle adhérence sur $E$ une application $\overline{\cdot}\colon\mathfrak{P}(E)\rightarrow\mathfrak{P}(E)$, dont les images sont appelées les adhérences, telle que:
	\begin{itemise}
		\item $X\subset\overline{X}$
		\item $\overline{\overline{X}}=\overline{X}$
		\item $\overline{X\cup Y}=\overline{X}\cup\overline{Y}$
		\item $\overline{\vide}=\vide$
	\end{itemise}
\end{definition}

\begin{proposition}
	Les topologies d'adhérence sont en bijection avec les topologies de fermés.
\end{proposition}

\begin{demonstration}
	Posons $\Phi\colon\mathscr{T}\mapsto(X\mapsto\bigcap_{X\subset F\in\mathscr{T}}F)$.
	Vérifions que $\Phi$ est surjective.
	Soit $\overline{\cdot}$ une adhérence sur $E$.
	On utilisera plusieurs fois la croissance de cette application (si $X\subset Y$ alors $\overline{Y}=\overline{Y\setminus X\cup X}=\overline{Y\setminus X}\cup\overline{X}\supset\overline{X}$). \\
	Alors $\mathscr{T}_\text{fermé}=\{F\subset E|F=\overline{F}\}$ est un topologie de fermés.
	\begin{itemise}
		\item $\overline{\vide}=\vide$ donc $\vide\in\mathscr{T}_\text{fermé}$ et $E\subset\overline{E}\subset E$ donc $E\in\mathscr{T}_\text{fermé}$.
		\item Si $F, F'\in\mathscr{T}_\text{fermé}$ alors $\overline{F\cup F'}=\overline{F}\cup\overline{F'}=F\cup F'$.
		\item Si $F_{i\in I}\in\mathscr{T}_\text{fermé}$ alors $F_j=\overline{F_j}\supset\overline{\bigcap_{i\in I}F_i}$ donc $\bigcap_{j\in I}F_j\supset\overline{\bigcap_{i\in I}F_i}$.
	\end{itemise}
	Posons $\underline{\cdot}=\Phi(\mathscr{T}_\text{fermé})$.
	Montrons que $\underline{X}=\overline{X}$.
	\begin{itemise}
		\item[$\subset$] $\overline{\overline{X}}=\overline{X}$ donc $\overline{X}\in\mathscr{T}_\text{fermé}$ et $X\subset\overline{X}$ puis $\underline{X}=\bigcap_{X\subset F\in\mathscr{T}_\text{fermé}}F\subset\overline{X}$.
		\item[$\supset$] Par croissance $\underline{X}=\bigcap_{X\subset F\in\mathscr{T}_\text{fermé}}F=\bigcap_{\overline{X}\subset\overline{F}\in\mathscr{T}_\text{fermé}}F\supset\overline{X}$.
	\end{itemise}
	Vérifions que $\Phi$ soit injective.
	Soit $\mathscr{T_\text{fermé}}$ et $\mathscr{T}_\text{fermé}'$ tels que $\Phi(\mathscr{T}_\text{fermé})=\Phi(\mathscr{T}_\text{fermé}')$.
	Soit $F\in\mathscr{T}_\text{fermé}$ alors $F=\Phi(\mathscr{T}_\text{fermé})(F)=\Phi(\mathscr{T}_\text{fermé}')(F)=\bigcap_{F\subset F'\in\mathscr{T}_\text{fermé}'}F'\in\mathscr{T}_\text{fermé}'$.
\end{demonstration}

\begin{definition}
	On appelle intérieur sur $E$ une application $\cdot^\circ\colon\mathfrak{P}(E)\rightarrow\mathfrak{P}(E)$, dont les images sont appelées les intérieurs, telle que:
	\begin{itemise}
		\item $X^\circ\subset X$
		\item $X^{\circ\circ}=X^\circ$
		\item $(X\cap Y)^\circ=X^\circ\cap Y^\circ$
		\item $E^\circ=E$
	\end{itemise}
\end{definition}

\begin{proposition}
	Les topologies d'intérieur sont en bijection avec les topologies d'ouverts.
\end{proposition}

\begin{demonstration}
	Posons $\Phi\colon\mathscr{T}\mapsto(X\mapsto\bigcup_{X\supset O\in\mathscr{T}}O)$.
	\begin{itemise}
		\item Vérifions que $\Phi$ est surjective.
		Soit $\cdot^\circ$ un intérieur sur $E$.
		On utilisera plusieur fois la croissance de cette application (si $X\subset Y$ alors $X^\circ=(X\cap Y)^\circ=X^\circ\cap Y^\circ\subset Y^\circ$). \\
		Alors $\mathscr{T}_\text{ouvert}=\{O\subset E|O=O^\circ\}$ est un topologie d'ouverts.
		\begin{itemise}
			\item $\vide^\circ=\vide$ donc $\vide\in\mathscr{T}_\text{ouvert}$ et $E=E^\circ$ donc $E\in\mathscr{T}_\text{ouvert}$.
			\item Si $O, O'\in\mathscr{T}_\text{ouvert}$
			alors $(O\cap O')^\circ=O^\circ\cap{O'}^\circ=O\cap O'$.
			\item Si $O_{i\in I}\in\mathscr{T}_\text{ouvert}$
			alors $O_j=O_j^\circ\subset(\bigcup_{i\in I}O_i)^\circ$
			donc $\bigcup_{j\in I}O_j\subset(\bigcup_{i\in I}O_i)^\circ$.
		\end{itemise}
		Posons $\cdot_\circ=\Phi(\mathscr{T}_\text{ouvert})$.
		Montrons que $X_\circ=X^\circ$.
		\begin{itemise}
			\item[$\supset$] $X^{\circ\circ}=X^\circ$ donc $X^\circ\in\mathscr{T}_\text{ouvert}$ et $X\supset X^\circ$ puis $X^\circ=\bigcup_{X\supset O\in\mathscr{T}_\text{ouvert}}O\supset X^\circ$.
			\item[$\subset$] Par croissance $X_\circ=\bigcup_{X\supset O\in\mathscr{T}_\text{ouvert}}O=\bigcup_{X^\circ\supset O^\circ\in\mathscr{T}_\text{ouvert}}O\subset X^\circ$.
		\end{itemise}
		\item Vérifions que $\Phi$ soit injective.
		Soit $\mathscr{T_\text{ouvert}}$ et $\mathscr{T}_\text{ouvert}'$ tels que $\Phi(\mathscr{T}_\text{ouvert})=\Phi(\mathscr{T}_\text{ouvert}')$.
		Soit $O\in\mathscr{T}_\text{Ouvert}$ alors $O=\Phi(\mathscr{T}_\text{ouvert})(O)=\Phi(\mathscr{T}_\text{ouvert}')(O)=\bigcup_{O\supset O'\in\mathscr{T}_\text{ouvert}'}O'\in\mathscr{T}_\text{ouvert}'$.
	\end{itemise}
\end{demonstration}

\section{Définitions}

La définition élégante d'une topologie par les ouverts est celle donnée par Bourbaki.
En pratique on utilise la caractérisation suivante, provenant de la convention qu'une intersection d'aucune partie de $E$ est $E$ lui même, et que l'union d'aucune partie de $E$ est $\vide$.
Un espace topologique est un ensemble munie d'une topologie.

\begin{definition}
	Une topologie $\mathscr{T}$ est aussi définie par les propriétés:
	\begin{itemise}
		\item $E$ et $\vide$ sont dans $\mathscr{T}$
		\item Si $O$ et $O'$ sont dans $\mathscr{T}$ alors $O\cap O'$ aussi.
		\item Si les $O_i$ sont dans $\mathscr{T}$ alors $\bigcup_iO_i$ aussi.
	\end{itemise}
\end{definition}

\begin{exemple}
	Voici quelques topologies dont on peut toujours munir un ensemble quelconque $E$:
	\begin{itemise}
		\item la topologie grossière $\mathscr{T}=\{\vide, E\}$.
		\item la topologie discrète $\mathscr{T}=\mathfrak{P}(E)$.
		\item la topologie cofinie $\mathscr{T}=\{\vide\}\cup\{O\subset E\ |\ \complement O\text{ fini}\}$.
		Démontrons que c'est bien une topologie sur $E$.
		Par définition $\vide\in\mathscr{T}$ et $\complement E=\vide$ est fini donc $E\in\mathscr{T}$.
		Si $O$ et $O'$ sont dans $\mathscr{T}$ alors $\complement O$ et $\complement O'$ sont finis donc $\complement O\cup\complement O'=\complement(O\cap O')$ est fini donc $O\cap O'$ est dans $\mathscr{T}$.
		Si les $O_i$ sont dans $\mathscr{T}$ alors $\complement O_i$ sont finis donc $\bigcap_i\complement O_i=\complement\bigcup_iO_i$ est fini donc $\bigcup_iO_i$ est dans $\mathscr{T}$.
	\end{itemise}
\end{exemple}

\section{Système fondamental de voisinage, Base d'une topologie}

\begin{definition}
	On appelle système fondamentale de voisinage d'un point $x$ un ensemble de voisinages $\mathfrak{S}$ tel que $\forall V\in\mathfrak{V}(x), \exists W\in\mathfrak{S}, W\subset V$. On peut de même définir un système fondamental de voisinage d'une partie de $E$.
\end{definition}

\begin{exemple}
	\begin{itemise}
		\item Dans un espace discret, l'ensemble $\{x\}$ (qui est ouvert donc un voisinage) est une base fondamentale de voisinage de $x$.
		\item Sur la droite rationnelle $[q-\frac{1}{n}, q+\frac{1}{n}]$ ou $]q-\frac{1}{n}, q+\frac{1}{n}[$ avec $n\in\N^*$ sont des systèmes fondamentaux de voisinages de $q\in\Q$.
		\item Il en va de même pour le droite réelle.
	\end{itemise}
\end{exemple}

\begin{definition}
	On appelle base de la topologie $\mathscr{T}$ toute collection $\mathfrak{B}\subset\mathfrak{P}(E)$ telle que tout ouvert soit réunion d'éléments de $\mathfrak{B}$, id est $\forall O\in\mathscr{T}, \exists X_{i\in I}\in\mathfrak{B}, O=\bigcup_{i\in I} X_i$.
	On dit que $\mathfrak{B}$ engendre par union les ouverts.
\end{definition}

\begin{proposition}
	Il existe une topologie dont $\mathfrak{B}$ est une base si et seulement si:
	\begin{itemise}
		\item $\mathfrak{B}$ recouvre $E$ c'est à dire $E=\bigcup_{X\in\mathfrak{B}}X$
		\item $\mathfrak{B}$ engendre ses intersections finies id est $X, Y\in\mathfrak{B}\implies\exists Z_{i\in I}\in\mathfrak{B}, X\cap Y=\bigcup_{i\in I}Z_i$
	\end{itemise}
	Cette topologie $\mathscr{T}$ est alors unique et ses ouverts sont les réunions d'éléments de $\mathfrak{B}$.
\end{proposition}

\begin{demonstration}
	\begin{itemise}
		\item[$\Rightarrow$] Pour le sens direct, si $\mathfrak{B}$ est une base d'une topologie $\mathscr{T}$ quelconque, alors les deux conditions sont vérifiée par application directe des propriétés de la topologie.
		Puisque $\mathscr{T}$ doit être stable par union, alors c'est la seul topologie engendrée par $\mathfrak{B}$.
		\item[$\Leftarrow$] Soit $\mathfrak{B}$ vérifiant les propriétés.
		\begin{itemise}
			\item $E=\bigcup_{X\in\mathfrak{B}}X$ et $\vide=\bigcup_{X\in\vide}X$ donc $\vide, E\in\mathscr{T}$.
			\item Si $X_i=\bigcup_{j\in J_i}Z_j\in\mathscr{T}$ alors $\bigcup_{i\in I}X_i=\bigcup_{i\in I}\bigcup_{j\in J_i}Z_j\in\mathscr{T}$.
			\item Si $(X, Y)=(\bigcup_{i\in I}Z_i, \bigcup_{j\in J}Z_j)\in\mathscr{T}$ alors $Z_i\cap Z_j=\bigcup_{k\in K_{ij}}Z_k$ car $Z_\bullet\in\mathfrak{B}$ donc $X\cap Y=\bigcup_{i\in I}\bigcup_{j\in J}\bigcup_{k\in K_{ij}}Z_k\in\mathscr{T}$.
		\end{itemise}
	\end{itemise}
\end{demonstration}

\begin{proposition}
	L'intersection quelconque de topologies $(\mathscr{T}_i)_{i\in I}$ sur $E$ est une topologie.
\end{proposition}

\begin{demonstration}
	\begin{itemise}
		\item $E, \vide\in\mathscr{T}_{i\in I}$ donc $E, \vide\in\bigcap_{i\in I}\mathscr{T}_i$
		\item Si $U, V\in\bigcap_{i\in I}\mathscr{T}_i$ alors $U, V\in\mathscr{T}_i, \forall i\in I$ donc $U\cap V\in\mathscr{T}_i, \forall i\in I$ puis $U\cap V\in\bigcap_{i\in I}\mathscr{T}_i$
		\item Si $U_{j\in J}\in\bigcap_{i\in I}\mathscr{T}_i$ alors $U_{j\in J}\in\mathscr{T}_i, \forall i\in I$ donc $\bigcup_{j\in J} U_j\in\mathscr{T}_{i\in I}, \forall i\in I$ puis $\bigcup_{j\in J} U_j\in\bigcap_{i\in I}\mathscr{T}_i$
	\end{itemise}
\end{demonstration}

\begin{definition}
	On pose donc $\sigma_\text{topologie}(\mathfrak{A})=\bigcap_{\mathfrak{A}\subset\mathscr{T}\text{topologie}}\mathscr{T}$ la plus petite topologie contenant $\mathfrak{A}$ nommée topologie engendré par $\mathfrak{A}$.
\end{definition}

\begin{remarque}
	Il est facile de vérifier que si $\mathfrak{B}$ est une base de la topologie $\mathscr{T}$ alors $\mathscr{T}=\sigma_\text{topologie}(\mathfrak{B})$.
\end{remarque}


\section{Premières propriétés}

\begin{proposition}
	\begin{itemise}
		\item $X^\circ=\complement\overline{\complement X}$
		\item $\overline{X}=\complement(\complement X)^\circ$
		\item $\overline{X\cap Y}\subset\overline{X}\cap\overline{Y}$
		\item $(X\cup Y)^\circ\supset X^\circ\cup Y^\circ$
	\end{itemise}
\end{proposition}

\begin{demonstration}
	$X^\circ=\bigcup_{X\supset O\in\mathscr{T}}O=\complement\bigcap_{\complement X\subset\complement O\in\complement\mathscr{T}}\complement O=\complement\overline{\complement X}$.

	$\overline{X}=\bigcap_{X\subset F\in\complement\mathscr{T}}O=\complement\bigcup_{\complement X\subset\complement O\in\mathscr{T}}\complement F=\complement(\complement X)^\circ$.

	$X\cap Y\subset\overline{X}\cap\overline{Y}$ qui est fermé donc par minimalité.

	$X\cup Y\supset X^\circ\cup Y^\circ$ qui est ouvert donc par maximalité.
\end{demonstration}

\begin{remarque}
	\begin{itemise}
		\item Une intersection infinie d'ouverts peut ne pas être un ouvert : $\bigcap_{x>0}]-x,x[=\{0\}$.
		\item Une union infinie de fermés peut ne pas être fermée : $\bigcup_{0<x<1}[x,1]=]0, 1]$.
		\item les inclusions précédentes peuvent être strictes : $\overline{]-\infty, 0[\cap]0, +\infty[}=\overline{\vide}=\vide$ tandis que  $\overline{]-\infty, 0[}\cap\overline{]0, +\infty[}=]-\infty, 0]\cap\overline[0, +\infty[=\{0\}$.
	\end{itemise}
\end{remarque}

\begin{proposition}
	$\overline{X}=\{x\in E| \forall V\in\mathfrak{V}(x), V\cap X\neq\vide\}$
\end{proposition}

\begin{demonstration}
	$\overline{X}
		=\complement(\complement X)^\circ
		=\complement\bigcup_{\complement X\supset O\in\mathscr{T}}O
		=\complement\{x\in E| \exists O\in\mathscr{T}, x\in O\subset\complement X\}
		=\complement\{x\in E| \exists V\in\mathfrak{V}(x), V\cap X=\vide\}
		=\{x\in E| \forall V\in\mathfrak{V}(x), V\cap X\neq\vide\}$
\end{demonstration}

\begin{definition}
	On dit que $X$ est dense dans $E$ si et seulement si $\overline{X}=E$.
\end{definition}

\begin{proposition}
	$X$ est dense si et seulement si les ouverts non vides rencontrent $X$
\end{proposition}

\begin{demonstration}
	La dernière équivalence vient de l'indépendance de $O\cap X\neq\vide$ à $x$.
	$$
		\begin{aligned}
			\overline{X}=E
				&\Leftrightarrow(\forall x\in E, \forall V\in\mathfrak{V}(x), V\cap X\neq\vide)\\
				&\Leftrightarrow(\forall x\in E, \forall O\in\mathscr{T}, x\in O\Rightarrow O\cap X\neq\vide)\\
				&\Leftrightarrow(\forall O\in\mathscr{T}, \forall x\in E, x\in O\Rightarrow O\cap X\neq\vide)\\
				&\Leftrightarrow(\forall O\in\mathscr{T}, O\neq\vide\Rightarrow O\cap X\neq\vide)
		\end{aligned}
	$$
\end{demonstration}


\section{Continuité}

Les espaces $E, F$ sont munis des topologies $\mathscr{T}, \mathscr{U}$ des voisinages $\mathfrak{V}, \mathfrak{W}$.
On se donne une application $f\colon E\rightarrow F$.

\begin{definition}
	$f$ est continue en $x$ si et seulement si tout voisinage de $f(x)$ inclut l'image par $f$ d'un voisinage de $x$ id est $\forall W\in\mathfrak{W}(f(x)), \exists V\in\mathfrak{V}(x) f(V)\subset W$.
\end{definition}

\begin{proposition}
	$f$ est continue en $x$ si et seulement si $\forall W\in\mathfrak{W}(f(x)), \inv{f}(W)\in\mathfrak{V}(x)$.
\end{proposition}

\begin{demonstration}
	Par double implication.
	\begin{itemise}
		\item[$\Rightarrow$] Supposons $f$ continue.
		Soit $W\in\mathfrak{W}(f(x))$, alors par continuité de $f$, $\exists V\in\mathfrak{V}(x), f(V)\subset W$.
		Ainsi $V\subset\inv{f}(f(V))=\inv{f}(W)$.
		Donc $W$ est un voisinage de $x$.
		\item[$\Leftarrow$] Soit $x\in E$ et $W\in\mathfrak{W}(f(x))$.
		Alors $\inv{f}(W)\in\mathfrak{V}(x)$, et $f(\inv{f}(W))=W\subset W$, donc $f$ est continue en $x$.
	\end{itemise}
\end{demonstration}

\begin{proposition}
	Si $f$ est continue en $x$ et $g$ en $f(x)$ alors $g\circ f$ l'est en $x$.
\end{proposition}

\begin{demonstration}
	Soit $W$ un voisinage de $(g\circ f)(x)=g(f(x))$.
	Par continuité de $g$, $\inv{g}(W)$ est un voisinage de ouvert de $f(x)$.
	Par continuité de $f$, $\inv{f}(\inv{g}(W))=\inv{(g\circ f)}(W)$ est un voisinage de $x$.
\end{demonstration}

\begin{definition}
	Soient $E$ et $F$ deux espaces topologiques.
	L'application $f\colon E\rightarrow F$ est continue si et seulement si elle est continue tout point.
	L'ensemble des applications continues de $E$ dans $F$ est noté $\mathcal{C}(E, F)$.
	%S'il n'y a pas d'ambiguïté, on notera $f\in\cont(E, F)$ sans préciser les topologies.
\end{definition}

\begin{proposition}
	Les propositions suivantes sont équivalentes:
	\begin{itemise}
		\item $f$ est continue; $f\in\cont(E, F)$
		\item $\forall X\in\mathfrak{P}(E), f(\overline{X})\subset\overline{f(X)}$
		\item l'image réciproque d'un fermé est fermée; $\forall F\in\complement\mathscr{U}, \inv{f}(F)\in\complement\mathscr{T}$
		\item l'image réciproque d'un ouvert est ouverte; $\forall O\in\mathscr{U}, \inv{f}(O)\in\mathscr{T}$
	\end{itemise}
\end{proposition}

\begin{demonstration}
	Par implication circulaire:
	\begin{itemise}
		\item Soit $x\in\overline{X}$ et $W\in\mathfrak{W}(f(x))$.
		Par continuité $\inv{f}(W)\in\mathfrak{V}(x)$ donc par adhérence de $x$ $\exists y\in\inv{f}(W)\cap X\neq\vide$.
		Ainsi $f(y)\in W\cap f(X)$ soit $W\cap f(X)\neq\vide$ donc $f(x)\in\overline{f(X)}$.
		\item Soit $G\in\complement\mathscr{U}$ et $F=\inv{f}(G)$.
		Alors $f(\overline{F})\subset\overline{f(F)}\subset\overline{G}=G$ donc $\overline{F}\subset\inv{f}(G)=F$.
		\item Il suffit d'utiliser $\inv{f}(O)=\inv{f}(\complement F)=\complement\inv{f}(F)\in\complement\mathscr{T}$.
		\item Soit $x\in E$ et $W\in\mathfrak{W}(f(x))$.
		Alors $\exists O\in\mathscr{U}, f(x)\in O\subset W$.
		Ainsi $x\in\inv{f}(O)$ et par hypothèse $\inv{f}(O)\in\mathscr{T}$ donc $\inv{f}(O)\in\mathfrak{V}(x)$.
	\end{itemise}
\end{demonstration}

\begin{proposition}
	Si $f$ est continue alors $\forall X\in\mathfrak{P}(E), \overline{\inv{f}(X)}\subset\inv{f}(\overline{X})$.
\end{proposition}

\begin{demonstration}
	Par continuité de $f$, $\inv{f}(\overline{X})$ est un fermé contenant $\inv{f}(X)$.
	Par minimalité de $\overline{\inv{f}(X)}$ on obtient le résultat.
\end{demonstration}

\begin{exemple}
	Il faut cependant ce méfier :
	\begin{itemise}
		\item L'image directe d'un ouvert par une application continue n'est pas nécessairement ouvert (on dit alors que $f$ est ouverte).
		Par exemple l'image de $\R$ par $x\mapsto\frac{1}{1+x^2}$ est $]0,1]$ qui n'est ni fermé ni ouvert.
		\item Une bijection continue n'est pas nécessairement à réciproque continue (on dit bicontinue) comme le montre l'identité de $\Q$ muni la topologie discrète dans la droite rationnelle.
	\end{itemise}
\end{exemple}

\section{Comparaison de topologies}

Soient $(E, \mathscr{T}_1), (E, \mathscr{T}_2), (F, \mathscr{U})$ des espace topologiques et $f\colon E\rightarrow F$.

\begin{definition}
	On dit que $\mathscr{U}$ est plus fine que $\mathscr{T}$ si l'identité $\id\colon(E, \mathscr{U})\rightarrow(E, \mathscr{T})$ est continue.
\end{definition}

\begin{proposition}
	Les propositions suivantes sont équivalentes:
	\begin{itemise}
		\item $\mathscr{U}$ est plus fine que $\mathscr{T}$.
		\item $\mathscr{T}\subseteq\mathscr{U}$.
		\item $\id\colon(E, \mathscr{T})\rightarrow(E, \mathscr{U})\text{ ouverte}$.
		\item tout voisinage de $x$ pour $\mathscr{T}$ est un voisinage de $x$ pour $\mathscr{U}$.
		\item l'adhérence de $X$ pour $\mathscr{U}$ est contenue dans l'adhérence de $X$ pour $\mathscr{T}$.
		\item un fermé de $\mathscr{T}$ est fermé pour $\mathscr{U}$.
		\item un ouvert de $\mathscr{T}$ est ouvert pour $\mathscr{U}$.
	\end{itemise}
\end{proposition}

\begin{demonstration}
	Toutes ces propriétés découlent du théorème précédent, à l'exception de l'inclusion.
	Or si $O\in\mathscr{T}$ par continuité de $\id\colon(E, \mathscr{U})\rightarrow(E, \mathscr{T})$, $\inv{\id}(O)=O\in\mathscr{U}$.
\end{demonstration}

\begin{remarque}
	Il est possible qu'entre deux topologies, aucune ne soit plus fine que l'autre, puisqu'il est possible qu'aucune des deux n'inclut l'autre.
	On dit alors qu'elles ne sont pas comparables.
\end{remarque}

\begin{proposition}
	\begin{itemise}
		\item $f$ reste continue si l'on remplace la topologie sur $F$ par une topologie moins fine ou si on remplace la topologie sur $E$ par une topologie plus fine.
		\item $f$ reste ouverte si l'on remplace la topologie sur $F$ par une topologie plus fine ou si on remplace la topologie sur X par une topologie moins fine.
	\end{itemise}
\end{proposition}

\begin{demonstration}
	Elle est immédiate.
	De façon imagé, plus la topologie d'arrivée est fine, plus elle possède d'ouverts, plus il y a de propriétés à vérifier pour être une application continue.
\end{demonstration}

\begin{proposition}
	Si $f\colon E\rightarrow F$ est une application continue qui est soit ouverte, soit fermée, alors:
	\begin{itemise}
		\item si f est une surjection, f est une application quotient, c'est-à-dire que Y a la topologie la plus fine pour laquelle f est continue ;
		\item si f est une injection, c'est un plongement topologique, c'est-à-dire que X et f(X) sont homéomorphes (par f) ;
		\item si f est une bijection, c'est un homéomorphisme.
	\end{itemise}
	Dans les deux premiers cas, être ouvert ou fermé n'est qu'une condition suffisante ; c'est également une condition nécessaire dans le dernier cas.
\end{proposition}

\begin{demonstration}
	\textcolor{red}{???}
\end{demonstration}

\chapter{Topologies de références}

\section{Trousse à topologies}

Soit $E$ est un espace topologique et $F$ une partie de $E$.

\begin{proposition}
	La topologie cofinie sur $E$ est $\mathscr{T}=\{\vide\}\cup\{X\subset E|\complement X\text{ fini}\}$.
\end{proposition}

\begin{demonstration}
	\begin{itemise}
		\item $\vide\in\mathscr{T}$ et $\complement E=\vide$ est fini donc $E\in\mathscr{T}$.
		\item Si $X_{i\in I}\in\mathscr{T}$ alors $\complement\bigcup_{i\in I}X_i=\bigcap_{i\in I}\complement X_i$ est fini par intersection de fini.
		\item Si $X_{1\infegal i\infegal n}\in\mathscr{T}$ alors $\complement\bigcap_{1\infegal i\infegal n}X_i=\bigcup_{1\infegal i\infegal n}\complement X_i$ est fini par union finie de finis.
	\end{itemise}
\end{demonstration}

\begin{proposition}
	La topologie codén\up{ble} sur $E$ est $\mathscr{T}=\{\vide\}\cup\{X\subset E|\complement X\text{ dénombrable}\}$.
\end{proposition}

\begin{demonstration}
	\begin{itemise}
		\item $\vide\in\mathscr{T}$ et $\complement E=\vide$ est fini donc dénombrable donc $E\in\mathscr{T}$.
		\item Si $X_{i\in I}\in\mathscr{T}$ alors $\complement\bigcup_{i\in I}X_i=\bigcap_{i\in I}\complement X_i$ dén\up{ble} par intersection de dénombrables.
		\item Si $X_{1\infegal i\infegal n}\in\mathscr{T}$ alors $\complement\bigcap_{1\infegal i\infegal n}X_i=\bigcup_{1\infegal i\infegal n}\complement X_i$ dén\up{ble} par union finie de dén\up{ble}.
	\end{itemise}
\end{demonstration}

\begin{proposition}
	la topologie classique sur $\R$ est $\mathscr{T}=\{]a, b[|a<b\}$
\end{proposition}

\section{Topologie induite}

\begin{proposition}
	Les ouverts de la topologie induite par $E$ sur $F$ sont les traces sur $F$ des ouverts de $E$.
\end{proposition}

\begin{remarque}
	Cette notion est d’usage fréquent, c’est en particulier à elle qu’on fait appel pour définir la continuité d’une fonction sur un intervalle fermé $[a, b]$ de $\R$.
	On dira que f est continue sur $[a, b]$ si elle est continue pour la toplogie induite par $\R$ sur cet intervalle, topologie pour laquelle $[a, b]$ est en fait ouvert !
	Les notions de continuité à droite ou à gauche aux bornes de l’intervalle font alors partie intégrante de la notion générale de continuité sur l’espace topologique $[a, b]$.
\end{remarque}

\section{Topologie quotient}
Soit $E, F$ topologiques, $\mathcal{R}$ une relation d’équivalence sur $E$, $\pi\colon E\rightarrow E/\mathcal{R}$ la surjection canonique et $f\colon E/\mathcal{R}\rightarrow F$ une application.

\begin{definition}
	La topologie quotient sur $E/\mathcal{R}$ est formée des ouverts $O$ tels que $\inv{\pi}(O)$ soit un ouvert de $E$.
\end{definition}

\begin{proposition}
	La topologie quotient est la topologie finale associée à $\pi$, c'est à dire la plus fine rendant $\pi$ continue.
\end{proposition}

\begin{demonstration}
	$\pi$ est continue par construction.
	Soit $\mathscr{U}$ une topologie sur $E/\mathcal{R}$ rendant $\pi$ continue.
	Si $O\in\mathscr{U}$ alors $\inv{\pi}(O)$ est un ouvert de $E$ par continuité de $\pi$, donc $O$ est dans la topologie quotient, qui est donc plus fine que $\mathscr{U}$.
\end{demonstration}

\begin{proposition}
	$f\in\cont(E/\mathcal{R}, F)\iff f\circ\pi\in\cont(E, F)$
\end{proposition}

\begin{demonstration}
	Si $O$ est un ouvert de $F$, alors $\inv{f}(O)$ est un ouvert de $E/\mathcal{R}$ si et seulement si $\inv{\pi}(\inv{f}(O))$ est ouvert dans $E$, c'est à dire $\inv{(f\circ\pi)}(O)$.
\end{demonstration}

\section{Topologie produit}

Soient $E, (F_i, \mathscr{U}_i)_{i\in I}$ des espaces topologiques.
On note $F=\prod_{i\in I} F_i$ le produit cartésien des $F_i$ et $\pi_i\colon F\rightarrow F_i$ les projections canoniques.

\begin{definition}
	On appelle topologie produit la topologie initiale associées aux projections canoniques $\pi_i\colon F\rightarrow F_i$ c'est à dire la topologie la moins fine rendant les $\pi_i$ continues.
\end{definition}

\begin{proposition}
	L'ensemble $\mathfrak{B}=\{\prod_{k=1}^nO_{i_k}\times\prod_{j\in I\setminus\{i_1, ..., i_n\}}F_j|n\in\N^*, O_{i_k}\in\mathscr{U}_{i_k}\}$ des pavés formés d'ouverts différents de $F_i$ dans un ensemble $\{i_1,...,i_n\}$ fini de directions est une base de topologie de $F$.
\end{proposition}

\begin{demonstration}
	Pour assurer la continuité des projections, la topologie produit $\mathscr{U}$ doit contenir toutes ces images réciproque $\inv{\pi_i}(O_i)=O_i\times\prod_{i\neq j\in I}F_j$.
	Elle inclue donc $\mathfrak{A}=\{O_i\times\prod_{i\neq j\in I}F_j|i\in I, O_i\in\mathscr{U}_i\}$.
	En tant que topologie, $\mathscr{U}$ est stable par intersection finie.
	Elle inclue donc $\mathfrak{B}=\{\bigcap_{k=1}^nA_k|n\in\N^*, A_k\in\mathfrak{A}\}$.
	Soit $B=\bigcap_{k=1}^nA_k\in\mathfrak{B}$.
	Certain $A_k=O_k\times\prod_{i_k\neq j\in I}F_j$ ont des $i_k$ égaux, c'est à dire des $O_k$ venant d'un même topologie.
	Soit $i'_{1\infegal l\infegal m}$ une renumérotation unique des topologies visités et $K_l=\{k|i_k=i'_l\}$ l'ensemble des indices dont les ouverts proviennent du même espace $F_{i'_l}$.
	Alors $B=\bigcap_{l=1}^m\bigcap_{k\in K_l}A_l$ avec chaque intersection finie.
	De même $O'_l=\bigcap_{k\in K_l}O_k\in\mathscr{U}_{i'_l}$ et
	$$
		\bigcap_{k\in K_l}A_l
		=\bigcap_{k\in K_l}(O_k\times\prod_{i_k\neq j\in I}F_j)
		=\bigcap_{k\in K_l}(O_k\times\prod_{i'_l\neq j\in I}F_j)
		=(\bigcap_{k\in K_l}O_k)\times\prod_{i'_l\neq j\in I}F_j
		=O'_l\times\prod_{i'_l\neq j\in I}F_j
	$$
	Ainsi $B=\bigcap_{l=1}^m(O'_l\times\prod_{i'_l\neq j\in I}F_j)=\prod_{l=1}^mO'_l\times\prod_{j\in I\setminus\{i'_1,...,i'_m\}}F_j$.
	La topologie produit inclue donc bien la famille $\mathfrak{B}$ annoncée.
	C'est bien une base puisque pour $n=0$ on obtient $F\in\mathfrak{B}$ et qu'elle est par contruction stable par intersection finie.
\end{demonstration}

\begin{proposition}
	$f\in\cont(E, F)\iff\forall i\in I, \pi_i\circ f\in\cont(E, F_i)$
\end{proposition}

\begin{demonstration}
	Par double implication.
	\begin{itemise}
		\item[$\Rightarrow$] Soit $O_i$ un ouvert de $F_i$.
		$\pi_i$ est continue donc $\inv{\pi_i}(O_i)$ est un ouvert de $F$.
		$f$ est continue donc $\inv{f}(\inv{\pi_i}(O_i))=\inv{(\pi_i\circ f)}(O_i)$ est un ouvert de $E$.
		\item[$\Leftarrow$] Soit $O$ un ouvert de $F$ alors $O=\bigcup_{j\in J}\bigcap_{k=1}^n\inv{\pi_{i_{jk}}}(O_{j_k})$ d'après la base trouvée précédemment.
		$\inv{f}(O)=\bigcup_{j\in J}\bigcap_{k=1}^n\inv{f}(\inv{\pi_{i_{jk}}}(O_{jk}))=\bigcup_{j\in J}\bigcap_{k=1}^n\inv{(\pi_{i_{jk}}\circ f)}(O_{jk})$ est une union d'intersection finie d'ouverts de $E$, donc est un ouvert.
	\end{itemise}
\end{demonstration}

\chapter{Séparation}

On considère ici un espace topologique $(E, \mathscr{T})$.

\section{Vocabulaire}

\begin{definition}
	Il existe de nombreuses notions de séparation :
	\begin{itemise}
		\item $T_0:\forall x, y\in E, x\neq y\Rightarrow\exists O\in\mathscr{T}, (x\in O\et y\notin O)\ou(y\in O\et x\notin O)$
		\item $T_1:\forall x, y\in E, x\neq y\Rightarrow\exists U\in\mathscr{T}, x\in U\et y\notin U$
		\item $T_2:\forall x, y\in E, x\neq y\Rightarrow\exists U, V\in\mathscr{T}, x\in U\et y\in V\et U\cap V=\vide$
		\item $T_{2\frac{1}{2}}:\forall x, y\in E, x\neq y\Rightarrow\exists U, V\in\mathscr{T}, x\in U\et y\in V\et\overline{U}\cap\overline{V}=\vide$
		\item $T_{2\frac{3}{4}}:\forall x, y\in E, x\neq y\Rightarrow\exists f\in\cont(E, [0, 1]), f(x)=0\et f(y)=1$
		\item $T_3:\forall x, F\in E\times\complement\mathscr{T}, x\notin F\Rightarrow\exists U, V\in\mathscr{T}, x\in U\et F\subset V\et U\cap V=\vide$
		\item $T_{3\frac{1}{2}}:\forall x, F\in E\times\complement\mathscr{T}, x\notin F\Rightarrow\exists f\in\mathcal{C}(E, [0, 1]), f(x)=1\et f(F)=\{0\}$
		\item $T_4:\forall F, G\in\complement\mathscr{T}, F\cap G=\vide\Rightarrow\exists U, V\in\mathscr{T}, F\subset U\et G\subset V\et U\cap V=\vide$
		\item $T_5:\forall X, Y\subset E, X\cap\overline{Y}=\overline{X}\cap Y=\vide\Rightarrow\exists U, V\in\mathscr{T}, X\subset U\et Y\subset V\et U\cap V=\vide$
		\item $G:\text{tout fermé est un $G_\delta$ (ouvert-intersection=Gebiet-Durchschnitt dénombrable)}$
	\end{itemise}
\end{definition}

\begin{proposition}
	$T_0$ est équivalente à $\forall x, y\in E, x\neq y\implies x\notin\overline{\{y\}}\ou y\notin\overline{\{x\}}$
\end{proposition}

\begin{demonstration}
	$x\notin\overline{\{y\}}
		\iff\exists F\in\complement\mathscr{T}, y\in F\et x\notin F
		\iff\exists O\in\mathscr{T}, x\in O, y\notin O$
\end{demonstration}

\begin{proposition}
	$T_1$ est équivalente à :
	\begin{itemise}
		\item $\forall x, y\in E, x\neq y\implies\exists U, V\in\mathscr{T}, (x\in U\et y\notin U)\et(y\in V\et x\notin V)$.
		\item les singletons de $E$ sont des fermés.
		\item tout point de $E$ est l'intersection de ses voisinage.
	\end{itemise}
\end{proposition}

\begin{demonstration}
	Par implication circulaire.
	\begin{itemise}
		\item Soit $x$ et $y$ différents, on obtient le résultat en appliquant $T_1$ à $x, y$ puis à $y, x$.
		\item Soit $x$ et $y$ différents $\exists F_y\in\complement\mathscr{T}, x\in F_y\et y\notin F_y$, donc $\overline{\{x\}}\subset\bigcap_{y\neq x}F_y=\{x\}$.
		\item $\{x\}\subset\bigcap_{V\in\mathfrak{V}(x)}V\subset\bigcap_{V\in\mathfrak{V}(x)}\overline{V}\bigcap_{x\in F\in\complement\mathscr{T}}F=\overline{\{x\}}=\{x\}$ implique l'égalité partout.
		\item Soit $y\neq x$ on a $y\notin\{x\}=\bigcap_{V\in\mathfrak{V}(x)}V$ donc $\exists U\in\mathscr{T}, x\in U\et y\notin U$.
	\end{itemise}
\end{demonstration}

\begin{proposition}
	$T_2$ est équivalente à :
	\begin{itemise}
		\item tout point de $E$ est l'intersection de ses voisinage fermés.
		\item quelque soit $I$ la diagonale de $E^I$ est fermée.
		\item la diagonale de $E\times E$ est fermée.
	\end{itemise}
\end{proposition}

\begin{demonstration}
	Par implication circulaire.
	\begin{itemise}
		\item Si $y\neq x$ alors $\exists U, V\in\mathscr{T}, x\in U\et y\in V\et U\cap V=\vide$ donc $y\notin\overline{U}$.
		\item Soit $\Delta$ la diagonale.
		Si $(x_i)_{i\in I}\notin\Delta$ alors $\exists j, k\in I, x_j\neq x_k$, donc $\exists V_j, V_k\in\mathfrak{V}(x_j)\times\mathfrak{V}(x_k)$ fermés tels que $x_j\notin V_k\et x_k\notin V_j=\vide$.
		Ainsi $O_j=V_j^\circ\setminus V_k$ et $O_k=V_k^\circ\setminus V_j$ sont des ouverts disjoints contenant $x_j$ et $x_k$ respectivement.
		Donc $O_j\times O_k\times\prod_{i\in I\setminus\{j, k\}}E$ est un ouvert de $E^I$ ne rencontrant pas la diagonale et contenant $(x_i)_{i\in I}$.
		\item Il suffit de prendre $I=\{0, 1\}$.
		\item Soit $\Delta$ la diagonale.
		Si $x\neq y$ alors $(x, y)\notin\Delta$.
		Or $\Delta$ est fermée dans $E^2$, donc $\complement\Delta$ est un ouvert.
		Par définition de la base de la topologie produit, $\exists U_{i\in I}, V_{i\in I}\in\mathscr{T}, \complement\Delta=\bigcup_{i\in I} U_i\times V_i$.
		Puisque $(x, y)\notin\Delta$ je peux extraire du recouvrement $U, V$ tels que $(x, y)\in U\times V$, c'est à dire $x\in U\et y\in V$.
		Or $V\times W\subset\complement\Delta$, donc $U\cap V=\vide$.
	\end{itemise}
\end{demonstration}

\begin{proposition}
	$T_3$ équivaut à tout fermé de $E$ est l'intersection de ses voisinages ouverts.
\end{proposition}

\begin{demonstration}
	Par double implication.
	\begin{itemise}
		\item[$\Rightarrow$] Soit $F\in\complement\mathscr{T}$ et $x\notin F$ alors $\exists V_x\in\mathscr{T}, F\subset V_x\et x\notin V_x$.
		Donc $F\subset\bigcap_{x\notin F}V_x\subset F$.
		\item[$\Leftarrow$] Soit $F\in\complement\mathscr{T}$ et $x\notin F$.
		Supposons $\forall U, V\in\mathscr{T}, (x\in U\et F\subset V)\implies U\cap V\neq\vide$.
		Soit $U$ un ouvert contenant $x$.
		Par hypothèse $\bigcap_{F\subset V\in\mathscr{T}}V=F$.
		Donc $U\cap F=U\cap\bigcap_{F\subset V\in\mathscr{T}}V=\bigcap_{F\subset V\in\mathscr{T}}U\cap V\neq\vide$.
		Donc $x\in\widetilde{F}=F$ car $F$ est fermé, ce qui est absurde.
	\end{itemise}
\end{demonstration}

\begin{proposition}
	$T_4$ équivaut à $\forall O, F\in\mathscr{T}\times\complement\mathscr{T}, F\subset O\implies\exists U\in\mathscr{T}, F\subset U\subset\overline{U}\subset O$.
\end{proposition}

\begin{demonstration}
	Par double implication.
	\begin{itemise}
		\item[$\Rightarrow$] Soit $O$ un ouvert et $F\subset O$ un fermé.
		Puisque $\complement O$ est un fermé disjoint de $F$, $\exists U, V\in\mathscr{T}, F\subset U\et\complement O\subset V\et U\cap V=\vide$.
		Alors $F\subset u\subset\overline{U}=\bigcap_{U\subset F\in\complement\mathscr{T}}F\subset\complement V\subset O$.
		\item[$\Leftarrow$] Soient $F, G\in\complement\mathscr{T}$ disjoints.
		Puisque $\complement G$ est ouvert et $F\subset\complement G$ alors $\exists U\in\mathscr{T}, F\subset U\subset\overline{U}\subset\complement G$.
		Puisque $\complement\overline{U}$ est ouvert, que $G\subset\complement\overline{U}$ et que $U\cap\complement\overline{U}\subset U\cap\complement U=\vide$, $T_4$ est montré pour $V=\complement\overline{U}$.
	\end{itemise}
\end{demonstration}

\section{Classification}

\begin{proposition}
	\begin{center}
		\begin{tikzpicture}[xscale=2.5, yscale=1.5]
			\node (T0) at (0, 0) {$T_0$};
			\node (T1) at (1, 0) {$T_1$};
			\node (T2) at (2, 0) {$T_2$};
			\node (T21/2) at (3, 0) {$T_{2\frac{1}{2}}$};
			\node (T23/4) at (4, 0) {$T_{2\frac{3}{4}}$};
			\node (T3) at (0.5, 1) {$T_3$};
			\node (T31/2) at (1.5, 1) {$T_{3\frac{1}{2}}$};
			\node (T1+T4) at (2.5, 0.75) {$T_1+T_4$};
			\node (T0+T3) at (3.5, 1) {$T_0+T_3$};
			\node (T0+T31/2) at (4.5, 1) {$T_0+T_{3\frac{1}{2}}$};
			\node (T3+T4) at (2.5, 1.25) {$T_3+T_4$};
			\node (T4) at (0, 2) {$T_4$};
			\node (T5) at (1, 2) {$T_5$};
			\node (T2+T4+G) at (2, 2) {$T_2+T_4+G$};
			% implications
			\draw[<-] (T0) to[bend left=20] (T1);
			\draw[<-] (T1) to[bend left=20] (T2);
			\draw[<-] (T2) to[bend left=20] (T21/2);
			\draw[<-] (T21/2) to[bend left=20] (T23/4);
			\draw[<-] (T3)--(T31/2);
			\draw[<-] (T2)--(T1+T4);
			\draw[<-] (T21/2)--(T0+T3);
			\draw[<-] (T23/4)--(T0+T31/2);
			\draw[<-, green] (T31/2) to[bend right=20] (T1+T4);
			\draw[<-, green] (T31/2) to[bend left=20] (T3+T4);
			\draw[<-] (T4)--(T5);
			\draw[<-, green] (T5)--(T2+T4+G);
			\draw[<-, red] (T5) to[bend left=20] (T1+T4);
			% contre-exemples
			\draw[->, dashed] (T4) to[bend right=20] node {$\times$} (T0);
			\draw[->, dashed] (T3) to[bend right=20] node {$\times$} (T0);
			\draw[->, dashed] (T0) to[bend right=20] node {$\times$} (T1);
			\draw[->, dashed] (T1) to[bend right=20] node {$\times$} (T2);
			\draw[->, dashed] (T2) to[bend right=20] node {$\times$} (T21/2);
			\draw[->, dashed, green] (T21/2) to[bend right=20] node {$\times$} (T23/4);
			\draw[->, dashed] (T23/4) -- node {$\times$} (T3);
			\draw[->, dashed] (T0+T3) -- node {$\times$} (T31/2);
		\end{tikzpicture}
	\end{center}
\end{proposition}

\begin{demonstration}
	\begin{itemise}
		\item[$T_5\Rightarrow T_4$ et $T_{2\frac{1}{2}}\Rightarrow T_2\Rightarrow T_1\Rightarrow T_0$:] Évident
		\item[$T_1+T_4\Rightarrow T_{3\frac{1}{2}}$:] {\color{green} ???}
		\item[$T_3+T_4\Rightarrow T_{3\frac{1}{2}}$:] {\color{green} ???}
			{\color{red}
				\item[$T_1+T_4\Rightarrow T_5$:] Soient $X, Y\subset E$ tels que $X\cap\overline{Y}=\overline{X}\cap Y=\vide$ et $x\in X$.
				Par caractérisation de $T_1$, $\{x\}$ est fermé et $\{x\}\cap\overline{Y}\subset X\cap\overline{Y}=\vide$.
				Par $T_4$: $\exists U_x, V\in\mathscr{T}, \{x\}\subset U_x\et\overline{Y}\subset V\et U_x\cap V=\vide$.
				Ainsi $X=\bigcup_{x\in X}\{x\}\subset\bigcup_{x\in X}U_x$ est ouvert, $Y\subset\overline{Y}\subset V$ ouvert et $V\cap\bigcup_{x\in X}U_x=\bigcup_{x\in X}V\cap U_x=\vide$.
			}
		\item[$T_{3\frac{1}{2}}\Rightarrow T_3$:] Soient $x, F$ tels que $x\notin F$.
		Par $T_{3\frac{1}{2}}$: $\exists f\in\cont(E, [0, 1]), f(x)=1, f(F)=\{0\}$.
		Posons $U=\inv{f}(]1-\varepsilon, 1])$ et $V=\inv{f}([0, \varepsilon[)$ pour $0<\varepsilon<\frac{1}{2}$ .
		Puisque $[0, 1]$ est munie de la topologie trace de $\R$, $]1-\varepsilon, 1]$ et $[0, \varepsilon[$ sont des ouverts.
		Ainsi $U$ et $V$ sont des ouverts par continuité de $f$.
		De $f(x)=1$ et $f(F)=0$ on a $x\in\inv{f}(]1-\varepsilon, 1])=U$ et $F\subset\inv{f}([0, \varepsilon[)=V$.
		Pour conclure $U\cap V=\inv{f}(]1-\varepsilon, 1])\cap\inv{f}([0, \varepsilon[)=\inv{f}(]1-\varepsilon, 1]\cap[0, \varepsilon[)=\inv{f}(\vide)=\vide$.
		\item[$T_0+T_{3\frac{1}{2}}\Rightarrow T_{2\frac{3}{4}}$:] Soient $x, y\in E$ différents.
		Quitte à échanger le rôle de $x$ et $y$ on a par $T_0$: $\exists O\in\mathscr{T}, y\in O\et x\notin O$.
		Puisque $\complement O$ est fermé et $y\notin\complement O$, on a par $T_{3\frac{1}{2}}$: $\exists f\in\cont(E, [0, 1]), f(y)=1\et f(\complement O)=\{0\}$.
		Donc $f(x)=0$ et $f(y)=1$.
		\item[$T_0+T_3\Rightarrow T_{2\frac{1}{2}}$:] Soient $x, y\in E$ différents.
		Quitte à échanger $x$ et $y$ dans la formule de $T_0$ on a $\exists O\in\mathscr{T}, x\in O\et y\notin O$, soit $x\in O\not\ni y$.
		Puisque $x\notin\complement O$ et $\complement O$ est fermé $T_3$ donne :
		$\exists U, V\in\mathscr{T}, x\in U\et\complement O\subset V\et U\cap V=\vide$, équivalent à
		$x\in U\et\complement V\subset O\et U\subset\complement V$, c'est à dire
		$x\in U\subset\complement V\subset O\not\ni y$.
		De même puisque $x\notin\complement U$ et $\complement U$ est fermé $T_3$ donne :
		$\exists W, T\in\mathscr{T}, x\in W\et\complement U\subset T\et W\cap T=\vide$, équivalent à
		$x\in W\et\complement T\subset U\et W\subset\complement T$, c'est à dire
		$x\in W\subset\complement T\subset U\subset\complement V\subset O\not\ni y$.
		Puisque $\overline{W}$ est le plus petit fermé contenant $W$, $\overline{W}\subset\complement T$.
		De même $\overline{V}\subset\complement U$.
		Ainsi $x\in W, y\in V$ et $\overline{W}\cap\overline{V}\subset\complement T\cap\complement U=\vide$.
		\item[$T_1+T_4\Rightarrow T_2$:] Soient $x, y\in E$ différents.
		Par la caractérisation de $T_1$, les singletons $\{x\}$ et $\{y\}$ sont fermés.
		Par $T_4$: $\exists U, V\in\mathscr{T}, \{x\}\subset U\et\{y\}\subset V\et U\cap V=\vide$.
		\item[$T_{2\frac{3}{4}}\Rightarrow T_{2\frac{1}{2}}$:] Soient $x, y\in E$ différents.
		Par $T_{2\frac{3}{4}}$: $\exists f\in\cont(E, [0, 1]), f(x)=1, f(y)=0$.
		Posons $U=\inv{f}(]1-\varepsilon, 1])$ et $V=\inv{f}([0, \varepsilon[)$ pour $0<\varepsilon<\frac{1}{2}$ .
		Puisque $[0, 1]$ est munie de la topologie trace de $\R$, $]1-\varepsilon, 1]$ et $[0, \varepsilon[$ sont des ouverts.
		Ainsi $U$ et $V$ sont des ouverts par continuité de $f$.
		De $f(x)=1$ on a $x\in\inv{f}(]1-\varepsilon, 1])=U$ et de $f(y)=0$ on a $y\in\inv{f}([0, \varepsilon[)=V$.
		Pour conclure
		$\overline{U}\cap\overline{V}
			=\overline{\inv{f}(]1-\varepsilon, 1])}\cap\overline{\inv{f}([0, \varepsilon[)}
			\subset\inv{f}(\overline{]1-\varepsilon, 1]})\cap\inv{f}(\overline{[0, \varepsilon[})
			=\inv{f}([1-\varepsilon, 1]\cap[0, \varepsilon])
			=\inv{f}(\vide)
			=\vide$.
		\item[$T_0\not\Rightarrow T_1$:] $E=\{a, b\}$ munie de la topologie $\mathscr{T}=\{\vide, \{a\}, E\}$ est $T_0$ car $a\in\{a\}\not\ni b$ mais le seul ouvert contenant $b$ est $O=\{a, b\}$ et $a\in O$.
		\item[$T_3\not\Rightarrow T_0$ et $T_4\not\Rightarrow T_0$] un ensemble à plus de deux éléments munit de sa topologie grossière est $T_3$ et $T_4$ mais pas $T_0$.
		\item[$T_1\not\Rightarrow T_2$:] Un espace fini $T_1$ est discret.
		En effet, sa topologie doit pouvoir exclure chaque point, donc possède chaque singleton.
		Par stabilité par union, cette topologie est discrète.
		Un espace $T_1$ non $T_2$ est donc infini.
		La topologie cofinie sur un espace infinie est $T_1$ mais non $T_2$.
		En effet soit $X, Y$ deux ouverts cofinis voisins de $x$ et $y$ respectivement.
		Alors $\complement(X\cap Y)=\complement X\cup\complement Y$ est fini, donc ne peut couvrir $E$ donc $X\cap Y\neq\vide$.
		\item[$T_2\not\Rightarrow T_{2\frac{1}{2}}$:] Soit un triangle équilatéral $T$ de sommets notés $S_i$.
		Considérons $E=T^\circ\cup\{S_1, S_2, S_3\}$.
		On le munit de la topologie trace de $\R^2$ sur $T^\circ$ à laquelle on ajoute les croissants $\{S_i\}\cup(T^\circ\cap B(\rho M_i, | \rho M_i-S_i|))$ avec $M_i=\frac{S_i+S_{i+1}}{2}$ le milieu des deux sommets consécutifs et $\rho>1$.
		On représente par exemple un voisinage du sommet $A$ en gris.
		\begin{center}
			\begin{tikzpicture}[scale=0.75]
				\draw[very thick] (1.0, 0.000) node[anchor=west]{$A$}
				-- (-0.5, 0.866025403784) node[anchor=east]{$C$}
				-- (-0.5,-0.866025403784) node[anchor=east]{$B$} -- cycle;
				\filldraw[black!20] (1.0, 0.000) circle (2pt);
				\filldraw[black!20] (-0.5, 0.866025403784) arc (210:270:1.75)--(-0.5, 0.866025403784) ;
			\end{tikzpicture}
		\end{center}
		Les points de $T^\circ$ sont séparés car la topologie de $\R^2$ est séparée.
		Les points de $\{S_1, S_2, S_3\}$ sont séparé de ceux dans $T^\circ$ en prenant $\rho$ suffisamment grand.
		Idem entre les points de $\{S_1, S_2, S_3\}$.
		Cette topologie est donc $T_2$.
		Elle n'est pas $T_{2\frac{1}{2}}$ car l'adhérence des croissants voisins de $S_i$ contiennent toujours $S_{i+1}$.
		\item[$T_0+T_3\not\Rightarrow T_{3\frac{1}{2}}$:] Considérons $E=(\R\times\R^+)\cup\{\infty\}$ munit:
		\begin{itemise}
			\item de la topologie discrète sur $\R\times\R_+^*$
			\item des voisinages $\{(x, 0)\}\cup R_x\setminus T_x$ de $(x, 0)$ où $R_x=V_x\cup O_x$ avec $V_x$ le segment verticale $\{(x, y)|y\in[0,2[\}$, $O_x$ l'oblique $\{(x+y, y)|y\in [0, 2[\}$ et $T_x$ un ensemble finis de « trous » dans $R_x$.
			\item des voisinages $U_n=\{\infty\}\cup(\{x>n\}\times\R^+)$ de $\infty$
		\end{itemise}
		Cet espace est $T_2$ comme le montre le premier schéma, puisqu'il suffit de placer des « trous » aux bons endroits dans $R_x$ ou de choisir $n$ suffisamment grand pour avoir des voisinages disjoints.
		\begin{center}
			\begin{tikzpicture}[scale=0.75]
				\filldraw[black!10, very thick] (5,0) rectangle (7,2.5);
				\draw[dashed] (5,0)--(5,2.5);
				\node[anchor=north] at (5,0) {$n$};
				\node at (6,1.5) {$\infty$};
				\draw[->] (-1, 0)--(6, 0);
				\draw[->] (0, -0.5)--(0, 2.5);
				\node[anchor=north] at (1,0) {$x_1$};
				\draw[black!20!blue, very thick] (1, 2)--(1, 0)--(3, 2);
				\filldraw[white] (1.0, 0.5) circle (2pt);
				\filldraw[white] (1.0, 1.5) circle (2pt);
				\filldraw[white] (1.0, 1.25) circle (2pt);
				\filldraw[white] (2.75, 1.75) circle (2pt);
				\filldraw[white] (2.3, 1.3) circle (2pt);
				\node[anchor=north] at (2.3,0) {$x_2$};
				\draw[black!20!red, very thick] (2.3, 2)--(2.3, 0)--(4.3, 2);
				\filldraw[white] (2.3, 0.7) circle (2pt);
				\filldraw[white] (2.3, 1.73) circle (2pt);
				\node[anchor=east] at (1.0, 1.5) {$x_3$};
				\filldraw[black!20!green] (1.0, 1.5) circle (1.25pt);
			\end{tikzpicture}
			\hfill
			\begin{tikzpicture}[scale=0.75]
				\node at (5,1.5) {$\infty$};
				\draw[->] (-1, 0)--(5, 0);
				\draw[->] (0, -0.5)--(0, 2.5);
				\draw[black!20!blue, very thick] (0.1, 2)--(0.1, 0)--(2.1, 2);
				\draw[black!20!blue, very thick] (0.5, 2)--(0.5, 0)--(2.5, 2);
				\draw[black!20!blue, very thick] (0.75, 2)--(0.75, 0)--(2.75, 2);
				\node[anchor=north] at (1,0) {$x$};
				\draw (1,-0.1)--(1,0.1);
				\draw[black!20!red, very thick] (1.5, 2)--(1.5, 0)--(3.5, 2);
				\draw[black!20!red, very thick] (1.75, 2)--(1.75, 0)--(3.75, 2);
				\draw[black!20!red, very thick] (2.3, 2)--(2.3, 0)--(4.3, 2);
			\end{tikzpicture}
		\end{center}

		Cet espace est $T_3$.
		Montrons $\forall x, O\in E\times\mathscr{T}, x\in O\Rightarrow\exists G\in\complement\mathscr{T}, x\in G\subset O$
		\begin{itemise}
			\item pour $x\in\R\times\R_+^*$ alors $F=\{x\}$ est un fermé vérifiant la propriété
			\item pour $x\in\R\times\{0\}$ comme le montre le second schéma $\bigcup_{x\notin V\in\mathscr{T}}V=E\setminus\{x\}$ est un ouvert en tant que réunion d'ouverts, donc $\{x\}$ fermé vérifie la propriété
			\item pour $x=\infty$ et $U_n$ un voisinage alors on remarque que seul les points de $]n+1, n+2]\times\{0\}$ hors de $U_{n+2}$ peuvent lui adhérer (grâce à branche à oblique des voisinages) donc $\overline{U_{n+2}}\subset U_n$ vérifie la propriété
		\end{itemise}
		En prenant $F=\complement O$, $U=O$ et $V=\complement G$ dans la définition de $T_3$ on conclue.

		Cet espace n'est pas $T_{3\frac{1}{2}}$.
		Soit $F=[0, 1]\times\{0\}$ et $f\in\cont(E, [0, 1])$ où $f(F)=\{0\}$.
		$\complement F=(\R\times\R_+^*)\cup\bigcup_{x\notin[0, 1]}R_x\in\mathscr{T}$ donc $F$ est fermé.
		Posons $C_0=F$ et montrons par récurrence $\forall n\in\N, \exists C_n\subset [n, n+1[\times\{0\}\text{ indénombrable}, f(C_n)=\{0\}$.
		Soit $c\in C_n$ alors $[0, \varepsilon[$ est ouvert pour la topologie trace de $\R$ sur $[0, 1]$.
		Par continuité $\inv{f}([0, \varepsilon[)$ est ouvert et contient $c$, donc $O_c\setminus T_c^\varepsilon$ avec $T_c^\varepsilon$ fini, puis $\inv{f}(\{0\})=\bigcap_{q\in\N^*}\inv{f}([0,\frac{1}{q}[)\supset\bigcap_{q\in\N^*}\bigcup_{c\in C_n}O_c\setminus T_c^{1/q}=\bigcup_{c\in C_n}O_c\setminus\bigcup_{q\in\N^*}\bigcap_{c\in C_n}T_c^{1/q}$ qui est indénombrable car $\bigcup_{c\in C_n}O_c$ l'est et $\bigcup_{q\in\N^*}\bigcap_{c\in C_n}T_c^{1/q}$ ne l'est pas.
		Sa projection $P_n$ sur l'axe des abscisses, et donc $C_{n+1}=P_n\cap[n+1, n+2[\times\{0\}$ est indénombrable.
		Soit $x\in C_{n+1}$ alors $x$ a « au dessus de lui » une infinité indénombrable de points d'annulation de $f$, donc adhère à $\inv{f}(\{0\})$ qui est fermé en tant qu'image réciproque d'un fermé par une application continue.
		Ainsi $f(C_{n+1})=0$ donc par récurrence, on a montré le lemme.
		Ainsi $\infty$ adhère à $\inv{f}(\{0\})$ donc $f(\infty)=0$.
		\item[$T_{2\frac{1}{2}}\not\Rightarrow T_3$ et $T_{2\frac{1}{2}}\not\Rightarrow T_{2\frac{3}{4}}$:]
		Soit $K=\{1/n|n\in\N^*\}$.
		On munie $\R$ de la la $K$-topologie dont une base est formée des $]a, b[$ et des $]a, b[\setminus K$.
						Elle est plus fine que la topologie classique, préservant donc la continuité des fonctions $f:\R\rightarrow[0, 1]$.
						En particulier, une fonction montrant la $T_{2\frac{3}{4}}$ séparation de $\R$ (par exemple une interpolation) montre que $(\R, K)$ l'est aussi.
						Cet espace n'est pas $T_3$.
						Soit $]a, 2[, ]-1, b[$ des ouverts contenant respectivement $K$ et $\{0\}$.
						De $0\in]-1, b[$ on a $b>0$.
						Si $a>0$ alors $\frac{1}{a}\infegal \lceil\frac{1}{a}\rceil=n\in\N^*$ donc $\frac{1}{n}<a$ et $K\ni\frac{1}{n}\notin]a, 2[$ absurde.
		Ainsi $a\infegal0<b$, forçant tout voisinages de la base de $0$ et $K$ à s'intersecter.
	\end{itemise}
\end{demonstration}

\section{Théorèmes de représentation}

Soit $(E, \mathscr{T})$ et $(F_i, \mathscr{U}_i)_{i\in I}$ des espaces topologiques $F=\prod_{i\in I}F_i$ leur produit cartésien munie de sa topologie produit $\mathscr{U}$ et $f_i:E\rightarrow F_i$ des applications continues.

\begin{definition}
	Si $\forall x, y\in E, x\neq y\Rightarrow\exists i\in I, f_i(x)\neq f_i(y)$ alors on dit que la famille $f_{i\in I}$ sépare les points.
\end{definition}

\begin{definition}
	Si $\forall x, F\in E\times\complement\mathscr{T}, x\notin F\Rightarrow\exists i\in I, f_i(x)\notin\overline{f_i(F)}$ alors on dit que la famille $f_{i\in I}$ sépare les points des fermés.
\end{definition}

\begin{proposition}
	Soit $e:E\rightarrow F$ telle que $e(x)_i=f_i(x)$.
	\begin{itemise}
		\item l'application $e$ est continue.
		\item l'application est injective si et seulement si $f_{i\in I}$ sépare les points de $E$.
		\item si la famille $f_{i\in I}$ sépare les points et les points des fermés alors $e$ est ouverte.
	\end{itemise}
\end{proposition}

\begin{demonstration}
	\begin{itemise}
		\item Contraposer l'injectivié de $e$ s'écrivant $e(x)=e(y)\Rightarrow x=y$ donne la séparation.
		\item Par définition, $\pi_i\circ e=f_i$  qui est continue, rendant $e$ continue.
		\item Soit $U\in\mathscr{T}$ non vide (sinon $e(U)=\vide$ évidement ouvert) et $x\in U$.
		Par séparation des fermés $\exists i\in I, f_i(x)\notin\overline{f_i(\complement U)}$.
		Ainsi $f_i(x)\in O_i=\complement\overline{f_i(\complement U)}\in\mathscr{U}_i$ puis $e(x)\in O_i\times\prod_{i\neq j\in I}F_i\in\mathscr{U}$.
		{\color{red} ???}
	\end{itemise}
\end{demonstration}

\begin{definition}
	Posons $Q=[0, 1]$ munie de la topologie $\{[0, t[|t\in[0, 1]\}$ appelée topologie supérieur.
	On appelle quasi-cube les espace de la forme $Q^I$ pour un certain $I\neq\vide$.
\end{definition}

\begin{proposition}
	L'espace $Q$ est $T_0$, quasi-compact mais pas $T_1$.
\end{proposition}

\begin{demonstration}
	\begin{itemise}
		\item Soit $x, y\in Q$ différents, supposons par exemple $y>x$.
		Alors pour $t=\frac{x+y}{2}$ on a $x\in[0, t[$ et $y\notin[0, t[$.
		\item Soit $U=\bigcup_{i\in I}[0, t_i[$ une réunion d'ouverts.
		Alors pour $t=\max_{i\in I}t_i$ on a $U=[0,t[$.
		\item L'intersection de deux ouverts contient toujours 0.
	\end{itemise}
\end{demonstration}

\begin{definition}
	Une application $f:E\rightarrow[0, 1]$ est semi-continue supérieurement si et seulement si $f:E\rightarrow Q$ est continue.
\end{definition}

\begin{proposition}
	Tout espace $T_0$ est homéomorphe à un sous-espace d'un quasi-cube.
\end{proposition}

\begin{demonstration}
	Soit $E$ un espace topologique $T_0$ et $f_{i\in I}:E\rightarrow [0, 1]$ la famille des applications semi-continues supérieurement de $E$.
	L'application $e:E\rightarrow \prod_{i\in I}Q$ définie par $e(x)_i=f_i(x)$ est continue d'après la proposition précédente.
	Soient $x, y\in E$ différents.
	Par $T_0$ quitte à échanger les rôles de $x$ et $y$, $\exists U\in\mathscr{T}, x\in U\not\ni y$.
	La fonction $\1_{\complement U}$ est semi-continue supérieurement car $\inv{\1_{\complement U}}([0, t[)=\inv{\1_{\complement U}}(\{0\})=U$ et sépare $x$ et $y$ car $\1_{\complement U}(x)=0\neq1=\1_{\complement U}(y)$.
	Soit $x, F\in E\times\complement\mathscr{T}$ tels que $x\notin F$.
	De même $\1_F$ est semi-continue supérieurement et sépare $x$ et $F$.
	Ainsi $f_{i\in I}$ sépare les points et les points des fermés.
	D'après la proposition précédente c'est une application ouverte et injective.
	D'après {\color{red} ???} c'est un homéomorphisme.
\end{demonstration}

\begin{remarque}
	D'après le théorème de Tychonoff, un produit d'espaces quasi-compacts est quasi-compact.
	En outre tout fermé d'un espace quasi-compact  est quasi-compact (l'inverse est faux, voir {\color{red} ???}).
	Ainsi si $E$ est $T_0$, $\overline{e(E)}$ (la fermeture de $e(E)$ dans $Q^F$) est un quasi-compact dans lequel $e(E)$ est dense: c'est une quasi-compactification de $E$.
\end{remarque}

\begin{remarque}
	Dans le cas d'un espace métrique séparable de distance $d$, il suffit de prendre $f_{n\in\N}=d(x_n, \cdot)$ avec $x_{n\in\N}$ une famille dénombrable dense.
	Cette famille satisfait aux hypothèses.
	Un espace métrique séparable est homéomorphe à un sous-espace de $[0, 1]^\N$.
\end{remarque}

\begin{proposition}
	Tout espace est $T_0+T_{3\frac{1}{2}}$ si et seulement s'il est homéomorphe à un sous-espace d'un cube $[0, 1]^I$.
\end{proposition}

\begin{demonstration}
	\begin{itemise}
		\item[$\Rightarrow$] Soit $E$ un espace topologique $T_0$ et $T_{3\frac{1}{2}}$ et $f_{i\in I}:E\rightarrow [0, 1]$ la famille des applications continues de $E$.
		L'application $e:E\rightarrow \prod_{i\in I}Q$ définie par $e(x)_i=f_i(x)$ est continue d'après la proposition précédente.
		Soient $x, y\in E$ différents.
		Par $T_0$ quitte à échanger les rôles de $x$ et $y$, $\exists U\in\mathscr{T}, x\in U\not\ni y$.
		Par $T_{3\frac{1}{2}}$ il existe une fonction continue $f$ telle que $f(x)=1$ et $f(\complement U)=\{0\}$ c'est à dire séparant $x$ et $y$.
		Soit $x, F\in E\times\complement\mathscr{T}$ tels que $x\notin F$.
		De même il existe une fonction continue et séparant $x$ et $F$.
		Ainsi $f_{i\in I}$ sépare les points et les points des fermés.
		D'après la proposition précédente c'est une application ouverte et injective.
		D'après {\color{red} ???} c'est un homéomorphisme.
		\item[$\Leftarrow$] Montrons d'abord que $\forall I\neq\vide, [0, 1]^I$ est $T_0+T_{3\frac{1}{2}}$.
		$[0, 1]$ est séparé car munie de sa topologie usuelle (métrique de $|\cdot|$).
		Par produit de séparés, $[0, 1]^I$ est séparé.
		Soit $x, F\in[0, 1]^I\times\complement\mathscr{T}$ tels que $x\notin F$.
		$O=\complement F$ est un ouvert de $[0, 1]^I$ contenant $x$.
		Par définition de la topologie produit et de la topologie usuelle de $[0, 1]$, il existe un ouvert $V$ tels que $x\in V\subset U$ et $V=\prod_{k=1}^n]x_{i_k}-\varepsilon, x_{i_k}+\varepsilon[\times\prod_{i\in I\setminus\{i_1, ..., i_n\}}[0, 1]$.
					Soit $f$ une fonction chapeau en $x_{i_1}$ nulle hors $]x_{i_1}-\varepsilon, x_{i_1}+\varepsilon[$.
		Alors $f\circ\pi_{i_1}$ montre que l'espace est $T_{3\frac{1}{2}}$.

		Soit maintenant un sous-espace $X$ de $[0, 1]^I$.
		Toute partie d'un espace séparé est elle-même séparé (on dit que $T_2$ est une propriété héréditaire) donc $X$ est $T_0+T_{3\frac{1}{2}}$.
	\end{itemise}
\end{demonstration}

\begin{remarque}
	D'après le théorème de Tychonoff, un produit d'espaces quasi-compacts est quasi-compact.
	En outre le produit d'espaces séparés est séparé.
	Ainsi si $E$ est $T_0+T_{3\frac{1}{2}}$ l'adhérence de $e(E)$ dans $[0, 1]^I$ est un compact dans lequel $e(E)$ est dense: c'est la compactification de Stone-Čech de $E$.
\end{remarque}


\section{Les pièges dans les espaces non séparés}

\textbf{Se méfier des sous-espaces fermés:}
Quand $E$ n'est pas séparé ($T_2$) certains espaces d'habitude fermé ne le sont plus: comme par exemple la diagonale $\Delta_E$.
Par ailleurs, certaines propriétés, vraies pour les espaces fermés et séparé ne la sont plus: dans un espace $T_2+T_4$, un sous-espace fermé est $T_2+T_4$.
Mais un sous-espace fermé d'un $T_4$ n'est pas nécessairement $T_4$...

La parade consiste à remplacer les sous-espace fermés par des rétracts: un sous-espace $X$ du topologique $E$ est un rétract s'il existe une application continue $f:E\rightarrow X$ telle que $f_{\vert X}=\id_{\vert X}$.
On dit alors que $f$ est une rétraction $E\rightarrow X$.

\begin{exemple}
	Si un produit $\prod_{i\in I}E_i$ d'espaces topologiques est $T_2+T_4$ alors chaque $E_i$ est $T_2+T_4$.
	Pour le démontrer on remarque que chaque $E_i$ est homéomorphe à un sous espace fermé du produit.
	Si les espaces ne sont pas séparés cette démonstration ne fonctionne plus.
	On peut néanmoins utiliser les rétracts: par les projections $\pi_i:\prod_{i\in I}E_i\rightarrow E_i$ chaque $E_i$ est un rétract du produit.
	S'il est $T_4$ il en va de même pour chaque $E_i$.
\end{exemple}


\begin{exemple}
	Si $E$ n'est pas séparé la diagonale $\Delta_E$ n'est pas fermée.
	Mais le graphe d'une application continue $f:E\rightarrow F$ est un rétract de $E\times F$ grâce à la rétraction $E\times F\rightarrow\Gamma(f); (x, y)\mapsto(x, f(x))$.
	Si $E$ et $F$ sont séparé alors $\Gamma(f)$ est fermé ce qui montre $\Delta_E$ fermé avec $f=\id$.
\end{exemple}

\textbf{Se méfier des applications continues:}
Si $E$ et $F$ sont séparés $f\in\cont(E, F)$ a un graphe fermé.
Si les espaces ne sont pas séparé c'est à priori faux.
Comme c'est bien souvent le graphe fermé qui est important, un remède est de considérer les applications dont le graphe est fermé plutôt que les applications continues.

\begin{exemple}
	Si $E$ est compact, $F$ séparé et $f\in\cont(E, F)$ alors $f$ est fermée.
	Si $F$ n'est pas séparé, c'est faux.
	Cependant, si $E$ est compact et que $f:E\rightarrow F$ a un graphe fermé alors $f$ est fermée.
\end{exemple}

\begin{exemple}
	Si $E$ et $F$ sont des espaces vectoriels topologiques sur $\R$ et si $f$ est additive et a graphe fermé alors $f$ est linéaire.
	En effet $f(\frac{n}{d}x)=nf(\frac{x}{d})=\frac{n}{d}df(\frac{x}{d})=\frac{n}{d}f(x)$.
	Ainsi $\forall q\in\Q, f(qx)=qf(x)$.
	Si l'application était continue on conclurait directement.
	Sinon pour $\lambda\in\R$ et $r\in\Q$ on fait tendre $r\rightarrow\lambda$ ainsi $rx\rightarrow\lambda x$ et $f(rx)=rf(x)\rightarrow\lambda f(x)$ par continuité de la multiplication externe.
	Or $(rx, rf(x))\in\Gamma(f)$ fermé on a $(\lambda x, \lambda f(x))\in\Gamma(f)$ id est $f(\lambda x)=\lambda f(x)$.
\end{exemple}

\textbf{Se méfier des espace localement compacts:}
« Tout point de $E$ appartient à un ouvert dont la fermeture est compacte » est équivalent dans un espace séparé à « tout point de $E$ a un voisinage compact » c'est à dire:
$$
	\forall x\in E, \exists O\in\mathscr{T}, x\in O\et\overline{O}\text{ compact}
	\iff
	\forall x\in E, \exists V\in\mathfrak{V}(x), V\text{ compact}
$$
Si l'on essaye de généraliser cette définition aux quasi-compact on perd l'équivalence, ce qui donne deux définitions de la local-quasi-compacité.
On a $\Rightarrow$ car $\overline{O}$ est un voisinage de $x$ compact.
Montrons que la réciproque peut être fausse.
Pour $E=\N$ numie de la topologie $\{\vide, \{0\}, \{0, 1\}, \{0, 1, 2\}, ...\}$ on obtient bine un espace localement quasi-compact au second sens (pour $n\in\N$ il suffit de prendre $\intervalle{0, n}$).
Il ne l'est pas au premier sens car avec cette topologie, les fermés ne sont pas borné (donc impossibles à recouvrir avec des ouverts qui sont bornés dans cette topologie).
Cette espace est de plus $T_0$ mais pas $T_1$ (mais totu de même $T_4$...).

Le remède aux espaces localement-quasi-compacts est simple: préciser clairement la définition utilisée.

\textbf{Se méfier les espaces quasi-compacts:}
Le muricain utilise \textit{compact} pour signifier \textit{quasi-compact}, tandis qu'il faut lui dire \textit{compact-Hausdorff} en lieu et place de \textit{compact}.
Beaucoup de propriétés vraies dans les espaces compacts deviennent fausses dans les espaces quasi-compacts.

\chapter{Filtres}

On le verra plus tard, mais dans un espace métrique, les suites caractérisent les fermés, donc les ouverts, donc la topologie de l’espace.
Toute propriété topologique peut donc s’écrire en termes de suites convergentes.
Dans un espace topologique général (non métrisable), ce n’est plus le cas.

Soit $\mathscr{T}_1$ la topologie codénombrable de $\R$ et $(u_n)$ une suite convergente pour $\mathscr{T}_1$ de limite $x$.
On note $U=\{u_n|n\in\N\}$ et $V=\{x\}\cup\complement U$.
Puisque $\complement V\subset U$ et que $U$ est dénombrable, $V$ est un ouvert.
De plus $x\in V$ donc $V$ est un voisinage de $x$.
En utilisant le caractérisation séquentielle $\exists n_0\in\N, n\supegal n_0\Rightarrow u_n\in V$.
Or $\forall n, u_n\in U$, donc $\forall n\supegal n_0, u_n\in U\cap V=\{x\}$.
Ainsi $u_n$ est stationnaire à partir d'un certain rang.

Munissons maintenant $\R$ de la topologie discrète $\mathscr{T}_2$
Pour cette topologie, $\{x\}$ est un voisinage de $x$.
De même la caractérisation séquentielle donne qu'une suite convergente vers $x$ sera constante à partir d'un certain rang.

On a donc un problème: $\mathscr{T}_1$ et $\mathscr{T}_2$ ont les même suites convergentes sans êtres les mêmes topologies.
Cela implique que certaines propriétés topologiques ne peuvent plus s’exprimer en termes de suites.
Par exemple, l’injection canonique $i : (\R, \mathscr{T}_1)\rightarrow (\R, \mathscr{T}_2)$ n’est pas continue car pour $\{x\}$ ouvert de $\mathscr{T}_2$ mais pas de $\mathscr{T}_1$, on a $\inv{i}(\{x\})=\{x\}$.
Pourtant, pour toute suite $(u_n)$ convergente vers $x$ pour $\mathscr{T}_1$, $(u_n)$ est donc stationnaire en $x$ à partir d'un certain rang, ce qui impose que $i(u_n)$ soit aussi stationnaire, et donc converge aussi vers $x$ dans $\mathscr{T}_2$.
Sans distance sur $E$, il faut traiter la convergence avec un outil plus puissant que les suites, et ça sera les filtres.

\section{Définitions}

Les espaces $E, F$ sont munis des topologies $\mathscr{T}, \mathscr{U}$ des voisinages $\mathfrak{V}, \mathfrak{W}$.
On se donne une application $f:E\rightarrow F$.

\begin{definition}
	On appelle filtre sur $E$ une collection $\mathfrak{F}\subset\mathfrak{P}(E)$ telle que:
	\begin{itemise}
		\item $\vide\notin\mathfrak{F}$ et $\mathfrak{F}\neq\vide$
		\item $U, V\in\mathfrak{F}\implies U\cap V\in\mathfrak{F}$
		\item $U\in\mathfrak{F}\et U\subset V\implies V\in\mathfrak{F}$
	\end{itemise}
\end{definition}

\begin{exemple}
	Soit $(E, \mathfrak{V})$ un espace topologique, $\forall x\in E, \mathfrak{V}(x)$ est un filtre sur $E$.
\end{exemple}

\begin{demonstration}
	$\vide\neq V\in\mathfrak{V}(x)$ car $x\in V$.
	Le reste est vrai par définition.
\end{demonstration}

\begin{definition}
	Si $\mathfrak{B}$ fait de $\mathfrak{F}=\{U|\exists X\in\mathfrak{B}, X\subset U\}$ un filtre on dit que $\mathfrak{B}$ est une base de filtre.
\end{definition}

\begin{proposition}
	$\mathfrak{B}\subset\mathfrak{P}(E)$ est une base de filtre si et seulement si:
	\begin{itemise}
		\item $\vide\notin\mathfrak{B}$ et $\mathfrak{B}\neq\vide$
		\item $X, Y\in\mathfrak{B}\implies\exists Z\in\mathfrak{B}, Z\subset X\cap Y$
	\end{itemise}
\end{proposition}

\begin{demonstration}
	Par double implication.
	\begin{itemise}
		\item[$\Rightarrow$] Soit $\mathfrak{B}$ une base de filtre $\mathfrak{F}$.
		\begin{itemise}
			\item si $\vide\in\mathfrak{B}$ alors $\vide\in\mathfrak{F}$
			\item soit $X, Y\in\mathfrak{B}$ alors $X, Y\in\mathfrak{F}$ donc $X\cap Y\in\mathfrak{F}$ donc $\exists Z\in\mathfrak{B}, Z\subset X\cap Y$
		\end{itemise}
		\item[$\Leftarrow$] Soit $\mathfrak{B}$ vérifiant les propriétés.
		\begin{itemise}
			\item si $\vide\in\mathfrak{F}$ alors $\vide\in\mathfrak{B}$
			\item si $U, V\in\mathfrak{F}$ alors $\exists X, Y\in\mathfrak{B}, X\subset U, Y\subset V$ donc $\exists Z\in\mathfrak{B}, Z\subset X\cap Y\subset U\cap V$ donc $U\cap V\in\mathfrak{F}$
			\item si $U\in\mathfrak{F}$ et $U\subset V$ alors $\exists X\in\mathfrak{B}, X\subset U\subset V$ donc $V\in\mathfrak{F}$
		\end{itemise}
	\end{itemise}
\end{demonstration}

\begin{proposition}
	L'intersection d'une famille de filtre $(\mathfrak{F}_i)_{i\in I}$ sur $E$ est un filtre sur $E$.
\end{proposition}

\begin{demonstration}
	\begin{itemise}
		\item $\forall i\in I, \vide\notin\mathfrak{F}_i$ donc $\vide\notin\bigcap_{i\in I}\mathfrak{F}_i$.
		\item Si $X, Y\in\bigcap_{i\in I}\mathfrak{F}_i$ alors $X, Y\in\mathfrak{F}_i$ donc $X\cap Y\in\mathfrak{F}_i$ puis $X\cap Y\in\bigcap_{i\in I}\mathfrak{F}_i$.
		\item Si $X\in\bigcap_{i\in I}\mathfrak{F}_i$ et $X\subset Y$ alors $X\in\mathfrak{F}_i$ et $X\subset Y$ donc $Y\in\mathfrak{F}_i$ puis $Y\in\bigcap_{i\in I}\mathfrak{F}_i$.
	\end{itemise}
\end{demonstration}

\begin{definition}
	On pose donc $\sigma_\text{filtre}(\mathfrak{A})=\bigcap_{\mathfrak{A}\subset\mathfrak{F}\text{ filtre}}\mathfrak{F}$ le plus petit filtre contenant $\mathfrak{A}$ nommée filtre engendré par $\mathfrak{A}$.
\end{definition}

\begin{remarque}
	Il est facile de vérifier que si $\mathfrak{B}$ est une base du filtre $\mathfrak{F}$ alors $\mathfrak{F}=\sigma_\text{filtre}(\mathfrak{B})$.
\end{remarque}

\begin{proposition}
	Soit $\mathfrak{B}$ une base de filtre de $E$.
	Alors $f(\mathfrak{B})$ est une base de filtre de $F$, appelé image de $\mathfrak{B}$ par $f$.
\end{proposition}

\begin{demonstration}
	\begin{itemise}
		\item $f$ est une application donc définie sur tout $E$ donc si $C\in\mathfrak{B}$ alors $f(C)\neq\vide$, donc $\vide\notin f(\mathfrak{B})$ et $f(\mathfrak{B})\neq\vide$.
		\item Si $X, Y\in f(\mathfrak{B})$ alors $\exists A, Y\in\mathfrak{B}, X=f(A)$ et $Y=f(B)$.
		$\mathfrak{B}$ étant une base de filtre $\exists C\in\mathfrak{B}, C\subset X\cap B$ donc $X\cap Y=f(A)\cap f(B)\supset f(X\cap B)\supset f(C)\in f(\mathfrak{B})$.
	\end{itemise}
\end{demonstration}

\section{Convergence}

\begin{definition}
	Soit $(E, \mathfrak{V})$ un espace topologique.
	On dit que le filtre $\mathfrak{F}$ converge vers $x\in E$ si et seulement s'il est plus fin que $\mathfrak{V}(x)$.
	On notera $\mathfrak{F}\rightarrow x$.
\end{definition}

\begin{proposition}
	Un espace topologique est $T_2$ si et seulement si tout filtre a au plus une limite.
\end{proposition}

\begin{demonstration}
	Par double implication.
	\begin{itemise}
		\item[$\Rightarrow$] Par l'absurde, soit $\mathfrak{F}$ un filtre convergent vers $x$ et $y$ distincts.
		Par définition de la convergence d'un filtre, $\mathfrak{V}(x)\infegal\mathfrak{F}$ et $\mathfrak{V}(y)\infegal\mathfrak{F}$.
		Soient $X$ et $Y$ des voisinages disjoints de $x$ et $y$ respectivement.
		Alors $X\in\mathfrak{V}(x)\subset\mathfrak{F}$ et $Y\in\mathfrak{V}(y)\subset\mathfrak{F}$.
		Par stabilité par intersection finie d'un filtre $X\cap Y=\vide\in\mathfrak{F}$ ce qui est absurde.
		\item[$\Leftarrow$] Soient $x\neq y$.
		Supposons $\forall V, W\in\mathfrak{V}(x)\times\mathfrak{V}(y), V\cap W\neq\vide$.
		Alors $V\cap W$ forment une base de filtre dont le filtre associé converge vers $x$ et $y$, ce qui est absurde.
	\end{itemise}
\end{demonstration}

\begin{proposition}
	$x$ adhère à $X$ si et seulement si c'est la limite d'un filtre sur $X$.
\end{proposition}

\begin{demonstration}
	Par double implication.
	\begin{itemise}
		\item[$\Rightarrow$] La trace $\mathfrak{B}$ des voisinages sur $X$ forme une base de filtre car :
		\begin{itemise}
			\item $\forall V\in\mathfrak{V}(x), V\cap X\supset\{x\}\neq\vide$ donc $\vide\notin\mathfrak{B}$
			\item si $Y, Z\in\mathfrak{B}$ alors $\exists V, W\in\mathfrak{V}(x), Y=X\cap V, Z=X\cap W$ donc $Y\cap Z=X\cap(V\cap W)\in\mathfrak{V}(x)$ par stabilité par intersection des voinsinages.
		\end{itemise}
		Soit $\mathfrak{F}$ le filtre engendré par $\mathfrak{B}$
		Soit $V\in\mathfrak{V}(x)$, alors $V\cap X\in\mathfrak{B}$ par construction, donc appartient au filtre sur $X$ engendré par $\mathfrak{B}$ de par la caractérisation.
		On a donc un exemple de filtre qui convient.
		\item[$\Leftarrow$] Soit $V\in\mathfrak{V}(x)$.
		Par convergence vers $x$, on a que $V\in\mathfrak{F}$.
		Par croissance des filtre, puisque $V\subset X\in\mathfrak{P}(X)$, alors $X\in\mathfrak{F}$.
		Par stabilité par intersection des filtres, $V\cap X\in\mathfrak{F}$, donc n'est pas vide.
		Ainsi $x\in\overline{X}$.
	\end{itemise}
\end{demonstration}

\begin{proposition}
	$f\text{ continue en }x\iff\forall\mathfrak{F}\text{ filtre},\quad\mathfrak{F}\rightarrow x\Rightarrow\sigma_\text{filtre}(f(\mathfrak{F}))\rightarrow f(x)$
\end{proposition}

\begin{demonstration}
	Par double implication.
	\begin{itemise}
		\item[$\Rightarrow$] Soit $\mathfrak{F}$ convergent en $x$ et $W\in\mathfrak{W}(f(x))$.
		Par continuité $\exists V\in\mathfrak{V}(x), f(V)\subset W$.
		Par convergence $V\in\mathfrak{F}$.
		Ainsi $\exists W'\in f(\mathfrak{F}), W'\subset W$, c'est à dire $W\in\sigma_\text{filtre}(f(\mathfrak{F}))$.
		Donc $\sigma_\text{filtre}(f(\mathfrak{F}))$ est plus fin que $\mathfrak{W}(f(x))$.
		\item[$\Leftarrow$] Soit $W\in\mathfrak{W}(f(x))$.
		Prenons $\mathfrak{F}=\mathfrak{V}(x)$ alors $\mathfrak{F}\rightarrow x$ donc $\sigma_\text{filtre}(f(\mathfrak{F}))\rightarrow f(x)$ id est $\sigma_\text{filtre}(f(\mathfrak{F}))\supset\mathfrak{W}(f(x))$.
		Ainsi $W\in\sigma_\text{filtre}(f(\mathfrak{F}))$.
		Or $f(\mathfrak{F})$ est une base de ce filtre donc $\exists V\in\mathfrak{F}=\mathfrak{V}(x), f(V)\subset W$.
	\end{itemise}
\end{demonstration}

\section{Ultrafiltres}

\begin{definition}
	Soit $E$ un espace topologique.
	On dit qu'un filtre $\mathfrak{F}$ est plus fin qu'un filtre $\mathfrak{G}$ si et seulement si $\mathfrak{F}\supset\mathfrak{G}$.
\end{definition}

\begin{definition}
	On dit qu'un filtre $\mathfrak{U}$ sur est un ultrafiltre s'il n'existe pas de filtre strictement plus fin (c'est à dire que $\forall\mathfrak{F}\text{ filtre}, \mathfrak{F}\supset\mathfrak{U}\Rightarrow\mathfrak{F}=\mathfrak{U}$).
\end{definition}

\begin{proposition}
	$\mathfrak{U}\text{ est un ultrafiltre}\iff\forall X\subset E, (X\in\mathfrak{U})\ou(\complement X\in\mathfrak{U})$
\end{proposition}

\begin{demonstration}
	Par double implication.
	\begin{itemise}
		\item[$\Rightarrow$] Soit $\mathfrak{U}$ un ultrafiltre et $X\in\mathfrak{P}(E)$ tel que $X\notin\mathfrak{U}$.
		Soit $V\in\mathfrak{U}$.
		Par l'absurde, si $V\subset X$ alors par croissance des filtres, $X\in\mathfrak{U}$.
		Donc $V\not\subset X$, c'est à dire que $V\cap\complement X\neq\vide$.
		$\mathfrak{B}=\{V\cap\complement X | V\in\mathfrak{U}\}$ est une base de filtre car $\mathfrak{U}$ est un filtre.
		Alors le filtre $\sigma_\text{filtre}(\mathfrak{B})$ est plus fin que $\mathfrak{U}$.
		En effet soit $V\in\mathfrak{U}$ puisque $V\cap\complement X\subset V$ avec $V\cap\complement X\in\mathfrak{B}$, alors $V\in\sigma_\text{filtre}(\mathfrak{B})$.
		Puisque $\mathfrak{U}$ est un ultrafiltre, $\mathfrak{U}=\sigma_\text{filtre}(\mathfrak{B})$.
		Or $\complement X=E\cap\complement X\in\mathfrak{B}$ puisque $E\in\mathfrak{U}$.
		Donc $\complement X\in\sigma_\text{filtre}(\mathfrak{B})=\mathfrak{U}$.
		\item[$\Leftarrow$] Par l'absurde, supposons que $\mathfrak{U}$ n'est pas un ultrafiltre.
		Il existe alors un filtre $\mathfrak{F}$ strictement plus fin, et donc une partie $X$ de $E$ dans $\mathfrak{F}$ qui n'est pas dans $\mathfrak{U}$.
		Puisque $X\in\mathfrak{U}\ou\complement X\in\mathfrak{U}$ est vrai par hypothèse, on en déduit que $\complement X\in\mathfrak{U}$.
		Puisque $\mathfrak{F}$ est plus fin que $\mathfrak{U}$, alors $\complement X\in\mathfrak{F}$.
		Par stabilité par intersection de $\mathfrak{F}$, on a $\vide=X\cap\complement X\in\mathfrak{F}$ ce qui est absurde car $\mathfrak{F}$ est un filtre.
	\end{itemise}
\end{demonstration}

\begin{proposition}
	Soit $f:E\rightarrow F$ une application et $\mathfrak{B}$ une base d'un ultrafiltre filtre de $E$.
	Alors $f(\mathfrak{B})$ est une base d'ultrafiltre sur $F$.
\end{proposition}

\begin{demonstration}
	$f(\mathfrak{B})$ est une base de filtre car $\mathfrak{B}$ est une base de filtre.
	On peut donc poser $\mathfrak{U}=\sigma_\text{filtre}(\mathfrak{B})$ et $\mathfrak{V}=\sigma_\text{filtre}(f(\mathfrak{B}))$.
	Soit $Y\subset F$.
	Puisque $\inv{f}(Y)$ et $\inv{f}(\complement Y)$ sont complémentaires dans $E$, d'après le caractérisation précédente, l'un des deux appartient à l'ultrafiltre engendré par $\mathfrak{B}$.
	Si c'est $\inv{f}(Y)$ alors $Y=f(\inv{f}(Y))$ appartient au filtre engendré par $f(\mathfrak{B})$.
	Si c'est $\inv{f}(\complement Y)$ alors $\complement Y=f(\inv{f}(\complement Y))$ appartient au filtre engendré par $f(\mathfrak{B})$.
	Ainsi $\forall Y\subset E, Y\in\mathfrak{V}\ou\complement V\in\mathfrak{V}$.
\end{demonstration}

\begin{proposition}
	Tout filtre est contenu dans un ultrafiltre.
\end{proposition}

\begin{demonstration}
	Soit $F$ un filtre sur $E$ et $X$ l'ensemble des filtres sur $E$ plus fins que $\mathfrak{F}$.
	On va appliquer le lemme de Zorn à $X$ pour montrer d'existence d'un ultrafiltre contenant $\mathfrak{F}$.
	$X$ n'est pas vide, car il contient au moins les $\mathfrak{F}$.
	Montrons que $X$ est inductif pour la relation d'ordre $\infegal$.
	Soit $\{\mathfrak{F}_i | i\in I\}\subset X$ une partie totalement ordonnée.
	Posons $\mathfrak{G}=\{X\subset E | \exists i\in I, X\in\mathfrak{F}_i\}$.
	C'est un filtre sur $E$ car:
	\begin{itemise}
		\item Si $X\in\mathfrak{G}$ alors $X\in\mathfrak{F}_i$ pour un certain $i\in I$.
		Puisque $\mathfrak{F}_i$ est un filtre, $Y\neq\vide$.
		\item Si $X, Y\in\mathfrak{G}$ alors $X\in\mathfrak{F}_i$ et $Y\in\mathfrak{F}_j$ pour des certains $i, j\in I$.
		$X$ étant totalement ordonné, $\mathfrak{F}_i\subset\mathfrak{F}_j$ (ou inversement).
		Mais alors $X, Y\in\mathfrak{F}_j$, donc $X\cap Y\in\mathfrak{F}_j$ puisque c'est un filtre.
		Ainsi $X\cap Y\in\mathfrak{G}$.
		\item Si $X\in\mathfrak{G}$ alors $X\in\mathfrak{F}_i$ pour un certain $i\in I$.
		Si $X\subset Y$ alors $Y\in\mathfrak{F}_i$ puisque c'est un filtre.
		Ainsi $Y\in\mathfrak{G}$.
	\end{itemise}
	C'est de plus un majorant de $X$ car si $x\in\mathfrak{F}_i$ alors $x\in\mathfrak{G}$, ce qui implique $\forall i\in I, \mathfrak{F}_i\subset\mathfrak{G}$.
	D'après le lemme de Zorn, il existe un élément maximal $\mathfrak{U}$ à $X$.
	$\mathfrak{U}$ est un ultrafiltre car s'il admettait un majorant, alors ce dernier serait plus fin que $\mathfrak{F}$, et serait donc dans $X$, impliquant qu'il soit égale à $\mathfrak{U}$.
	$\mathfrak{U}$ est plus fin que $\mathfrak{F}$, car c'est un élément maximal de $X$, donc appartient à $X$, qui est l'ensemble des filtres plus fins que $\mathfrak{F}$.
\end{demonstration}

\section{Suites}

\begin{definition}
	On appelle suite sur $E$ une application $x:\N\rightarrow E\in E^\N$.
\end{definition}

\begin{proposition}
	Les suites sur un espace métrique sont un cas particulier de filtres.
\end{proposition}

\begin{demonstration}
	Soit $(x_n)$ une suite de $E$.
	Posons $X_n=\{x_p | p\supegal n\}$ et vérifions que $\mathfrak{B}_x=\{X_p | p\in\N\}$ est une base de filtre.
	\begin{itemise}
		\item une suite étant une application $X_p$ n'est jamais vide.
		\item Soit $X_m, X_n\in\mathfrak{B}$ alors $X_m\cap X_n=X_{\max(m, n)}\in\mathfrak{B}$.
	\end{itemise}
\end{demonstration}

\begin{proposition}
	Soit $(x_n)$ une suite et $(x_{\phi(n)})$ une suite extraite.
	Le filtre associé à $(x_{\phi(n)})$ est plus fin que celui de $(x_n)$.
\end{proposition}

\begin{demonstration}
	Soit $X_n$ un élément de la base de filtre $\mathfrak{B}_x$.
	Par croissance de $\phi$, $X_{\phi(n)}\subset X_n$, avec $X_{\phi(n)}\in\mathfrak{B}_{x\circ\phi}$.
	Ainsi $\mathfrak{B}_x\subset\sigma_\text{filtre}(\mathfrak{B}_{x\circ\phi})$, puis $\sigma_\text{filtre}(\mathfrak{B}_x)\subset\sigma_\text{filtre}(\mathfrak{B}_{x\circ\phi})$.
\end{demonstration}

\begin{definition}
	Soit $E$ un espace topologique.
	Une suite $(x_n)$ converge vers $x$ si et seulement si son filtre associé converge vers $x$.
\end{definition}

\begin{proposition}
	Soit $E$ un espace topologique.
	$$
		x_n\rightarrow x
		\iff
		\forall V\in\mathfrak{V}(x), \exists N, \forall n\supegal N, x_n\in V
	$$
\end{proposition}

\begin{demonstration}
	Par double implication.
	\begin{itemise}
		\item[$\Rightarrow$] Soit $V\in\mathfrak{V}(x)$.
		Par convergence de $(x_n)$ vers $x$, $V$ appartient au filtre associé à $(x_n)$.
		Par caractérisation du filtre engendré, $\exists X\in\mathfrak{B}_x, X\subset V$.
		Par définition de la base de filtre associée à la suite, $\exists N, X_N=X\subset V$, c'est à dire $\exists N, \forall n\supegal N\implies x_n\in V$.
		\item[$\Leftarrow$] Soit $V\in\mathfrak{V}(x)$.
		Par hypothèse, $\exists N, X_N\in\mathfrak{B}_x, X_N\subset V$.
		Par caractérisation du filtre engendré $V\in\sigma_\text{filtre}(\mathfrak{B}_x)$, c'est à dire que $(x_n)$ converge vers $x$.
	\end{itemise}
\end{demonstration}

\begin{definition}
	Soit $X$ une partie d'un espace topologique $E$.
	La fermeture séquentielle de $X$ est $\overline{X}^\text{seq}=\{x | \exists(x_n)\in X^\N, x_n\rightarrow x\}$.
\end{definition}

\begin{proposition}
	Soit $X$ une partie d'un espace topologique $E$.
	Alors $\overline{X}\supset\overline{X}^\text{seq}$.
\end{proposition}

\begin{demonstration}
	On a démontré qu'une suite définissait un filtre, et que la suite convergeait si et seulement si ce filtre convergeait.
	Par caractérisation de l'adhérence en terme de filtre, on obtient l'inclusion.
\end{demonstration}

\begin{proposition}
	Soient $E$ et $F$ deux espaces topologiques, $f:E\rightarrow F$.
	$$
		\text{$f$ continue en $x$}
		\implies
		\left[\forall(x_n)\in E^\N, x_n\rightarrow x\implies f(x_n)\rightarrow f(x)\right]
	$$
\end{proposition}

\begin{demonstration}
	On a démontré qu'une suite définissait un filtre, et que la suite convergeait si et seulement si ce filtre convergeait.
	Par caractérisation de la continuité en terme de filtre, on obtient l'implication.
\end{demonstration}

\begin{proposition}
	Sur espace topologique compact de toute suite on peut extraire une sous suite convergente.
\end{proposition}

\begin{demonstration}
	On a démontré qu'une suite définissait un filtre, et que la suite convergeait si et seulement si ce filtre convergeait.
	Par caractérisation de la compacité en terme de filtre, on obtient l'implication.
\end{demonstration}

\chapter{Structures uniformes}

Les espaces métriques forment une classe importante d'espaces topologiques, qui est néanmoins insuffisante pour englober tout les espaces de l'analyse.
Une généralisation satisfaisante est donnée par les espaces uniformes.
Les groupes topologiques, en particulier les espaces vectoriels topologiques, sont uniformisables.
Il y a toutefois des espaces topologiques non uniformisables.
Une question naturelle pour le topologue est donc de déterminer quelles sont les topologies que l'on peut définir à partir de familles d'entourages de la diagonale.
Pour obtenir une classe d'espaces plus lage que la classe des espaces uniformes, il faut affaiblir les exigences que l'on met sur cette famille d'entourages.
L'affaiblissement convenable consiste à omettre l'axiome de symétrie.
On parle alors de quasi-uniformité, et tout espace est quasi-uniformisable: c'est donc cet axiome de symétrie qui caractérise véritablement les espaces uniformisable parmi les espaces topologiques, et le procédé consistant à définir une topologie à partir d'entourages de la diagonale est tout à fait général.
Aussi ce procédé doit-il être ajouté au catalogue des méthodes générales pour définir les topologies: par les ouverts (Sierpinski) par les fermé (Alexandroff) par les voisinages (Hausdorff) par les fermetures (Kuratowski) par les filtres convergents (Bourbaki) et cætera...

\begin{figure}[h]
	\centering
	\begin{tikzpicture}[scale=4]
		\pgfmathsetmacro{\x}{0.65}
		\pgfmathsetmacro{\y}{0.3}
		\pgfmathsetmacro{\l}{0.25}
		\pgfmathsetmacro{\xx}{0.7}
		\pgfmathsetmacro{\yy}{0.6}
		\pgfmathsetmacro{\yyy}{0.5}
		\draw[black!30!white] (0, 0) rectangle (1, 1);
		\draw[dashed] (0, 0)--(1, 1);
		\draw (\x, \y)--(\x-\l, \y+\l)--(\x-2*\l, \y)--(\x-\l, \y-\l) -- cycle;
		\draw (\xx, \yy)--(\xx+\yy-\yyy, \yyy)--(\xx+2*\yy-2*\yyy, \yy) -- cycle;
		%\draw[dashed] (\xx, \yy)--(\yy, \yy)--(\yy, \y+\x-\yy)--(\yy, \y-\x+\yy);
		%\draw[dashed] (\xx+\yy-\yyy, \yyy)--(\yyy, \yyy)--(\yyy, \y+\x-\yyy)--(\yyy, \y-\x+\yyy);
		\filldraw[black!30!white]
		(\xx,\y+\x-\yy)--(\xx+\yy-\yyy,\y+\x-\yyy)--
		(\xx+2*\yy-2*\yyy,\y+\x-\yy)--(\xx+2*\yy-2*\yyy,\y-\x+\yy)--
		(\xx+\yy-\yyy,\y-\x+\yyy)--(\xx,\y-\x+\yy)--cycle;
		\node[anchor=east] at (\x, \y) {$F$};
		\node[anchor=south] at (\xx+\yy-\yyy, \yy) {$G$};
		\node[anchor=north] at (\xx+\yy-\yyy,\y-\x+\yyy) {$F\circ G$};
	\end{tikzpicture}
	\caption{$F\circ G$}
\end{figure}

\section{Définitions}

\begin{definition}
	Soit $X$ un ensemble et $F, G\in\mathfrak{P}(X^2)$.
	On pose :
	\begin{itemise}
		\item $F\circ G=\{(x, z)\in X^2 | \exists y\in X, (x, y)\in G, (y, z)\in F\}$.
		\item $F[x]=\{y\in X | (x, y)\in F\}$
		\item $\inv{F}=\{(y, x) | (x, y)\in F\}$
		\item $\Delta_X=\{(x, x) | x\in X\}$
	\end{itemise}
\end{definition}

\begin{definition}
	On appelle (quasi-)uniformité sur $E$ une collection $\mathscr{U}\subset\mathfrak{P}(E)$, dont ses éléments sont appelés les entourages, telle que:
	\begin{itemise}
		\item $U\in\mathscr{U}\Rightarrow\Delta_E\subset U$
		\item $U, V\in\mathscr{U}\Rightarrow U\cap V\in\mathscr{U}$
		\item $U\in\mathscr{U}\et U\subset V\Rightarrow V\in\mathscr{U}$
		\item $U\in\mathscr{U}\Rightarrow\exists V\in\mathscr{U}, V\circ V\subset U$
		\item Si $\mathscr{U}$ est une uniformité: $U\in\mathscr{U}\Rightarrow\inv{U}\in\mathscr{U}$
	\end{itemise}
\end{definition}

\begin{definition}
	Si $\mathfrak{B}$ fait de $\mathscr{U}=\{U\in\mathfrak{P}(E^2) | \exists X\in\mathfrak{B}, X\subset U\}$ une (quasi-)uniformité on dit que $\mathfrak{B}$ est une base de (quasi-)uniformité.
\end{definition}

\begin{proposition}
	$\mathfrak{B}\subset\mathfrak{P}(E)$ est une base de (quasi-)uniformité si et seulement si:
	\begin{itemise}
		\item $X\in\mathfrak{B}\Rightarrow\Delta_E\subset X$
		\item $X, Y\in\mathfrak{B}\Rightarrow\exists Z\in\mathfrak{B}, Z\subset X\cap Y$
		\item $X\in\mathfrak{B}\Rightarrow\exists Y\in\mathfrak{B}, Y\circ Y\subset X$
		\item Si $\mathfrak{B}$ est une base d'uniformité: $X\in\mathfrak{B}\Rightarrow\exists Y\in\mathfrak{B}, Y\subset x^{-1}$
	\end{itemise}
\end{proposition}

\begin{demonstration}
	Par double implication.
	\begin{itemise}
		\item[$\Rightarrow$] Soit $\mathfrak{B}$ une base de $\mathscr{U}$.
		\begin{itemise}
			\item Si $X\in\mathfrak{B}\subset\mathscr{U}$ alors $\Delta_E\subset X$.
			\item Si $X, Y\in\mathfrak{B}\subset\mathscr{U}$ alors $X\cap Y\in\mathscr{U}$ donc $\exists Z\in\mathfrak{B}, Z\subset X\cap Y$.
			\item Si $X\in\mathfrak{B}\subset\mathscr{U}$ alors $\exists V\in\mathscr{U}, V\circ V\subset X$ donc $\exists Y\in\mathfrak{B}, Y\subset V$ donc $Y\circ Y\subset V\circ V\subset X$.
			\item Si $\mathfrak{B}$ est une base d'uniformité: si $X\in\mathfrak{B}\subset\mathscr{U}$ alors $ x^{-1}\in\mathscr{U}$ donc $\exists Y\in\mathscr{B}, Y\subset x^{-1}$.
		\end{itemise}
		\item[$\Leftarrow$] Soit $\mathfrak{B}$ vérifiant les propriétés.
		\begin{itemise}
			\item Si $U\in\mathscr{U}$ alors $\exists X\in\mathfrak{B}, X\subset U$ donc $\Delta_E\subset X\subset U$.
			\item Si $U, V\in\mathscr{U}$ alors $\exists X, Y\in\mathfrak{B}, X\subset U, Y\subset V$ donc $\exists Z\in\mathfrak{V}, Z\subset X\cap Y\subset U\cap V\in\mathscr{U}$.
			\item Si $U\in\mathscr{U}$ et $U\subset V$ alors $\exists X\in\mathfrak{B}, X\subset U\subset V$ donc $V\in\mathscr{U}$.
			\item Si $U\in\mathscr{U}$ alors $\exists X\in\mathfrak{B}, X\subset U$ donc $\exists Y\in\mathfrak{B}\subset\mathscr{U}, Y\circ Y\subset X\subset U$.
			\item Si $\mathfrak{B}$ vérifie le dernier axiome: si $U\in\mathscr{U}$ alors $\exists X\in\mathfrak{B}, X\subset U$ donc $\exists Y\in\mathfrak{B}, Y\subset x^{-1}\subset\inv{U}$ donc $\inv{U}\in\mathscr{U}$.
		\end{itemise}
	\end{itemise}
\end{demonstration}

\begin{proposition}
	L'intersection d'une famille de (quasi-)uniformités $\mathscr{U}_{i\in I}$ est une (quasi-)uniformité.
\end{proposition}

\begin{demonstration}
	\begin{itemise}
		\item Si $U\in\bigcap_{i\in I}\mathscr{U}_i$ alors $U\in\mathscr{U}$ donc $\Delta_E\in U$.
		\item Si $U, V\in\bigcap_{i\in I}\mathscr{U}_i$ alors $U, V\in\mathscr{U}_i$ donc $U\cap V\in\mathscr{U}_i$ donc $U\cap V\in\bigcap_{i\in I}\mathscr{U}_i$.
		\item Si $U\in\bigcap_{i\in I}\mathscr{U}_i\et U\subset V$ alors $U\in\mathscr{U}_i\et U\subset V$ donc $V\in\mathscr{U}_i$ donc $V\in\bigcap_{i\in I}\mathscr{U}_i$.
		\item Si $U\in\bigcap_{i\in I}\mathscr{U}_i$ alors $U\in\mathscr{U}_i$ donc $\exists V_i\in\mathscr{U}_i, V_i\circ V_i\subset U$.
		Considérons {\color{red} ???}
		\item Si $\mathscr{U}$ est une uniformité: Si $U\in\bigcap_{i\in I}\mathscr{U}_i$ alors $U\in\mathscr{U}_i$ donc $\inv{U}\in\mathscr{U}_i$ donc $\in{U}\in\bigcap_{i\in I}\mathscr{U}_i$.
	\end{itemise}
\end{demonstration}

\begin{definition}
	On pose donc $\sigma_\text{quasi}(\mathfrak{A})=\bigcap_{X\subset\mathscr{U}\text{ quasi}}\mathscr{U}$ la plus petite (quasi-)uniformité contenant $\mathfrak{A}$ nommée (quasi-)uniformité engendré par $\mathfrak{A}$.
\end{definition}

\begin{remarque}
	Il est facile de vérifier que si $\mathfrak{B}$ est une base de la (quasi-)uniformité $\mathscr{U}$ alors $\mathscr{U}=\sigma_\text{quasi}(\mathfrak{B})$.
\end{remarque}

\begin{proposition}
	Les quasi-uniformités forment un espace plus gros que les topologies.
\end{proposition}

\begin{demonstration}
	Posons $\Phi:\mathscr{T}\mapsto\sigma_\text{quasi}(\mathfrak{B})$ avec $\mathfrak{B}$ la famille des intersections finies de $\mathfrak{B}'=\{O^2\cup\complement O\times E | O\in\mathscr{T}\}$ (appelée prébase de Pervin).
	\begin{center}
		\begin{tikzpicture}[scale=1.75]
			\fill[black!30!white] (0.5, 0.5) rectangle (0.75, 0.75);
			\fill[black!20!white] (0, 0) rectangle (0.5, 1);
			\fill[black!20!white] (0.75, 0) rectangle (1, 1);
			\draw (0.5, 0) -- node [below, midway] {$O$} (0.75, 0);
			\draw (0, 0) -- node [left] {$E$} (0, 1);
			\draw (0, 1)--(0.5, 1) (0.75, 1) -- node [above] {$\complement O$} (1, 1);
			\draw[dashed] (0, 0) -- node[above, left] {$\Delta_E$} (1, 1);
		\end{tikzpicture}
	\end{center}
	Montrons que $\mathfrak{B}$ est une base de quasi-uniformité.
	\begin{itemise}
		\item Si $X\in\mathfrak{B}'$ alors $\Delta_E\subset X$.
		C'est encore vrai après intersection, donc pour $\mathfrak{B}$.
		\item Par définition $\mathfrak{B}$ est stable par intersection finie.
		\item Soit $X'=O^2\cup\complement O\times E\in\mathfrak{B}'$ et $(x, z)\in X'\circ X'$ alors $\exists y\in E, (x, y), (y, z)\in X'$. \\
		Si $x\in O$ alors $(x, y)\notin\complement O\times E$ donc $y\in O$ de même $z\in O$ donc $(x, z)\in O^2\subset X'$. \\
		Si $x\in\complement O$ alors $(x, y)\in\complement O\times E=X'$.
		Ainsi $\forall X'\in\mathfrak{B}', X'\circ X'\subset X'$.
		Cette propriété passe aux intersections finies donc si $X\in\mathfrak{B}, X\circ X\subset X$.
	\end{itemise}
	Posons $\Psi:\mathscr{U}\mapsto\{U[x]|U\in\mathscr{U}\}$ pour $U[x]=\{y\in E|(x, y)\in U\}$.
	Montrons que pour $\mathscr{U}$ une quasi-uniformité non vide $\mathfrak{V}=\Psi(\mathscr{U})$ est une voisinagination.
	\begin{itemise}
		\item Soit $U\in\mathscr{U}$.
		Puisque $U\subset E\times E$ alors $E\times E\in\mathscr{U}$, donc $E=(E\times E)[x]\in\mathfrak{V}(x)$.
		\item Si $U[x]\in\mathfrak{V}(x)$, puisque $\Delta_E\subset U$ alors $(x, x)\in U$ donc $x\in U[x]$.
		\item Si $U[x], V[x]\in\mathfrak{V}(x)$ alors $U[x]\cap V[x]=(U\cap V)[x]\in\mathfrak{B}$ car $U\cap V\in\mathscr{U}$.
		\item Si $U[x]\in\mathfrak{V}(x)\et U[x]\subset T$ en posant $V=U\cup\{x\}\times T$ on a $U\subset V$ donc $V\in\mathscr{U}$ et $V[x]=(U\cup\{x\}\times T)[x]=U[x]\cup T=T\in\mathfrak{V}(x)$ car $U[x]\subset T$.
		\item Si $U[x]\in\mathfrak{V}(x)$ alors $\exists V\in\mathscr{U}, V\circ V\subset U$.
		De $y\in V[x]\et z\in V[y]$ on a $(x, y), (y, z)\in V$ donc $(x, z)\in V\circ V\subset U$.
		Ainsi $z\in U[x]$ puis $V[y]\subset U[x]$.
		Or $V[y]\in\mathfrak{V}(y)$ donc d'après le point précédent $U[x]\in\mathfrak{V}(y)$
	\end{itemise}
	Montrons que $\Psi\circ\Phi\cong\id$, c'est à dire $\mathscr{T}\cong\mathfrak{V}$ (l'ismorphisme canonique défini en permière partie) par double inclusion.
	\begin{itemise}
		\item[$\subset$] Si $O\in\mathscr{T}$ alors $O^2\cup\complement O\times E\in\mathfrak{B}'$ donc $O=(O^2\cup\complement O\times E)[x]\in\mathfrak{V}(x)$.
		\item[$\supset$] Si $U[x]\in\mathfrak{V}(x)$ alors $\exists X_{1\infegal i\infegal n}\in\mathfrak{B}', \bigcap_{i=1}^nX_i\subset U$ avec $X_i=O_i^2\cup\complement O_i\times E$ et $O_i\in\mathscr{T}$.
		Si $x\in O_i$ alors $X_i[x]=O_i$ sinon $X_i[x]=E$.
		En posant $J=\{j\in I|x\in O_j\}$ qui est fini on a $(\bigcap_{i=1}^nX_i)[x]=\bigcap_{i=1}^nX_i[x]=\bigcap_{j\in J}O_j\in\mathscr{T}$ et $(\bigcap_{i=1}^nX_i)[x]\subset U[x]$.
	\end{itemise}
\end{demonstration}

\begin{remarque}
	On a ainsi montré qu'on peut construire une unique topologie à partir d'une quasi-uniformité.
	L'inverse est faux: une topologie engendrée par une quasi-uniformité est en générale engendrée par toute une famille de quasi-uniformité.
	On verre au chapitre suivant la structure des uniformités engendrant une topologie uniformisable.
\end{remarque}

\begin{center}
	\begin{tikzpicture}[scale=1.5]
		\node at (0,-0.25) {$\mathscr{T}$};
		\node at (1,-0.25) {$\mathscr{U}$};
		\node at (2,-0.25) {$\mathscr{T}$};

		\node (t1) at (0,0) {};
		\node (t2) at (0,1) {};
		\node (u1) at (1,0) {};
		\node (u2) at (1,1) {};
		\node (u3) at (1,-0.5) {};
		\node (u4) at (1, 1.5) {};
		\node (v1) at (2,0) {};
		\node (v2) at (2,1) {};

		\draw[thick,->] (-0.4,0.5)--(2.4,0.5) node[right] {id};
		\node[circle,fill=white,inner sep=0pt,minimum size=5pt] at (0.15,0.5) {};
		\node[circle,fill=white,inner sep=0pt,minimum size=5pt] at (1.15,0.5) {};
		\node[circle,fill=white,inner sep=0pt,minimum size=5pt] at (2.15,0.5) {};

		\draw[very thick] (0,0.5) ellipse (0.15 and 0.5);
		\draw[very thick] (1,0.5) ellipse (0.15 and 0.5);
		\draw[very thick] (1,0.5) ellipse (0.3 and 1.0);
		\draw[very thick] (2,0.5) ellipse (0.15 and 0.5);

		\draw[->] (0,0)--(1,0);
		\draw[->] (0,1)--(1,1);
		\draw[->] (1,0)--(2,0);
		\draw[->] (1,1)--(2,1);
		\draw[] (1,-0.5)--(2,0);
		\draw[] (1,1.5)--(2,1);
	\end{tikzpicture}
\end{center}

\section{Séparation}

\begin{proposition}
	Un espace topologie $E$ est $T_3$ si et seulement si sa quasi-uniformité de Pervin vérifie $\forall U\in\mathscr{U}, \forall x\in E, \exists V\in\mathscr{U}, V=\inv{V}\et (V\circ V)[x]\subset U[x]$
\end{proposition}

\begin{demonstration}
	Par double implication.
	\begin{itemise}
		\item[$\Rightarrow$] %{\color{red} ???}
		Soit $U\in\mathscr{U}$ et $x\in E$.
		$U[x]$ est un voisinage de $x$ donc $\exists O\in\mathscr{T}, x\in O\subset U[x]$.
		Or $x\notin\complement O$ fermé donc par $T_3$ $\exists U, \complement F\in\mathscr{T}, x\in U\et\complement O\subset\complement F\et U\cap\complement F=\vide$ ce qui s'écrit aussi $x\in U\subset F\subset O$.
		Posons $V=F^2\cup\complement^2F\cup(O\setminus U)^2$ est symétrique.
		\begin{itemise}
			\item $(O^2\cup\complement O\times E)\cap(\complement^2F\cup F\times E)=[O^2\cap...]\cup[\complement^2F\cap...]\cup[\cancel{(\complement O\cap F)}\times E]\subset V$ et le membre de gauche est dans la base de Pervin montrant $V\in\mathscr{U}$.
			\item Si $(x, y)\in(F\setminus U)\times\complement F$ et $t\in O\setminus F$ (rectangle et ligne rouge) on a $(x, t)\in(F\setminus U)\times(O\setminus F)\subset(O\setminus U)^2\subset V$ et $(t, y)\in(O\setminus F)\times\complement F\subset(\complement F)^2\subset V$ montrant $(F\setminus U)\times\complement F\subset V\circ V$.
			Or $V$ est symétrique donc $V\circ V$ aussi donc $\complement F\times(F\setminus U)\subset V\circ V$.
			Puisque $V\subset V\circ V$ ainsi $(F\setminus U)^2\subset F^2\subset V\circ V$ et $\complement^2 F\subset V\circ V$.
			Ainsi $(F\setminus U)^2\cup(F\setminus U\times\complement F)\cup(\complement F\times F\setminus U)\cup\complement^2F=[(F\setminus U)\cup\complement F]^2=\complement^2U\subset V\circ V$.
			\item Idem pour $(x, y)\in(O\setminus F)\times F$ et $t\in F\setminus U$ (en bleu) montrant $O^2\subset V\circ V$.
			\item En développant on a $\complement(O^2\cup\complement^2 U)=(U\times\complement O)\cup(\complement O\times U)$.
			Par symétrie limitons nous à $(x, y)\in U\times\complement O$  (en noir).
			Pour avoir $(x, t), (t, y)\in V$ il faudrait $(x, t)\in F^2$ et $(t, y)\in\complement^2F$ soit $t\in\vide$ ce qui est impossible donc $U\times\complement O\not\subset V\circ V$.
		\end{itemise}
		\begin{center}
			\begin{tikzpicture}[scale=3]
				\pgfmathsetmacro{\x}{0.5};
				\pgfmathsetmacro{\d}{0.2};
				\fill[black!10!white] (0,0) rectangle (\x,\x);
				\fill[black!20!white] (\x,\x) rectangle (1,1);
				\fill[black!30!white] (\x-\d,\x-\d) rectangle (\x+\d,\x+\d);
				\draw[dotted] (\x-\d, 0)--(\x-\d, 1) (\x, 0)--(\x, 1) (\x+\d, 0)--(\x+\d, 1);
				\draw[dotted] (0, \x-\d)--(1, \x-\d) (0, \x)--(1, \x) (0, \x+\d)--(1, \x+\d);
				\draw[thin] (0,-0.0)--(\x-\d,-0.0) node[right] {$U$};
				\draw[thin] (0,-0.1)--(\x   ,-0.1) node[right] {$F$};
				\draw[thin] (0,-0.2)--(\x+\d,-0.2) node[right] {$O$};
				\draw[thick] (0,\x+\d) rectangle (\x-\d,1);
				\draw[very thick, black!20!red] (\x-\d,\x) rectangle (\x,1) (\x,\x)--(\x+\d,\x+\d);
				\draw[very thick, black!20!blue] (\x,0) rectangle (\x+\d,\x) (\x,\x)--(\x-\d,\x-\d);
				\pgfmathsetmacro{\x}{0.4};
				\pgfmathsetmacro{\y}{0.9};
				\pgfmathsetmacro{\t}{0.6};
				\draw[dashed, black!20!red] (\x,\y)--(\x,\t)--(\t,\t)--(\t,\y)--cycle;
			\end{tikzpicture}
		\end{center}
		Ainsi $V\circ V=O^2\cup\complement^2 U$ donc $(V\circ V)[x]=O\subset U[x]$ car $x\notin\complement U$.
		\item[$\Leftarrow$] Soit $\mathscr{U}$ vérifiant la propriété, $\mathscr{T}$ la topologie engendrée par $\mathscr{U}$, $F\in\complement\mathscr{T}$ et $x\notin F$.
		Ainsi $\complement F$ est un voisinage de $x$ donc par définition $\exists U\in\mathscr{U}, \complement F=U[x]$.
		D'après la propriété $\exists V\in\mathscr{U}, V=\inv{V}\et V\circ V[x]\subset U[x]$.
		De $y\in\overline{V[x]}$ c'est à dire $y$ adhère à $V[x]$ alors tout voisinage de $y$ rencontre $V[x]$.
		Or $V[y]$ est un voisinage de $y$ donc $V[x]\cap V[y]\neq\vide$.
		Soit $z\in V[x]\cap V[y]$ alors $(x, z), (y, z)\in V$ donc par symétrie $(x, z), (z, y)\in V$ soit $y\in V\circ V[x]$.
		Ainsi $\overline{V[x]}\subset U[x]=\complement F$ donc $x\in V[x], F\subset\complement\overline{V[x]}, V[x]\cap\complement\overline{V[x]}=\vide$ faisant de l'espace un $T_3$.
	\end{itemise}
\end{demonstration}

\section{Équivalence}

\begin{definition}
	On appelle écart sur $E$ toute application $d:E\times E\rightarrow\overline{\R^+}$ vérifiant:
	\begin{itemise}
		\item $\forall x\in E, d(x, x)=0$
		\item $\forall x, y\in E, d(x, y)=d(y, x)$
		\item $\forall x, y, z\in E, d(x, z)\infegal d(x, y)+d(y, z)$
	\end{itemise}
	Une jauge sur un ensemble $E$ est une famille $(d_i)_{i\in I}$ d'écarts sur $E$.
\end{definition}

\begin{proposition}
	Soit $d_{i\in I}$ une jauge sur $E$.
	La famille $\mathfrak{B}'=\{\inv{d_i}([0, \varepsilon[)|\varepsilon\in\R^+, i\in I\}$ est une prébase d'uniformité.
\end{proposition}

\begin{demonstration}
	\begin{itemise}
		\item Si $(x, x)\in\Delta_E$ alors $d_i(x, x)=0$ donc $(x, x)\in\inv{d_i}([0, \varepsilon[)\in\mathfrak{B}'$.
		%\item Par construction Si $(x, y)\in\inv{d_i}([0, \varepsilon[)$ et $(x', y')\in\inv{d_{i'}}([0, \varepsilon'[)$
		\item Si $\inv{d_i}([0, \varepsilon[)\in\mathfrak{B}'$ alors $\inv{d_i}([0, \varepsilon/2[)\circ \inv{d_i}([0, \varepsilon/2[)\subset\inv{d_i}([0, \varepsilon[)$.
		\item Par symétrie des $d_i$ $\inv{d_i}([0, \varepsilon[)=\inv{\inv{d_i}([0, \varepsilon[)}$
	\end{itemise}
\end{demonstration}

\begin{proposition}
	Soit $d_{i\in I}$ une jauge sur $E$.
	La famille $\mathfrak{B}'=\{\inv{d_i}(x, [0, \varepsilon[)|x\in E, \varepsilon\in\R^+, i\in I\}$ est une prébase de topologie.
	C'est exactement celle engendrée par l'uniformité définie au dessus.
\end{proposition}

\begin{demonstration}
\end{demonstration}

\begin{proposition}
	La topologie engendrée par la jauge $d_{i\in I}$ est la même que celle engendrée par $\{\max(d_{i_1}, ..., d_{i_n})|i_1, ..., i_n\in I\}$ appelée saturation de la jauge.
\end{proposition}

\begin{demonstration}
\end{demonstration}

\begin{proposition}
	Pour un espace topologique on a a l'équivalence entre:
	\begin{itemise}
		\item être $T_{3\frac{1}{2}}$
		\item être uniformisable
		\item être engendré par une famille d'écarts
	\end{itemise}
\end{proposition}

\begin{demonstration}
	\begin{itemise}
		\item Soit $\mathscr{U}$ une uniformité et $U\in\mathscr{U}$. {\color{red} ???}
		Construisons une suite décroissante d'entourages symétriques par $V_0=U\in\mathscr{U}$ et $V_{n+1}\circ V_{n+1}\circ V_{n+1}\subset V_n$ qui existe car par construction $V_n\in\mathscr{U}$.
		Soient:
		\begin{align*}
			f_U(x, y) & =
			\left\{\begin{array}{lll}
				1      & \text{si} & (x, y)\notin V_0               \\
				2^{-n} & \text{si} & (x, y)\in V_{n-1}\setminus V_n \\
				0      & \text{si} & (x, y)\in\bigcap_{n\in\N} V_n
			\end{array}\right.\\
			d_U(x, y) & =\inf\{\sum_{k=0}^nf_U(a_k, a_{k+1})|a_0=x, a_k\in E, a_{n+1}=y\}
		\end{align*}
		Vérifions que $d_U$ est un écart.
		\begin{itemise}
			\item $d_U(x,x)=0$ en prenant $a_0=a_1=x$.
			\item Chaque $V_n$ étant symétrique $f_U$ puis $d_U$ l'est.
			\item Soient $x, y, z\in E$ et $\varepsilon>0$.
			Par définition de l'inf il existe $a_{1\infegal k\infegal m}$ et $b_{1\infegal k\infegal n}$ tels que $\sum_{k=0}^mf_U(a_k, a_{k+1})\infegal d_U(x, y)+\frac{\varepsilon}{2}\et \sum_{k=0}^nf_U(b_k, b_{k+1})\infegal d_U(x, y)+\frac{\varepsilon}{2}$
			En concaténant les chemins $a_k$ et $b_k$ on obtient $c_k$ reliant $x$ à $z$ et montrant que $d_U(x, z)\infegal\sum_{k=0}^mf_U(a_k, a_{k+1})+\sum_{k=0}^nf_U(b_k, b_{k+1})$ donc $d_U(x, z)\infegal d_U(x, y)+d_U(y, z)+\varepsilon$.
			Étant vrai pour tout $\varepsilon$ on a $d_U(x, y)\infegal d_(x, y)+d_U(y, z)$.
		\end{itemise}
		Montrons $V_n\subset\{(x, y)|d_U(x, y)<2^{-n}\}=\inv{d_U}([0,2^{-n}[)\subset V_{n-1}$.
		\begin{itemise}
			\item Soit $(x, y)\in V_n$ alors $f_U(x, y)\infegal 2^{-(n+1)}<2^{-n}$ donc $V_n\subset\inv{d_U}([0,2^{-n}[)$.
			\item Montrons par récurrence forte:
			\begin{itemise}
				\item $\forall q\in\N, \exists a_{1\infegal k\infegal q}\text{ de }x\text{ à }y, \sum_{k=0}^qf_U(a_k, a_{k+1})<2^{-n}\Rightarrow (x, y)\in V_{n-1}$
				\item Pour $q=0$ si $f_U(x, y)<2^{-n}$ donc $f_U(x, y)\infegal2^{-n-1}$ soit $(x, y)\in V_{n-1}$.
				\item Soit $\alpha=\sum_{k=0}^{q+1}f_U(a_k, a_{k+1})<2^{-n}$.
				Il existe deux rangs $p$ et $p'$ tels que $\sum_{k=0}^pf_U(a_k, a_{k+1}), \sum_{k=p+1}^{p'}f_U(a_k, a_{k+1}), \sum_{k=p'+1}^{q+1}f_U(a_k, a_{k+1})<\frac{\alpha}{2}<2^{-n-1}$.
				Par hypothèse de récurrence $(a_0, a_{p+1}), (a_{p+1}, a_{p'+1}), (a_{p'+1}, a_{q+1})\in V_n$.
				Par construction $(a_0, a_{q+1})=(x, y)\in V_n\circ V_n\circ V_n\subset V_{n-1}$.
			\end{itemise}
			\item Soit $(x, y)\in\inv{d_U}([0, 2^{-n}[)$.
			Alors $\exists a_{1\infegal k\infegal q}\text{ de }x\text{ à }y, \sum_{k=0}^qf_U(a_k, a_{k+1})<2^{-n}$.
			D'après la récurrence précédente $(x, y)\in V_{n-1}$.
		\end{itemise}
		Ainsi la famille d'écarts saturés engendre l'uniformité, et donc la même topologie.
		\item[écarts$\Rightarrow T_{3\frac{1}{2}}$] Soit $\mathscr{T}$ une topologie engendrée par la jauge $d_{i\in I}$, $x, F\in E\times\complement\mathscr{T}$ avec $x\notin F$.
		Alors $x\in\complement F$ est un voisinage de $x$ donc $\exists U\in\mathscr{U}, \complement F=U[x]$.
		Par définition de l'uniformité engendrée par une jauge il existe $\varepsilon\in\R^+$ et $i\in I$ tels que $U=\inv{d_i}([0, \varepsilon[)$ donc $\complement F=\{y|d_i(x, y)<\varepsilon\}$.
	\end{itemise}
\end{demonstration}

\section{Continuité uniforme}

Les espaces $E, F$ sont munis des uniformités $\mathscr{U}, \mathscr{V}$.
On se donne une application $f:E\rightarrow F$.

\begin{definition}
	$f$ est uniformément continue si et seulement si tout entourage $V\in\mathscr{V}$ il existe un entourage $U\in\mathscr{U}$ tel que $\forall (x, y)\in U, (f(x), f(y))\in V$.
\end{definition}

\begin{proposition}
	$f$ est uniformément continue si et seulement si $\forall V\in\mathscr{V}, \inv{f\times F}(V)\in\mathscr{U}$.
\end{proposition}




\chapter{Compacité}

La compacité est une propriété particulièrement importante qui confère aux espaces topologiques des qualités proches de celles d’un ensemble fini.
Ceci est particulièrement clair des propositions ci-dessous, qui deviennent évidentes dans le cas d’un espace composé d’un nombre fini de points.

\section{Vocabulaire}

\begin{definition}
	Un espace topologique est quasi-compact si de tout recouvrement par des ouverts, on peut extraire un sous recouvrement fini.
	Par passage au complémentaire, il revient au même de dire que de toute famille de fermés d’intersection vide on peut extraire une sous-famille finie d’intersection vide.
\end{definition}

\begin{definition}
	Un espace topologique est compact s'il est séparé $T_2$ et quasi-compact.
\end{definition}

\begin{definition}
	Un espace topologique est localement compact s'il est séparé et admet des voisinages compacts pour tous ses points.
\end{definition}

\begin{proposition}
	Une partie quasi-compact d'un espace topologique séparé est fermé.
\end{proposition}

\begin{demonstration}
	Soit $K$ un compact et $x, y\in\complement K\times K$.
	Par séparabilité $\exists U_{xy}, V_{xy}\in\mathscr{T}, x\in U_{xy}, y\in V_{xy}, U_{xy}\cap V_{xy}=\vide$.
	Ainsi $\bigcup_{y\in K}V_{xy}$ est un recouvrement ouvert de $K$ donc par quasi-compacité on peut en extraire un sous recouvrement fini $V_{xy^k}$.
	Ainsi $\bigcap_{k=1}^nU_{xy^k}$ est un ouvert disjoint de $K$ puis $\bigcup_{x\in\complement K}\bigcap_{k=1}^nU_{xy^k}=\complement K$ est un ouvert.
\end{demonstration}

\begin{proposition}
	Un fermé d'un espace topologique quasi-compact est quasi-compact.
\end{proposition}

\begin{demonstration}
	Soit $F$ un fermé de $E$ quasi-compact et $\bigcup_{i\in I}O_i$ un recouvrement de $F$ par des ouverts.
	Puisque $\complement F$ est un ouvert, $\complement F\cup\bigcup_{i\in I}O_i$ est un recouvrement de $E$ par des ouverts.
	Par compacité, on peut en extraire un sous recouvrement fini.
	Ce recouvrement fini couvre $F$, donc $F$ est quasi-compact.
\end{demonstration}

\begin{proposition}
	Un espace topologique est quasi-compact si et seulement si de tout filtre on peut en extraire un filtre plus fin convergent.
\end{proposition}

\begin{demonstration}
	Par double implication.
	\begin{itemise}
		\item[$\Rightarrow$] Soit $\mathfrak{F}$ un filtre sur $E$ quasi-compact.
		Par l'absurde supposons $\bigcap_{X\in\mathfrak{F}}\overline{X}=\vide$.
		Alors $\bigcup_{X\in\mathfrak{F}}\complement\overline{X}=E$ donc par quasi-compacité $\exists X_{1\infegal i\infegal n}\in\mathfrak{F}, \bigcup_{i=1}^n\complement\overline{X_i}=E$.
		Ainsi $\bigcap_{i=1}^n\overline{X_i}=\vide$ en particulier $\bigcap_{i=1}^nX_i=\vide\in\mathfrak{F}$ par stabilité par intersection finie des filtres, ce qui est absurde.
		Soit donc $x\in\bigcap_{X\in\mathfrak{F}}\overline{X}$ et $\mathfrak{B}=\{V\cap X | V\in\mathfrak{V}(x), X\in\mathfrak{F}\}$.
		C'est une base de filtre.
		En effet les $V\cap X$ sont toujours non vides car $x$ adhère à tout les $X\in\mathfrak{F}$ donc $\vide\notin\mathfrak{B}$ et les $V\cap X$ forment les voisinages de $x$ donc $\mathfrak{V}(x)\subset\mathfrak{B}$.
		De plus cette famille est stable par intersection car $\mathfrak{V}(x)$ et $\mathfrak{F}$ le sont.
		Ainsi le filtre engendré par $\mathfrak{B}$ converge vers $x$.
		\item[$\Leftarrow$] Par contraposée soit $(F_i)_{i\in I}$ une famille de fermés de $E$ dont les intersections finies sont non vides.
		Posons $\mathfrak{B}$ la collection de ces intersections.
		C'est une base de filtre car ses éléments sont non vides et qu'elle est stable par intersection deux à deux.
		Par hypothèse, il existe donc un filtre $\mathfrak{F}$ convergent plus fin que $\sigma_\text{filtre}(\mathfrak{B})$.
		Soit $x$ sa limite.
		Ainsi $\sigma_\text{filtre}(\mathfrak{B})$ est un filtre sur $F_i$ (qui est une intersection finie des $F_{i\in I}$) convergent vers $x$ donc $x$ adhère à tout les $F_i$ fermés donc $\forall i\in I, x\in F_i$ soit $\bigcap_{i\in I}F_i\neq\vide$.
	\end{itemise}
\end{demonstration}

\begin{proposition}
	Un espace est quasi-compact si et seulement si tout ultrafiltre converge.
\end{proposition}

\begin{demonstration}
	Par double implication.
	\begin{itemise}
		\item[$\Rightarrow$] Supposons $E$ quasi-compact.
		Soit $\mathfrak{U}$ un ultrafiltre sur $E$.
		Par le théorème précédent on peut extraire un filtre de $\mathfrak{U}$ convergent.
		Puisque $\mathfrak{U}$ est un ultrafiltre, $\mathfrak{U}$ converge.
		\item[$\Leftarrow$] Supposons que tout ultrafiltre sur $E$ converge.
		Soit un filtre $\mathfrak{F}$.
		Il est contenu dans un ultrafiltre d'après le lemme de Zorn, qui converge par hypothèse.
	\end{itemise}
\end{demonstration}

\begin{proposition}
	Tout produit d’espaces quasi-compacts est quasi-compact.
\end{proposition}

\begin{demonstration}
	Soit $\mathfrak{U}$ un ultrafiltre sur $E=\prod_{i\in I}E_i$ produit d'espaces compacts de projections $\pi_i$.
	Alors $\pi_i(\mathfrak{U})$ est la base d'un ultrafiltre $\mathfrak{U}_i$ sur $E_i$, qui converge d'après le théorème précédent.
	Appelons $x_i\in E_i$ sa limite et posons $x=(x_i)_{i\in I}\in E$.
	Soit $U=\prod_{k=1}^nO_{i_k}\times\prod_{i\in I\setminus\{i_1, ..., i_n\}}E_i$ un ouvert élémentaire de $E$ contenant $x$.
	Comme $\mathfrak{U}_i\rightarrow x_i$ on a $O_{i_k}\in\mathfrak{U}_{i_k}=\sigma_\text{filtre}(\pi_{i_k}(\mathfrak{U}))$.
	Puisque $\pi_{i_k}(\mathfrak{U})$ est une base $\exists V\in\mathfrak{U}, \pi_{i_k}(V)\subset O_{i_k}$ soit $V\subset\inv{\pi_{i_k}}(O_{i_k})$.
	Par croissance des filtres $\inv{\pi_{i_k}}(O_{i_k})=O_{i_k}\times\prod_{i_k\neq j\in I}E_j\in\mathfrak{U}$.
	Par stabilité des filtres par intersection finie $\bigcap_{k=1}^n\inv{\pi_{i_k}}(O_{i_k})=U\in\mathfrak{U}$.
	L’ultrafiltre $\mathfrak{U}$ est donc convergent vers $x$, ce qui prouve que $E$ est compact.
\end{demonstration}

\section{Application aux fonctions continues}

\begin{proposition}
	L’image continue d’un quasi-compact est quasi-compacte.
\end{proposition}

\begin{demonstration}
	Soit $E$ un quasi-compact et $f:E\rightarrow F$ une application continue.
	Soit $\bigcup_{i\in I}O_i$ un recouvrement ouvert de $f(E)$.
	Ainsi $\bigcup_{i\in I}\inv{f}(O_i)=\inv{f}(\bigcup_{i\in I}O_i)=\inv{f}(f(E))=E$ est un recouvrement ouvert de $E$.
	Par compacité, on peut en extraire un sous recouvrement fini $O_{i_k}$.
	Alors $f(E)=f(\bigcup_{k=1}^n\inv{f}(O_{i_k}))=\bigcup_{k=1}^nf(\inv{f}(O_{i_k}))=\bigcup_{k=1}^nO_{i_k}$ est un recouvrement par des ouverts de l'image qui est donc quasi-compact.
\end{demonstration}

\begin{remarque}
	Voici maintenant une application de la compacité aux problèmes d’optimisation, le résultat qui suit est en effet est à la base des démonstrations qui prouvent l’existence d’un optimum.
	De façon générale, il s’agit de trouver quel jeu de valeurs d’un certain nombre de paramètres rend optimal le résultat d’une opération donnée ; il peut s’agir de son coût, de sa rapidité, de la résistance à la flexion d’une poutre en fonction de sa forme géométrique Ce dernier exemple est d’ailleurs plutôt instructif, puisqu’on peut montrer qu’un tel maximum n’existe pas dans l’ensemble des géométries qui conservent le volume et la longueur de la poutre !
	C’est ce qui explique que les structures légères et résistantes soient du type ‘nid d’abeilles’ et sortent du cadre des formes géométriques régulières.
\end{remarque}

\begin{proposition}
	Une fonction $f:K\rightarrow\R$ (dite fonction numérique) continue sur un compact est bornée, elle atteint son maximum et son minimum.
\end{proposition}

\begin{demonstration}
	D’après la proposition précédente l’image de $f$ est compacte, elle est donc bornée car $\R$ est métrique.
	Ainsi $M=\sup_{x\in K}f(x)$ existe.
	Soit $z_n$ une suite croissante (dite suite maximisante) de limite $M$ et  M pour limite et $F_n=\{x\in K | z_n\infegal f(x)\infegal M\}$.
	Les $F_n$ constituent une suite décroissante de fermés non vides.
	Par compacité leur intersection n’est pas vide.
	Soit $x\in\bigcap_{n\in\N}F_n$ alors $\forall n\in\N, z_n\infegal f(x)\infegal M$ donc $f(x)=M$.
	Le maximum est atteint ; il en est de même pour le minimum.
\end{demonstration}

\begin{remarque}
	La compacité est une propriété particulièrement agréable mais rare ; en témoigne la proposition suivante, qui montre qu’une topologie d’espace compact ne peut être affaiblie sans perdre la propriété de séparation, en particulier, au sein d’une famille de topologies séparées comparables sur un ensemble donné, l’une au plus est une topologie d’espace compact.
\end{remarque}

\begin{proposition}
	Si $E$ est quasi-compact pour une topologie il ne peut pas être muni d’une topologie séparé moins fine.
\end{proposition}

\begin{demonstration}
	Soit $\id:(E, \mathscr{T}_1)\rightarrow(E, \mathscr{T}_2)$ avec $(E, \mathscr{T}_1)$ compact $(E, \mathscr{T}_2)$ séparé et $\mathscr{T}_2\subset\mathscr{T}_1$.
	Par comparaison $\id$ est continue.
	Soit $F$ un fermé de $(E, \mathscr{T}_1)$ quasi-compact alors $F$ est quasi-compact.
	Sont image $\id(F)=F$ est donc quasi-compact dans $(E, \mathscr{T}_2)$ séparé donc $F\in\complement\mathscr{T}_2$.
	Ainsi $\mathscr{T}_1\subset\mathscr{T}_2$.
\end{demonstration}

\part{Et après ?}

\chapter{Groupes topologiques}

\section{Définitions et premières propriétés}

\begin{definition}
	On appelle groupe topologique $(G, \star)$ un groupe munie d'une topologie pour laquelle les applications $\star:G^2\rightarrow G$ et $\inv{}:G\rightarrow G$ sont continues.
\end{definition}

\begin{proposition}
	Un groupe $(G, \star)$ munie d'une topologie est un groupe topologique si et seulement si $x, y\longmapsto x\star\inv{y}$ est continue.
\end{proposition}

\begin{demonstration}
	Par double implication.
	\begin{itemise}
		\item[$\Rightarrow$] Le sens directe vient de la composition d'applications continues.
		\item[$\Leftarrow$] En prenant $x=e$ puis $\inv{y}$ dans $x, y\longmapsto x\star\inv{y}$, on vérifie que c'est un groupe topologique.
	\end{itemise}
\end{demonstration}

%\section{Mesure de Haar}

\begin{definition}
	Sur tout groupe topologique localement compact, il existe une et une seule mesure de Borel quasi-régulière non nulle (à coefficient multiplicateur près) invariante par les translations à gauche $x\longmapsto y*x$ : la mesure de Haar.
\end{definition}

\begin{demonstration}
	Cf théorie de la mesure.
\end{demonstration}

\begin{proposition}
	Les translations $x\longmapsto x\star a$ et $x\longmapsto a\star x$ sont des homéomorphismes.
\end{proposition}

\begin{demonstration}
	Par continuité de $\star$ ces deux applications sont continues et leurs réciproques $x\longmapsto x\star\inv{a}$ et $x\longmapsto\inv{a}\star x$ sont continues.
\end{demonstration}

\begin{proposition}
	Dans un groupe topologique, les voisinages d’un point sont les translatés des voisinage de $e$.
\end{proposition}

\begin{demonstration}
	Soit $W$ un voisinage de $x$.
	Comme $y\longmapsto(x, y)$ et $(x, y)\longmapsto x\star y$ sont continues, par composition $\phi:y\longmapsto x\star y$ l'est aussi.
	Or $\phi(e)=x$ donc $V=\inv{\phi}(W)$ est un voisinage de $e$.
	Ainsi $x+V=\phi(V)=\phi(\inv{\phi}(W))=W$ par surjectivité de $\phi$.
\end{demonstration}

\begin{proposition}
	Soit $G$ et $H$ deux groupes topologiques, $f:G\rightarrow H$ un morphisme de groupe.
	$$
		f\in\cont(G, H)\iff f\in\cont(e_G, H)
	$$
\end{proposition}

\begin{demonstration}
	Par double implication
	\begin{itemise}
		\item[$\Rightarrow$] Le sens direct est évident.
		\item[$\Leftarrow$] Soit $x\in G$ et $V$ un voisinage de $y=f(x)$.
		D'après la proposition précédente il existe un voisinage $W$ de $e_H$ tel que $V=y\star_HW$.
		Puisque $f$ est un morphisme de groupe, $f(e_G)=e_H$.
		Par continuité de $f$ en $e_G$, $\inv{f}(W)$ est un voisinage de $e_G$.
		Par continuité des translations, $x\star_G\inv{f}(W)$ est un voisinage de $x$.
		Or $x\star_G\inv{f}(W)\subset\inv{f}(y\star_HW)$.
		En effet si $x\star_Gz\in x\star_G\inv{f}(W)$ alors $f(x\star_Gz)=f(x)\star_Hf(z)\in y\star_HW$.
		Donc $\inv{f}(V)$ contient un voisinage de $x$.
	\end{itemise}
\end{demonstration}

\begin{proposition}
	Un groupe topologique est séparé si et seulement si le singleton ${e}$ est fermé.
\end{proposition}

\begin{demonstration}
	Par double implication.
	\begin{itemise}
		\item[$\Rightarrow$] Soit $y\in\widetilde{\{x\}}$.
		Alors $\exists\mathfrak{F}\text{ filtre sur $\widetilde{\{x\}}$}, \mathfrak{F}\rightarrow y$.
		Cette convergence a lieu pour la topologie trace c'est à dire pour l'ensemble des voisinages $\mathfrak{V}'(y)=\{V\cap\{x\}|V\in\mathfrak{V}(y)\}$.
		Par séparation, si $y\neq x$, il existerait un voisinage de $y$ de rencontrant pas $\{x\}$ et donc $\vide\in\mathfrak{V}'(y)\subset\mathfrak{F}$, ce qui est impossible pour un filtre.
		Ainsi les singletons sont fermé, notamment $\{E\}$.
		\item[$\Leftarrow$] Réciproquement, si $\{E\}$ est fermé alors $G\setminus\{E\}$ est ouvert.
		Par continuité de $\phi:x, y\longmapsto x\star\inv{y}$, $\inv{\phi}(G\setminus\{E\})=G^2\setminus\{(x, x)|x\in G\}$ est un ouvert de $G$.
		Ainsi pour $(x, y)\in G^2$ avec $x\neq y$, il existe un voisinage pour la topologie produit de $(x, y)$ ne rencontrant pas la diagonale.
		Par définition de cette topologie, il existe un voisinage $V_x$ de $x$ et $V_y$ de $y$ tel que $(V_x\times V_y)\cap\{(x, x)|x\in G\}=\vide$, c'est à dire $V_x\cap V_y=\vide$.
	\end{itemise}
\end{demonstration}

\begin{proposition}
	Soit $O$ un ouvert et $X$ une partie d'un groupe topologique.
	Alors $O\star X$ et $O\star X$ sont ouverts.
\end{proposition}

\begin{demonstration}
	$O\star X=\bigcup_{x\in X}O\star x$ est ouvert car $O\star x$ l'est car c'est l'image réciproque de l'ouvert $O$ par l'application continue $y\longmapsto y\star x^{-1}$.
	Idem pour l'autre sens.
\end{demonstration}

\begin{proposition}
	Tout groupe quotient $G/H$ d'un groupe topologique $G$ par un sous-groupe disingué $H$ est encore un groupe topologique, lorsque $G/H$ est muni de la topologie quotient.
	De plus, $G/H$ est séparé si et seulement si $H$ est fermé.
\end{proposition}

\begin{demonstration}
	Soit $OH$ un voisinage de $eH$ dans $G/H$.
	Par définition de la topologie quotient, $\inv{\pi}(OH)=O$ est un voisinage de $e$ dans $G$.
	Par continuité, $\inv{\cdot}(O)$ est un voisinage de $(e, e)$ dans $G\times G$.
	Par définition de la topologie produit, il existes deux voisinages $O_1$ et $O_2$ de $e$ tels que $\inv{\cdot}(O)=O_1\times O_2$.
	Puisque $\inv{\pi}(\pi(O_i))\subset O_i$, $\inv{\pi}(\pi(O_i))$ est un voisinage de $e$ dans $G$.
	Par définition de la topologie quotient, $\pi(O_i)$ est un voisinage de $eH$ dans $G/H$.
	Par définition de la topologie produit, $\pi(O_1)\times\pi(O_2)$ est un voisinage de $(e, e)$ dans $(G/H)^2$.

	De plus $\inv{*}(OH)\supset\pi(O_1)\times\pi(O_2)$.
	En effet, soit $(x_1H, x_2H)\in\pi(O_1)\times\pi(O_2)$, c'est à dire $(x_1, x_2)\in O_1\times O_2=\inv{\cdot}(O)$, alors $x_1H*x_2H=(x_1\cdot x_2)H\in OH$.
	Donc $*$ est continue en $(e, e)$ en tant que morphisme de groupe de $(G/H)^2$ dans $G/H$.
	Elle est donc continue.
	\begin{center}
		\begin{tikzpicture}[xscale=3, yscale=1.5]
			% Noeuds (MODIFIABLES : Styles et Coefficients d'InterFeuilles)
			\node (GG) at (0, 0) {$G\times G$};
			\node (G) at (1, 0) {$G$};
			\node (QQ) at (0, 1) {$G/H\times G/H$};
			\node (Q) at (1, 1) {$G/H$};
			% Arcs (MODIFIABLES : Styles)
			\draw[black, -stealth] (GG)-- node [above, midway] {$\cdot$} (G);
			\draw[black, -stealth] (QQ)-- node [above, midway] {$*$} (Q);
			\draw[black, -stealth] (GG)-- node [left, midway] {$(\pi, \pi)$} (QQ);
			\draw[black, -stealth] (G)-- node [left, midway] {$\pi$} (Q);
		\end{tikzpicture}
	\end{center}
\end{demonstration}

\chapter{Espace vectoriel topologique}

\part{Analyse}

\chapter{Espaces localement convexes}

\section{Définitions et premières propriétés}

\begin{definition}
	Un $\R$-espace vectoriel topologique $E$ est un $\R$-espace vectoriel dont l'addition $+:E\times E\rightarrow E$ et la multiplication externe $\cdot:\R\times E\rightarrow E$ sont continues.
\end{definition}

\begin{proposition}
	Dans un $\R$-espace vectoriel topologique, les voisinages d’un point sont les translatés des voisinage de l'origine.
\end{proposition}

\begin{demonstration}
	Soit $W$ un voisinage de $x$.
	Comme $y\longmapsto(x, y)$ et $(x, y)\longmapsto x+y$ sont continues, par composition $\phi:y\longmapsto x+y$ l'est aussi.
	Or $\phi(0)=x$ donc $V=\inv{\phi}(W)$ est un voisinage de 0.
	Ainsi $x+V=\phi(V)=\phi(\inv{\phi}(W))=W$ par surjectivité de $\phi$.
\end{demonstration}

\begin{proposition}
	Soit $E$ un $\R$-espace vectoriel topologique.
	\begin{itemise}
		\item Un voisinage $V$ de 0 est absorbant: $\forall x\in E, \exists\Lambda>0, \forall\lambda\in[-\Lambda, \Lambda], \lambda x\in V$.
		\item l'ensemble des voisinages de 0 est invariant par homothétie.
		\item Un voisinage $V$ de 0 contient un voisinage $W$ équilibré: $\forall\lambda\in[-1, 1], \lambda W\subset W$.
		\item Une base de voisinage de 0 est formée de voisinages ouverts équilibrés et absorbants.
		\item tout voisinage $V$ de 0 contient un voisinage $W$ tel que $W+W\subset V$.
	\end{itemise}
\end{proposition}

\begin{demonstration}
	Soit $V$ un voisinage de $0_E$.
	\begin{itemise}
		\item À $x$ fixé, par continuité en $0_E$ de $\lambda\longmapsto\lambda x$, il existe un ouvert $I$ de $0_\R$ vérifiant la propriété.
		$I$ peut être non borné, mais contient un voisinage de la forme $[-\Lambda, \Lambda]$.
		\item À $\lambda\neq0$ fixé, par continuité de $x\longmapsto\frac{1}{\lambda}x$, il existe un voisinage $U$ de $0_E$ tel que $\frac{1}{\lambda}U\subset V$ id est $U\subset\lambda V$, faisant de $\lambda V$ un voisinage de $0_E$.
		\item Par continuité de $\lambda, x\longmapsto\lambda x$, il existe un voisinage $[-M, M]$ de $0_\R$ et $U$ de $0_E$ tels que $\forall\mu, x\in[-M, M]\times U, \mu x\in V$.
		Il suffit de poser $W=\bigcup_{\mu\in[-m, M]}\mu U$.
		\item Il suffit de prendre l'ensemble des $W^\circ$ de la démonstration précédente.
		\item Par continuité de $x, y\longmapsto x+y$, il existe deux voisinages $U_1$ et $U_2$ de $0_E$ tels que $\forall x1, x_2\in U_1\times U_2, x_1+x_2\in V$.
		Il suffit de prendre $W=U_1\cap U_2$ qui est encore un voisinage de $0_E$ par stabilité par intersection finie.
	\end{itemise}
\end{demonstration}

\begin{proposition}
	Soit $E$ et $F$ deux $\R$-espaces vectoriels topologiques, et $f:E\rightarrow F$ linéaire.
	$$
		f\in\cont(0_E, F)\implies f\in\cont(E, F)
	$$
\end{proposition}

\begin{demonstration}
	Soit $x\in E$, $y=f(x)$ et $W$ un voisinage de $y$.
	Il existe alors un voisinage $V$ de $0_F$ tel que $W=y+V$
	Par translation des voisinage $\inv{f}(W)=\inv{f}(y+V)\supset x+\inv{f}(V)$.
	Par continuité en $0_E$ de $f$, $\inv{f}(V)$ est un voisinage de $0_E$.
	Ainsi $\inv{f}(W)$ est un voisinage de $x$.
\end{demonstration}

\begin{proposition}
	Soit $E$ un ensemble qui soit un espace $\R$-vectoriel et un espace topologique.
	S'il vérifie les conclusions des deux premières propositions, alors réciproquement c'est un un espace vectoriel topologique.
\end{proposition}

\begin{demonstration}
	Il suffit d'utiliser le proposition précédente sur $+$ et $\times$, les hypothèses assurant leur continuité en $(0_E, 0_E)$ et $(0_\R, 0_E)$.
\end{demonstration}

\begin{definition}
	Une partie d'un $\R$-espace vectoriel topologique est bornée si et seulement si elle est absorbée par tout voisinage de 0.
\end{definition}

\begin{proposition}
	Les compacts d'un $\R$-espace vectoriel topologique sont bornés.
\end{proposition}

\begin{demonstration}
	Soient $K$ un compact et $V$ un voisinage ouvert de 0.
	Soit $x\in K$, par absorption de $K$ par $V$, $\exists n\in\N^*, \forall\lambda\in[-\frac{1}{n}, \frac{1}{n}], \lambda x\in V$, c'est à dire $\exists n\in\N^*, x\in nV$.
	Ainsi $K\subset\bigcup_{n\in\N^*}nV$.
	Par compacité de $K$ on peut extraire de ce recouvrement ouvert un sous recouvrement fini.
	Soit $N$ le plus grand indice de ce recouvrement fini, alors $K\subset NV$, ce qui fait de $K$ une partie bornée.
\end{demonstration}

\chapter{Espaces métriques}

\section{Définitions et premières propriétés}

\begin{definition}
	On appelle distance sur $E$ toute application $d:E\times E\rightarrow\R^+$ qui soit séparable, symétrique et Minkovskienne, c'est à dire:
	\begin{itemise}
		\item $\forall x, y\in E, d(x, y)=0\iff x=y$
		\item $\forall x, y\in E, d(x, y)=d(y, x)$
		\item $\forall x, y, z\in E, d(x, z)\infegal d(x, y)+d(y, z)$
	\end{itemise}
\end{definition}

\begin{proposition}
	Soit $(E, d)$ un espace métrique.
	C'est un espace séparable pour la topologie engendrée par les boules ouvertes $B_\epsilon(x)=\{y|d(x, y)<\epsilon\}=\inv{d(x, \cdot)}([0, \epsilon[)$.
\end{proposition}

\begin{demonstration}
	Soit $x, y\in E$ deux points différents de $E$.
	Alors $\epsilon=d(x, y)>0$ par séparabilité de la distance.
	Donc $B_{\epsilon/2}(x)$ et $B_{\epsilon/2}(y)$ sont deux voisinages disjoins de $x$ et $y$ respectivement.
\end{demonstration}

\begin{definition}
	Soit $E$ un espace topologique.
	S’il existe une distance permettant de construire une topologie, on dit que la topologie découle d’une distance et on parle d’espace métrisable.
\end{definition}

\begin{remarque}
	Différentes distances peuvent être à l’origine de la même topologie, c’est pourquoi on distingue espace métrique (où la distance est donnée) et espace métrisable (où seule la topologie est donnée).
\end{remarque}

\begin{proposition}
	Soient $E$ et $F$ deux espaces métriques définis par $d$ et $\delta$, $f:E\rightarrow F$.
	$$
		f\in\cont(x, F)
		\iff
		\forall\epsilon>0, \exists\eta, \forall y\in E, d(x, y)<\eta\implies\delta(f(x), f(y))<\epsilon
	$$
\end{proposition}

\begin{demonstration}
	On reformule simplement la définition avec les voisinages.
	\begin{itemise}
		\item[$\Rightarrow$] Soit $\epsilon>0$.
		Puisque $B_\epsilon(f(x))$ est un voisinage de $x$, $\exists V\in\mathfrak{V}(x), f(V)\subset B_\epsilon(f(x))$.
		Les boules ouvertes étant par construction une base de voisinages, $\exists\eta>0, B_\eta(x)\subset V$.
		On a donc bien que si $d(x, y)<\eta$, alors $y\in V$, donc $f(y)\in B_\epsilon(f(x))$, soit $\delta(f(x), f(y))<\epsilon$.
		\item[$\Leftarrow$] Soit $W\in\mathfrak{W}(f(x))$.
		Alors pour $\epsilon$ assez petit, $B_\epsilon(x)\subset W$.
		Donc $V=B_\eta(x)$ est un voisinage de $x$ tel que si $y\in V$, alors $d(x, y)<\eta$, donc $\delta(f(x), f(y))<\epsilon$, soit $f(y)\in B_\epsilon(f(x))$.
		Donc $f(V)\subset W$, assurant la continuité de $f$ en $x$.
	\end{itemise}
\end{demonstration}

\section{Les suites suffisent}

\begin{proposition}
	Soit $E$ un espace métrique.
	$$
		(x_n)\rightarrow x
		\iff
		\forall\epsilon>0, \exists N, \forall n\supegal N, d(x_n, x)<\epsilon
	$$
\end{proposition}

\begin{demonstration}
	On reformule simplement la définition avec les voisinages.
	\begin{itemise}
		\item[$\Rightarrow$] Soit $\epsilon>0$.
		Puisque $B_\epsilon(x)$ est un voisinage de $x$, $\exists N, \forall n\supegal N, x_n\in B_\epsilon(x)$, c'est à dire $\exists N, \forall n\supegal N, d(x_n, x)\epsilon$.
		\item[$\Leftarrow$] Soit $V\in\mathfrak{V}(x)$.
		Alors pour $\epsilon$ assez petit, $B_\epsilon(x)\subset V$.
		Or $\exists N, \forall n\supegal N, x_n\in B_\epsilon(x)$, donc $\exists N, X_N\in V$.
		Par caractérisation du filtre engendré par une base de filtre, $(x_n)$ converge.
	\end{itemise}
\end{demonstration}

\begin{proposition}
	Soit $X$ une partie d'un espace métrique $E$.
	Alors $\overline{X}=\overline{X}^\text{seq}$
\end{proposition}

\begin{demonstration}
	L'inclusion réciproque à déjà été prouvée précédemment.
	Soit $x\in\overline{X}$ et $V$ un voisinage de $x$.
	Alors $V$ contient une boule ouverte centrée en $x$, c'est à dire $\exists\epsilon>0, B_\epsilon(x)\subset V$.
	En prenant $x_n\in B_{1/n}(x)\cap X$ qui est non vide car $x$ est dedans, avec $1/n<\epsilon$, on obtient une suite $(x_n)$ de $X$ qui converge vers $x$.
\end{demonstration}

\begin{proposition}
	Soient $E$, $F$ métriques, $f:E\rightarrow F$ une application.
	$f$ est continue en $x$ si et seulement si la convergence d'une suite $(x_n)$ vers $x$ implique la convergence de $f(x_n)$ vers $f(x)$.
\end{proposition}

\begin{demonstration}
	L'implication directe à déjà été prouvée précédemment.
	Soit $W$ un voisinage de $f(x)$.
	Alors $W$ continent une boule ouverte centrée en $f(x)$, c'est à dire $\exists\epsilon>0, B^F_\epsilon(f(x))\subset V$.
	Soit $(x_n)$ une suite de $\inf{f}(W)$ telle que $x_n\in B^E_{1/n}(x)$.
	Alors $x_n\rightarrow x$ donc $f(x_n)\rightarrow f(x)$, donc $\exists N, \forall n\supegal N, f(x_n)\in B^F_\epsilon(f(x))\subset W$.
	Ainsi $f(B^E_{1/n}(x))\subset W$ et $B^E_{1/n}(x)$ est un voisinage de $x$.
\end{demonstration}

\section{Compacité}

\begin{proposition}
	Un espace topologique est compact si et seulement si de toute suite on peut en extraire une sous suite convergente.
\end{proposition}

\begin{demonstration}
	L'implication directe à déjà été prouvée précédemment.
	Soit $\bigcup_{i\in I}O_i$ un recouvrement ouvert de $E$.

	Montrons $\exists\epsilon>0, \forall x_\epsilon\in E, \exists i\in I, B_\epsilon(x_\epsilon)\subset O_i$.
	Raisonnons par l'absurde.
	En prenant $\epsilon=1/n$, on construit une suite $(x_n)$ telle que $\forall i\in I, B_{1/n}(x_n)\not\subset O_i$.
	Par hypothèse, on peut en extraire une sous suite $(y_n)$ convergente, de limite $y$.
	Puisque $\bigcup_{i\in I}O_i$ est un recouvrement de $E$ et que $y\in E$, $\exists i\in I, y\in O_i$.
	Les ouverts étant engendrés par les boules ouvertes, $\exists\eta>0, B_\eta(y)\subset O_i$.
	Par convergence de $(y_n)$ vers $y$, $\exists N, \forall n\supegal N, d(y_n, y)<\eta/2$.
	Prenons maintenant $n\supegal\max(N, 2/\eta)$ et $z\in B_{1/n}(y_n)$ c'est à dire $d(z, y_n)<1/n$.
	Alors $d(y, z)\infegal d(y, y_n)+d(y_n, z)<\eta/2+1/n\infegal\eta$, c'est à dire $B_{1/n}(y_n)\subset B_\eta(y)\subset O_i$, ce qui est impossible.

	Montrons que $\forall\eta>0$, il existe un recouvrement fini de $E$ par des boules ouvertes de rayon $\eta$.
	Raisonnons par l'absurde.
	Soit $x_0\in E$ et $R_0=\vide$.
	Définissons par récurrence $x_{n+1}=E\setminus R_n$ et $R_{n+1}=R_n\cup B_\eta(x_{n+1})$.
	Ces suites sont bien définies car $R_0$ ne recouvre pas $E$, et que si $R_n$ ne recouvre pas $E$, alors $x_{n+1}$ existe, et par hypothèse $R_{n+1}$ est une unions finie de boules ouvertes donc ne peut recouvrir $E$.
	En particulier on a que $d(x_{n+1}, x_n)\supegal\eta$.
	Par hypothèse, on peut en extraire une sous suite $(y_n)$ convergente, de limite $y$.
	Par convergence de $(y_n)$ vers $y$, $\exists N, \forall n\supegal N, d(y_n, y)<\eta/2$.
	En particulier, $d(y_{N+1}, y_N)\infegal d(y_{N+1}, y)+d(y, y_n)<\eta$, ce qui est impossible.

	Ainsi, on peut recouvrir $E$ par un nombre fini $n$ de boules ouvertes de rayon $\epsilon$.
	D'après la première partie, elles sont chacune incluses dans un ouvert du recouvrement de $E$.
	Ces ouverts forment donc un recouvrement fini de $E$, ce qui montre la compacité de $E$.
\end{demonstration}

\begin{definition}
	Une partie d'un espace métrique est bornée si et seulement si la distance entre chacun de ses point est majorée par une constante.
\end{definition}

\begin{proposition}
	Une partie compact d'un espace métrique est fermée et bornée.
\end{proposition}

\begin{demonstration}
	Un espace métrique est topologique séparé, et une partie compact d'un espace topologique séparé est fermé.
	De plus d'après la démonstration précédente, cette partie compact peut être recouverte par un nombre fini $n$ de boules ouvertes de rayon 1.
	Par inégalité triangulaire, on majore la distance entre n'importe quel bipoint par $1+n+1$.
\end{demonstration}

\begin{proposition}
	Une fonction numérique réelle d'un compact est bornée et atteint ses bornes.
\end{proposition}

\begin{demonstration}
	L’image d'un compact par une application continue est compacte.
	Elle est donc fermée et bornée d'après la proposition précédente.
	Par bornitude, elle admet un élément maximal $y^+$.
	Par fermeture séquentielle, il existe une suite $(y_n)$ de l'image de limite $y^+$, la convergence étant prise pour la distance $d(x, y)=|x-y|$.
	Posons $F_n=\{x|y_n\infegal f(x)\infegal y^+\}$ une suite décroissante pour l'inclusion de fermés de l'image.
	Par l'absurde, si $\bigcap_{n\in\N}F_n=\vide$, alors par compacité de l'image, $\exists N, \bigcap_{n\infegal N}F_n=\vide$.
	C'est absurde car par décroissance des $F_n$, on aurait $\bigcap_{n\infegal N}F_n=F_{\max(N)}=\vide$.
	Soit alors $x\in\bigcap_{n\in\N}F_n$, il vient $\forall n\in\N, y_n\infegal f(x)\infegal y^+$, donc $|f(x)-y^+| \infegal |y_n-y^+|$.
	Par convergence de $(y_n)$ vers $y$, $\forall\epsilon>0, |f(x)-y^+|<\epsilon$, donc $f(x)=y^+$.
	Le maximum est donc atteint.
	On procède de même pour le minimum.
\end{demonstration}

\chapter{Espaces vectoriels normés}

\chapter{JSP}

\paragraph{Définition:} Distances équivalentes: $\exists\alpha_-\\alpha_+\quad\alpha_- d_1\le d_2\le\alpha_+ d_1 $.
C'est une relation d'équivalence.

\paragraph{Théorème:}
$
	\begin{array}{rcl}
		\text{$d_1$ est équivalente à $d_2$}
		  & \implies & \forall x, y\in E\quad\forall r > 0,
		\left\{
		\begin{array}{cc}
			y\in B_2(x, r)\implies y\in B_1(x, r/\alpha_-) \\
			y\in B_1(x, r)\implies y\in B_2(x, r/\alpha_+) \\
		\end{array}
		\right.\\
		  & \implies & \text{les voisinages issus de $d_1$ sont des voisinage pour $d_2$ et inversement} \\
		  & \implies & \text{les topologies sont égales}                                                 \\
	\end{array}
$

\paragraph{Définition:} Suite de Cauchy: $ d(x_m, x_n)\underset{m, n\rightarrow +\infty}{\rightarrow} 0$,
soit: $\quad\forall\varepsilon > 0, \exists N, \quad m, n\ge N\implies d(x_m, x_n) <\varepsilon$.

\paragraph{Définition:} Espace complet: toutes suites de Cauchy est convergente (réciproque triviale).

\section{Norme, Espace vectoriel normé}

\paragraph{Définition:}
$
	\text{$|| \cdot||:E\rightarrow\mathbb{R}^+$ est une norme sur $E$}:
	\left\{
	\begin{array}{ll}
		\text{$|| \cdot||$ est séparable:} & ||x||=0\iff x=0                 \\
		\text{$|| \cdot||$ est homogène:}  & || \lambda x||=| \lambda| ||x|| \\
		\text{inégalité triangulaire:}     & ||x+z|| \le ||x+y||+||y+z||     \\
	\end{array}
	\right.
$

\paragraph{Théorème:} Un espace vectoriel normé est un espace métrique en posant $d(x, y)=||x-y||$.

\paragraph{Définition:} Normes équivalentes: $\exists\alpha_-\\alpha_+\quad\alpha_- || \cdot||_1\le || \cdot||_2\le\alpha_+ || \cdot||_1 $. C'est une relation d'équivalence.

\paragraph{Théorème:} Normes équivalentes $\iff$ Topologies issues des distances issues des normes égales.

\fbox{$\implies$} Cf théorème pour les espace métriques.

\fbox{$\Longleftarrow$}
$
	\begin{array}{rcl}
		\text{Topologies égales}
		  & \implies & \forall O_1\in\mathscr{T}_1, \exists O_2\in\mathscr{T}_2, \quad O_2\subset O_1                                                \\text{(et inversement)}\\
		  & \implies & \forall r_1, \exists r_2, \quad B_2(x, r_2)\subset B_1(x, r_1)                                                                \\text{(et inversement)}\\
		  & \implies & \forall r_1, \exists r_2, \quad ||x-y||_2<r_2\implies ||x-y||_1<r_1                                                           \\text{(et inversement)}\\
		  & \implies & \forall r_1, \exists r_2, \quad ||y||_2<r_2\implies ||y||_1<r_1                                                               \\text{(et inversement)}\\
		  & \implies & \forall r_1, \exists r_2, \quad ||x\frac{r_2}{\varepsilon+||x||_2}||_2<r_2\implies ||x\frac{r_2}{\varepsilon+||x||_2}||_1<r_1 \\text{(et inversement)}\\
		  & \implies & \forall r_1, \exists r_2, \quad ||x||_2<\varepsilon+||x||_2\implies ||x||_1<\frac{r_1}{r_2}(\varepsilon+||x||_2)              \\text{(et inversement)}\\
		  & \implies & \exists r_2, \quad ||x||_1 <\frac{1}{r_2}||x||_2                                                                              \\text{(et inversement)}\\
		  & \implies & \text{$|| \cdot||_1$ et $|| \cdot||_2$ sont équivalentes}
	\end{array}
$

\paragraph{Définition:} Espace de Banach: espace vectoriel normé et complet.
\paragraph{Définition:} Espace de Hilbert: espace vectoriel préhilbertien (avec produit scalaire) normé par celui-ci et complet.

\paragraph{Théorème:}
$
	\begin{array}{rcl}
		\text{$f$ linéaire entre evn continue en $0$}
		  & \implies & \forall r > 0, \exists\rho > 0\quad ||x||_E <\rho\implies ||f(x)||_F < r                     \\
		  & \implies & \forall r > 0, \exists\rho > 0\quad ||x-y||_E <\rho\implies ||f(x-y)||_F=||f(x)-f(y)||_F < r \\
		  & \implies & \text{$f$ est continue sur tout $E$}
	\end{array}
$

\paragraph{Théorème:}
$
	\begin{array}{rcl}
		\text{$f$ est continue sur tout $E$}
		  & \implies & \forall r > 0, \exists\rho > 0\quad ||x||_E <\rho\implies ||f(x)||_F < r                                                      \\
		  & \implies & \exists\rho > 0\quad ||x||_E <\rho\implies ||f(x)||_F < 1                                                                     \\
		  & \implies & \exists\rho > 0\quad ||y\frac{\rho}{\varepsilon + ||y||_E}||_E <\rho\implies ||f(y\frac{\rho}{\varepsilon + ||y||_E})||_F < 1 \\
		  & \implies & \exists\rho > 0\quad ||y||_E <\varepsilon + ||y||_E\implies ||f(y)||_F <\frac{1}{\rho}(\varepsilon + ||y||_E)                 \\
		  & \implies & \exists\rho > 0\quad ||f(y)||_F <\frac{1}{\rho}||y||_E                                                                        \\
		  & \implies & \text{$f$ lipschitzienne en zéro}
	\end{array}
$

\paragraph{Théorème:}
$
	\begin{array}{rcl}
		\text{$f$ lipschitzienne en zéro}
		  & \implies & \exists\rho > 0\quad ||f(y)||_F <\frac{1}{\rho}||y||_E                                            \\
		  & \implies & \exists\rho > 0, \quad\forall r > 0, ||y||_E < r/\rho\implies ||f(y)||_F <\frac{1}{\rho}||y||_E=r \\
		  & \implies & \text{$f$ est continue en $0$}
	\end{array}
$

\paragraph{Théorème:Hahn-Banach} $f$ forme linéaire partielle et majorée pas une distance Minkovskienne (application sous linéaire) $\implies$ $f$ prolongeable sur tout $E$ avec la même inégalité.

\chapter{Théorème de Hahn-Banach et dualité}

\chapter{Théorème de Baire et ses conséquences}

\chapter{Exemples d'espaces fonctionnels}

\chapter{Topologie faible}

\end{document}











\usepackage{tikz}  % dessins
\usetikzlibrary{arrows}
\usetikzlibrary{calc}
\usetikzlibrary{decorations.pathreplacing}
\tikzset{%
	show curve controls/.style={
			postaction={
					decoration={
							show path construction,
							curveto code={
									\draw [blue]
									(\tikzinputsegmentfirst)--(\tikzinputsegmentsupporta)
									(\tikzinputsegmentlast)--(\tikzinputsegmentsupportb);
									\fill [red, opacity=0.5]
									(\tikzinputsegmentsupporta) circle [radius=.5ex]
									(\tikzinputsegmentsupportb) circle [radius=.5ex];
								}
						},
					decorate
				}}}

\begin{tikzpicture}
	\draw [show curve controls] (-3, 4) .. controls ++(135:-1) and ++(135:1) .. (0, 4);
	\draw [show curve controls] (0, 0)
	.. controls ++(0:-1) and ++(240: 1) .. (3, 2)
	.. controls ++(240:-1) and ++(165:-1) .. (2, 4)
	.. controls ++(165: 1) and ++(175:-2) .. (-1, 2)
	.. controls ++(175: 2) and ++(165: 1) .. (0, 0);
\end{tikzpicture}

\begin{center}
	\begin{tikzpicture}[scale=0.5]
		\draw [help lines] (0, 0) grid (3, 3);
		\draw [show curve controls] (0, 0)
		.. controls ++(45:-1) and ++(45:1) .. (0, 2)
		.. controls (1, 2) .. (2, 2)
		.. controls (2, 1) .. (2, 0)
		.. controls (1, 0) .. (0, 0);
	\end{tikzpicture}
\end{center}

\begin{tikzpicture}
	\begin{scope}[scale=0.5]
		\draw (-3, 0.6) .. controls +(1, 0) and +(-1, 0) .. (0, 1.8)
		.. controls +(1, 0) and +(0, -3) .. (5, 3.2)
		.. controls +(0, 2) and +(2, 0)  .. (0, 5.2)
		.. controls +(-1, 0) and +(0, 3) .. (-4.5, 2.2)
		.. controls +(0, -1) and +(-1, 0).. (-3, 0.6);
		\begin{scope}  % pour limiter la portée du clip
			\clip (-3, 0.6) .. controls +(1, 0) and +(-1, 0) .. (0, 1.8)
			.. controls +(1, 0) and +(0, -3) .. (5, 3.2)
			.. controls +(0, 2) and +(2, 0)  .. (0, 5.2)
			.. controls +(-1, 0) and +(0, 3) .. (-4.5, 2.2)
			.. controls +(0, -1) and +(-1, 0).. (-3, 0.6);
			\draw (-5, 0) grid[step=1](5, 6);
		\end{scope}
		\draw[<->] (-5, 2)--(-5, 3) node[midway, left]{$\varepsilon$};
		\node at (0, 1) {$\Omega$};
	\end{scope}
\end{tikzpicture}
}

\begin{proposition}
	Si $\mathfrak{B}$ est une base $\sigma_\text{topologie}(\mathfrak{B})=\{U\subset E| \exists X_{i\in I}\in\mathfrak{B}, U=\bigcup_{i\in I}X_i\}$
\end{proposition}

\begin{demonstration}
	Notons l'ensemble de droite $\mathscr{T}=\{U\subset E| \exists X_{i\in I}\in\mathfrak{B}, U=\bigcup_{i\in I}X_i\}$ et montrons $\mathscr{T}=\sigma_\text{topologie}(\mathfrak{B})$ par double inclusion.
	\begin{itemise}
		\item[$\subset$] Soit $O\in\mathscr{T}$ alors $\exists X_{i\in I}\in\mathfrak{B}, O=\bigcup_{i\in I}X_i$.
		Or $\sigma_\text{topologie}(\mathfrak{B})$ continent $\mathfrak{B}$ donc $X_{i\in I}\in\sigma_\text{topologie}(\mathfrak{B})$, et est une topologie donc $O=\bigcup_{i\in I}X_i\in\sigma_\text{topologie}(\mathfrak{B})$.
		\item[$\supset$] Montons que $\mathscr{T}$ est une topologie.
		\begin{itemise}
			\item de $E\subset\bigcup_{X\in\mathfrak{B}}X$ on a $E\in\mathscr{T}$.
			Pour $I=\vide$ on obtient $\vide\in\mathscr{T}$
			\item Soit $U, V\in\mathscr{T}$ alors $\exists X_{i\in I}, X_{j\in J}\in\mathfrak{B}, U=\bigcup_{i\in I}X_i$ et $V=\bigcup_{j\in J}X_j$.
			Puisque $\mathfrak{B}$ est une base $\exists Z^{ij}_{k\in K}, X_i\cap X_j=\bigcup_{k\in K}Z^{ij}_k, \forall i, j\in I\times J$.
			Donc $U\cap V=\bigcup_{i\in I, j\in J}X_i\cap X_j=\bigcup_{i\in I, j\in J, k\in K}Z^{ij}_k\in\mathscr{T}$.
			\item Si $O_{i\in I}\in\mathscr{T}$ alors $\exists X^i_{j\in J}\in\mathfrak{B}, O_i=\bigcup_{i\in I, j\in J}X^i_j$ donc $\bigcup_{i\in I}O_i=\bigcup_{i\in I, j\in J}X^i_j\in\mathscr{T}$.
		\end{itemise}
		De plus $\mathscr{T}$ contient $\mathfrak{B}$.
		Or $\sigma_\text{topologie}(\mathfrak{B})$ est la plus petite topologie contenant $\mathfrak{B}$.
		Donc $\mathscr{T}\supset\sigma_\text{topologie}(\mathfrak{B})$.
	\end{itemise}
\end{demonstration}

Soit $\mathfrak{V}$ une voisinagination de $E$.
Vérifions que $\mathscr{T}_\text{ouvert}=\Phi(\mathfrak{V})$ est une topologie d'ouverts.
\begin{itemise}
	\item $\forall x\in\vide, \vide\in\mathfrak{V}(x)$ est vrai donc $\vide\in\mathscr{T}_\text{ouvert}$ et $E\in\mathscr{T}_\text{ouvert}$ par hypothèse.
	\item Si $O, O'\in\mathscr{T}_\text{ouvert}$ alors $x\in O\cap O'$ implique $x\in O\et x\in O'$ donc $O\in\mathfrak{V}(x)\et O'\in\mathfrak{V}(x)$, soit $O, O'\in\mathfrak{V}(x)$.
	Par définition des voisinaginations $O\cap O'\in\mathfrak{V}(x)$ soit $O\cap O'\in\mathscr{T}_\text{ouvert}$.
	\item Si $O_{i\in I}\in\mathscr{T}_\text{ouvert}$ alors $x\in\bigcup_{i\in I} O_i$ implique $\exists i\in I, x\in O_i$.
	Ainsi $O_i\in\mathfrak{V}(x)\et O_i\subset\bigcup_{i\in I} O_i$, donc par définition des voisinaginations $\bigcup_{i\in I} O_i\in\mathfrak{V}(x)$ soit $\bigcup_{i\in I} O_i\in\mathscr{T}_\text{ouvert}$.
\end{itemise}

$$
	\mathscr{T}_\text{ouvert}\underset{\Psi}{\longmapsto}\mathfrak{V}\underset{\Phi}{\longmapsto}\mathscr{T}_\text{ouvert}'
$$

Vérifions que $\Psi\circ\Phi=\id$.
Soit $\mathscr{T}_\text{ouvert}$ une topologie d'ouverts de $E$.
Vérifions que $\mathscr{T}_\text{ouvert}'=\mathscr{T}_\text{ouvert}$.

$$
	\mathfrak{V}\underset{\Phi}{\longmapsto}\mathscr{T}_\text{ouvert}\underset{\Psi}{\longmapsto}\mathfrak{V}'
$$
Soit $\mathfrak{V}$ une voisinagination de $E$.
Vérifions que $\mathfrak{V}'=\mathfrak{V}$.
\begin{itemise}
	\item Si $V\in\mathfrak{V}(x)$ posons $O=\{y\in E|V\in\mathfrak{V}(y)\}$.
	Puisque $x\in V$ alors $x\in O$.
	Si $y\in O$ alors $V\in\mathfrak{V}(y)$ donc $y\in V$ soit $O\subset V$, et $\exists W\in\mathfrak{V}(y), \forall z\in W, V\in\mathfrak{V}(z)$ soit $z\in O$  donc $W\subset O$ puis $O\in\mathfrak{V}(y)$ par définition de voisinagination, donc $O\in\Psi(\mathfrak{V})=\mathscr{T}_\text{ouvert}$.
	Ainsi $\exists O\in\mathscr{T}_\text{ouvert}, x\in O\subset V$ donc $V\in\Psi(\mathscr{T}_\text{ouvert})(x)=\mathfrak{V}'(x)$.
	\item Si $V'\in\mathfrak{V}'(x)$ alors $\exists O\in\mathscr{T}_\text{ouvert}, x\in O\subset V'$.
	Or $\mathscr{T}_\text{ouvert}=\Phi(\mathfrak{V})$ donc $O\in\mathfrak{V}(x)$, puis $V'\in\mathfrak{V}(x)$ par définition d'une voisinagination.
\end{itemise}


Soit $O$ un ouvert non vide et $x\in O$.
Alors $O$ est un voisinage de $x$, et $x\in\overline{X}$ donc $O\cap X\neq\vide$.

Soit $O$ un ouvert de $E$ inclus dans $\complement X$.
Par l'absurde supposons $O\neq\vide$.
Alors $O\cap X\neq\vide$, puis $\vide\neq O\cap X\subset\complement X\cap X=\vide$.
C'est absurde, donc $O$ est vide, puis $\complement\overline{X}=\vide$, soit $\overline{X}=E$.

\begin{definition}
	Soient $(E_i)_{i\in I}$ des espaces topologiques, $E=\prod_{i\in I}E_i$ leur produit cartésien.
	La topologie produit sur $E$ est celle engendrée par la base précédente des pavés formés d'ouverts différents de $E_i$ dans un ensemble de directions $J$ fini.
	%Il est alors facile de voir qu’une base de voisinages de $x=(x_i)_{i\in I}$ est formée des pavés $\prod_{i\in I} V_i$ où les $V_i\in\mathfrak{V}(x_i)$ sont tous égaux à $E_i$ , sauf un nombre fini d’entre eux.
\end{definition}

\begin{remarque}
	Cette définition de la topologie produit peut sembler surprenante, car les voisinages ne sont susceptibles de manifester de restriction que dans un nombre fini de directions (la fameuse tranche Napolitaine).
	C’est en fait la proposition suivante qui justifie cette définition.
\end{remarque}

\begin{proposition}
	La topologie produit est la topologie initiale associée aux projections canoniques $\pi_i:E\rightarrow E_i$, c'est à dire la moins fine rendant les $\pi_i$ continues.
\end{proposition}

\begin{demonstration}
	Soit $O_i$ un ouvert de $E_i$ , alors $\inv{\pi_i}(O_i)=O_i\times\prod_{j\neq i} E_j$ est un ouvert de $E$, rendant $\pi_i$ continue.
	Soient $\mathscr{T}$ une topologie sur $E$ rendant les $p_i$ continues.
	Soit $O=\prod_{j\in J}O_j\times\prod_{i\in I\setminus J}E_i$ dans $\mathfrak{B}$.
	Remarquons que :
	$$
		p_i\left(\bigcap_{j\in J}\inv{p_j}(O_j)\right)
		=\bigcap_{j\in J}p_i(\inv{p_j}(O_j))
		=\bigcap_{j\in J}\left\{
		\begin{array}{cc}
			O_i & \text{si $i=j$}     \\
			E_i & \text{si $i\neq j$}
		\end{array}
		\right.
		=\left\{
		\begin{array}{cc}
			O_i & \text{si $i\in J$}     \\
			E_i & \text{si $i\not\in J$}
		\end{array}
		\right.
		=O_i
	$$
	Puisque $X=\prod_{i\in I}p_i(X)$ pour tout $X\subset E$ on obtient:
	$$
		O
		=\prod_{i\in I}O_i
		=\prod_{i\in I}p_i\left(\bigcap_{j\in J}\inv{p_j}(O_j)\right)
		=\bigcap_{j\in J}\inv{p_j}(O_j)
	$$
	Par continuité de $p_i$, $\inv{p_j}(O_j)\in\mathscr{T}$ et l'intersection étant finie $O\in\mathscr{T}$.
	Par minimalité de la topologie engendrée, la topologie produit est incluse dans $\mathscr{T}$.
\end{demonstration}
