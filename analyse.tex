\documentclass[a4paper, 11pt, french]{book}
\synctex=1

%------- Packages ------- 
\usepackage[french]{babel}
\usepackage[T1]{fontenc}
\usepackage[utf8]{inputenc}
\usepackage{icomma} % pour les virgules françaises
\usepackage[inner=3cm, outer=2cm, top=2cm, bottom=2cm]{geometry}
\usepackage{amsmath, amssymb} % subset, boxed..., nabla...
\usepackage{cancel} % pour simplifier des termes
\usepackage{mathrsfs} % cursives
\usepackage{dsfont} % indicatrice
\usepackage{xcolor} % pour colorier du texte
\usepackage{graphicx} % les graphiques
\usepackage{bussproofs} % démonstration formelle
\usepackage{tikz}
\usepackage{hyperref}
\usepackage{tikz-cd}
\usetikzlibrary{patterns}
%\usepackage[backend=biber, style=alphabetic, maxnames=15, sorting=ynt]{biblatex}
%\addbibresource{sample.bib} % la bibilographie
%\setlength\parindent{0pt}
\tikzcdset{scale cd/.style={every label/.append style={scale=#1}, cells={nodes={scale=#1}}}}

%------- Commande en Anglais -------
\newenvironment{centre}{\center}{\endcenter}
\newenvironment{itemise}{\itemize}{\enditemize}
\let\centring=\centering
\let\colourlet=\colorlet
\let\definecolour=\definecolor
\let\colour=\color
\let\textcolour=\textcolor
\colourlet{grey}{gray}

%------ Paragraphes particuliers -------
\usepackage{amsthm}

\theoremstyle{plain} % plain = boldface title, italicised body
\newtheorem{theoreme}{Théorème}
\newtheorem{proposition}{Proposition}
\newtheorem{exercice}{Exercice}

\theoremstyle{definition} % definition = boldface title, Roman body
\newtheorem{definition}{Définition}
\newtheorem{remarque}{Remarque}

\theoremstyle{remark} % remark = italicized title, Roman body
\newtheorem{exemple}{Exemple}
\newtheorem{correction}{Correction}
\newtheorem*{demonstration}{Démonstration}
\newtheorem*{lemme}{Lemme}

%------- Paragraphes particuliers ------- 
\usepackage{titlesec}
%\setlength{\parindent}{0pt} % pas d'alinéa
%\setlength{\parskip}{0.5em} % distance entre les paragraphes \par
%\titlespacing{\paragraph}{0em}{1em}{0.5em}[] % espace à gauche, en haut, à droite
\titleformat{\chapter}[display]{\normalfont\bfseries}{}{-3cm}{\Huge Chapitre \thechapter\ : }
\titleformat{\section}[display]{\normalfont\bfseries}{}{-0.75cm}{\Large\thesection\quad}

% Raccourcis
\renewcommand{\setminus}{\backslash}
\newcommand{\1}{\mathds{1}}
\newcommand{\id}{\mathrm{id}}
\newcommand\vide{\varnothing}
\newcommand{\inv}[1]{#1^{-1}}
\newcommand{\et}{\mathrel{\mathrm{et}}}
\newcommand{\ou}{\mathrel{\mathrm{ou}}}

\newcommand{\N}{\mathbb{N}}
\newcommand{\Z}{\mathbb{Z}}
\newcommand{\Q}{\mathbb{Q}}
\newcommand{\R}{\mathbb{R}}
\newcommand{\C}{\mathbb{C}}

\renewcommand{\cal}[1]{\mathcal{#1}}
\renewcommand{\frak}[1]{\mathfrak{#1}}
\newcommand{\scr}[1]{\mathscr{#1}}
\renewcommand{\rm}[1]{\mathrm{#1}}
\renewcommand{\bf}[1]{\mathbf{#1}}

\newcommand\ens[2]{\{#1 \ |\ #2\}}
\newcommand\equivalence[3]{
	\begin{demonstration}
		#1
		\begin{itemise}
			\item[$\Longrightarrow$] #2
			\item[$\Longleftarrow$] #3
		\end{itemise}
	\end{demonstration}
}

%----- Pour l'instant pas utile -----
\newcommand\transp{\, {}^t}
\newcommand\diag{\mathrm{diag}}
\newcommand\rang{\mathrm{rg}}
\newcommand\image{\mathrm{Im}}
\newcommand\noyau{\mathrm{Ker}}
\newcommand\Vect{\mathrm{Vect}}
\newcommand\crochet[1]{\langle #1\rangle}
\newcommand\classe[1]{\mathcal{C}^#1}
\newcommand\intervalle[1]{[\![#1]\!]}
\newcommand\diff{\mathrm{d}}

\title{Analyse\\Mathématiques sophistiquées expliquées par un simple}
\author{Paul \textsc{Trillat}}
\date{\today}

\begin{document}
\maketitle
\tableofcontents

\chapter*{Introduction}
\begin{centre}
	\begin{tikzpicture}[xscale=3, yscale=2]
		\node (G) at (1, 1) {$(G, \star)$};
		\node (M) at (2, 1) {$(\Omega, \mu)$};
		\begin{scope}[local bounding box=partieI]
			\node (T) at (0, 1) {$(E, \scr{T})$};
			\node (EU) at (0, 0) {structure uniforme};
		\end{scope}
		\draw (partieI.south west) rectangle (partieI.north east);
		\node[left] at (partieI.west) {Partie I};
		\begin{scope}[local bounding box=partieII]
			\node (GT) at (1, 0) {$(G, \star, \scr{T})$};
			\node (EVT) at (1, -1) {$(E, +, \cdot, \scr{T})$};
		\end{scope}
		\draw (partieII.south west) rectangle (partieII.north east);
		\node[right] at (partieII.east) {Partie II};
		\begin{scope}[local bounding box=partieIII]
			\node (ELC) at (0, -2) {loc. convexe};
			\node (EM) at (0, -3) {$(E, d)$};
			\node (EN) at (0, -4) {$(E, || \dots||)$};
		\end{scope}
		\node[left] at (partieIII.west) {Partie III};
		\draw (partieIII.south west) rectangle (partieIII.north east);
		\draw[black, -stealth] (T)--(EU);
		\draw[black, -stealth] (T)--(GT);
		\draw[black, -stealth] (EVT)--(ELC);
		\draw[black, -stealth] (EU)--(ELC);
		\draw[black, -stealth] (G)--(GT);
		\draw[black, -stealth] (M)--(GT);
		\draw[black, -stealth] (GT)--(EVT);
		\draw[black, -stealth] (ELC)--(EM);
		\draw[black, -stealth] (EM)--(EN);
	\end{tikzpicture}
\end{centre}

\chapter{Ensembles, en grosse balle}

\begin{definition}
	Une relation binaire $\preccurlyeq$ est une relation d'ordre sur $E$ si elle est réflexive, antisymétrique et transitive.
	On dit alors que $(E, \preccurlyeq)$ est ordonné.
	Cela veut dire:
	\begin{itemise}
		\item $\forall x\in E, x\preccurlyeq x$
		\item si $x\preccurlyeq y$ et $y\preccurlyeq x$ alors $x=y$
		\item si $x\preccurlyeq z$ et $y\preccurlyeq z$ alors $x\preccurlyeq z$
	\end{itemise}
\end{definition}


\begin{definition}
	On dit qu'un partie $X$ d'un ensemble $(E, \preccurlyeq)$ ordonné est totalement ordonnée si et seulement si $\forall x, y\in X, (x\preccurlyeq y)\ou(y\preccurlyeq x)$.
\end{definition}


\begin{definition}
	On dit que $x^+\in E$ est un majorant d'une partie $X$ d'un ensemble $(E, \preccurlyeq)$ ordonné si et seulement si $\forall x\in X, x\preccurlyeq x^+$.
\end{definition}

\begin{definition}
	On dit que $x^*\in X$ est un élément maximal d'une partie $X$ d'un ensemble $(E, \preccurlyeq)$ ordonné si et seulement si $\forall x\in X, (x^*\preccurlyeq x)\Rightarrow(x^*=x)$.
\end{definition}


\begin{definition}
	On dit qu'un ensemble ordonné est inductif si toute partie totalement ordonnée admet un majorant.
\end{definition}

\begin{theoreme}[Lemme de Zorn]
	Tout ensemble ordonné, non vide et inductif admet un élément maximal.
\end{theoreme}

\chapter{Groupes, en grosse balle}

\section{Définitions et premières propriétés}

\begin{definition}
	On appelle $(G, \star)$ un groupe si et seulement si
	\begin{itemise}
		\item $\star$ est une opération binaire interne $\forall x, y\in G, x\star y\in G$.
		\item $\star$ admet un neutre $e$, $\forall x\in G, x\star e=e\star x=x$.
		\item Il existe toujours un symétrique par $\star$: $\forall x\in G, \exists y\in G, x\star y=y\star x=e$.
		\item $G$ est associative pour $\star$: $\forall x, y, z\in G, (x\star y)\star z=x\star(y\star z)$.
	\end{itemise}
\end{definition}

\begin{proposition}
	On peut alléger les axiomes en supposant que $e$ est juste un neutre à gauche et que le symétrique existe à gauche.
\end{proposition}

\begin{demonstration}
	Soit $e$ un neutre à gauche de $G$, $x$ dans $G$, $y$ un symétrique à gauche de $x$, $z$ un symétrique à gauche de $y$.
	$x\star y=e\star x\star y=z\star y\star x\star y=z\star e\star y=z\star y=e$.
	Ainsi un symétrique à gauche est aussi un symétrique à droite.
	$x\star e=x\star y\star x=e\star x=x$.
	Ainsi un neutre à gauche est aussi neutre à droite.
\end{demonstration}

\begin{proposition}
	Dans un groupe, il n'existe qu'un neutre et qu'un symétrique.
	Pour un élément $x$ de $G$, on notera $ \inv{x}$ ou $-x$ ce symétrique.
	De plus $\inv{(\inv{x})}=x$
\end{proposition}

\begin{demonstration}
	Soit $e$ et $e'$ deux neutres, alors $e\star e'=e'\star e=e=e'$.
	Soit $x$ un élément de $G$, $y$ et $y'$ deux symétriques, alors $y'=e\star y'=y\star x\star y'=y\star e=y$.
	De plus $\inv{(\inv{x})}=e\star\inv{(\inv{x})}=(x\star \inv{x})\star\inv{(\inv{x})}=x\star(\inv{x}\star\inv{(\inv{x})})=x\star e=x$.
\end{demonstration}

\begin{definition}
	On appelle sous-groupe d'un groupe $G$ une partie $H$ de $G$ sur laquelle la loi de $G$ défini une structure de groupe.
\end{definition}

\begin{proposition}
	$H$ est un sous groupe de $G$ si et seulement si $H\neq\vide\et\forall x, y\in H, x\star\inv{y}\in H$.
\end{proposition}

\equivalence{Par double implication.}
{
	Par définition $e\in H\neq\vide$, et il suffit de composer l'existence d'un inverse avec le produit.
}{
	Puisque $H$ est non vide $\exists x\in H$, donc $x\star \inv{x}=e\in H$.
	Ainsi $\forall y\in H, e\star\inv{y}=\inv{y}\in H$.
	C'est donc une opération interne car $x\star y=x\star\inv{(\inv{y})}\in H$, admettant $e$ pour neutre car $x\star e=x\in H$.
}

\begin{proposition}
	L'intersection d'une famille de sous-groupe $(H_i)_{i\in I}$ est un sous-groupe.
\end{proposition}

\begin{demonstration}
	$\forall i\in I, e\in H_i$ donc $e\in\bigcap_{i\in I}H_i\neq\vide$.
	Soient $x, y\in\bigcap_{i\in I}H_i$ alors $\forall i\in I, x\star\inv{y}\in H_i$, donc $x\star y\in\bigcap_{i\in I}H_i$.
\end{demonstration}

\begin{definition}
	On appelle $\sigma_\rm{groupe}(\frak{B})=\bigcap_{\substack{H\text{ sous-groupe}\\\frak{B}\subset H}}H$, le sous-groupe engendré par $\frak{B}$.
\end{definition}

\begin{proposition}
	$\sigma_\rm{groupe}(\frak{B})=\ens{x_1^{\epsilon_1}\star\dots\star x_n^{\epsilon_n}}{n\in\N, x_i\in\frak{B}, \epsilon_i=\pm1}$
\end{proposition}

\begin{demonstration}
	\colour{red}{À démontrer}
\end{demonstration}

\begin{definition}
	Un groupe est dit de type fini lorsqu'il admet une partie génératrice finie.
\end{definition}

\begin{proposition}
	Tout sous-groupe de $(\Z, +)$ est de la forme $a\Z$, pour un unique $a\geqslant0$.
\end{proposition}

\equivalence{Par double implication.}{
	$a\Z$ est toujours un sous-groupe de $\Z$.
}{
	Si $G=\{0\}=0\Z$ le résultat est vrai.
	Soit $G$ un sous-groupe de $\Z$ avec un élément $x$ non nul.
	Quite à prendre $-x\in G$ on peut supposer $x>0$
	D'après l'algorithme d'Euclide, $\exists q\in\Z, \exists r\in[\![0, a-1]\!], x = qa + r$.
	Mais puisque $G$ est un groupe $x-qa=r\in G$.
	Si $r$ était non nul, ce serait un élément de $G$ strictement positif et strictement plus petit que $a$, ce qui est impossible.
	Ainsi $r=0$, et donc $x\in a\Z$.
}

\begin{proposition}
	Tout sous-groupe de $(\R, +)$ est de la forme $a\Z$, pour un unique $a\geqslant0$, ou dense.
\end{proposition}

\begin{demonstration}
	Soit $G$ un sous-groupe de $\R$.
	Si $G = \{ 0 \}$ alors $G = 0\Z$ donc le résultat est vérifié.
	Sinon il existe alors un $x_0\in G$ non nul et quitte à prendre $-x_0$ on peut supposer que $x_0$ est strictement positif.
	Ainsi l’ensemble $\ens{x\in G}{x > 0}$ est non vide et minoré par 0, il admet donc une borne inférieure $a=\inf\ens{x\in G}{x > 0}$.
	\begin{itemise}
		\item Si $a = 0$ les éléments de $\ens{x\in G}{x > 0}$ sont moralement aussi petit qu'on veut.
		On veut donc montrer que $G$ est dense dans $\R$.
		Soit $x\in\R$ et $\epsilon > 0 = a$.
		Puisque $\epsilon$ n'est pas un minorant il existe $x_\epsilon\in G$ tel que $0 < x_\epsilon < \epsilon$.
		Puisque $x_\epsilon$ est non nul il est possible de couper $\R$ en bandes de tailles $x_\epsilon$ par $\bigcup_{n\in\Z}[nx_\epsilon, (n+1)x_\epsilon[$.
		Ainsi $x$ est dans l'une de ces bandes c'est-à-dire qu'il existe $n\in\Z$ tel que $nx_\epsilon\leqslant x < (n + 1)x_\epsilon=nx_\epsilon + x_\epsilon < nx_\epsilon + \epsilon$.
		Donc $0\leqslant x - nx_\epsilon < \epsilon$ et puisque $nx_\epsilon = x_\epsilon + \cdots + x_\epsilon\in G$ alors $G$ est dense dans $\R$.
		\item Si $a>0$ cela veut dire que $G$ ne peut pas décrire des nombres aussi petits que l'on veut.
		En fait on va montrer que le plus petit pas que l'on peut faire dans $G$ est exactement $a$.
		Puis que $G$ étant un groupe ses autres éléments seront donc des multiples de $a$.
		\begin{itemise}
			\item Montrons que $a\in G$ par l'absurde.
			Puisque $2a$ n'est pas un minorant il existe $x\in G$ tel que $a \leqslant x < 2a$.
			Mais $a\notin G$ donc l'inégalité est stricte montrant que $x$ n'est pas un minorant.
			De même il existe $y\in G$ tel que $a < y < x < 2a$ ce qui montre $0 < x - y < a$.
			Puisque $G$ est un groupe $0 < x - y\in G$ ce qui est absurde car $a$ est la borne inférieure.
			Ainsi $a\in G$ et donc $a\Z\subset G$.
			\item Montrons que $G\subset a\Z$.
			Soit $x\in G$ en coupant $\R$ en bandes de taille $a$ il existe $n\in\Z$ tel que $na\leqslant x<(n+1)a$ donc $0\leqslant x - na < a$.
			Or $x - na\in G$ car $G$ est un groupe et $n$ est entier.
			Par défintion le seul élément de $G$ plus petit que $a$ est $0$ ce qui montre que $x=na\in a\Z$.
		\end{itemise}
	\end{itemise}
\end{demonstration}

\begin{definition}
	Soit $x$ un élément d'un groupe $G$ et $H$ un sous-groupe.
	On appelle classe à droite de $x$ l'ensemble $xH = \ens{xh}{h\in H}$.
	On définit de même la classe à gauche.
\end{definition}

\begin{proposition}
	Soit $H$ un sous-groupe de $G$.
	La relation $\sim_H$ définie par $x\sim_H y\iff xH = yH$  sur $G$ est une relation d'équivalence.
\end{proposition}

\begin{demonstration}
	Vérifions tout les axiomes d'une relation d'équivalence.
	\begin{itemise}
		\item $x\sim_H x$ par réflexivité de $xH = xH$.
		\item de $x\sim_H y$ on a $y\sim_H x$ par symétrie de $xH = yH$.
		\item de $x\sim_H y\et y\sim_H z$ on a $x\sim_H z$ par transitivité de $xH = yH\et yH\sim_H zH$.
	\end{itemise}
\end{demonstration}

\begin{definition}
	Un sous-groupe est disingué si ses classes à gauches sont égales à ses classes à droites.
	On note alors $H \trianglelefteq G$.
\end{definition}

\begin{proposition}
	Un sous-groupe $H$ est disingué si et seulement s'il est compatible avec la relation d'équivalence $\sim_H$
	(id est de $x, y, z, t\in G$ tels que $x\sim_H y$ et $z\sim_H t$ alors $xz\sim_H yt$).
\end{proposition}

\equivalence{Par double implication.}{
	Supposons $H$ disingué.
	Soient $x, y, z, t\in H$ tels que $x\sim_H y$ et $z\sim_H t$.
	En utilisant l'associativité $(xz)H = x(zH) = x(tH) = x(Ht) = (xH)t = (yH)t = (yt)H$.
}{
	Supposons $H$ compatible avec $\sim_H$ et prenons $x\in G$.
	Si $y\in xH$ est dans une classe à droite alors $y\sim_H x$.
	Par réfléxivité $\inv{x}\sim_H \inv{x}$ donc par compatibilité $y\inv{x}\sim_H x\inv{x}=e$.
	Ainsi il existe $h\in H$ tel que $ = eh = h$ donc $y=hx$ c'est-à-dire $y\in Hx$.
	De même pour l'inclusion réciproque.
}

\begin{definition}
	Le quotient d'un groupe $G$ par $H\trianglelefteq G$ noté $G/H$ est l'ensemble des classes de $H$.
	C'est un groupe pour la loi $xH, yH\longmapsto (xH)\star(yH)=(xy)H$.
\end{definition}

\begin{proposition}
	Théorème de factorisation.
	\colour{red}{À compléter !}
\end{proposition}

\part{Topologie}

\chapter{Espaces topologiques généraux}

\section{Équivalences}

Ayant été étudiés par de nombreuses personnes, il existe plusieurs façons équivalentes de définir un espace topologique.
On donne dans la suite les définitions les plus classiques pour montrer qu'elles sont équivalentes.

\begin{definition}
	On appelle topologie d'ouverts sur $E$ une collection $\scr{O}\subset\frak{P}(E)$ telle que:
	\begin{itemise}
		\item Toute réunion d'ensembles de $\scr{O}$ est un ensemble de $\scr{O}$.
		\item Toute intersection finie d'ensembles de $\scr{O}$ est un ensemble de $\scr{O}$.
	\end{itemise}
	Les éléments de $\scr{O}$ sont appelés les ouverts de $E$.
\end{definition}

\begin{definition}
	On appelle topologie de fermés sur $E$ une collection $\scr{F}\subset\frak{P}(E)$ telle que:
	\begin{itemise}
		\item Toute réunion finie d'ensembles de $\scr{F}$ est un ensemble de $\scr{F}$.
		\item Toute intersection d'ensembles de $\scr{F}$ est un ensemble de $\scr{F}$.
	\end{itemise}
	Les éléments de $\scr{F}$ sont appelés les fermés de $E$.
\end{definition}

\begin{proposition}
	L'ensemble des topologies d'ouverts est en bijection avec celui des fermés.
\end{proposition}

\begin{demonstration}
	L'application $\complement\colon X\longmapsto E\setminus X$ est une bijection car c'est une involution.
	Soit $\scr{O}$ une topologie d'ouverts et montrons que $\scr{F}=\complement\scr{O}$ est une topologie de fermés.
	\begin{itemise}
		\item Soit $(F_i)_{1\leqslant i\leqslant n}$ une famille finie de $\scr{F}$ alors
		$\bigcup_{i=1}^nF_i=\bigcup_{i=1}^n\complement O_i=\complement\bigcap_{i=1}^nO_i\in\scr{F}$.
		\item Soit $(F_i)_{i\in I}$ une famille quelconque de $\scr{F}$ alors
		$\bigcap_{i\in I}F_i=\bigcap_{i\in I}\complement O_i=\complement\bigcup_{i\in I}O_i\in\scr{F}$.
	\end{itemise}
	On montre de même que si $\scr{F}$ une topologie de fermés alors $\scr{O}=\complement\scr{F}$ est une topologie d'ouverts.
\end{demonstration}

\begin{definition}
	On appelle intérieur sur $E$ une application $\cdot^\circ\colon\frak{P}(E)\rightarrow\frak{P}(E)$ telle que :
	\begin{itemise}
		\item $\cdot^\circ$ est une involution.
		\item L'intérieur d'une intersection finie est l'intersection des intérieurs.
		\item L'intérieur d'une partie est inclue dans celle-ci.
	\end{itemise}
\end{definition}

\begin{proposition}
	Les fonctions d'intérieur sont en bijection avec les topologies d'ouverts.
\end{proposition}

\begin{demonstration}
	On utilisera plusieurs fois dans la démonstration la croissance des fonctions intérieures.
	En effet si $X\subset Y$ alors $X^\circ=(X\cap Y)^\circ=X^\circ\cap Y^\circ\subset Y^\circ$).

	Soit $\Phi\colon\scr{O}\mapsto(X\mapsto\bigcup_{X\supset O\in\scr{O}}O)$ allant le l'ensemble des topologies d'ouverts vers l'ensemble des fonctions « intérieur ».
	Montrons sa bijectivité.
	\begin{itemise}
		\item Montrons que $\Phi$ est injective.
		Soient deux topologies $\scr{O}_1$ et $\scr{O}_2$ telles que $\Phi(\scr{O}_1)=\Phi(\scr{O}_2)$.
		On veut montrer que $\scr{O}_1=\scr{O}_2$.
		Soit donc $O_1\in\scr{O}_1$ alors $O_1=\bigcup_{O_1\supset O\in\scr{O}_1}O=\Phi(\scr{O}_1)(O_1)$.
		Ainsi $O_1=\Phi(\scr{O}_2)(O_1)=\bigcup_{O_1\supset O\in\scr{O}_2}O\in\scr{O}_2$ puisque c'est une union d'ouverts de $\scr{O}_2$.
		Donc $\scr{O}_1\subset\scr{O}_2$.
		L'argument étant symétrique on a $\scr{O}_1=\scr{O}_2$ puis que $\Phi$ est injective.
		\item Vérifions que $\Phi$ est injective.
		Soit $\scr{T_\rm{ouvert}}$ et $\scr{O}'$ tels que $\Phi(\scr{O})=\Phi(\scr{O}')$.
		Soit $O\in\scr{O}$ alors $O=\Phi(\scr{O})(O)=\Phi(\scr{O}')(O)=\bigcup_{O\supset O'\in\scr{O}'}O'\in\scr{O}'$.
		\item Montrons que $\Phi$ est surjective en montrant qu'un antécédent par $\Phi$ de $\cdot^\circ$ une fonction d'intérieur est $\scr{O}=\ens{O\subset E}{O=O^\circ}$.
		\begin{itemise}
			\item Vérifions que $\scr{O}$ est une topologie d'ouverts.
			\begin{itemise}
				\item Soit $(O_i)_{i\in I}$ une famille de $\scr{O}$ et $O=\bigcup_{i\in I}O_i$.
				Par croissance $O_i^\circ\subset O^\circ$ donc $O=\bigcup_{i\in I}O_i=\bigcup_{i\in I}O_i^\circ\subset O^\circ$.
				Puisque $O^\circ\subset O$ est un axiome on a $O\in\scr{O}$.
				\item Soit $(O_i)_{1\leqslant i\leqslant n}$ une famille finie de $\scr{O}$ et $O=\bigcap_{i=1}^nO_i$.
				Alors $O^\circ=(\bigcap_{i=1}^nO_i)^\circ=\bigcap_{i=1}^nO_i^\circ=\bigcap_{i=1}^nO_i=O$ donc $O\in\scr{O}$.
			\end{itemise}
			\item Montrons que $\Phi(\scr{O})=\cdot^\circ$ en montrant $\Phi(\scr{O})(X)=X^\circ$ par double inclusion.
			\begin{itemise}
				\item $X^\circ\in\scr{O}$ car $X^{\circ\circ}=X^\circ$ et $X\supset X^\circ$ donc $\Phi(\scr{O})(X)=\bigcup_{X\supset O\in\scr{O}}O\supset X^\circ$.
				\item $X\supset O\in\scr{O}$ implique $O=O^\circ\subset X^\circ$ par croissance donc $\Phi(\scr{O})(X)=\subset X^\circ$.
			\end{itemise}
		\end{itemise}
	\end{itemise}
\end{demonstration}

\begin{definition}
	On appelle adhérence sur $E$ une application $\overline{\cdot}\colon\frak{P}(E)\rightarrow\frak{P}(E)$ telle que :
	\begin{itemise}
		\item $\overline{\cdot}$ est une involution.
		\item L'adhérence d'une union finie est l'union des adhérences.
		\item Toute partie est include dans son adhérence.
	\end{itemise}
\end{definition}

\begin{proposition}
	Les fonctions d'adhérence sont en bijection avec les topologies de fermés.
\end{proposition}

\begin{demonstration}
	Il suffit de remarquer que ces axiomes passés au complémentaire deviennent ceux des intérieurs.
	Il ne reste plus qu'à concaténer les deux démonstrations précédentes.
\end{demonstration}

\begin{definition}
	On appelle voisinagination sur $E$ une fonction $\frak{V}\colon E\rightarrow\frak{P}(\frak{P}(E))$ telle que :
	\begin{itemise}
		\item Toute partie de $E$ qui contient un ensemble de $\frak{V}(x)$ appartient à $\frak{V}(x)$.
		\item Toute intersection finie d'ensembles de $\frak{V}(x)$ appartient à $\frak{V}(x)$.
		\item Tout ensemble de $\frak{V}(x)$ contient $x$.
		\item Si $V$ appartient à $\frak{V}(x)$, il existe un ensemble $W$ appartenant à $\frak{V}(x)$ et tel que, pour tout $y\in W$, $V$ appartienne à $\frak{V}(y)$.
	\end{itemise}
\end{definition}

\begin{proposition}
	Les topologies d'ouverts sont en bijection avec les voisinaginations.
\end{proposition}

\begin{demonstration}
	Soit $\Phi\colon\scr{O}\mapsto(x\mapsto\ens{V\subset E}{\exists O\in\scr{O},\ x\in O\subset V})$ allant de l'ensemble des topologies d'ouverts vers l'ensemble des voisinaginations.
	Montrons sa bijectivité.
	\begin{itemise}
		\item Montrons que $\Phi$ est injective.
		Soient deux topologies $\scr{O}_1$ et $\scr{O}_2$ telles que $\Phi(\scr{O}_1)=\Phi(\scr{O}_2)$.
		On veut donc montrer que $\scr{O}_1=\scr{O}_2$.
		Soit donc $O_1\in\scr{O}_1$.
		Pour $x\in O_1$ on a $O_1\in\Phi(\scr{O}_1)(x)$ par définition donc $O_1\in\Phi(\scr{O}_2)(x)$ par hypothèse.
		Ainsi il existe un ouvert $O_2^x\in\scr{O}_2$ tel que $x\in O_2^x\subset O_1$.
		Par axiome de passage à l'union des ouverts $\bigcup_{x\in O_1}O_2^x\in\scr{O}_2$ et on a les inclusions $O_1=\bigcup_{x\in O_1}\{x\}\subset\bigcup_{x\in O_1}O_2^x\subset O_1$.
		Cela montre que $O_1\in\bigcup_{x\in O_1}O_2^x\in\scr{O}_2$ puis $\scr{O}_1\subset\scr{O}_2$.
		L'argument étant symétrique on a donc l'égalité puis l'injectivité de $\Phi$.
		\item Montrons que $\Phi$ est surjective en montrant qu'un antécédent par $\Phi$ de $\frak{V}$ une voisinagination est $\scr{O}=\ens{O\subset E}{\forall x\in O, O\in\frak{V}(x)}$.
		\begin{itemise}
			\item Vérifions donc que $\scr{O}$ est une topologie d'ouverts.
			\begin{itemise}
				\item Soit $(O_i)_{i\in I}$ une famille de $\scr{O}$ et $O=\bigcup_{i\in I}O_i$.
				Montrons que $O\in\scr{O}$ en prenant $x\in O$.
				Il existe donc $O_j$ tel que $x\in O_j\in\scr{O}$ ce qui montre $O_j\in\frak{V}(x)$.
				Or $O_j\subset O$ donc par axiome croissance des voisinages $O\in\frak{V}(x)$.
				Ainsi $O=\bigcup_{i\in I}O_i\in\scr{O}$.
				\item Soit $(O_i)_{1\leqslant i\leqslant n}$ une famille finie de $\scr{O}$ et $O=\bigcap_{i=1}^nO_i$.
				Montrons que $O\in\scr{O}$ en prenant $x\in O$.
				Pour tout $i\in[\![1, n]\!]$ on a $x\in O_i\in\scr{O}$ ce qui montre $O_i\in\frak{V}(x)$.
				Par intersection finie de voisinages $O\in\frak{V}(x)$.
				Ainsi $O=\bigcap_{i=1}^nO_i\in\scr{O}$.
			\end{itemise}
			\item Montrons que $\Phi(\scr{O})=\frak{V}$ en montrant $\Phi(\scr{O})(x)=\frak{V}(x)$ par double inclusion.
			\begin{itemise}
				\item Soit $V\in\Phi(\scr{O})(x)$ il exsite $O\in\scr{O}$ tel que $x\in O\subset V$.
				Par définition de $\scr{O}$ on a $O\in\frak{V}(x)$ donc par axiome croissance des voisinages $V\in\frak{V}(x)$.
				\item Soit $V\in\frak{V}(x)$ et posons $O=\ens{y\in E}{V\in\frak{V}(y)}$.
				\begin{itemise}
					\item Montrons que $O\in\scr{O}$ en prenant $y\in O$.
					Alors $V\in\frak{V}(y)$ donc par axiome de compatibilité des voisinages il existe $W\in\frak{V}(y)$ tel que $\forall z\in W, V\in\frak{V}(z)$.
					Par définition $\forall z\in W, z\in O$ ce qui montre $W\subset U$.
					Par axiome de croissance des voisinages $U\in\frak{V}(y)$.
					Puisque $y\in U$ est quelconque $O\in\scr{O}$.
					\item Par défintion $O\subset V$ et $x\in V$ donc $x\in O\subset V$.
				\end{itemise}
				Cela montre que $\exists O\in\scr{O}, x\in O\subset V$ c'est-à-dire $V\in\Phi(\scr{O})(x)$.
			\end{itemise}
		\end{itemise}
	\end{itemise}
\end{demonstration}

\section{Définitions}

Les définitions élégantes d'une topologie par les ouverts et des voisinaginations sont celles données par Nicolas Bourbaki.
Elles utilisent toutes les deux la convention qu'une intersection vide de parties de $E$ est $E$ lui-même, et qu'une union vide est $\vide$.
En pratique on remplace l'intersection finie par une intersection de deux éléments (puisqu'une intersection finie n'est autre qu'une répétition d'intersection de deux éléments) et un axiome pour décrire cette intersection vide.
On reformule de plus ces axiomes avec des symboles, plus utilisés et concis.
Les définitions précédentes étant équivalentes, on se permettra dans la suite de définir des topologies à partir d'une de ces constructions, voir d'un mélange de plusieurs d'entre elles.

\begin{definition}
	Une topologie $\scr{T}$ sur $E$ est aussi définie par les propriétés:
	\begin{itemise}
		\item $\vide, E\in\scr{T}$.
		\item $\forall U, V\in\scr{T}, U\cap V\in\scr{T}$.
		\item $\forall (U_i)_{i\in I}\in\scr{T}, \bigcup_{i\in I}U_i\in\scr{T}$.
	\end{itemise}
	Un espace topologique $(E, \scr{T})$ est un ensemble munie d'une topologie.
\end{definition}

\begin{definition}
	L'ensemble des voisinages $\frak{V}(x)$ de $x$ sont caractérisée par : 
	\begin{itemise}
		\item $\forall V\in\frak{V}(x), V\subset W\subset E\implies W\in\frak{V}(x)$.
		\item $\forall V, W\in\frak{V}(x), V\cap W\in\frak{V}(x)$.
		\item $\forall V\in\frak{V}(x), x\in V$.
		\item $\forall V\in\frak{V}(x)\quad\exists W\in\frak{V}(x)\quad\forall y\in W\quad V\in\frak{V}(y)$.
		\item $E\in\frak{V}(x)$.
	\end{itemise}
\end{definition}

\begin{exemple}
	Voici quelques topologies dont on peut toujours munir un ensemble quelconque $E$:
	\begin{itemise}
		\item la topologie grossière $\{\vide, E\}$.
		\item la topologie discrète $\frak{P}(E)$.
		\item la topologie cofinie $\{\vide\}\cup\ens{X\subset E}{\text{$\complement X$ fini}}$.
		\item la topologie codénombrable $\{\vide\}\cup\ens{X\subset E}{\complement X\text{ dénombrable}}$.
	\end{itemise}
\end{exemple}

\begin{demonstration}
	Les deux premières collections sont bien des topologies, car les unions et les intersections sont par définition dans la topologie.
	Vérifions que la troisième construction est bien une topologie.
	La quatrième s'en déduit en remplaçant « fini » par « dénombrable ».
	\begin{itemise}
		\item Soit $(O_i)_{i\in I}$ une famille de $\scr{T}$ et $O=\bigcup_{i\in I}O_i$.
		S'ils sont tous vides alors $O=\vide\in\scr{T}$.
		Sinon il existe un $O_j\in\scr{T}$ non vide, ce qui implique $\complement O_j$ est fini.
		De plus $\complement O=\bigcap_{i\in I}\complement O_i\subset\complement O_j$ fini donc $\complement O$ est fini donc $O\in\scr{T}$.
		\item Soit $(O_i)_{1\leqslant i\leqslant n}$ une famille finie de $\scr{T}$ et $O=\bigcap_{i=1}^nO_i$.
		Si un des $O_i$ est vide alors $O=\vide\in\scr{T}$.
		Sinon chaque $O_i\in\scr{T}$ est non vide, ce qui implique $\complement O_i$ est fini.
		Donc $\complement O=\bigcup_{i=1}^n\complement O_i$ est une union finie d'ensemble finis donc est finie donc $O\in\scr{T}$.
	\end{itemise}
\end{demonstration}

\section{Base d'une topologie, système fondamental de voisinage}

\begin{proposition}
	L'intersection d'une famille de topologies est une topologie.
\end{proposition}

\begin{demonstration}
	Soient $(\scr{T}_i)_{i\in I}$ des topologies de $E$ et $\scr{T}=\bigcap_{i\in I}\scr{T}_i$.
	\begin{itemise}
		\item Soit $(O_j)_{j\in J}$ une famille de $\scr{T}$ alors c'est une famille de chaque $\scr{T}_i$ donc pour chaque $i\in I$ on a par union $\bigcup_{j\in J}O_j\in\scr{T}_i$ donc $\bigcup_{j\in J}O_j\in\scr{T}$.
		\item Soit $(O_j)_{1\leqslant j\leqslant n}$ une famille fini de $\scr{T}$ alors c'est une famille de chaque $\scr{T}_i$ donc pour chaque $i\in I$ on a par intersection finie $\bigcap_{j=1}^nO_j\in\scr{T}_i$ donc $\bigcap_{j=1}^nO_j\in\scr{T}$.
	\end{itemise}
\end{demonstration}

\begin{definition}
	Pour toute partie $\frak{A}$ on appelle topologie engendrée par $\frak{A}$ l'intersection des topologie de $E$ incluant $\frak{A}$.
	Elle est notée $\sigma_\rm{topologie}(\frak{A})$ et on nomme alors $\frak{A}$ une pré-base de cette topologie.
\end{definition}

On peut en donner une approche plus constructive en faisant émerger une topologie à partir d'une famille quelconque.
Pour cela on utilise deux intermédiaires : les bases et pré-bases d'une topologie.
Ces notions sont utiles, car de nombreuses propriétés d'une topologie se ramènent à des énoncés sur une de ses bases et beaucoup de topologies sont faciles à définir par la donnée d'une base. 

\begin{definition}
	Une partie $\frak{B}$ de la topologie $\scr{T}$ est appelée une base si tout voisinage inclus un élément de $\frak{B}$.
\end{definition}

\begin{proposition}
	Il existe une topologie dont $\frak{B}$ est une base si et seulement si l'intersection de deux éléments de $\frak{B}$ inclut un élément de $\frak{B}$.
	Cette topologie est alors unique et ses voisinages d'un point $x$ sont les ensembles contenant un élément de $\frak{B}$ contenant $x$.
\end{proposition}

\equivalence{Par double implication.}{
	Supposons que $\frak{B}$ soit une base d'une topologie $\scr{T}$.
	Les intersections finies d'éléments de $\frak{B}$ sont des ouverts de $\scr{T}$ car $\frak{B}\subset\scr{T}$ et que $\scr{T}$ est stable par intersection finie.
	Or $\frak{B}$ génère par union les ouverts de $\scr{T}$ donc en particulier ses intersections finies.
	De même les unions quelconques d'éléments de $\frak{B}$ sont des ouverts de $\scr{T}$ et réciproqument chaque ouvert de $\scr{T}$ est une union d'éléments de $\frak{B}$.
	Ainsi $\scr{T}=\ens{\bigcup_{i\in I}U_i}{(U_i)_{i\in I}\in\frak{B}}$.
}{
	Supponsons que $\frak{B}$ génère ses intersections finies.
	La stabilité par union quelconque de ce $\scr{T}$ trouvé est claire.
	Il ne reste qu'à vérifier sa stabilité par intersections finies.
	Soit $(O_i)_{1\leqslant i\leqslant n}$ une famille finie de $\scr{T}$.
	Par construction pour tout $i\in I$ il exsite des $(U_i^j)_{j\in J_i}$ tels que $O_i=\bigcup_{j\in J_i}U_i^j$.
	Ainsi en utilisant la distributivité de l'intersection sur l'union $\bigcap_{i=1}^nO_i
	=\bigcap_{i=1}^n\bigcup_{j\in J_i}U_i^j
	=\bigcup_{j_1\in J_1}\cdots\bigcup_{j_n\in J_n}\bigcap_{i=1}^nU_i^{j_i}$.
	Par hypothèse $\frak{B}$ génère ses intersections finies donc le dernier ensemble est une grosse unions d'éléments de $\frak{B}$ qui est donc dans $\scr{T}$.
}

\begin{definition}
	Une partie $\frak{A}$ d'une topologie $\scr{T}$ est appelée une sous-base si l’ensemble des intersections finies de $\frak{A}$ est une base de $\scr{T}$.
\end{definition}

\begin{definition}
	On appelle système fondamentale de voisinage d'un point $x$ un ensemble de voisinages $\frak{S}$ tel que $\forall V\in\frak{V}(x), \exists W\in\frak{S}, W\subset V$. On peut de même définir un système fondamental de voisinage d'une partie de $E$.
\end{definition}

\begin{exemple}
	En anticipant un peu sur la définition d'ensemble discret :
	\begin{itemise}
		\item Dans un espace discret, l'ensemble $\{x\}$ (qui est ouvert donc un voisinage) est une base fondamentale de voisinage de $x$.
		\item Sur la droite rationnelle $[q-\frac{1}{n}, q+\frac{1}{n}]$ ou $]q-\frac{1}{n}, q+\frac{1}{n}[$ avec $n\in\N^*$ sont des systèmes fondamentaux de voisinages de $q\in\Q$.
		\item Il en va de même pour la droite réelle.
	\end{itemise}
\end{exemple}

% \begin{remarque}
% 		En explicitant le cas de l'intersection vide on obtient deux conditions :
% 		\begin{itemise}
% 			\item $E=\bigcup_{X\in\frak{B}}X$
% 			\item $\forall X, Y\in\frak{B}, \exists \mathcal{B}\subset\frak{B}, X\cap Y=\bigcup\frak{B}$
% 		\end{itemise}
% \end{remarque}

% \begin{remarque}
% 	Il est facile de vérifier que :
% 	$$
% 	\text{$\frak{B}$ est une base de $\scr{T}$}
% 	\iff
% 	\scr{T}=\sigma_\rm{topologie}(\frak{B})
% 	$$
% \end{remarque}

\section{Premières propriétés}

\begin{proposition}
	Les intérieurs et les fermés sont en relations par :
	\begin{itemise}
		\item $X^\circ=\complement\overline{\complement X}$ et $\overline{X}=\complement(\complement X)^\circ$.
		\item $\overline{X\cap Y}\subset\overline{X}\cap\overline{Y}$ et $(X\cup Y)^\circ\supset X^\circ\cup Y^\circ$
	\end{itemise}
\end{proposition}

\begin{demonstration}
	Appliquons les définitions :
	\begin{itemise}
		\item $X^\circ
			=\bigcup\ens{O}{X\supset O\in\scr{T}}
			=\complement\bigcap\ens{\complement O}{X\supset O\in\scr{T}}
			=\complement\overline{\complement X}$.
		\item $\overline{X\cap Y}$ est le plus petit fermé incluant $X\cap Y$ or $\overline{X}\cap\overline{Y}$ est fermé et l'inclut aussi.
	\end{itemise}
	Les autres relations s'obtiennent en passant au complémentaire.
\end{demonstration}

\begin{remarque}
	Il faut faire un peu attention.
	\begin{itemise}
		\item Une intersection infinie d'ouverts peut ne pas être un ouvert : $\bigcap_{x>0}]-x,x[=\{0\}$.
		\item Une union infinie de fermés peut ne pas être fermée : $\bigcup_{0<x<1}[x,1]=]0, 1]$.
		\item Les inclusions précédentes peuvent être strictes :
		\begin{itemise}
			\item $\overline{]-\infty, 0[\cap]0, +\infty[}=\overline{\vide}=\vide$
			\item $\overline{]-\infty, 0[}\cap\overline{]0, +\infty[}=]-\infty, 0]\cap[0, +\infty[=\{0\}$
		\end{itemise}
	\end{itemise}
\end{remarque}

\begin{proposition}
	$\overline{X}=\ens{x\in E}{\forall V\in\frak{V}(x), V\cap X\neq\vide}$
\end{proposition}

\begin{demonstration}
	\begin{align*}
		\overline{X}
		&=\complement(\complement X)^\circ
		=\complement\bigcup\ens{O\in\scr{T}}{O\subset\complement X} \\
		&=\complement\ens{x\in E}{\exists O\in\scr{T}, x\in O\subset\complement X} \\
		&=\complement\ens{x\in E}{\exists V\in\frak{V}(x), V\cap X=\vide} \\
		&=\ens{x\in E}{\forall V\in\frak{V}(x), V\cap X\neq\vide} \\
	\end{align*}
\end{demonstration}

\begin{definition}
	On dit que $X$ est dense dans $E$ si et seulement si $\overline{X}=E$.
\end{definition}

\begin{proposition}
	$X$ est dense si et seulement si les ouverts non vides rencontrent $X$
\end{proposition}

\begin{demonstration}
	La dernière équivalence vient de l'indépendance de $O\cap X\neq\vide$ à $x$.
	\begin{align*}
		\overline{X}=E
		&\iff\forall x\in E, \forall V\in\frak{V}(x), V\cap X\neq\vide \\
		&\iff\forall x\in E, \forall O\in\scr{T}, x\in O\Rightarrow O\cap X\neq\vide \\
		&\iff\forall O\in\scr{T}, O\neq\vide\Rightarrow O\cap X\neq\vide
	\end{align*}
\end{demonstration}

\section{Continuité}

Le premier exemple de fonctions continues concerne des fonctions réelles définies sur un intervalle et dont le graphe peut se tracer sans lever le crayon.
Cette première approche donne une idée de la notion (la fonction ne saute pas) mais n'est pas suffisante pour la définir, d'autant plus que certains graphes de fonctions pourtant continues ne peuvent pas se tracer de cette manière, telles par exemple des courbes ayant des propriétés fractales comme l'escalier de Cantor.

Historiquement définie pour des fonctions de la variable réelle, la notion de continuité se généralise à des fonctions entre espaces métriques ou entre espaces topologiques, sous une forme locale et sous une forme globale.
On énonce donc la continuité dans le cas de deux espaces topologiques $(E, \frak{V})$ et $(F, \frak{W})$, $f$ une application de $E$ dans $F$ et $x$ un point de $E$.

\begin{definition}
	La fonction $f$ est continue en $x$ si et seulement si tout voisinage de $f(x)$ inclut l'image par $f$ d'un voisinage de $x$ ce qui s'écrit :
	$$
	\forall W\in\frak{W}[f(x)], \exists V\in\frak{V}(x) f(V)\subset W
	$$
	Il suffit pour cela que cette propriété soit vérifiée pour tout $W$ d'une base de voisinages de $f(x)$, par exemple pour tout $W$ ouvert contenant $f(x)$. 
\end{definition}

\begin{proposition}
	La fonction $f$ est continue en $x$ si et seulement si :
	$$
	\forall W\in\frak{W}[f(x)], \inv{f}(W)\in\frak{V}(x)
	$$
\end{proposition}

\begin{demonstration}
	On sait que $f(V)\subset W$ est équivalent à $V\subset\inv{f}(W)$.
	Si $V$ est un voisinage, par croissance des voisinagination $\inv{f}(W)$ en est un ce qui montre le sens direct.
	Si $\inv{f}(W)$ est un voisinage noté $V$ alors $f(V)=f(\inv{f}(W))=W\subset W$, donc $f$ est continue en $x$ ce qui montre la réciproque.
\end{demonstration}

\begin{proposition}
	La composée de fonctions continues (en un point) est continue.
\end{proposition}

\begin{demonstration}
	Soit $g$ une application continue en $f(x)$.
	Utilisons la dernière caractérisation pour montrer que $g\circ f:E\rightarrow G$ est continue en $x$.
	Soit $W$ un voisinage de $(g\circ f)(x)=g(f(x))$.
	Par continuité de $g$ en $f(x)$ puis de $f$ en $x$ on a que $\inv{g}(W)$ est un voisinage d'ouvert de $f(x)$ et enfin que $\inv{f}(\inv{g}(W))=\inv{(g\circ f)}(W)$ est un voisinage de $x$.
\end{demonstration}

\begin{definition}
	Soient $E$ et $F$ deux espaces topologiques.
	L'application $f\colon E\rightarrow F$ est continue si et seulement si elle est continue tout point.
	L'ensemble des applications continues de $E$ dans $F$ est noté $\mathcal{C}(E, F)$.
\end{definition}

\begin{proposition}
	Les propositions suivantes sont équivalentes:
	\begin{itemise}
		\item $f$ est continue; $f\in\cal{C}(E, F)$
		\item $\forall X\in\frak{P}(E), f(\overline{X})\subset\overline{f(X)}$
		\item l'image réciproque d'un fermé est fermée; $\forall F\in\complement\scr{U}, \inv{f}(F)\in\complement\scr{T}$
		\item l'image réciproque d'un ouvert est ouverte; $\forall O\in\scr{U}, \inv{f}(O)\in\scr{T}$
	\end{itemise}
\end{proposition}

\begin{demonstration}
	Par implication circulaire:
	\begin{itemise}
		\item Soit $x\in\overline{X}$ et $W\in\frak{W}(f(x))$.
		Par continuité $\inv{f}(W)\in\frak{V}(x)$ donc par adhérence de $x$ $\exists y\in\inv{f}(W)\cap X\neq\vide$.
		Ainsi $f(y)\in W\cap f(X)$ soit $W\cap f(X)\neq\vide$ donc $f(x)\in\overline{f(X)}$.
		\item Soit $G\in\complement\scr{U}$ et $F=\inv{f}(G)$.
		Alors $f(\overline{F})\subset\overline{f(F)}\subset\overline{G}=G$ donc $\overline{F}\subset\inv{f}(G)=F$.
		\item Il suffit d'utiliser $\inv{f}(F)=\inv{f}(\complement O)=\complement\inv{f}(O)\in\complement\scr{T}$.
		\item Soit $x\in E$ et $W\in\frak{W}(f(x))$.
		Alors $\exists O\in\scr{U}, f(x)\in O\subset W$.
		Ainsi $x\in\inv{f}(O)$ et par hypothèse $\inv{f}(O)\in\scr{T}$ donc $\inv{f}(O)\in\frak{V}(x)$.
	\end{itemise}
\end{demonstration}

\begin{proposition}
	Si $f$ est continue alors $\forall X\in\frak{P}(E), \overline{\inv{f}(X)}\subset\inv{f}(\overline{X})$.
\end{proposition}

\begin{demonstration}
	Par continuité de $f$, $\inv{f}(\overline{X})$ est un fermé contenant $\inv{f}(X)$.
	Par minimalité de $\overline{\inv{f}(X)}$ on obtient le résultat.
\end{demonstration}

\begin{exemple}
	Il faut cependant se méfier :
	\begin{itemise}
		\item L'image directe d'un ouvert par une application continue n'est pas nécessairement ouvert (on dit alors que $f$ est ouverte).
		Par exemple l'image de $\R$ par $x\mapsto\frac{1}{1+x^2}$ est $]0,1]$ qui n'est ni fermé ni ouvert.
		\item Une bijection continue n'est pas nécessairement à réciproque continue (on dit bicontinue) comme le montre l'identité de $\Q$ muni la topologie discrète dans la droite rationnelle.
	\end{itemise}
\end{exemple}

\section{Comparaison de topologies}

Soient $(E, \scr{T}_1)$, $(E, \scr{T}_2)$, $(F, \scr{U})$ des espaces topologiques et $f\colon E\rightarrow F$.

\begin{definition}
	On dit que $\scr{U}$ est plus fine que $\scr{T}$ si l'identité $\id\colon(E, \scr{U})\rightarrow(E, \scr{T})$ est continue.
\end{definition}

\begin{proposition}
	Les propositions suivantes sont équivalentes:
	\begin{itemise}
		\item $\scr{U}$ est plus fine que $\scr{T}$.
		\item $\scr{T}\subseteq\scr{U}$.
		\item $\id\colon(E, \scr{T})\rightarrow(E, \scr{U})\text{ ouverte}$.
		\item tout voisinage de $x$ pour $\scr{T}$ est un voisinage de $x$ pour $\scr{U}$.
		\item l'adhérence de $X$ pour $\scr{U}$ est contenue dans l'adhérence de $X$ pour $\scr{T}$.
		\item un fermé de $\scr{T}$ est fermé pour $\scr{U}$.
		\item un ouvert de $\scr{T}$ est ouvert pour $\scr{U}$.
	\end{itemise}
\end{proposition}

\begin{demonstration}
	Toutes ces propriétés découlent du théorème précédent, à l'exception de l'inclusion.
	Or si $O\in\scr{T}$ par continuité de $\id\colon(E, \scr{U})\rightarrow(E, \scr{T})$, $\inv{\id}(O)=O\in\scr{U}$.
\end{demonstration}

\begin{remarque}
	Il est possible qu'entre deux topologies, aucune ne soit plus fine que l'autre, puisqu'il est possible qu'aucune des deux n'inclut l'autre.
	On dit alors qu'elles ne sont pas comparables.
\end{remarque}

\begin{remarque} 
	\text{}
	\begin{itemise}
		\item $f$ reste continue si l'on remplace la topologie sur $F$ par une topologie moins fine ou si on remplace la topologie sur $E$ par une topologie plus fine.
		\item $f$ reste ouverte si l'on remplace la topologie sur $F$ par une topologie plus fine ou si on remplace la topologie sur X par une topologie moins fine.
	\end{itemise}
	En effet, plus la topologie d'arrivée est fine, plus elle possède d'ouverts, plus il y a de propriétés à vérifier pour être une application continue.
\end{remarque}

\begin{proposition}
	Si $f\colon E\rightarrow F$ est une application continue qui est soit ouverte, soit fermée, alors:
	\begin{itemise}
		\item si f est une surjection, f est une application quotient, c'est-à-dire que Y a la topologie la plus fine pour laquelle f est continue ;
		\item si f est une injection, c'est un plongement topologique, c'est-à-dire que X et f(X) sont homéomorphes (par f) ;
		\item si f est une bijection, c'est un homéomorphisme.
	\end{itemise}
	Dans les deux premiers cas, être ouvert ou fermé n'est qu'une condition suffisante ; c'est également une condition nécessaire dans le dernier cas.
\end{proposition}

\begin{demonstration}
	\colour{red}{???}
\end{demonstration}

\section{Génération de topologie}

\begin{definition}
	Soit $E$ est un espace topologique et $F$ une partie de $E$.
	Les ouverts de la topologie induite par $E$ sur $F$ sont les traces sur $F$ des ouverts de $E$.
\end{definition}

\begin{remarque}
	Cette notion est d’usage fréquent, c’est en particulier à elle qu’on fait appel pour définir la continuité d’une fonction sur un intervalle fermé $[a, b]$ de $\R$.
	On dira que f est continue sur $[a, b]$ si elle est l’est pour la topologie induite par $\R$ sur cet intervalle, topologie pour laquelle $[a, b]$ est en fait ouvert !
	Les notions de continuité à droite ou à gauche aux bornes de l’intervalle font alors partie intégrante de la notion générale de continuité sur l’espace topologique $[a, b]$.
\end{remarque}

\begin{definition}
	Soit $E$ et $F$ deux espaces topologiques, $\mathcal{R}$ une relation d’équivalence sur $E$ de surjection canonique $\pi\colon E\rightarrow E/\mathcal{R}$.
	La topologie quotient sur $E/\mathcal{R}$ est formée des ouverts $O$ tels que $\inv{\pi}(O)$ soit un ouvert de $E$.
\end{definition}

\begin{proposition}
	La topologie quotient est la topologie finale associée à $\pi$, c'est-à-dire la plus fine rendant $\pi$ continue.
\end{proposition}

\begin{demonstration}
	Par construction $\pi$ est continue.
	Soit $\scr{U}$ une topologie sur $E/\mathcal{R}$ rendant $\pi$ continue.
	Si $O\in\scr{U}$ alors $\inv{\pi}(O)$ est un ouvert de $E$ par continuité de $\pi$, donc $O$ est dans la topologie quotient, qui est donc plus fine que $\scr{U}$.
\end{demonstration}

\begin{proposition}
	Une application $f\colon E/\mathcal{R}\rightarrow F$ est continue si et seulement si $f\circ\pi$ l'est.
\end{proposition}

\begin{demonstration}
	Si $O$ est un ouvert de $F$, alors $\inv{f}(O)$ est un ouvert de $E/\mathcal{R}$ si et seulement si $\inv{\pi}(\inv{f}(O))$ est ouvert dans $E$, c'est-à-dire $\inv{(f\circ\pi)}(O)$.
\end{demonstration}

\begin{definition}
	Soit $(E_i)_{i\in I}$ des espaces topologiques.
	On note $E=\prod_{i\in I} E_i$ le produit cartésien des $E_i$ et $\pi_i\colon E\rightarrow E_i$ les projections canoniques.
	La topologie produit est la topologie initiale associées aux projections canoniques, c'est-à-dire la topologie la moins fine rendant les $\pi_i\colon E\rightarrow E_i$ continues.
\end{definition}

\begin{proposition}
	L'ensemble $\frak{B}=\{\prod_{k=1}^nO_{i_k}\times\prod_{j\in I\setminus\ens{i_1, ..., i_n\}}F_j}{n\in\N^*, O_{i_k}\in\scr{U}_{i_k}}$ des pavés formés d'ouverts différents de $F_i$ dans un ensemble $\{i_1,...,i_n\}$ fini de directions est une base de topologie de $F$.
\end{proposition}

\begin{demonstration}
	Pour assurer la continuité des projections, la topologie produit $\scr{U}$ doit contenir toutes ces images réciproque $\inv{\pi_i}(O_i)=O_i\times\prod_{i\neq j\in I}F_j$.
	Elle inclut donc $\frak{A}=\ens{O_i\times\prod_{i\neq j\in I}F_j}{i\in I, O_i\in\scr{U}_i}$.
	En tant que topologie, $\scr{U}$ est stable par intersection finie.
	Elle inclut donc $\frak{B}=\ens{\bigcap_{k=1}^nA_k}{n\in\N^*, A_k\in\frak{A}}$.
	Soit $B=\bigcap_{k=1}^nA_k\in\frak{B}$.
	Certain $A_k=O_k\times\prod_{i_k\neq j\in I}F_j$ ont des $i_k$ égaux, c'est-à-dire des $O_k$ venant d'un même topologie.
	Soit $i'_{1\leqslant l\leqslant m}$ une renumérotation unique des topologies visités et $K_l=\ens{k}{i_k=i'_l}$ l'ensemble des indices dont les ouverts proviennent du même espace $F_{i'_l}$.
	Alors $B=\bigcap_{l=1}^m\bigcap_{k\in K_l}A_l$ avec chaque intersection finie.
	De même $O'_l=\bigcap_{k\in K_l}O_k\in\scr{U}_{i'_l}$ et
	$$
		\bigcap_{k\in K_l}A_l
		=\bigcap_{k\in K_l}(O_k\times\prod_{i_k\neq j\in I}F_j)
		=\bigcap_{k\in K_l}(O_k\times\prod_{i'_l\neq j\in I}F_j)
		=(\bigcap_{k\in K_l}O_k)\times\prod_{i'_l\neq j\in I}F_j
		=O'_l\times\prod_{i'_l\neq j\in I}F_j
	$$
	Ainsi $B=\bigcap_{l=1}^m(O'_l\times\prod_{i'_l\neq j\in I}F_j)=\prod_{l=1}^mO'_l\times\prod_{j\in I\setminus\{i'_1,...,i'_m\}}F_j$.
	La topologie produit inclue donc bien la famille $\frak{B}$ annoncée.
	C'est bien une base puisque pour $n=0$ on obtient $F\in\frak{B}$ et qu'elle est par contruction stable par intersection finie.
\end{demonstration}

\begin{proposition}
	$f\in\cal{C}(E, F)\iff\forall i\in I, \pi_i\circ f\in\cal{C}(E, F_i)$
\end{proposition}

\equivalence{Par double implication.}{
	Soit $O_i$ un ouvert de $F_i$.
		$\pi_i$ est continue donc $\inv{\pi_i}(O_i)$ est un ouvert de $F$.
		$f$ est continue donc $\inv{f}(\inv{\pi_i}(O_i))=\inv{(\pi_i\circ f)}(O_i)$ est un ouvert de $E$.
}{
	Soit $O$ un ouvert de $F$ alors $O=\bigcup_{j\in J}\bigcap_{k=1}^n\inv{\pi_{i_{jk}}}(O_{j_k})$ d'après la base trouvée précédemment.
		$\inv{f}(O)=\bigcup_{j\in J}\bigcap_{k=1}^n\inv{f}(\inv{\pi_{i_{jk}}}(O_{jk}))=\bigcup_{j\in J}\bigcap_{k=1}^n\inv{(\pi_{i_{jk}}\circ f)}(O_{jk})$ est une union d'intersection finie d'ouverts de $E$, donc est un ouvert.
}

\subsection{Approche catégorique}

On peut trouver \href{https://www.youtube.com/watch?v=xMmQrqdOkwI}{ici} une justification catégorique à cette définition.
Soit $A\subset X$.
On souhaite éviter de changer la continuité de $f\colon\bf{Top}(Y, A)$ en la voyant comme une application de $Y$ dans $A$ ou comme une application de $Y$ dans $X$.
Cela revient à demander l'équivalence entre la continuité de $f$ et celle de $\iota\circ f$ où $\iota\colon A\hookrightarrow X$ est l'injection naturelle.
\begin{align*}
\forall Y\colon\bf{Top}, \forall f\colon\bf{Top}(Y, A),\quad f\in\bf{Top}(Y, A)\iff \iota\circ f\in\bf{Top}(Y, X)
\end{align*}
La topologie induite $\scr{T}_\rm{induite}$ vérifie cette propriété car $\inv{(\iota\circ f)}(O)=\inv{f}(\inv{\iota}(O))=\inv{f}(A\cap O)$.
C'est aussi la seul topologie la vérifiant.
En effet soit $\scr{T}$ une autre topologie sur $A$ la vérifiant cette propriété.
Prenons $Y=(A, \scr{T})$ et $f=\id$ qui est alors continue.
Ainsi d'après le sens direct $\iota\circ\id=\iota\colon(A, \scr{T})\rightarrow X$ continue donc $\scr{T}_\rm{induite}\subset\scr{T}$.
Réciproquement prenons $Y=(A, \scr{T}_\rm{induite})$ et $f=\id$.
Puisque $\iota\circ\id=\iota$ est continue l'implication réciproque donne $\id\colon(A, \scr{T}_\rm{induite})\rightarrow(A, \scr{T})$ continue donc $\scr{T}\subset\scr{T}_\rm{induite}$.

\begin{centre}
	\begin{tikzcd}[scale cd=1.5] % [row sep=3pc, column sep=3pc]
		Y \ar[rr, bend right=20, "\iota\circ f"'] \arrow[r, "f"] & (A, \scr{T}) \arrow[r, "\iota"] & X \\
	\end{tikzcd}
\end{centre}

\chapter{Séparation}

On considère ici un espace topologique $(E, \scr{T})$.

\section{Vocabulaire}

\begin{definition}
	Il existe de nombreuses notions de séparation :
	\begin{itemise}
		\item $T_0:\forall x, y\in E, x\neq y\Rightarrow\exists O\in\scr{T}, (x\in O\et y\notin O)\ou(y\in O\et x\notin O)$
		\item $T_1:\forall x, y\in E, x\neq y\Rightarrow\exists U\in\scr{T}, x\in U\et y\notin U$
		\item $T_2:\forall x, y\in E, x\neq y\Rightarrow\exists U, V\in\scr{T}, x\in U\et y\in V\et U\cap V=\vide$
		\item $T_{2\frac{1}{2}}:\forall x, y\in E, x\neq y\Rightarrow\exists U, V\in\scr{T}, x\in U\et y\in V\et\overline{U}\cap\overline{V}=\vide$
		\item $T_{2\frac{3}{4}}:\forall x, y\in E, x\neq y\Rightarrow\exists f\in\cal{C}(E, [0, 1]), f(x)=0\et f(y)=1$
		\item $T_3:\forall x, F\in E\times\complement\scr{T}, x\notin F\Rightarrow\exists U, V\in\scr{T}, x\in U\et F\subset V\et U\cap V=\vide$
		\item $T_{3\frac{1}{2}}:\forall x, F\in E\times\complement\scr{T}, x\notin F\Rightarrow\exists f\in\mathcal{C}(E, [0, 1]), f(x)=1\et f(F)=\{0\}$
		\item $T_4:\forall F, G\in\complement\scr{T}, F\cap G=\vide\Rightarrow\exists U, V\in\scr{T}, F\subset U\et G\subset V\et U\cap V=\vide$
		\item $T_5:\forall X, Y\subset E, X\cap\overline{Y}=\overline{X}\cap Y=\vide\Rightarrow\exists U, V\in\scr{T}, X\subset U\et Y\subset V\et U\cap V=\vide$
		\item $G:\text{tout fermé est un $G_\delta$ (ouvert-intersection=Gebiet-Durchschnitt dénombrable)}$
	\end{itemise}
\end{definition}

\begin{proposition}
	$T_0$ est équivalente à $\forall x, y\in E, x\neq y\implies x\notin\overline{\{y\}}\ou y\notin\overline{\{x\}}$
\end{proposition}

\begin{demonstration}
	$x\notin\overline{\{y\}}
		\iff\exists F\in\complement\scr{T}, y\in F\et x\notin F
		\iff\exists O\in\scr{T}, x\in O, y\notin O$
\end{demonstration}

\begin{proposition}
	$T_1$ est équivalente à :
	\begin{itemise}
		\item $\forall x, y\in E, x\neq y\implies\exists U, V\in\scr{T}, (x\in U\et y\notin U)\et(y\in V\et x\notin V)$.
		\item les singletons de $E$ sont des fermés.
		\item tout point de $E$ est l'intersection de ses voisinage.
	\end{itemise}
\end{proposition}

\begin{demonstration}
	Par implication circulaire.
	\begin{itemise}
		\item Soit $x$ et $y$ différents, on obtient le résultat en appliquant $T_1$ à $x, y$ puis à $y, x$.
		\item Soit $x$ et $y$ différents $\exists F_y\in\complement\scr{T}, x\in F_y\et y\notin F_y$, donc $\overline{\{x\}}\subset\bigcap_{y\neq x}F_y=\{x\}$.
		\item $\{x\}\subset\bigcap_{V\in\frak{V}(x)}V\subset\bigcap_{V\in\frak{V}(x)}\overline{V}\bigcap_{x\in F\in\complement\scr{T}}F=\overline{\{x\}}=\{x\}$ implique l'égalité partout.
		\item Soit $y\neq x$ on a $y\notin\{x\}=\bigcap_{V\in\frak{V}(x)}V$ donc $\exists U\in\scr{T}, x\in U\et y\notin U$.
	\end{itemise}
\end{demonstration}

\begin{proposition}
	$T_2$ est équivalente à :
	\begin{itemise}
		\item tout point de $E$ est l'intersection de ses voisinages fermés.
		\item quel que soit $I$ la diagonale de $E^I$ est fermée.
		\item la diagonale de $E\times E$ est fermée.
	\end{itemise}
\end{proposition}

\begin{demonstration}
	Par implication circulaire.
	\begin{itemise}
		\item Si $y\neq x$ alors $\exists U, V\in\scr{T}, x\in U\et y\in V\et U\cap V=\vide$ donc $y\notin\overline{U}$.
		\item Soit $\Delta$ la diagonale.
		Si $(x_i)_{i\in I}\notin\Delta$ alors $\exists j, k\in I, x_j\neq x_k$, donc $\exists V_j, V_k\in\frak{V}(x_j)\times\frak{V}(x_k)$ fermés tels que $x_j\notin V_k\et x_k\notin V_j=\vide$.
		Ainsi $O_j=V_j^\circ\setminus V_k$ et $O_k=V_k^\circ\setminus V_j$ sont des ouverts disjoints contenant $x_j$ et $x_k$ respectivement.
		Donc $O_j\times O_k\times\prod_{i\in I\setminus\{j, k\}}E$ est un ouvert de $E^I$ ne rencontrant pas la diagonale et contenant $(x_i)_{i\in I}$.
		\item Il suffit de prendre $I=\{0, 1\}$.
		\item Soit $\Delta$ la diagonale.
		Si $x\neq y$ alors $(x, y)\notin\Delta$.
		Or $\Delta$ est fermée dans $E^2$, donc $\complement\Delta$ est un ouvert.
		Par définition de la base de la topologie produit, $\exists U_{i\in I}, V_{i\in I}\in\scr{T}, \complement\Delta=\bigcup_{i\in I} U_i\times V_i$.
		Puisque $(x, y)\notin\Delta$ je peux extraire du recouvrement $U, V$ tels que $(x, y)\in U\times V$, c'est-à-dire $x\in U\et y\in V$.
		Or $V\times W\subset\complement\Delta$, donc $U\cap V=\vide$.
	\end{itemise}
\end{demonstration}

\begin{proposition}
	$T_3$ équivaut à tout fermé de $E$ est l'intersection de ses voisinages ouverts.
\end{proposition}

\equivalence{Par double implication.}{
	Soit $F\in\complement\scr{T}$ et $x\notin F$ alors $\exists V_x\in\scr{T}, F\subset V_x\et x\notin V_x$.
		Donc $F\subset\bigcap_{x\notin F}V_x\subset F$.
}{
	Soit $F\in\complement\scr{T}$ et $x\notin F$.
		Supposons $\forall U, V\in\scr{T}, (x\in U\et F\subset V)\implies U\cap V\neq\vide$.
		Soit $U$ un ouvert contenant $x$.
		Par hypothèse $\bigcap_{F\subset V\in\scr{T}}V=F$.
		Donc $U\cap F=U\cap\bigcap_{F\subset V\in\scr{T}}V=\bigcap_{F\subset V\in\scr{T}}U\cap V\neq\vide$.
		Donc $x\in\widetilde{F}=F$ car $F$ est fermé, ce qui est absurde.
}

\begin{proposition}
	$T_4$ équivaut à $\forall O, F\in\scr{T}\times\complement\scr{T}, F\subset O\implies\exists U\in\scr{T}, F\subset U\subset\overline{U}\subset O$.
\end{proposition}

\equivalence{Par double implication.}{
	Soit $O$ un ouvert et $F\subset O$ un fermé.
		Puisque $\complement O$ est un fermé disjoint de $F$, $\exists U, V\in\scr{T}, F\subset U\et\complement O\subset V\et U\cap V=\vide$.
		Alors $F\subset u\subset\overline{U}=\bigcap_{U\subset F\in\complement\scr{T}}F\subset\complement V\subset O$.
}{
	Soient $F, G\in\complement\scr{T}$ disjoints.
		Puisque $\complement G$ est ouvert et $F\subset\complement G$ alors $\exists U\in\scr{T}, F\subset U\subset\overline{U}\subset\complement G$.
		Puisque $\complement\overline{U}$ est ouvert, que $G\subset\complement\overline{U}$ et que $U\cap\complement\overline{U}\subset U\cap\complement U=\vide$, $T_4$ est montré pour $V=\complement\overline{U}$.
}

\section{Classification}

\begin{proposition}
	\begin{centre}
		\begin{tikzpicture}[xscale=2.5, yscale=1.5]
			\node (T0) at (0, 0) {$T_0$};
			\node (T1) at (1, 0) {$T_1$};
			\node (T2) at (2, 0) {$T_2$};
			\node (T21/2) at (3, 0) {$T_{2\frac{1}{2}}$};
			\node (T23/4) at (4, 0) {$T_{2\frac{3}{4}}$};
			\node (T3) at (0.5, 1) {$T_3$};
			\node (T31/2) at (1.5, 1) {$T_{3\frac{1}{2}}$};
			\node (T1+T4) at (2.5, 0.75) {$T_1+T_4$};
			\node (T0+T3) at (3.5, 1) {$T_0+T_3$};
			\node (T0+T31/2) at (4.5, 1) {$T_0+T_{3\frac{1}{2}}$};
			\node (T3+T4) at (2.5, 1.25) {$T_3+T_4$};
			\node (T4) at (0, 2) {$T_4$};
			\node (T5) at (1, 2) {$T_5$};
			\node (T2+T4+G) at (2, 2) {$T_2+T_4+G$};
			% implications
			\draw[<-] (T0) to[bend left=20] (T1);
			\draw[<-] (T1) to[bend left=20] (T2);
			\draw[<-] (T2) to[bend left=20] (T21/2);
			\draw[<-] (T21/2) to[bend left=20] (T23/4);
			\draw[<-] (T3)--(T31/2);
			\draw[<-] (T2)--(T1+T4);
			\draw[<-] (T21/2)--(T0+T3);
			\draw[<-] (T23/4)--(T0+T31/2);
			\draw[<-, green] (T31/2) to[bend right=20] (T1+T4);
			\draw[<-, green] (T31/2) to[bend left=20] (T3+T4);
			\draw[<-] (T4)--(T5);
			\draw[<-, green] (T5)--(T2+T4+G);
			\draw[<-, red] (T5) to[bend left=20] (T1+T4);
			% contre-exemples
			\draw[->, dashed] (T4) to[bend right=20] node {$\times$} (T0);
			\draw[->, dashed] (T3) to[bend right=20] node {$\times$} (T0);
			\draw[->, dashed] (T0) to[bend right=20] node {$\times$} (T1);
			\draw[->, dashed] (T1) to[bend right=20] node {$\times$} (T2);
			\draw[->, dashed] (T2) to[bend right=20] node {$\times$} (T21/2);
			\draw[->, dashed, green] (T21/2) to[bend right=20] node {$\times$} (T23/4);
			\draw[->, dashed] (T23/4) -- node {$\times$} (T3);
			\draw[->, dashed] (T0+T3) -- node {$\times$} (T31/2);
		\end{tikzpicture}
	\end{centre}
\end{proposition}

\begin{demonstration}
	\begin{itemise}
		\item[$T_5\Rightarrow T_4$ et $T_{2\frac{1}{2}}\Rightarrow T_2\Rightarrow T_1\Rightarrow T_0$:] Évident
		\item[$T_1+T_4\Rightarrow T_{3\frac{1}{2}}$:] {\color{green} ???}
		\item[$T_3+T_4\Rightarrow T_{3\frac{1}{2}}$:] {\color{green} ???}
			{\color{red}
				\item[$T_1+T_4\Rightarrow T_5$:] Soient $X, Y\subset E$ tels que $X\cap\overline{Y}=\overline{X}\cap Y=\vide$ et $x\in X$.
				Par caractérisation de $T_1$, $\{x\}$ est fermé et $\{x\}\cap\overline{Y}\subset X\cap\overline{Y}=\vide$.
				Par $T_4$: $\exists U_x, V\in\scr{T}, \{x\}\subset U_x\et\overline{Y}\subset V\et U_x\cap V=\vide$.
				Ainsi $X=\bigcup_{x\in X}\{x\}\subset\bigcup_{x\in X}U_x$ est ouvert, $Y\subset\overline{Y}\subset V$ ouvert et $V\cap\bigcup_{x\in X}U_x=\bigcup_{x\in X}V\cap U_x=\vide$.
			}
		\item[$T_{3\frac{1}{2}}\Rightarrow T_3$:] Soient $x, F$ tels que $x\notin F$.
		Par $T_{3\frac{1}{2}}$: $\exists f\in\cal{C}(E, [0, 1]), f(x)=1, f(F)=\{0\}$.
		Posons $U=\inv{f}(]1-\varepsilon, 1])$ et $V=\inv{f}([0, \varepsilon[)$ pour $0<\varepsilon<\frac{1}{2}$ .
		Puisque $[0, 1]$ est munie de la topologie trace de $\R$, $]1-\varepsilon, 1]$ et $[0, \varepsilon[$ sont des ouverts.
		Ainsi $U$ et $V$ sont des ouverts par continuité de $f$.
		De $f(x)=1$ et $f(F)=0$ on a $x\in\inv{f}(]1-\varepsilon, 1])=U$ et $F\subset\inv{f}([0, \varepsilon[)=V$.
		Pour conclure $U\cap V=\inv{f}(]1-\varepsilon, 1])\cap\inv{f}([0, \varepsilon[)=\inv{f}(]1-\varepsilon, 1]\cap[0, \varepsilon[)=\inv{f}(\vide)=\vide$.
		\item[$T_0+T_{3\frac{1}{2}}\Rightarrow T_{2\frac{3}{4}}$:] Soient $x, y\in E$ différents.
		Quitte à échanger le rôle de $x$ et $y$ on a par $T_0$: $\exists O\in\scr{T}, y\in O\et x\notin O$.
		Puisque $\complement O$ est fermé et $y\notin\complement O$, on a par $T_{3\frac{1}{2}}$: $\exists f\in\cal{C}(E, [0, 1]), f(y)=1\et f(\complement O)=\{0\}$.
		Donc $f(x)=0$ et $f(y)=1$.
		\item[$T_0+T_3\Rightarrow T_{2\frac{1}{2}}$:] Soient $x, y\in E$ différents.
		Quitte à échanger $x$ et $y$ dans la formule de $T_0$ on a $\exists O\in\scr{T}, x\in O\et y\notin O$, soit $x\in O\not\ni y$.
		Puisque $x\notin\complement O$ et $\complement O$ est fermé $T_3$ donne :
		$\exists U, V\in\scr{T}, x\in U\et\complement O\subset V\et U\cap V=\vide$, équivalent à
		$x\in U\et\complement V\subset O\et U\subset\complement V$, c'est-à-dire
		$x\in U\subset\complement V\subset O\not\ni y$.
		De même puisque $x\notin\complement U$ et $\complement U$ est fermé $T_3$ donne :
		$\exists W, T\in\scr{T}, x\in W\et\complement U\subset T\et W\cap T=\vide$, équivalent à
		$x\in W\et\complement T\subset U\et W\subset\complement T$, c'est-à-dire
		$x\in W\subset\complement T\subset U\subset\complement V\subset O\not\ni y$.
		Puisque $\overline{W}$ est le plus petit fermé contenant $W$, $\overline{W}\subset\complement T$.
		De même $\overline{V}\subset\complement U$.
		Ainsi $x\in W, y\in V$ et $\overline{W}\cap\overline{V}\subset\complement T\cap\complement U=\vide$.
		\item[$T_1+T_4\Rightarrow T_2$:] Soient $x, y\in E$ différents.
		Par la caractérisation de $T_1$, les singletons $\{x\}$ et $\{y\}$ sont fermés.
		Par $T_4$: $\exists U, V\in\scr{T}, \{x\}\subset U\et\{y\}\subset V\et U\cap V=\vide$.
		\item[$T_{2\frac{3}{4}}\Rightarrow T_{2\frac{1}{2}}$:] Soient $x, y\in E$ différents.
		Par $T_{2\frac{3}{4}}$: $\exists f\in\cal{C}(E, [0, 1]), f(x)=1, f(y)=0$.
		Posons $U=\inv{f}(]1-\varepsilon, 1])$ et $V=\inv{f}([0, \varepsilon[)$ pour $0<\varepsilon<\frac{1}{2}$ .
		Puisque $[0, 1]$ est munie de la topologie trace de $\R$, $]1-\varepsilon, 1]$ et $[0, \varepsilon[$ sont des ouverts.
		Ainsi $U$ et $V$ sont des ouverts par continuité de $f$.
		De $f(x)=1$ on a $x\in\inv{f}(]1-\varepsilon, 1])=U$ et de $f(y)=0$ on a $y\in\inv{f}([0, \varepsilon[)=V$.
		Pour conclure
		$\overline{U}\cap\overline{V}
			=\overline{\inv{f}(]1-\varepsilon, 1])}\cap\overline{\inv{f}([0, \varepsilon[)}
			\subset\inv{f}(\overline{]1-\varepsilon, 1]})\cap\inv{f}(\overline{[0, \varepsilon[})
			=\inv{f}([1-\varepsilon, 1]\cap[0, \varepsilon])
			=\inv{f}(\vide)
			=\vide$.
		\item[$T_0\not\Rightarrow T_1$:] $E=\{a, b\}$ munie de la topologie $\scr{T}=\{\vide, \{a\}, E\}$ est $T_0$ car $a\in\{a\}\not\ni b$ mais le seul ouvert contenant $b$ est $O=\{a, b\}$ et $a\in O$.
		\item[$T_3\not\Rightarrow T_0$ et $T_4\not\Rightarrow T_0$] un ensemble à plus de deux éléments munit de sa topologie grossière est $T_3$ et $T_4$ mais pas $T_0$.
		\item[$T_1\not\Rightarrow T_2$:] Un espace fini $T_1$ est discret.
		En effet, sa topologie doit pouvoir exclure chaque point, donc possède chaque singleton.
		Par stabilité par union, cette topologie est discrète.
		Un espace $T_1$ non $T_2$ est donc infini.
		La topologie cofinie sur un espace infinie est $T_1$ mais non $T_2$.
		En effet soit $X, Y$ deux ouverts cofinis voisins de $x$ et $y$ respectivement.
		Alors $\complement(X\cap Y)=\complement X\cup\complement Y$ est fini, donc ne peut couvrir $E$ donc $X\cap Y\neq\vide$.
		\item[$T_2\not\Rightarrow T_{2\frac{1}{2}}$:] Soit un triangle équilatéral $T$ de sommets notés $S_i$.
		Considérons $E=T^\circ\cup\{S_1, S_2, S_3\}$.
		On le munit de la topologie trace de $\R^2$ sur $T^\circ$ à laquelle on ajoute les croissants $\ens{S_i\}\cup(T^\circ\cap B(\rho M_i,}{\rho M_i-S_i|))$ avec $M_i=\frac{S_i+S_{i+1}}{2}$ le milieu des deux sommets consécutifs et $\rho>1$.
		On représente par exemple un voisinage du sommet $A$ en gris.
		\begin{centre}
			\begin{tikzpicture}[scale=0.75]
				\draw[very thick] (1.0, 0.000) node[anchor=west]{$A$}
				-- (-0.5, 0.866025403784) node[anchor=east]{$C$}
				-- (-0.5,-0.866025403784) node[anchor=east]{$B$} -- cycle;
				\filldraw[black!20] (1.0, 0.000) circle (2pt);
				\filldraw[black!20] (-0.5, 0.866025403784) arc (210:270:1.75)--(-0.5, 0.866025403784) ;
			\end{tikzpicture}
		\end{centre}
		Les points de $T^\circ$ sont séparés car la topologie de $\R^2$ est séparée.
		Les points de $\{S_1, S_2, S_3}$ sont séparé de ceux dans $T^\circ$ en prenant $\rho$ suffisamment grand.
		Idem entre les points de $\{S_1, S_2, S_3\}$.
		Cette topologie est donc $T_2$.
		Elle n'est pas $T_{2\frac{1}{2}}$ car l'adhérence des croissants voisins de $S_i$ contiennent toujours $S_{i+1}$.
		\item[$T_0+T_3\not\Rightarrow T_{3\frac{1}{2}}$:] Considérons $E=(\R\times\R^+)\cup\{\infty\}$ munit:
		\begin{itemise}
			\item de la topologie discrète sur $\R\times\R_+^*$
			\item des voisinages $\{(x, 0)\}\cup R_x\setminus T_x$ de $(x, 0)$ où $R_x=V_x\cup O_x$ avec $V_x$ le segment vertical $\ens{(x, y)}{y\in[0,2[}$, $O_x$ l'oblique $\ens{(x+y, y)}{y\in [0, 2[}$ et $T_x$ un ensemble finis de « trous » dans $R_x$.
			\item des voisinages $U_n=\{\infty\}\cup(\{x>n\}\times\R^+)$ de $\infty$
		\end{itemise}
		Cet espace est $T_2$ comme le montre le premier schéma, puisqu'il suffit de placer des « trous » aux bons endroits dans $R_x$ ou de choisir $n$ suffisamment grand pour avoir des voisinages disjoints.
		\begin{centre}
			\begin{tikzpicture}[scale=0.75]
				\filldraw[black!10, very thick] (5,0) rectangle (7,2.5);
				\draw[dashed] (5,0)--(5,2.5);
				\node[anchor=north] at (5,0) {$n$};
				\node at (6,1.5) {$\infty$};
				\draw[->] (-1, 0)--(6, 0);
				\draw[->] (0, -0.5)--(0, 2.5);
				\node[anchor=north] at (1,0) {$x_1$};
				\draw[black!20!blue, very thick] (1, 2)--(1, 0)--(3, 2);
				\filldraw[white] (1.0, 0.5) circle (2pt);
				\filldraw[white] (1.0, 1.5) circle (2pt);
				\filldraw[white] (1.0, 1.25) circle (2pt);
				\filldraw[white] (2.75, 1.75) circle (2pt);
				\filldraw[white] (2.3, 1.3) circle (2pt);
				\node[anchor=north] at (2.3,0) {$x_2$};
				\draw[black!20!red, very thick] (2.3, 2)--(2.3, 0)--(4.3, 2);
				\filldraw[white] (2.3, 0.7) circle (2pt);
				\filldraw[white] (2.3, 1.73) circle (2pt);
				\node[anchor=east] at (1.0, 1.5) {$x_3$};
				\filldraw[black!20!green] (1.0, 1.5) circle (1.25pt);
			\end{tikzpicture}
			\hfill
			\begin{tikzpicture}[scale=0.75]
				\node at (5,1.5) {$\infty$};
				\draw[->] (-1, 0)--(5, 0);
				\draw[->] (0, -0.5)--(0, 2.5);
				\draw[black!20!blue, very thick] (0.1, 2)--(0.1, 0)--(2.1, 2);
				\draw[black!20!blue, very thick] (0.5, 2)--(0.5, 0)--(2.5, 2);
				\draw[black!20!blue, very thick] (0.75, 2)--(0.75, 0)--(2.75, 2);
				\node[anchor=north] at (1,0) {$x$};
				\draw (1,-0.1)--(1,0.1);
				\draw[black!20!red, very thick] (1.5, 2)--(1.5, 0)--(3.5, 2);
				\draw[black!20!red, very thick] (1.75, 2)--(1.75, 0)--(3.75, 2);
				\draw[black!20!red, very thick] (2.3, 2)--(2.3, 0)--(4.3, 2);
			\end{tikzpicture}
		\end{centre}

		Cet espace est $T_3$.
		Montrons $\forall x, O\in E\times\scr{T}, x\in O\Rightarrow\exists G\in\complement\scr{T}, x\in G\subset O$
		\begin{itemise}
			\item pour $x\in\R\times\R_+^*$ alors $F=\{x\}$ est un fermé vérifiant la propriété
			\item pour $x\in\R\times\{0\}$ comme le montre le second schéma $\bigcup_{x\notin V\in\scr{T}}V=E\setminus\{x\}$ est un ouvert en tant que réunion d'ouverts, donc $\{x\}$ fermé vérifie la propriété
			\item pour $x=\infty$ et $U_n$ un voisinage alors on remarque que seul les points de $]n+1, n+2]\times\{0\}$ hors de $U_{n+2}$ peuvent lui adhérer (grâce à branche à oblique des voisinages) donc $\overline{U_{n+2}}\subset U_n$ vérifie la propriété
		\end{itemise}
		En prenant $F=\complement O$, $U=O$ et $V=\complement G$ dans la définition de $T_3$ on conclue.

		Cet espace n'est pas $T_{3\frac{1}{2}}$.
		Soit $F=[0, 1]\times\{0\}$ et $f\in\cal{C}(E, [0, 1])$ où $f(F)=\{0\}$.
		$\complement F=(\R\times\R_+^*)\cup\bigcup_{x\notin[0, 1]}R_x\in\scr{T}$ donc $F$ est fermé.
		Posons $C_0=F$ et montrons par récurrence $\forall n\in\N, \exists C_n\subset [n, n+1[\times\{0\}\text{ indénombrable}, f(C_n)=\{0\}$.
		Soit $c\in C_n$ alors $[0, \varepsilon[$ est ouvert pour la topologie trace de $\R$ sur $[0, 1]$.
		Par continuité $\inv{f}([0, \varepsilon[)$ est ouvert et contient $c$, donc $O_c\setminus T_c^\varepsilon$ avec $T_c^\varepsilon$ fini, puis $\inv{f}(\{0\})=\bigcap_{q\in\N^*}\inv{f}([0,\frac{1}{q}[)\supset\bigcap_{q\in\N^*}\bigcup_{c\in C_n}O_c\setminus T_c^{1/q}=\bigcup_{c\in C_n}O_c\setminus\bigcup_{q\in\N^*}\bigcap_{c\in C_n}T_c^{1/q}$ qui est indénombrable car $\bigcup_{c\in C_n}O_c$ l'est et $\bigcup_{q\in\N^*}\bigcap_{c\in C_n}T_c^{1/q}$ ne l'est pas.
		Sa projection $P_n$ sur l'axe des abscisses, et donc $C_{n+1}=P_n\cap[n+1, n+2[\times\{0\}$ est indénombrable.
		Soit $x\in C_{n+1}$ alors $x$ a « au-dessus de lui » une infinité indénombrable de points d'annulation de $f$, donc adhère à $\inv{f}(\{0\})$ qui est fermé en tant qu'image réciproque d'un fermé par une application continue.
		Ainsi $f(C_{n+1})=0$ donc par récurrence, on a montré le lemme.
		Ainsi $\infty$ adhère à $\inv{f}(\{0\})$ donc $f(\infty)=0$.
		\item[$T_{2\frac{1}{2}}\not\Rightarrow T_3$ et $T_{2\frac{1}{2}}\not\Rightarrow T_{2\frac{3}{4}}$:]
		Soit $K=\ens{1/n}{n\in\N^*}$.
		On munit $\R$ de la la $K$-topologie dont une base est formée des $]a, b[$ et des $]a, b[\setminus K$.
		Elle est plus fine que la topologie classique, préservant donc la continuité des fonctions $f:\R\rightarrow[0, 1]$.
		En particulier, une fonction montrant la $T_{2\frac{3}{4}}$ séparation de $\R$ (par exemple une interpolation) montre que $(\R, K)$ l'est aussi.
		Cet espace n'est pas $T_3$.
		Soit $]a, 2[, ]-1, b[$ des ouverts contenant respectivement $K$ et $\{0\}$.
		De $0\in]-1, b[$ on a $b>0$.
		Si $a>0$ alors $\frac{1}{a}\leqslant \lceil\frac{1}{a}\rceil=n\in\N^*$ donc $\frac{1}{n}<a$ et $K\ni\frac{1}{n}\notin]a, 2[$ absurde.
		Ainsi $a\leqslant0<b$, forçant tout voisinages de la base de $0$ et $K$ à s'intersecter.
	\end{itemise}
\end{demonstration}

\section{Théorèmes de représentation}

Soit $(E, \scr{T})$ et $(F_i, \scr{U}_i)_{i\in I}$ des espaces topologiques $F=\prod_{i\in I}F_i$ leur produit cartésien munie de sa topologie produit $\scr{U}$ et $f_i:E\rightarrow F_i$ des applications continues.

\begin{definition}
	Si $\forall x, y\in E, x\neq y\Rightarrow\exists i\in I, f_i(x)\neq f_i(y)$ alors on dit que la famille $f_{i\in I}$ sépare les points.
\end{definition}

\begin{definition}
	Si $\forall x, F\in E\times\complement\scr{T}, x\notin F\Rightarrow\exists i\in I, f_i(x)\notin\overline{f_i(F)}$ alors on dit que la famille $f_{i\in I}$ sépare les points des fermés.
\end{definition}

\begin{proposition}
	Soit $e:E\rightarrow F$ telle que $e(x)_i=f_i(x)$.
	\begin{itemise}
		\item l'application $e$ est continue.
		\item l'application est injective si et seulement si $f_{i\in I}$ sépare les points de $E$.
		\item si la famille $f_{i\in I}$ sépare les points et les points des fermés alors $e$ est ouverte.
	\end{itemise}
\end{proposition}

\begin{demonstration}
	Vérifions chaque propriété.
	\begin{itemise}
		\item Contraposer l'injectivié de $e$ s'écrivant $e(x)=e(y)\Rightarrow x=y$ donne la séparation.
		\item Par définition, $\pi_i\circ e=f_i$  qui est continue, rendant $e$ continue.
		\item Soit $U\in\scr{T}$ non vide (sinon $e(U)=\vide$ évidement ouvert) et $x\in U$.
		Par séparation des fermés $\exists i\in I, f_i(x)\notin\overline{f_i(\complement U)}$.
		Ainsi $f_i(x)\in O_i=\complement\overline{f_i(\complement U)}\in\scr{U}_i$ puis $e(x)\in O_i\times\prod_{i\neq j\in I}F_i\in\scr{U}$.
		{\color{red} ???}
	\end{itemise}
\end{demonstration}

\begin{definition}
	Posons $Q=[0, 1]$ munie de la topologie $\ens{[0, t[}{t\in[0, 1]}$ appelée topologie supérieur.
	On appelle quasi-cube les espace de la forme $Q^I$ pour un certain $I\neq\vide$.
\end{definition}

\begin{proposition}
	L'espace $Q$ est $T_0$, quasi-compact mais pas $T_1$.
\end{proposition}

\begin{demonstration}
	\begin{itemise}
		\item Soit $x, y\in Q$ différents, supposons par exemple $y>x$.
		Alors pour $t=\frac{x+y}{2}$ on a $x\in[0, t[$ et $y\notin[0, t[$.
		\item Soit $U=\bigcup_{i\in I}[0, t_i[$ une réunion d'ouverts.
		Alors pour $t=\max_{i\in I}t_i$ on a $U=[0,t[$.
		\item L'intersection de deux ouverts contient toujours 0.
	\end{itemise}
\end{demonstration}

\begin{definition}
	Une application $f:E\rightarrow[0, 1]$ est semi-continue supérieurement si et seulement si $f:E\rightarrow Q$ est continue.
\end{definition}

\begin{proposition}
	Tout espace $T_0$ est homéomorphe à un sous-espace d'un quasi-cube.
\end{proposition}

\begin{demonstration}
	Soit $E$ un espace topologique $T_0$ et $f_{i\in I}:E\rightarrow [0, 1]$ la famille des applications semi-continues supérieurement de $E$.
	L'application $e:E\rightarrow \prod_{i\in I}Q$ définie par $e(x)_i=f_i(x)$ est continue d'après la proposition précédente.
	Soient $x, y\in E$ différents.
	Par $T_0$ quitte à échanger les rôles de $x$ et $y$, $\exists U\in\scr{T}, x\in U\not\ni y$.
	La fonction $\1_{\complement U}$ est semi-continue supérieurement car $\inv{\1_{\complement U}}([0, t[)=\inv{\1_{\complement U}}(\{0\})=U$ et sépare $x$ et $y$ car $\1_{\complement U}(x)=0\neq1=\1_{\complement U}(y)$.
	Soit $x, F\in E\times\complement\scr{T}$ tels que $x\notin F$.
	De même $\1_F$ est semi-continue supérieurement et sépare $x$ et $F$.
	Ainsi $f_{i\in I}$ sépare les points et les points des fermés.
	D'après la proposition précédente c'est une application ouverte et injective.
	D'après {\color{red} ???} c'est un homéomorphisme.
\end{demonstration}

\begin{remarque}
	D'après le théorème de Tychonoff, un produit d'espaces quasi-compacts est quasi-compact.
	En outre tout fermé d'un espace quasi-compact  est quasi-compact (l'inverse est faux, voir {\color{red} ???}).
	Ainsi si $E$ est $T_0$, $\overline{e(E)}$ (la fermeture de $e(E)$ dans $Q^F$) est un quasi-compact dans lequel $e(E)$ est dense: c'est une quasi-compactification de $E$.
\end{remarque}

\begin{remarque}
	Dans le cas d'un espace métrique séparable de distance $d$, il suffit de prendre $f_{n\in\N}=d(x_n, \cdot)$ avec $x_{n\in\N}$ une famille dénombrable dense.
	Cette famille satisfait aux hypothèses.
	Un espace métrique séparable est homéomorphe à un sous-espace de $[0, 1]^\N$.
\end{remarque}

\begin{proposition}
	Tout espace est $T_0+T_{3\frac{1}{2}}$ si et seulement s'il est homéomorphe à un sous-espace d'un cube $[0, 1]^I$.
\end{proposition}

\begin{demonstration}
	\begin{itemise}
		\item[$\Rightarrow$] Soit $E$ un espace topologique $T_0$ et $T_{3\frac{1}{2}}$ et $f_{i\in I}:E\rightarrow [0, 1]$ la famille des applications continues de $E$.
		L'application $e:E\rightarrow \prod_{i\in I}Q$ définie par $e(x)_i=f_i(x)$ est continue d'après la proposition précédente.
		Soient $x, y\in E$ différents.
		Par $T_0$ quitte à échanger les rôles de $x$ et $y$, $\exists U\in\scr{T}, x\in U\not\ni y$.
		Par $T_{3\frac{1}{2}}$ il existe une fonction continue $f$ telle que $f(x)=1$ et $f(\complement U)=\ens{0\}$ c'est-à-dire séparant $x$ et $y$.
		Soit $x, F\in E\times\complement\scr{T}$ tels que $x\notin F$.
		De même il existe une fonction continue et séparant $x$ et $F$.
		Ainsi $f_{i\in I}$ sépare les points et les points des fermés.
		D'après la proposition précédente c'est une application ouverte et injective.
		D'après {\color{red} ???} c'est un homéomorphisme.
		\item[$\Leftarrow$] Montrons d'abord que $\forall I\neq\vide, [0, 1]^I$ est $T_0+T_{3\frac{1}{2}}$.
		$[0, 1]$ est séparé car munie de sa topologie usuelle (métrique de $}{\cdot|$).
		Par produit de séparés, $[0, 1]^I$ est séparé.
		Soit $x, F\in[0, 1]^I\times\complement\scr{T}$ tels que $x\notin F$.
		$O=\complement F$ est un ouvert de $[0, 1]^I$ contenant $x$.
		Par définition de la topologie produit et de la topologie usuelle de $[0, 1]$, il existe un ouvert $V$ tels que $x\in V\subset U$ et $V=\prod_{k=1}^n]x_{i_k}-\varepsilon, x_{i_k}+\varepsilon[\times\prod_{i\in I\setminus\{i_1, ..., i_n}}[0, 1]$.
					Soit $f$ une fonction chapeau en $x_{i_1}$ nulle hors $]x_{i_1}-\varepsilon, x_{i_1}+\varepsilon[$.
		Alors $f\circ\pi_{i_1}$ montre que l'espace est $T_{3\frac{1}{2}}$.

		Soit maintenant un sous-espace $X$ de $[0, 1]^I$.
		Toute partie d'un espace séparé est elle-même séparé (on dit que $T_2$ est une propriété héréditaire) donc $X$ est $T_0+T_{3\frac{1}{2}}$.
	\end{itemise}
\end{demonstration}

\begin{remarque}
	D'après le théorème de Tychonoff, un produit d'espaces quasi-compacts est quasi-compact.
	En outre le produit d'espaces séparés est séparé.
	Ainsi si $E$ est $T_0+T_{3\frac{1}{2}}$ l'adhérence de $e(E)$ dans $[0, 1]^I$ est un compact dans lequel $e(E)$ est dense: c'est la compactification de Stone-Čech de $E$.
\end{remarque}


\section{Les pièges dans les espaces non séparés}

\textbf{Se méfier des sous-espaces fermés :}
Quand $E$ n'est pas séparé ($T_2$) certains espaces d'habitude fermés ne le sont plus, par exemple la diagonale $\Delta_E$.
Par ailleurs, certaines propriétés, vraies pour les espaces fermés et séparé ne la sont plus: dans un espace $T_2+T_4$, un sous-espace fermé est $T_2+T_4$.
Mais un sous-espace fermé d'un $T_4$ n'est pas nécessairement $T_4$...

La parade consiste à remplacer les sous-espace fermés par des rétracts: un sous-espace $X$ du topologique $E$ est un rétract s'il existe une application continue $f:E\rightarrow X$ telle que $f_{\vert X}=\id_{\vert X}$.
On dit alors que $f$ est une rétraction $E\rightarrow X$.

\begin{exemple}
	Si un produit $\prod_{i\in I}E_i$ d'espaces topologiques est $T_2+T_4$ alors chaque $E_i$ est $T_2+T_4$.
	Pour le démontrer on remarque que chaque $E_i$ est homéomorphe à un sous espace fermé du produit.
	Si les espaces ne sont pas séparés cette démonstration ne fonctionne plus.
	On peut néanmoins utiliser les rétracts: par les projections $\pi_i:\prod_{i\in I}E_i\rightarrow E_i$ chaque $E_i$ est un rétract du produit.
	S'il est $T_4$ il en va de même pour chaque $E_i$.
\end{exemple}


\begin{exemple}
	Si $E$ n'est pas séparé la diagonale $\Delta_E$ n'est pas fermée.
	Mais le graphe d'une application continue $f\colon E\rightarrow F$ est un rétract de $E\times F$ grâce à la rétraction $E\times F\rightarrow\Gamma(f); (x, y)\mapsto(x, f(x))$.
	Si $E$ et $F$ sont séparé alors $\Gamma(f)$ est fermé ce qui montre $\Delta_E$ fermé avec $f=\id$.
\end{exemple}

\textbf{Se méfier des applications continues :}
Si $E$ et $F$ sont séparés $f\in\cal{C}(E, F)$ a un graphe fermé.
Si les espaces ne sont pas séparés c'est à priori faux.
Comme c'est bien souvent le graphe fermé qui est important, un remède est de considérer les applications dont le graphe est fermé plutôt que les applications continues.

\begin{exemple}
	Si $E$ est compact, $F$ séparé et $f\in\cal{C}(E, F)$ alors $f$ est fermée.
	Si $F$ n'est pas séparé, c'est faux.
	Cependant, si $E$ est compact et que $f:E\rightarrow F$ a un graphe fermé alors $f$ est fermée.
\end{exemple}

\begin{exemple}
	Si $E$ et $F$ sont des espaces vectoriels topologiques sur $\R$ et si $f$ est additive et a graphe fermé alors $f$ est linéaire.
	En effet $f(\frac{n}{d}x)=nf(\frac{x}{d})=\frac{n}{d}df(\frac{x}{d})=\frac{n}{d}f(x)$.
	Ainsi $\forall q\in\Q, f(qx)=qf(x)$.
	Si l'application était continue on conclurait directement.
	Sinon pour $\lambda\in\R$ et $r\in\Q$ on fait tendre $r\rightarrow\lambda$ ainsi $rx\rightarrow\lambda x$ et $f(rx)=rf(x)\rightarrow\lambda f(x)$ par continuité de la multiplication externe.
	Or $(rx, rf(x))\in\Gamma(f)$ fermé on a $(\lambda x, \lambda f(x))\in\Gamma(f)$ id est $f(\lambda x)=\lambda f(x)$.
\end{exemple}

\textbf{Se méfier des espaces localement compacts :}
« Tout point de $E$ appartient à un ouvert dont la fermeture est compacte » est équivalent dans un espace séparé à « tout point de $E$ a un voisinage compact » c'est-à-dire:
$$
	\forall x\in E, \exists O\in\scr{T}, x\in O\et\overline{O}\text{ compact}
	\iff
	\forall x\in E, \exists V\in\frak{V}(x), V\text{ compact}
$$
Si l'on essaye de généraliser cette définition au quasi-compact on perd l'équivalence, ce qui donne deux définitions de la local-quasi-compacité.
On a $\Rightarrow$ car $\overline{O}$ est un voisinage de $x$ compact.
Montrons que la réciproque peut être fausse.
Pour $E=\N$ numie de la topologie $\{\vide, \{0\}, \{0, 1\}, \{0, 1, 2\}, ...\}$ on obtient bien un espace localement quasi-compact au second sens (pour $n\in\N$ il suffit de prendre $\intervalle{0, n}$).
Il ne l'est pas au premier sens car avec cette topologie, les fermés ne sont pas bornés (donc impossibles à recouvrir avec des ouverts qui sont bornés dans cette topologie).
Cette espace est de plus $T_0$ mais pas $T_1$ (mais tout de même $T_4$...).

Le remède aux espaces localement-quasi-compacts est simple: préciser clairement la définition utilisée.

\textbf{Se méfier les espaces quasi-compacts :}
Beaucoup de propriétés vraies dans les espaces compacts deviennent fausses dans les espaces quasi-compacts.
La litérature anglo-saxonne utilise \textit{compact} pour le français \textit{quasi-compact} et \textit{compact-Hausdorff} pour \textit{compact}.

\chapter{Filtres}

On le verra plus tard, mais dans un espace métrique, les suites caractérisent les fermés, donc les ouverts, donc la topologie de l’espace.
Toute propriété topologique peut donc s’écrire en termes de suites convergentes.
Dans un espace topologique général (non métrisable), ce n’est plus le cas.

Soit $\scr{T}_1$ la topologie codénombrable de $\R$ et $(u_n)$ une suite convergente pour $\scr{T}_1$ de limite $x$.
On note $U=\ens{u_n}{n\in\N}$ et $V=\{x\}\cup\complement U$.
Puisque $\complement V\subset U$ et que $U$ est dénombrable, $V$ est un ouvert.
De plus $x\in V$ donc $V$ est un voisinage de $x$.
En utilisant le caractérisation séquentielle $\exists n_0\in\N, n\geqslant n_0\Rightarrow u_n\in V$.
Or $\forall n, u_n\in U$, donc $\forall n\geqslant n_0, u_n\in U\cap V=\{x\}$.
Ainsi $u_n$ est stationnaire à partir d'un certain rang.

Munissons maintenant $\R$ de la topologie discrète $\scr{T}_2$
Pour cette topologie, $\{x\}$ est un voisinage de $x$.
De même la caractérisation séquentielle donne qu'une suite convergente vers $x$ sera constante à partir d'un certain rang.

On a donc un problème: $\scr{T}_1$ et $\scr{T}_2$ ont les mêmes suites convergentes sans êtres les mêmes topologies.
Cela implique que certaines propriétés topologiques ne peuvent plus s’exprimer en termes de suites.
Par exemple, l’injection canonique $i : (\R, \scr{T}_1)\rightarrow (\R, \scr{T}_2)$ n’est pas continue car pour $\{x\}$ ouvert de $\scr{T}_2$ mais pas de $\scr{T}_1$, on a $\inv{i}(\{x\})=\{x\}$.
Pourtant, pour toute suite $(u_n)$ convergente vers $x$ pour $\scr{T}_1$, $(u_n)$ est donc stationnaire en $x$ à partir d'un certain rang, ce qui impose que $i(u_n)$ soit aussi stationnaire, et donc converge aussi vers $x$ dans $\scr{T}_2$.
Sans distance sur $E$, il faut traiter la convergence avec un outil plus puissant que les suites, et ça sera les filtres.

\section{Définitions}

Les espaces $E, F$ sont munis des topologies $\scr{T}, \scr{U}$ des voisinages $\frak{V}, \frak{W}$.
On se donne une application $f:E\rightarrow F$.
{\color{red} Ajouter la version de John Terilla !}

\begin{definition}
	On appelle filtre sur $E$ une collection $\frak{F}\subset\frak{P}(E)$ telle que :
	\begin{itemise}
		\item Toute partie de $E$ incluant un élément de $\frak{F}$ appartient à $\frak{F}$
		\item Toute intersection finie d'éléments de $\frak{F}$ appartient à $\frak{F}$
		\item L'ensemble vide n'appartient pas à $\frak{F}$
	\end{itemise}
\end{definition}

\begin{exemple}
	Soit $(E, \frak{V})$ un espace topologique, $\forall x\in E, \frak{V}(x)$ est un filtre sur $E$.
	$\vide\neq V\in\frak{V}(x)$ car $x\in V$.
	Le reste est vrai par définition.
\end{exemple}

\section{Base d'un filtre}

\begin{proposition}
	L'intersection d'une famille de filtre est un filtre.
\end{proposition}

\begin{demonstration}
	Soit $(\frak{F}_i)_{i\in I}$ une famille de filtres et $\frak{F}=\bigcap_{i\in I}\frak{F}_i$.
	\begin{itemise}
		\item Soit $X\in\frak{F}$ et $X\subset Y$ alors pour chaque $i\in I$ on a $X\in\frak{F}_i$ donc $Y\in\frak{F}_i$ puis $Y\in\frak{F}$.
		\item Soit $X, Y\in\frak{F}$ alors pour chaque $i\in I$ on a $X, Y\in\frak{F}_i$ donc $X\cap Y\in\frak{F}_i$ puis $X\cap Y\in\frak{F}$.
		\item L'ensemble vide n'appartient à aucun des $\frak{F}_i$ donc pas à l'intersection.
	\end{itemise}
\end{demonstration}

\begin{definition}
	Pour toute partie $\frak{A}$ non vide et ne contenant pas l'ensemble vide on appelle le filtre engendré par $\frak{A}$ l'intersection des filtres de $E$ contenant $\frak{A}$.
	Il est noté $\sigma_\rm{filtre}(\frak{A})$ et on nomme alors $\frak{A}$ une pré-base de ce filtre.
\end{definition}

Comme pour la notion de base d'une topologie on peut en donner une approche plus constructive.
Il faut cependant remarquer que contrairement aux topologies, certains axiomes commencent à restreindre la définition de pré-base.

\begin{definition}
	Une partie $\frak{B}$ d'un filtre $\frak{F}$ est appelée une base si tout élément de $\frak{F}$ inclus un élément de $\frak{B}$.
\end{definition}

\begin{proposition}
	Il existe un filtre dont $\frak{B}$ est une base si et seulement si $\frak{B}$ est non vide, ne contient pas l'ensemble vide et l'intersection de deux éléments de $\frak{B}$ inclut un élément de $\frak{B}$.
	Ce filtre est alors unique et ses éléments sont les ensembles incluant un élément de $\frak{B}$.
\end{proposition}

\equivalence{Par double implication.}{
	Supposons que $\frak{B}$ soit une base d'un filtre $\frak{F}$.
	Si $\frak{B}$ était vide (ou si $\vide\in\frak{B}$) alors $\frak{F}$ serait vide (ou $\vide\in\frak{F}$) ce qui est impossible.
	Soit $X, Y\in\frak{B}$ alors $X, Y\in\frak{F}$ donc $X\cap Y\in\frak{F}$ par définition des filtres.
	Puisque $\frak{B}$ est une base il exite $Z\in\frak{B}$ tel que $Z\subset X\cap Y$.
	Par définition tout élément de $\frak{F}$ inclut un élément de $\frak{B}$ et réciproquement tout ensemble incluant un élément de $\frak{B}$ inclut donc un élément de $\frak{F}$ dont appartient au filtre.
	Ainsi on a nécessairement $\frak{F}=\ens{U\in\frak{P}(E)}{\exists A\in\frak{B}, A\subset U}$.
}{
	Soit $\frak{B}$ vérifiant les propriétés et montrons que le $\frak{F}$ trouvé est un filtre.
	\begin{itemise}
		\item Soit $U\in\frak{F}$ et $U\subset V$ alors $\exists X\in\frak{B}, X\subset U\subset V$ donc $V\in\frak{F}$.
		\item Soit $U, V\in\frak{F}$ alors $\exists X, Y, Z\in\frak{B}, X\subset U, Y\subset V, Z\subset X\cap Y$ donc $Z\subset U\cap V\in\frak{F}$
		\item Pour que $\vide\in\frak{F}$ il faudrait $\vide\in\frak{B}$ ce qui est faux.
	\end{itemise}
}

\begin{definition}
	Une partie non vide $\frak{A}$ d'un filtre $\frak{F}$ est appelée une sous-base si l'ensemble des intersections finies de $\frak{A}$ est une base de $\frak{F}$.
\end{definition}

\section{Convergence}

\begin{definition}
	Soit $E$ et $F$ deux ensembles, $f$ une fonction de $E$ dans $F$ et $\frak{F}$ un filtre sur $E$.
	Le filtre image de $\frak{F}$ par $f$ est par définition l'ensemble des parties de $F$ dont l'image réciproque par $f$ appartient au filtre $\frak{F}$.
	Une base de ce filtre est l'ensemble $f(\frak{F})$ des images directes des éléments de $\frak{F}$.
\end{definition}

\begin{demonstration}
	Montrons que $\frak{G}=\ens{A}{\inv{f}(A)\in\frak{F}}$ est un filtre.
	\begin{itemise}
		\item Soit $A\in\frak{G}$ et $A\subset B$ alors $\frak{F}\ni \inv{f}(A)\subset \inv{f}(B)$ donc $B\in\frak{G}$.
		\item Soit $A, B\in\frak{G}$ alors $\inv{f}(A\cap B)=\inv{f}(A)\cap \inv{f}(B)\in\frak{F}$ donc $A\cap B\in\frak{G}$.
		\item Puisque $\inv{f}(\vide)=\vide\notin\frak{F}$ donc $\vide\notin\frak{G}$.
	\end{itemise}
	Montrons que $f(\frak{F})$ est une base de filtre.
	Cet ensemble est non vide, ne contient pas l'ensemble vide.
	Soit $A, B\in f(\frak{F})$ alors il existe $U, V\in\frak{F}$ tels que $A=f(U)$ et $B=f(V)$.
	Par propriété de l'image directe $A\cap B=f(U)\cap f(V)\supset f(U\cap V)\in f(\frak{F})$.
\end{demonstration}

\begin{definition}
	Soit $(E, \frak{V})$ un espace topologique.
	On dit que le filtre $\frak{F}$ converge vers $x\in E$ si et seulement s'il est plus fin que $\frak{V}(x)$.
	On notera $\frak{F}\rightarrow x$.
\end{definition}

\begin{proposition}
	Un espace topologique est $T_2$ si et seulement si tout filtre a au plus une limite.
\end{proposition}

\equivalence{Par double implication.}{
	Par l'absurde, soit $\frak{F}$ un filtre convergent vers $x$ et $y$ distincts.
		Par définition de la convergence d'un filtre, $\frak{V}(x)\leqslant\frak{F}$ et $\frak{V}(y)\leqslant\frak{F}$.
		Soient $X$ et $Y$ des voisinages disjoints de $x$ et $y$ respectivement.
		Alors $X\in\frak{V}(x)\subset\frak{F}$ et $Y\in\frak{V}(y)\subset\frak{F}$.
		Par stabilité par intersection finie d'un filtre $X\cap Y=\vide\in\frak{F}$ ce qui est absurde.
}{
	Soient $x\neq y$.
		Supposons $\forall V, W\in\frak{V}(x)\times\frak{V}(y), V\cap W\neq\vide$.
		Alors $V\cap W$ forment une base de filtre dont le filtre associé converge vers $x$ et $y$, ce qui est absurde.
}

\begin{proposition}
	$x$ adhère à $X$ si et seulement si c'est la limite d'un filtre sur $X$.
\end{proposition}

\equivalence{Par double implication.}{
	La trace $\frak{B}$ des voisinages sur $X$ forme une base de filtre car :
		\begin{itemise}
			\item $\forall V\in\frak{V}(x), V\cap X\supset\{x\}\neq\vide$ donc $\vide\notin\frak{B}$
			\item si $Y, Z\in\frak{B}$ alors $\exists V, W\in\frak{V}(x), Y=X\cap V, Z=X\cap W$ donc $Y\cap Z=X\cap(V\cap W)\in\frak{V}(x)$ par stabilité par intersection des voinsinages.
		\end{itemise}
		Soit $\frak{F}$ le filtre engendré par $\frak{B}$
		Soit $V\in\frak{V}(x)$, alors $V\cap X\in\frak{B}$ par construction, donc appartient au filtre sur $X$ engendré par $\frak{B}$ de par la caractérisation.
		On a donc un exemple de filtre qui convient.
}{
	Soit $V\in\frak{V}(x)$.
		Par convergence vers $x$, on a que $V\in\frak{F}$.
		Par croissance des filtres, puisque $V\subset X\in\frak{P}(X)$, alors $X\in\frak{F}$.
		Par stabilité par intersection des filtres, $V\cap X\in\frak{F}$, donc n'est pas vide.
		Ainsi $x\in\overline{X}$.
}

\begin{proposition}
	$f\text{ continue en }x\iff\forall\frak{F}\text{ filtre},\quad\frak{F}\rightarrow x\Rightarrow\sigma_\rm{filtre}(f(\frak{F}))\rightarrow f(x)$
\end{proposition}

\equivalence{Par double implication.}{
	Soit $\frak{F}$ convergent en $x$ et $W\in\frak{W}(f(x))$.
		Par continuité $\exists V\in\frak{V}(x), f(V)\subset W$.
		Par convergence $V\in\frak{F}$.
		Ainsi $\exists W'\in f(\frak{F}), W'\subset W$, c'est-à-dire $W\in\sigma_\rm{filtre}(f(\frak{F}))$.
		Donc $\sigma_\rm{filtre}(f(\frak{F}))$ est plus fin que $\frak{W}(f(x))$.
}{
	Soit $W\in\frak{W}(f(x))$.
		Prenons $\frak{F}=\frak{V}(x)$ alors $\frak{F}\rightarrow x$ donc $\sigma_\rm{filtre}(f(\frak{F}))\rightarrow f(x)$ id est $\sigma_\rm{filtre}(f(\frak{F}))\supset\frak{W}(f(x))$.
		Ainsi $W\in\sigma_\rm{filtre}(f(\frak{F}))$.
		Or $f(\frak{F})$ est une base de ce filtre donc $\exists V\in\frak{F}=\frak{V}(x), f(V)\subset W$.
}

\section{Ultrafiltres}

\begin{definition}
	Soit $E$ un espace topologique.
	On dit qu'un filtre $\frak{F}$ est plus fin qu'un filtre $\frak{G}$ si et seulement si $\frak{F}\supset\frak{G}$.
\end{definition}

\begin{definition}
	On dit qu'un filtre $\frak{U}$ sur est un ultrafiltre s'il n'existe pas de filtre strictement plus fin (c'est-à-dire que pour tout filtre $\frak{F}$ incluant $\frak{U}$ il s'avère que $\frak{F}=\frak{U}$).
\end{definition}

\begin{proposition}
	$\frak{U}$ est un ultrafiltre si et seulement si $\forall X\subset E, (X\in\frak{U})\ou(\complement X\in\frak{U})$.
\end{proposition}

\equivalence{Par double implication.}{
	Soit $\frak{U}$ un ultrafiltre et $X\in\frak{P}(E)$ tel que $X\notin\frak{U}$.
		Soit $V\in\frak{U}$.
		Par l'absurde, si $V\subset X$ alors par croissance des filtres, $X\in\frak{U}$.
		Donc $V\not\subset X$, c'est-à-dire que $V\cap\complement X\neq\vide$.
		$\frak{B}=\ens{V\cap\complement X}{V\in\frak{U}}$ est une base de filtre car $\frak{U}$ est un filtre.
		Alors le filtre $\sigma_\rm{filtre}(\frak{B})$ est plus fin que $\frak{U}$.
		En effet soit $V\in\frak{U}$ puisque $V\cap\complement X\subset V$ avec $V\cap\complement X\in\frak{B}$, alors $V\in\sigma_\rm{filtre}(\frak{B})$.
		Puisque $\frak{U}$ est un ultrafiltre, $\frak{U}=\sigma_\rm{filtre}(\frak{B})$.
		Or $\complement X=E\cap\complement X\in\frak{B}$ puisque $E\in\frak{U}$.
		Donc $\complement X\in\sigma_\rm{filtre}(\frak{B})=\frak{U}$.
}{
	Par l'absurde, supposons que $\frak{U}$ n'est pas un ultrafiltre.
		Il existe alors un filtre $\frak{F}$ strictement plus fin, et donc une partie $X$ de $E$ dans $\frak{F}$ qui n'est pas dans $\frak{U}$.
		Puisque $X\in\frak{U}\ou\complement X\in\frak{U}$ est vrai par hypothèse, on en déduit que $\complement X\in\frak{U}$.
		Puisque $\frak{F}$ est plus fin que $\frak{U}$, alors $\complement X\in\frak{F}$.
		Par stabilité par intersection de $\frak{F}$, on a $\vide=X\cap\complement X\in\frak{F}$ ce qui est absurde car $\frak{F}$ est un filtre.
}

\begin{proposition}
	Soit $f:E\rightarrow F$ une application et $\frak{B}$ une base d'un ultrafiltre filtre de $E$.
	Alors $f(\frak{B})$ est une base d'ultrafiltre sur $F$.
\end{proposition}

\begin{demonstration}
	$f(\frak{B})$ est une base de filtre car $\frak{B}$ est une base de filtre.
	On peut donc poser $\frak{U}=\sigma_\rm{filtre}(\frak{B})$ et $\frak{V}=\sigma_\rm{filtre}(f(\frak{B}))$.
	Soit $Y\subset F$.
	Puisque $\inv{f}(Y)$ et $\inv{f}(\complement Y)$ sont complémentaires dans $E$, d'après le caractérisation précédente, l'un des deux appartient à l'ultrafiltre engendré par $\frak{B}$.
	Si c'est $\inv{f}(Y)$ alors $Y=f(\inv{f}(Y))$ appartient au filtre engendré par $f(\frak{B})$.
	Si c'est $\inv{f}(\complement Y)$ alors $\complement Y=f(\inv{f}(\complement Y))$ appartient au filtre engendré par $f(\frak{B})$.
	Ainsi $\forall Y\subset E, Y\in\frak{V}\ou\complement V\in\frak{V}$.
\end{demonstration}

\begin{proposition}
	Tout filtre est contenu dans un ultrafiltre.
\end{proposition}

\begin{demonstration}
	Soit $F$ un filtre sur $E$ et $X$ l'ensemble des filtres sur $E$ plus fins que $\frak{F}$.
	On va appliquer le lemme de Zorn à $X$ pour montrer d'existence d'un ultrafiltre contenant $\frak{F}$.
	$X$ n'est pas vide, car il contient au moins les $\frak{F}$.
	Montrons que $X$ est inductif pour la relation d'ordre $\leqslant$.
	Soit $\ens{\frak{F}_i}{i\in I}\subset X$ une partie totalement ordonnée.
	Posons $\frak{G}=\ens{X\subset E}{\exists i\in I, X\in\frak{F}_i}$.
	C'est un filtre sur $E$ car:
	\begin{itemise}
		\item Si $X\in\frak{G}$ alors $X\in\frak{F}_i$ pour un certain $i\in I$.
		Puisque $\frak{F}_i$ est un filtre, $Y\neq\vide$.
		\item Si $X, Y\in\frak{G}$ alors $X\in\frak{F}_i$ et $Y\in\frak{F}_j$ pour des certains $i, j\in I$.
		$X$ étant totalement ordonné, $\frak{F}_i\subset\frak{F}_j$ (ou inversement).
		Mais alors $X, Y\in\frak{F}_j$, donc $X\cap Y\in\frak{F}_j$ puisque c'est un filtre.
		Ainsi $X\cap Y\in\frak{G}$.
		\item Si $X\in\frak{G}$ alors $X\in\frak{F}_i$ pour un certain $i\in I$.
		Si $X\subset Y$ alors $Y\in\frak{F}_i$ puisque c'est un filtre.
		Ainsi $Y\in\frak{G}$.
	\end{itemise}
	C'est de plus un majorant de $X$ car si $x\in\frak{F}_i$ alors $x\in\frak{G}$, ce qui implique $\forall i\in I, \frak{F}_i\subset\frak{G}$.
	D'après le lemme de Zorn, il existe un élément maximal $\frak{U}$ à $X$.
	$\frak{U}$ est un ultrafiltre car s'il admettait un majorant, alors ce dernier serait plus fin que $\frak{F}$, et serait donc dans $X$, impliquant qu'il soit égale à $\frak{U}$.
	$\frak{U}$ est plus fin que $\frak{F}$, car c'est un élément maximal de $X$, donc appartient à $X$, qui est l'ensemble des filtres plus fins que $\frak{F}$.
\end{demonstration}

\section{Suites}

\begin{definition}
	On appelle suite sur $E$ une application $x:\N\rightarrow E\in E^\N$.
\end{definition}

\begin{proposition}
	Les suites sur un espace métrique sont un cas particulier de filtres.
\end{proposition}

\begin{demonstration}
	Soit $(x_n)$ une suite de $E$.
	Posons $X_n=\ens{x_p}{p\geqslant n}$ et vérifions que $\frak{B}_x=\ens{X_p}{p\in\N}$ est une base de filtre.
	\begin{itemise}
		\item une suite étant une application $X_p$ n'est jamais vide.
		\item Soit $X_m, X_n\in\frak{B}$ alors $X_m\cap X_n=X_{\max(m, n)}\in\frak{B}$.
	\end{itemise}
\end{demonstration}

\begin{proposition}
	Soit $(x_n)$ une suite et $(x_{\phi(n)})$ une suite extraite.
	Le filtre associé à $(x_{\phi(n)})$ est plus fin que celui de $(x_n)$.
\end{proposition}

\begin{demonstration}
	Soit $X_n$ un élément de la base de filtre $\frak{B}_x$.
	Par croissance de $\phi$, $X_{\phi(n)}\subset X_n$, avec $X_{\phi(n)}\in\frak{B}_{x\circ\phi}$.
	Ainsi $\frak{B}_x\subset\sigma_\rm{filtre}(\frak{B}_{x\circ\phi})$, puis $\sigma_\rm{filtre}(\frak{B}_x)\subset\sigma_\rm{filtre}(\frak{B}_{x\circ\phi})$.
\end{demonstration}

\begin{definition}
	Soit $E$ un espace topologique.
	Une suite $(x_n)$ converge vers $x$ si et seulement si son filtre associé converge vers $x$.
\end{definition}

\begin{proposition}
	Soit $E$ un espace topologique.
	$$
	x_n\rightarrow x
	\iff
	\forall V\in\frak{V}(x), \exists N, \forall n\geqslant N, x_n\in V
	$$
\end{proposition}

\equivalence{Par double implication.}{
	Soit $V\in\frak{V}(x)$.
	Par convergence de $(x_n)$ vers $x$, $V$ appartient au filtre associé à $(x_n)$.
	Par caractérisation du filtre engendré, $\exists X\in\frak{B}_x, X\subset V$.
	Par définition de la base de filtre associée à la suite, $\exists N, X_N=X\subset V$, c'est-à-dire $\exists N, \forall n\geqslant N\implies x_n\in V$.
}{
	Soit $V\in\frak{V}(x)$.
	Par hypothèse, $\exists N, X_N\in\frak{B}_x, X_N\subset V$.
	Par caractérisation du filtre engendré $V\in\sigma_\rm{filtre}(\frak{B}_x)$, c'est-à-dire que $(x_n)$ converge vers $x$.	
}

\begin{definition}
	Soit $X$ une partie d'un espace topologique $E$.
	La fermeture séquentielle de $X$ est $\overline{X}^\rm{seq}=\ens{x}{\exists(x_n)\in X^\N, x_n\rightarrow x}$.
\end{definition}

\begin{proposition}
	Soit $X$ une partie d'un espace topologique $E$.
	Alors $\overline{X}\supset\overline{X}^\rm{seq}$.
\end{proposition}

\begin{demonstration}
	On a démontré qu'une suite définissait un filtre, et que la suite convergeait si et seulement si ce filtre convergeait.
	Par caractérisation de l'adhérence en termes de filtre, on obtient l'inclusion.
\end{demonstration}

\begin{proposition}
	Soient $E$ et $F$ deux espaces topologiques, $f:E\rightarrow F$.
	$$
		\text{$f$ continue en $x$}
		\implies
		\left[\forall(x_n)\in E^\N, x_n\rightarrow x\implies f(x_n)\rightarrow f(x)\right]
	$$
\end{proposition}

\begin{demonstration}
	On a démontré qu'une suite définissait un filtre, et que la suite convergeait si et seulement si ce filtre convergeait.
	Par caractérisation de la continuité en termes de filtre, on obtient l'implication.
\end{demonstration}

\begin{proposition}
	Sur espace topologique compact de toute suite on peut extraire une sous suite convergente.
\end{proposition}

\begin{demonstration}
	On a démontré qu'une suite définissait un filtre, et que la suite convergeait si et seulement si ce filtre convergeait.
	Par caractérisation de la compacité en termes de filtre, on obtient l'implication.
\end{demonstration}

\chapter{Structures uniformes}

Les espaces métriques forment une classe importante d'espaces topologiques, qui est néanmoins insuffisante pour englober toute l’analyse.
Une généralisation satisfaisante est donnée par les espaces uniformes.
Les groupes topologiques, en particulier les espaces vectoriels topologiques, sont uniformisables.
Il y a toutefois des espaces topologiques non uniformisables.
Une question naturelle pour le topologue est donc de déterminer quelles sont les topologies que l'on peut définir à partir de familles d'entourages de la diagonale.
Pour obtenir une classe d'espaces plus large que la classe des espaces uniformes, il faut affaiblir les exigences que l'on met sur cette famille d'entourages.
L'affaiblissement convenable consiste à omettre l'axiome de symétrie.
On parle alors de quasi-uniformité, et tout espace est quasi-uniformisable : c'est donc cet axiome de symétrie qui caractérise véritablement les espaces uniformisable parmi les espaces topologiques, et le procédé consistant à définir une topologie à partir d'entourages de la diagonale est tout à fait général.
Aussi ce procédé doit-il être ajouté au catalogue des méthodes générales pour définir les topologies: par les ouverts (Sierpinski) par les fermés (Alexandroff) par les voisinages (Hausdorff) par les fermetures (Kuratowski) par les filtres convergents (Bourbaki) et cætera...

\begin{remarque}
	Dans le « complément d'analyse » la notation de $F\circ G$ est inverse $\ens{(x, z)\in X^2}{\exists y\in X, (x, y)\in G, (y, z)\in F}$.
	L'interprétation des éléments de l'uniformité en tant que relation me fait pencher pour la notation de \href{https://projects.lsv.ens-paris-saclay.fr/topology/?page_id=2869}{ce super site}.
\end{remarque}

\begin{figure}
	\centring
	\begin{tikzpicture}[scale=6]
		\draw (0,0) rectangle (1,1);
		\draw[very thick] (0,0) -- (1,1);
		\node at (0.81, 0.91) {$\Delta$};
		% G
		\pgfmathsetmacro{\xf}{0.23}
		\pgfmathsetmacro{\yf}{0.28}
		\pgfmathsetmacro{\wf}{0.2}
		\pgfmathsetmacro{\hf}{0.2}
		\draw[thick] (\xf, \yf) -- (\xf+\hf, \yf+\hf) -- (\xf+\hf+\wf, \yf+\hf-\wf) -- (\xf+\wf, \yf-\wf) node[anchor=south, minimum height=1cm] {$G$} -- cycle;
		% F
		\pgfmathsetmacro{\xg}{0.6}
		\pgfmathsetmacro{\yg}{0.55}
		\pgfmathsetmacro{\hg}{0.15}
		\pgfmathsetmacro{\wg}{0.2}
		\draw[thick] (\xg, \yg) -- (\xg+\hg, \yg+\hg) node[anchor=north, minimum height=0.75cm] {$F$} -- (\xg+\hg+\wg, \yg+\hg) -- (\xg+\wg, \yg) -- cycle;
		% point de F°G
		\pgfmathsetmacro{\x}{0.75}
		\pgfmathsetmacro{\z}{0.25}
		\draw[dashed] (\x, 0) -- (\x, \z); \draw[very thick] (\x, -0.01) node[anchor=north west] {$x$} -- (\x, 0.01);
		\draw[dashed] (0, \z) -- (\x, \z); \draw[very thick] (-0.01, \z) node[anchor=north east] {$z$} -- (0.01, \z);
		\pgfmathsetmacro{\y}{0.58}
		\draw[dashed] (\x, \z) -- (\x, \y) -- (\y, \y) -- (\y, \z);
		\draw[very thick] (\x, -0.01) node[anchor=north west] {$x$} -- (\x, 0.01);
		\draw[very thick] (\y+0.01, \y-0.01) node[anchor=south east] {$(y, y)$} -- (\y-0.01, \y+0.01);
		\draw (\x, \y) circle (0.01);
		\draw (\y, \z) circle (0.01);
		\draw (\x, \z) circle (0.01);
		% bord de F°G
		\draw[thin, dotted] (\xg, \yg) -- (\yg, \yg) -- (\yg, \yf+\hf+\hf-\yg+\xf) -- (\xg, \yf+\hf+\hf-\yg+\xf) -- cycle;
		\draw[thin, dotted] (\yg, \yf+\hf+\hf-\yg+\xf) -- (\yg, \yf+\yg-\xf-\wf-\wf) -- (\xg, \yf+\yg-\xf-\wf-\wf);
		\draw[thin, dotted] (\yg, \yf+\hf+\hf-\yg+\xf) -- (\yg, \yf+\yg-\xf-\wf-\wf) -- (\xg, \yf+\yg-\xf-\wf-\wf);
		\draw[thin, dotted] (\xg+\wg+\hf+\wf-\yg+\xf, \yf+\hf-\wf) -- (\xf+\hf+\wf, \yf+\hf-\wf) -- (\xf+\hf+\wf, \xf+\hf+\wf) -- (\xg+\wg+\hf+\wf-\yg+\xf, \xf+\hf+\wf) -- cycle;
		\draw[very thick] (\xg, \yf+\hf+\hf-\yg+\xf) -- (\xg, \yf+\yg-\xf-\wf-\wf) -- (\xg+\wg, \yf+\yg-\xf-\wf-\wf) node[anchor=north west] {$F\circ G$} -- (\xg+\wg+\hf+\wf-\yg+\xf, \yf+\hf-\wf) -- (\xg+\wg, \yf+\hf+\hf-\yg+\xf) -- cycle;
	\end{tikzpicture}
	\caption{$F\circ G$}
\end{figure}

\section{Définition et motivation}

Une relation binaire $\cal{R}$ sur un ensemble $E$ est une partie de $E^2$.
Pour $(x, y)\in\cal{R}$ on note plus souvent $x\cal{R}y$ (penser à la relation d'égalité par exemple).
On peut définire une notion de composition de relation : $\cal{R}\circ\cal{S}$ est l'ensemble des paires $(x, z)$ tel qu'il existe $y$ vérifiant $x\cal{R}y$ et $y\cal{S}z$.
Notez que $\cal{R}\circ\cal{R}\subset\cal{R}$ si et seulement si $\cal{R}$ est transitive.
De même on montre qu'une relation est réfléxive si et seulement si elle contient la diagonale $\Delta=\ens{(x, x)}{x\in X}$.
\begin{align*}
	\cal{R}\circ\cal{R}\subset\cal{R}
	&\iff \forall x, z\in X, \quad x(\cal{R}\circ\cal{R})z\implies x\cal{R}z \\
	&\iff \forall x, z\in X, \quad (\exists y\in X, \quad x\cal{R}y\et y\cal{R}z) \implies x\cal{R}z \\
	&\iff \forall x, y, z\in X, \quad (x\cal{R}y\et y\cal{R}z) \implies x\cal{R}z \\	
	&\iff \text{$\cal{R}$ est transitive}
\end{align*}

Une quasi-uniformité $\scr{U}$ est un filtre de relation binaires sur $E$ réflexives vérifiant une forme faible de transitivité :
pour toute relation $\cal{S}$ de $\scr{U}$ il existe $\cal{R}$ de $\scr{U}$ telle que $\cal{R}\circ\cal{R}\subset\cal{S}$.
En utilisant une forme équivalente des axiomes d'un filtre on obtient la définition partique suivante.
\begin{definition}
	On appelle quasi-uniformité sur $E$ une collection $\scr{U}\subset\frak{P}(E^2)$ dont ses éléments sont appelés les entourages, telle que :
	\begin{itemise}
		\item $R\in\scr{U}\Rightarrow\Delta_E\subset R$
		\item $R, S\in\scr{U}\Rightarrow R\cap S\in\scr{U}$
		\item $R\in\scr{U}\et R\subset S\Rightarrow S\in\scr{U}$
		\item $R\in\scr{U}\Rightarrow\exists S\in\scr{U}, S\circ S\subset R$
		\item Si $\scr{U}$ est une uniformité: $R\in\scr{U}\Rightarrow\inv{R}\in\scr{U}$
	\end{itemise}
\end{definition}

\section{Base d'une uniformité}

Contrairement aux topologies et aux filtres, la catégorie des quasi-unfiormités n'admet pas de borne supérieure à cause de l'axiome de faible transitivité.
On ne pourra donc pas définir aussi simplement la quasi-uniformité par le limite décroissante d'une suite d'ensembles.
Cependant on peut conserver la construction explicite à partir des bases.

\begin{definition}
	Une partie $\frak{B}$ de la quasi-uniformité $\scr{U}$ est appelée une base si tout élément de $\scr{U}$ continent un élément de $\frak{B}$.
\end{definition}

\begin{proposition}
	Il existe une quasi-uniformité dont $\frak{B}$ est une base si et seulement si $\frak{B}$ est une base de filtre de relations binaires réflexives et faiblements transitives.
	Cette quasi-uniformité est alors unique et ses éléments sont les ensembles incluant un élément de $\frak{B}$.
\end{proposition}

\equivalence{Par double implication.}{
	Supposons que $\frak{B}$ soit une base d'une quasi-uniformité $\scr{U}$.
	Les deux premières conditions sont claires.
	Soit $R\in\frak{B}$ alors $R\in\scr{U}$ donc il existe $S\in\scr{U}$ tel que $S\circ S\subset R$.
	Puisque c'est une base il existe $T\in\frak{B}$ tel que $T\subset S$ et par croissance $T\circ T\subset S\circ S\subset R$.
	Par définition tout élément de $\scr{U}$ inclut un élément de $\frak{B}$ et réciproquement tout ensemble incluant un élément de $\frak{B}$ inclut donc un élément de $\scr{U}$ dont appartient à la quasi-uniformité.
	Ainsi $\scr{U}=\ens{R\in\frak{P}(E^2)}{\exists S\in\frak{B}, S\subset R}$.
}{
	Soit $\frak{B}$ vérifiant les propriétés et montrons que le $\scr{U}$ trouvé est une quasi-uniformité.
	Les axiomes de la diagonale, de l'intersection et de la croissance sont claires.
	Soit $R\in\scr{U}$ alors il existe $T\in\frak{B}$ inclut dans $R$ donc il existe $S\in\frak{B}$ $S\circ S\subset T\subset R$.
}

\begin{remarque}
	Pour une base d'uniformité il suffit de demander en plus à la base que pour toute relation $R$ de celle-ci il existe une relation $S$ telle que $S\subset R$.
	L'ensemble $\scr{U}$ trouvé devient alors une uniformité.
	En effet soit $R\in\scr{U}$ il existe $T$ dans $\frak{B}$ tel que $T\subset R$.
	D'après le dernier axiome il existe $S$ dans la base tel que $S\subset\inv{T}\subset\inv{R}$ ce qui montre que $\inv{R}$ est dans $\scr{U}$.
\end{remarque}

\begin{exemple}
	On appelle espace hémi-métrique un ensemble muni d'une hémi-distance $d\colon E\times E\rightarrow\overline{\R}$.
	C'est une fonction positive, nulle sur la diagonale et qui vérifie l'inégalité triangulaire.
	Généralisons la notion de boule d'un espace métrique avec les relations $\approx_r :=\ens{(x, y)}{d(x, y)<r}$.
	On montre que cette famille est une base engendrant donc une quasi-uniformité $\scr{U}_d$ dont les entourages contiennent au moins une relation $\approx_r$.

	Puisque $d$ est nulle sur la diagonale chaque $\approx_r$ contient $\Delta_E$.
	La famille de ces $\approx_r$ avec $r>0$ est stable par intersection finie.
	En effet l'intersection des deux relations $\approx_r$ et $\approx_s$ n'est autre que $\approx_{\min(r, s)}$ qui est bien dans la famille.
	D'aprs le théorème de caractérisation des bases de quasi-uniformité il ne reste plus qu'à montrer la transitivité faible.

	Soit $\approx_s$ avec $s>0$ une relation de la famille.
	Prenons trois points $x$, $y$ et $z$ vérifiant $x\approx_ry$ et $y\approx_rz$ avec $r=\frac{s}{2}$.
	Ainsi $d(x, y)<r$ et $d(y, z)<r$ donc par inégalité triangulaire $d(x,z)<s$ ce qui montre $x\approx_s z$.
	La transitivité faible est donc un avatar de l'inégalité triangulaire.
\end{exemple}

\section{Topologie engendrée}

Pour toute relation binaire $R$ sur $E$ on note $R[x]:=\ens{y\in E}{xRy}$ l'image de $x$ par $R$.

\begin{definition}
	À partir d'une structure quasi-uniforme $\mathscr{U}$ on définit une topologie dite topologie engendrée par $\mathscr{U}$ en appelant voisinage d'un point $x$ les ensembles de la forme $R[x]$ avec $R\in\mathscr{U}$.
	On vérifie facilement que les ensembles ainsi définis correspondent à une unique topologie (cf Bourbaki TG II.3 proposition 1).
\end{definition}

L'inverse est faux : une topologie engendrée par une structure quasi-uniforme est en général, engendrée par une famille de structures quasi-uniformes.
On verra au chapitre {\colour{red} ???} comment décrire cette famille engendrant une topologie uniformisable.

Nous allons résoudre ici le problème réciproque : étant donné une topologie, existe-t'il une structure quasi-uniforme l'engendrant ?

\begin{proposition}
	Soit $(E, \scr{T})$ un espace topologique.
	La famille suivante est une pré-base de quasi-uniformité appelée la pré-base de Pervin (1962) $\frak{A}_\rm{Pervin}=\ens{R_U:=(U\times U)\cup(\complement U\times E)}{U\in\scr{T}}$.
	Cette pré-base engendre la topologie d'origine $\scr{T}$.
\end{proposition}

\begin{remarque}
	En tant que relation binaire $xR_Uy$ n'est autre que $x\in U\implies y\in U$.
	Cela permet de démontrer facilement $R_U\circ R_U\subset R_U$ c'est-à-dire la transitivité de $R_U$.
	Soit donc $x$, $y$ et $z$ tels que $xR_Uy$ et $yR_Uz$.
	Ainsi $x\in U\implies y\in U$ et $y\in U\implies z\in U$.
	Par transitivité de l'implication on a donc $x\in U\implies z\in U$ soit $xR_Uz$.
\end{remarque}

\begin{remarque}
	% On se servira plusieurs fois dans la suite de la démarche suivante.
	Si $R\in\scr{U}_\rm{Pervin}$ il existe $S$ dans la prébase de Pervin tel que $S\subset R$.
	Par définition de celle-ci il existe donc une famille finie d'ouverts $(U_i)_{1\leqslant i\leqslant n}$ tel que $S=R_{U_1}\cap\dots\cap R_{U_n}=\bigcap_{i=1}^nR_{U_i}$.
	Pour tout $A, B\subset E^2$ on voit sur un dessin que $(A\cap B)[x]=A[x]\cap B[x]$ c'est-à-dire que la tranche en $x$ d'une intersection de patates c'est l'intersection des tranches de patates.
	Par récurrence on a ainsi $S[x]=(\bigcap_{i=1}^nR_{U_i})[x]=\bigcap_{i=1}^nR_{U_i}[x]$.
	Le schéma \ref{fig:pervin} montre que si $x\in U$ alors $R_U[x]=U$ et sinon $R_U[x]=E$.
	En collectionnant dans un ensemble $J$ les indices $j$ tels que $x\in U_j$ on a donc $S[x]=\bigcap_{j\in J}U_j$.
	C'est un ouvert en tant qu'intersection finie d'ouvert et il contient $x$ car construction.
	On a donc montré l'existence d'un ouvert $O$ tel que $x\in O\subset R[x]$ pour tout $x\in E$.
\end{remarque}

\begin{figure}
	\centring
	\begin{tikzpicture}[scale=6]
		\pgfmathsetmacro{\x}{0.23}
		\pgfmathsetmacro{\l}{0.23}
		\fill[black!10!white] (0, 0) rectangle (\x, 1);
		\fill[black!10!white] (\x+\l, 0) rectangle (1, 1);
		\fill[black!10!white] (\x,\x) rectangle (\x+\l, \x+\l);
		\draw[dashed] (\x, 0) -- (\x, 1);
		\draw[dashed] (\x+\l, 0) -- (\x+\l, 1);
		\draw[very thick] (\x, 0) node[anchor=north west] {$U$} -- (\x+\l, 0);
		\draw (0,0) rectangle (1,1);
		\draw[very thick] (0,0) -- (1,1);
		\node at (0.81, 0.91) {$\Delta$};
	\end{tikzpicture}
	\caption{Représentation d'un élément de la prébase de Pervin}
	\label{fig:pervin}
\end{figure}

\begin{demonstration}
	Montrons que les intersections finies $\frak{B}_\rm{Pervin}$ de $\frak{A}_\rm{Pervin}$ est une base d'uniformité.
	\begin{itemise}
		\item Chaque élément de la famille contient la diagonale ce qui passe à l'interserction.
		\item Les intersections de deux éléments de $\frak{B}_\rm{Pervin}$ sont par définition dans $\frak{B}_\rm{Pervin}$.
		\item Soit $S=R_{U_1}\cap\dots\cap R_{U_n}$ une intersection finie de relations de Pervin.
		On cherche $R$ une relation de $\frak{B}_\rm{Pervin}$ telle que $R\circ R\subset S$.
		Montrons que $S$ convient.
		Soit $(x, z)\in S\circ S$ il existe $y$ tel que $(x, y), (y, z)\in S\subset R_i$ pour un $i$ quelconque.
		Ainsi $(x, z)\in R_{U_i}\circ R_{U_i}$ et par transitivité $(x, z)\in R_{U_i}$.
		En passant à l'intersection $(x, z)\in S$.
	\end{itemise}
	Soit $\scr{U}_\rm{Pervin}$ la quasi-uniformité engendrée par $\frak{B}_\rm{Pervin}$ et $\scr{T}_\rm{Pervin}$ la topologie engendrée par $\scr{U}_\rm{Pervin}$.
	Montrons par double inclusion que $\scr{T}=\scr{T}_\rm{Pervin}$.
	\begin{itemise}
		\item[$\subset$] Soit $U\in\scr{T}$ et $x\in U$.
		Puisque $U=R_U[x]$ est un voisinage de $x$ d'après $\scr{U}_\rm{Pervin}$ alors $U$ est un ouvert de $\scr{T}_\rm{Pervin}$.
		\item[$\supset$] Soit $U\in\scr{T}_\rm{Pervin}$ et $x\in U$.
		La topologie engendrée par des voisinages dit que $U$ est un voisinage de $x$.
		Ainsi il existe $R\in\scr{U}$ tel que $U=R[x]$.
		D'après la remarque précédente il existe $O_x$ un ouvert de $\scr{T}$ tel que $x\in O_x\subset R[x]=U$.
		Ainsi $U=\bigcup_{x\in U}O_x$ est un ouvert de $\scr{T}$.
	\end{itemise}
\end{demonstration}

\section{Séparation}

\begin{proposition}
	Un espace topologie $E$ est $T_3$ si et seulement si sa quasi-uniformité de Pervin vérifie
	\begin{align*}
		\forall R\in\scr{U}_\rm{Pervin},\quad \forall x\in E,\quad \exists S\in\scr{U}_\rm{Pervin},\quad S=\inv{S}\et (S\circ S)[x]\subset R[x]
	\end{align*}
\end{proposition}

\equivalence{Par double implication.}{
	Supposons la topologie $T_3$ et prenons $R\in\scr{U}_\rm{Pervin}$ et $x\in E$.
	D'après la remarque précédente il existe $O$ un ouvert tel que $x\in O\subset R[x]$.
	Ainsi $\complement O$ est un fermé ne contenant pas $x$ donc par $T_3$ il existe deux ouverts $U$ et $U'$ tels que $x\in U$, $\complement O\subset U'$ et d'intesection vide.
	Cela montre $x\in U\subset\complement U'\subset O$ donc en particulier $x\in U\subset\overline{U}\subset O$.

	Posons $S_{OU} = O^2\cup(\complement\overline{U})^2$.
	C'est un élément de $\scr{U}_\rm{Pervin}$ car il inclut $R_O\cap R_{\complement\overline{U}}$.
	En effet chaque terme du développement est inclus dans $O^2$ dans $(\complement\overline{U})^2$ ou vides donc toujours dans $S_{OU}$.
	On peut répéter l'argument de séparation à partir de $x\in U$.
	Il existe donc un ouvert $V$ tel que $x\in V\subset\overline{V}\subset U$ que $S_{UV}=U^2\cup(\complement\overline{V})^2$ est un élément de $\scr{U}_\rm{Pervin}$.
	Ainsi $S=S_{OU}\cap S_{UV}$ est symétrique dans l'unformité de Pervin.

	Montrons qu'il est solution du problème.
	Soit $z\in(S\circ S)[x]$ il existe donc $y$ tel que $(x, y), (y, z)\in S=S_{OU}\cap S_{UV}$.
	En particulier $(x, y)\in S_{UV}$ et $(y, z)\in S_{OU}$.
	Puisque $x\in V$ alors $x\notin\complement\overline{V}$ donc $y\in U$ puis $y\notin\complement\overline{U}$ ce qui montre $z\in O$.
	Ainsi $(S\circ S)[x]\subset O\subset R[x]$.
	\begin{figure}
		\centring
		\begin{tikzpicture}[scale=6]
			\draw (0,0) rectangle (1,1);
			\pgfmathsetmacro{\e}{0.01}
			\pgfmathsetmacro{\x}{0.5}
			\draw[very thick] (\x, -0.01) node[anchor=south west] {$x$} -- (\x, +0.01);
			\pgfmathsetmacro{\o}{0.4}
			\pgfmathsetmacro{\u}{0.3}
			\pgfmathsetmacro{\v}{0.2}

			\fill[black!20!white] (0, 0) --
				(\x-\u, 0) -- (\x-\u, \x-\o) -- (\x-\v, \x-\o) -- (\x-\v, \x-\u) --
				(\x+\v, \x-\u) -- (\x+\v, \x-\o) -- (\x+\u, \x-\o) -- (\x+\u, 0) -- (1, 0) --
				(1, \x-\u) -- (\x+\o, \x-\u) -- (\x+\o, \x-\v) -- (\x+\u, \x-\v) --
				(\x+\u, \x+\v) -- (\x+\o, \x+\v) -- (\x+\o, \x+\u) -- (1, \x+\u) -- (1, 1) --
				(\x+\u, 1) -- (\x+\u, \x+\o) -- (\x+\v, \x+\o) --
				(\x+\v, \x+\u) -- (\x-\v, \x+\u) -- (\x-\v, \x+\o) -- (\x-\u, \x+\o) -- (\x-\u, 1) -- 
				(0, 1) -- (0, \x+\u) -- (\x-\o, \x+\u) -- (\x-\o, \x+\v) -- (\x-\u, \x+\v) --
				(\x-\u, \x-\v) -- (\x-\o, \x-\v) -- (\x-\o, \x-\u) -- (0, \x-\u) -- cycle;

			\draw[red, very thick] (\x-\o,\x-\o) rectangle (\x+\o, \x+\o);
			\draw[red, very thick] (\e, \e) rectangle (\x-\u, \x-\u);
			\draw[red, very thick] (\e, \x+\u) rectangle (\x-\u, 1-\e);
			\draw[red, very thick] (\x+\u, \x+\u) rectangle (1-\e, 1-\e);
			\draw[red, very thick] (\x+\u, \e) rectangle (1-\e, \x-\u);

			\draw[blue, very thick] (0, 0) rectangle (\x-\v, \x-\v);
			\draw[blue, very thick] (\x+\v, \x+\v) rectangle (1, 1);
			\draw[blue, very thick] (0, \x+\v) rectangle (\x-\v, 1);
			\draw[blue, very thick] (\x+\v, 0) rectangle (1, \x-\v);
			\draw[blue, very thick] (\x-\u,\x-\u) rectangle (\x+\u, \x+\u);

			\draw[very thick] (\x-\o, -0.02) -- (\x+\o, -0.02) node[anchor=north east] {$O$};
			\draw[very thick] (\x-\u, -0.04) -- (\x+\u, -0.04) node[anchor=north east] {$U$};
			\draw[very thick] (\x-\v, -0.06) -- (\x+\v, -0.06) node[anchor=north east] {$V$};
			
			\draw[very thick] (0,0) -- (1,1);
			\node[anchor=south east] at (0.5, 0.5) {$\Delta$};
		\end{tikzpicture}
		\caption{Représentation de $S_{OU}$ en rouge de $S_{UV}$ en bleu et de l'intersection $S$ en gris}
		\label{fig:pervin-T3}
	\end{figure}
}{
	Supposons que $\scr{U}_\rm{Pervin}$ vérifie la propriété.
	Soit $F$ un fermé et $x\notin F$ alors $\complement F$ est un ouvert contenant $x$.
	Par définition des voisiange par la base de Pervin il existe donc un entourage $R$ tel que $R[x]\subset\complement F$.
	Par hypothèse il existe un entourage $S$ symétrique tel que $(S\circ S)[x]\subset R[x]$.
	Montrons que $\overline{S[x]}\subset\complement F$.
	Soit $y\in\overline{S[x]}$ par définition $S[y]$ est un voisinage de $y$ donc $S[x]\cap S[y]$ est non vide.
	Prenons $z$ dedans alors $(x, z), (y, z)\in S$ donc par symétrie $(x, z), (z, y)\in S$.
	Ainsi $(x, y)\in S\circ S$ c'est-à-dire $y\in (S\circ S)[x]\subset R[x]\subset\complement F$.
	En passant au complémentaire on a donc $F\subset\complement\overline{S[x]}$ et $x\in S[x]$ avec $\complement\overline{S[x]}$ et $S[x]$ deux voisinages d'intersection vide.
	Cela montre $T_3$.
}

\section{Équivalence}

\begin{definition}
	On appelle écart sur $E$ toute application $d:E\times E\rightarrow\overline{\R^+}$ vérifiant:
	\begin{itemise}
		\item $\forall x\in E, d(x, x)=0$
		\item $\forall x, y\in E, d(x, y)=d(y, x)$
		\item $\forall x, y, z\in E, d(x, z)\leqslant d(x, y)+d(y, z)$
	\end{itemise}
	Une jauge sur un ensemble $E$ est une famille $(d_i)_{i\in I}$ d'écarts sur $E$.
\end{definition}

\begin{remarque}
	Une pseudo-distance est un écart à valeurs finies.
	On constate deux différences par rapport à la notion de distance :
	\begin{itemise}
		\item la première est mineure : un écart peut prendre la valeur $+\infty$.
		Mais on peut toujours remplacer $d$ par un écart équivalent (du point de vue de la topologie et de la structure uniforme) à valeurs finies, par exemple $\min(1, d)$.
		\item la seconde est essentielle : un écart ne vérifie pas nécessairement l'axiome de séparation pour les distances, qui est : $d(x, y) = 0\iff x = y$.
	\end{itemise}
\end{remarque}

\begin{proposition}
	Soit $(d_i)_{i\in I}$ une jauge sur $E$.
	La famille $\frak{B}'=\ens{\inv{d_i}([0, \varepsilon[)}{\varepsilon\in\R^+, i\in I}$ est une prébase d'uniformité.
\end{proposition}

{\colour{red} À reprendre}

\begin{demonstration}
	Vérifions les axiomes.
	\begin{itemise}
		\item Si $(x, x)\in\Delta_E$ alors $d_i(x, x)=0$ donc $(x, x)\in\inv{d_i}([0, \varepsilon[)\in\frak{B}'$.
		%\item Par construction Si $(x, y)\in\inv{d_i}([0, \varepsilon[)$ et $(x', y')\in\inv{d_{i'}}([0, \varepsilon'[)$
		\item Si $\inv{d_i}([0, \varepsilon[)\in\frak{B}'$ alors $\inv{d_i}([0, \varepsilon/2[)\circ \inv{d_i}([0, \varepsilon/2[)\subset\inv{d_i}([0, \varepsilon[)$.
		\item Par symétrie des $d_i$ $\inv{d_i}([0, \varepsilon[)=\inv{\inv{d_i}([0, \varepsilon[)}$
	\end{itemise}
\end{demonstration}

\begin{proposition}
	Soit $(d_i)_{i\in I}$ une jauge sur $E$.
	La famille $\frak{B}'=\ens{\inv{d_i}(x, [0, \varepsilon[)}{x\in E, \varepsilon\in\R^+, i\in I}$ est une prébase de topologie.
	C'est exactement celle engendrée par l'uniformité définie au-dessus.
\end{proposition}

\begin{demonstration}
	\colour{red}{À démontrer}
\end{demonstration}

\begin{proposition}
	La topologie engendrée par la jauge $(d_i)_{i\in I}$ est la même que celle engendrée par $\ens{\max(d_{i_1}, ..., d_{i_n})}{i_1, ..., i_n\in I}$ appelée saturation de la jauge.
\end{proposition}

\begin{demonstration}
	\colour{red}{À démontrer}
\end{demonstration}

\begin{proposition}
	Pour un espace topologique on a l'équivalence entre:
	\begin{itemise}
		\item être $T_{3\frac{1}{2}}$
		\item être uniformisable
		\item être engendré par une famille d'écarts
	\end{itemise}
\end{proposition}

\begin{demonstration}
	Par implications circulaires.
	\begin{itemise}
		\item[$ii\implies iii$] Soit $\scr{U}$ une uniformité.
		Montrons que les entourages symétriques forment une base.
		Soit $U\in\scr{U}$ alors $U'=U\cap\inv{U}\in\scr{U}$ par axiome de symétrie et intersection de l'uniformité.
		Ainsi $\frak{B}=\ens{U'}{U\in\scr{U}}$ est une base car pour tout $U$ il existe $U'\in\frak{B}$ tel que $U'\subset U$.
		Construisons une jauge $(d_U)_{U\in\frak{B}}$ puis montrons que sa topologie coïncide. % avec celle de $\scr{U}$.
		\begin{itemise}
			\item Soit $U\in\frak{B}$ et vérifions que la fonction suivante est une pseudo-distance :
			$$
			d_U(x, y)=\inf\left\{\frac{1}{n}\ \Big|\ n\in\N, \exists (a_0, a_1), \dots, (a_{n-1}, a_n)\in U, x=a_0, y=a_n\right\}
			$$
			\begin{itemise}
				\item Puisque $\Delta_E\subset U$ alors $(x, x)\in U$.
				En répetant $n$ fois ce couple on a $d_U(x, x)\leqslant 1/n$ donc en passant à la limite $d_U(x, x)=0$.
				\item Par symétrie de $U$ une chaîne de $x$ à $y$ se retourne en une chaîne de $y$ à $x$ ce qui montre $d_U(x, y)=d_U(y, x)$.
				\item Soient $x$ $y$ $z$ trois points et $\varepsilon>0$.
				Par définition de la borne inférieure il existe deux chaînes $x=a_0, \dots, a_m=y$ et $y=b_0, \dots, b_n=z$ dans $U$ telles que $\frac{1}{m}<d_U(x, y)+\frac{\varepsilon}{2}$ et $\frac{1}{n}<d_U(y, z)+\frac{\varepsilon}{2}$.
				Puisque la chaîne $a_0, \dots, a_m, b_0, \dots, b_n$ relie $x$ à $z$ en $m+n$ étapes on a :
				$$
				d_U(x, z)
				\leqslant\frac{1}{m+n}
				\leqslant\frac{1}{m}+\frac{1}{n}
				\leqslant d_U(x, y)+\frac{\varepsilon}{2} + d_U(y, z)+\frac{\varepsilon}{2}
				= d_U(x, y) + d_U(y, z) + \varepsilon
				$$
				On a utilisé l'inégalité $\frac{1}{m+n}\leqslant\frac{1}{m}+\frac{1}{n}$ qui s'obtient à partir de $m^2+mn+n^2\geqslant 0$.
				En passant à la limite on a donc montré l'inégalité triangulaire.
			\end{itemise}
			\item On a donc construit une jauge $(d_U)_{U\in\frak{B}}$.
			La base de la topologie issue de cette jauge est formée des boules ouvertes $B_U(x, r)=\ens{y\in E}{d_U(x, y)<r}$, tandis que les voisinages issus de l'uniformité $\scr{U}$ sont $U[x]=\ens{y\in E}{(x, y)\in U}$.
			Montrons que ces deux topologies sont équivalentes par doubles inclusions.
			\begin{itemise}
				\item Soit $U[x]$ un voisinage de l'uniformité.
				Il existe un entourage symétrique $V$ de la base $\frak{B}$ tel que $V\subset U$.
				Soit $y\in V[x]$ alors $y\in B_V(x, 1)$.
				Ainsi $B_V(x, 1)\subset U[x]$.
				\item Définisons par récurrence la suite $(U_n)$ d'entourages.
				Posons $U_0=U$ et d'après le dernier axiome des uniformités pour tout $U_n\in\scr{U}$ il existe $U_{n+1}\in\scr{U}$ tel que $U_{n+1}\circ U_{n+1}\subset U_n$.
				Ainsi $U_1\circ U_1\subset U_0=U$ et $U_2\circ U_2\subset U_1$ donc $U_2^{\circ 4}\subset U$ par associativité de $\circ$.
				Par récurrence on a donc $U_n^{\circ 2^n}\subset U$.
				Ainsi $U_2\circ U_2\subset U_1$ et
				En applicant $n$ fois le
			\end{itemise}
		\end{itemise}
	\end{itemise}
\end{demonstration}
% 		-------------------------------------
% 		Construisons une suite décroissante d'entourages symétriques par $V_0=U\in\scr{U}$ et $V_{n+1}\circ V_{n+1}\circ V_{n+1}\subset V_n$ qui existe car par construction $V_n\in\scr{U}$.
% 		Soient:
% 		\begin{align*}
% 			f_U(x, y) & =
% 			\left\{\begin{array}{lll}
% 				1      & \text{si} & (x, y)\notin V_0               \\
% 				2^{-n} & \text{si} & (x, y)\in V_{n-1}\setminus V_n \\
% 				0      & \text{si} & (x, y)\in\bigcap_{n\in\N} V_n
% 			\end{array}\right.\\
% 			d_U(x, y) & =\inf\ens{\sum_{k=0}^nf_U(a_k, a_{k+1})}{a_0=x, a_k\in E, a_{n+1}=y}
% 		\end{align*}
% 		Vérifions que $d_U$ est un écart.
% 		\begin{itemise}
% 			\item $d_U(x,x)=0$ en prenant $a_0=a_1=x$.
% 			\item Chaque $V_n$ étant symétrique $f_U$ puis $d_U$ l'est.
% 			\item Soient $x, y, z\in E$ et $\varepsilon>0$.
% 			Par définition de l'inf il existe $a_{1\leqslant k\leqslant m}$ et $b_{1\leqslant k\leqslant n}$ tels que $\sum_{k=0}^mf_U(a_k, a_{k+1})\leqslant d_U(x, y)+\frac{\varepsilon}{2}\et \sum_{k=0}^nf_U(b_k, b_{k+1})\leqslant d_U(x, y)+\frac{\varepsilon}{2}$
% 			En concaténant les chemins $a_k$ et $b_k$ on obtient $c_k$ reliant $x$ à $z$ et montrant que $d_U(x, z)\leqslant\sum_{k=0}^mf_U(a_k, a_{k+1})+\sum_{k=0}^nf_U(b_k, b_{k+1})$ donc $d_U(x, z)\leqslant d_U(x, y)+d_U(y, z)+\varepsilon$.
% 			Étant vrai pour tout $\varepsilon$ on a $d_U(x, y)\leqslant d_(x, y)+d_U(y, z)$.
% 		\end{itemise}
% 		Montrons $V_n\subset\ens{(x, y)}{d_U(x, y)<2^{-n}}=\inv{d_U}([0,2^{-n}[)\subset V_{n-1}$.
% 		\begin{itemise}
% 			\item Soit $(x, y)\in V_n$ alors $f_U(x, y)\leqslant 2^{-(n+1)}<2^{-n}$ donc $V_n\subset\inv{d_U}([0,2^{-n}[)$.
% 			\item Montrons par récurrence forte:
% 			\begin{itemise}
% 				\item $\forall q\in\N, \exists a_{1\leqslant k\leqslant q}\text{ de }x\text{ à }y, \sum_{k=0}^qf_U(a_k, a_{k+1})<2^{-n}\Rightarrow (x, y)\in V_{n-1}$
% 				\item Pour $q=0$ si $f_U(x, y)<2^{-n}$ donc $f_U(x, y)\leqslant2^{-n-1}$ soit $(x, y)\in V_{n-1}$.
% 				\item Soit $\alpha=\sum_{k=0}^{q+1}f_U(a_k, a_{k+1})<2^{-n}$.
% 				Il existe deux rangs $p$ et $p'$ tels que $\sum_{k=0}^pf_U(a_k, a_{k+1}), \sum_{k=p+1}^{p'}f_U(a_k, a_{k+1}), \sum_{k=p'+1}^{q+1}f_U(a_k, a_{k+1})<\frac{\alpha}{2}<2^{-n-1}$.
% 				Par hypothèse de récurrence $(a_0, a_{p+1}), (a_{p+1}, a_{p'+1}), (a_{p'+1}, a_{q+1})\in V_n$.
% 				Par construction $(a_0, a_{q+1})=(x, y)\in V_n\circ V_n\circ V_n\subset V_{n-1}$.
% 			\end{itemise}
% 			\item Soit $(x, y)\in\inv{d_U}([0, 2^{-n}[)$.
% 			Alors $\exists a_{1\leqslant k\leqslant q}\text{ de }x\text{ à }y, \sum_{k=0}^qf_U(a_k, a_{k+1})<2^{-n}$.
% 			D'après la récurrence précédente $(x, y)\in V_{n-1}$.
% 		\end{itemise}
% 		Ainsi la famille d'écarts saturés engendre l'uniformité, et donc la même topologie.
% 		\item[écarts$\Rightarrow T_{3\frac{1}{2}}$] Soit $\scr{T}$ une topologie engendrée par la jauge $(d_i)_{i\in I}$, $x, F\in E\times\complement\scr{T}$ avec $x\notin F$.
% 		Alors $x\in\complement F$ est un voisinage de $x$ donc $\exists U\in\scr{U}, \complement F=U[x]$.
% 		Par définition de l'uniformité engendrée par une jauge il existe $\varepsilon\in\R^+$ et $i\in I$ tels que $U=\inv{d_i}([0, \varepsilon[)$ donc $\complement F=\ens{y}{d_i(x, y)<\varepsilon}$.
% 	\end{itemise}
% \end{demonstration}

\section{Continuité uniforme}

Les espaces $E, F$ sont munis des uniformités $\scr{U}, \scr{V}$.
On se donne une application $f:E\rightarrow F$.

\begin{definition}
	$f$ est uniformément continue si et seulement si tout entourage $V\in\scr{V}$ il existe un entourage $U\in\scr{U}$ tel que $\forall (x, y)\in U, (f(x), f(y))\in V$.
\end{definition}

\begin{proposition}
	$f$ est uniformément continue si et seulement si $\forall V\in\scr{V}, \inv{f\times F}(V)\in\scr{U}$.
\end{proposition}

\chapter{Compacité}

La compacité est une propriété particulièrement importante qui confère aux espaces topologiques des qualités proches de celles d’un ensemble fini.
Ceci est particulièrement clair des propositions ci-dessous, qui deviennent évidentes dans le cas d’un espace composé d’un nombre fini de points.

\section{Vocabulaire}

\begin{definition}
	Un espace topologique est quasi-compact si de tout recouvrement par des ouverts, on peut extraire un sous recouvrement fini.
	Par passage au complémentaire, il revient au même de dire que de toute famille de fermés d’intersection vide on peut extraire une sous-famille finie d’intersection vide.
\end{definition}

\begin{definition}
	Un espace topologique est compact s'il est séparé $T_2$ et quasi-compact.
\end{definition}

\begin{definition}
	Un espace topologique est localement compact s'il est séparé et admet des voisinages compacts pour tous ses points.
\end{definition}

\begin{proposition}
	Une partie quasi-compact d'un espace topologique séparé est fermé.
\end{proposition}

\begin{demonstration}
	Soit $K$ un compact et $x, y\in\complement K\times K$.
	Par séparabilité $\exists U_{xy}, V_{xy}\in\scr{T}, x\in U_{xy}, y\in V_{xy}, U_{xy}\cap V_{xy}=\vide$.
	Ainsi $\bigcup_{y\in K}V_{xy}$ est un recouvrement ouvert de $K$ donc par quasi-compacité on peut en extraire un sous recouvrement fini $V_{xy^k}$.
	Ainsi $\bigcap_{k=1}^nU_{xy^k}$ est un ouvert disjoint de $K$ puis $\bigcup_{x\in\complement K}\bigcap_{k=1}^nU_{xy^k}=\complement K$ est un ouvert.
\end{demonstration}

\begin{proposition}
	Un fermé d'un espace topologique quasi-compact est quasi-compact.
\end{proposition}

\begin{demonstration}
	Soit $F$ un fermé de $E$ quasi-compact et $\bigcup_{i\in I}O_i$ un recouvrement de $F$ par des ouverts.
	Puisque $\complement F$ est un ouvert, $\complement F\cup\bigcup_{i\in I}O_i$ est un recouvrement de $E$ par des ouverts.
	Par compacité, on peut en extraire un sous recouvrement fini.
	Ce recouvrement fini couvre $F$, donc $F$ est quasi-compact.
\end{demonstration}

\begin{proposition}
	Un espace topologique est quasi-compact si et seulement si de tout filtre on peut en extraire un filtre plus fin convergent.
\end{proposition}

\equivalence{Par double implication.}{
	Soit $\frak{F}$ un filtre sur $E$ quasi-compact.
	Par l'absurde supposons $\bigcap_{X\in\frak{F}}\overline{X}=\vide$.
	Alors $\bigcup_{X\in\frak{F}}\complement\overline{X}=E$ donc par quasi-compacité $\exists X_{1\leqslant i\leqslant n}\in\frak{F}, \bigcup_{i=1}^n\complement\overline{X_i}=E$.
	Ainsi $\bigcap_{i=1}^n\overline{X_i}=\vide$ en particulier $\bigcap_{i=1}^nX_i=\vide\in\frak{F}$ par stabilité par intersection finie des filtres, ce qui est absurde.
	Soit donc $x\in\bigcap_{X\in\frak{F}}\overline{X}$ et $\frak{B}=\ens{V\cap X}{V\in\frak{V}(x), X\in\frak{F}}$.
	C'est une base de filtre.
	En effet les $V\cap X$ sont toujours non vides car $x$ adhère à tout les $X\in\frak{F}$ donc $\vide\notin\frak{B}$ et les $V\cap X$ forment les voisinages de $x$ donc $\frak{V}(x)\subset\frak{B}$.
	De plus cette famille est stable par intersection car $\frak{V}(x)$ et $\frak{F}$ le sont.
	Ainsi le filtre engendré par $\frak{B}$ converge vers $x$.
}{
	Par contraposée soit $(F_i)_{i\in I}$ une famille de fermés de $E$ dont les intersections finies sont non vides.
	Posons $\frak{B}$ la collection de ces intersections.
	C'est une base de filtre car ses éléments sont non vides et qu'elle est stable par intersection deux à deux.
	Par hypothèse, il existe donc un filtre $\frak{F}$ convergent plus fin que $\sigma_\rm{filtre}(\frak{B})$.
	Soit $x$ sa limite.
	Ainsi $\sigma_\rm{filtre}(\frak{B})$ est un filtre sur $F_i$ (qui est une intersection finie des $F_{i\in I}$) convergent vers $x$ donc $x$ adhère à tout les $F_i$ fermés donc $\forall i\in I, x\in F_i$ soit $\bigcap_{i\in I}F_i\neq\vide$.
}

\begin{proposition}
	Un espace est quasi-compact si et seulement si tout ultrafiltre converge.
\end{proposition}

\equivalence{Par double implication.}{
	Supposons $E$ quasi-compact.
	Soit $\frak{U}$ un ultrafiltre sur $E$.
	Par le théorème précédent on peut extraire un filtre de $\frak{U}$ convergent.
	Puisque $\frak{U}$ est un ultrafiltre, $\frak{U}$ converge.
}{
	Supposons que tout ultrafiltre sur $E$ converge.
	Soit un filtre $\frak{F}$.
	Il est contenu dans un ultrafiltre d'après le lemme de Zorn, qui converge par hypothèse.
}

\begin{proposition}
	Tout produit d’espaces quasi-compacts est quasi-compact.
\end{proposition}

\begin{demonstration}
	Soit $\frak{U}$ un ultrafiltre sur $E=\prod_{i\in I}E_i$ produit d'espaces compacts de projections $\pi_i$.
	Alors $\pi_i(\frak{U})$ est la base d'un ultrafiltre $\frak{U}_i$ sur $E_i$, qui converge d'après le théorème précédent.
	Appelons $x_i\in E_i$ sa limite et posons $x=(x_i)_{i\in I}\in E$.
	Soit $U=\prod_{k=1}^nO_{i_k}\times\prod_{i\in I\setminus\{i_1, ..., i_n\}}E_i$ un ouvert élémentaire de $E$ contenant $x$.
	Comme $\frak{U}_i\rightarrow x_i$ on a $O_{i_k}\in\frak{U}_{i_k}=\sigma_\rm{filtre}(\pi_{i_k}(\frak{U}))$.
	Puisque $\pi_{i_k}(\frak{U})$ est une base $\exists V\in\frak{U}, \pi_{i_k}(V)\subset O_{i_k}$ soit $V\subset\inv{\pi_{i_k}}(O_{i_k})$.
	Par croissance des filtres $\inv{\pi_{i_k}}(O_{i_k})=O_{i_k}\times\prod_{i_k\neq j\in I}E_j\in\frak{U}$.
	Par stabilité des filtres par intersection finie $\bigcap_{k=1}^n\inv{\pi_{i_k}}(O_{i_k})=U\in\frak{U}$.
	L’ultrafiltre $\frak{U}$ est donc convergent vers $x$, ce qui prouve que $E$ est compact.
\end{demonstration}

\section{Application aux fonctions continues}

\begin{proposition}
	L’image continue d’un quasi-compact est quasi-compacte.
\end{proposition}

\begin{demonstration}
	Soit $E$ un quasi-compact et $f:E\rightarrow F$ une application continue.
	Soit $\bigcup_{i\in I}O_i$ un recouvrement ouvert de $f(E)$.
	Ainsi $\bigcup_{i\in I}\inv{f}(O_i)=\inv{f}(\bigcup_{i\in I}O_i)=\inv{f}(f(E))=E$ est un recouvrement ouvert de $E$.
	Par compacité, on peut en extraire un sous recouvrement fini $O_{i_k}$.
	Alors $f(E)=f(\bigcup_{k=1}^n\inv{f}(O_{i_k}))=\bigcup_{k=1}^nf(\inv{f}(O_{i_k}))=\bigcup_{k=1}^nO_{i_k}$ est un recouvrement par des ouverts de l'image qui est donc quasi-compact.
\end{demonstration}

\begin{remarque}
	Voici maintenant une application de la compacité aux problèmes d’optimisation, le résultat qui suit est en effet est à la base des démonstrations qui prouvent l’existence d’un optimum.
	De façon générale, il s’agit de trouver quel jeu de valeurs d’un certain nombre de paramètres rend optimal le résultat d’une opération donnée ; il peut s’agir de son coût, de sa rapidité, de la résistance à la flexion d’une poutre en fonction de sa forme géométrique Ce dernier exemple est d’ailleurs plutôt instructif, puisqu’on peut montrer qu’un tel maximum n’existe pas dans l’ensemble des géométries qui conservent le volume et la longueur de la poutre !
	C’est ce qui explique que les structures légères et résistantes soient du type « nid d’abeilles » et sortent du cadre des formes géométriques régulières.
\end{remarque}

\begin{proposition}
	Une fonction $f:K\rightarrow\R$ (dite fonction numérique) continue sur un compact est bornée et atteint ses bornes.
\end{proposition}

\begin{demonstration}
	D’après la proposition précédente l’image de $f$ est compacte, elle est donc bornée car $\R$ est métrique.
	Ainsi $M=\sup_{x\in K}f(x)$ existe.
	Soit $z_n$ une suite croissante (dite suite maximisante) de limite $M$ et $F_n=\ens{x\in K}{z_n\leqslant f(x)\leqslant M}$.
	Les $F_n$ constituent une suite décroissante de fermés non vides.
	Par compacité leur intersection n’est pas vide.
	Soit $x\in\bigcap_{n\in\N}F_n$ alors $\forall n\in\N, z_n\leqslant f(x)\leqslant M$ donc $f(x)=M$.
	Le maximum est atteint ; il en est de même pour le minimum.
\end{demonstration}

\begin{remarque}
	La compacité est une propriété particulièrement agréable mais rare ; en témoigne la proposition suivante, qui montre qu’une topologie d’espace compact ne peut être affaiblie sans perdre la propriété de séparation, en particulier, au sein d’une famille de topologies séparées comparables sur un ensemble donné, l’une au plus est une topologie d’espace compact.
\end{remarque}

\begin{proposition}
	Si $E$ est quasi-compact pour une topologie il ne peut pas être muni d’une topologie séparé moins fine.
\end{proposition}

\begin{demonstration}
	Soit $\id:(E, \scr{T}_1)\rightarrow(E, \scr{T}_2)$ avec $(E, \scr{T}_1)$ compact $(E, \scr{T}_2)$ séparé et $\scr{T}_2\subset\scr{T}_1$.
	Par comparaison $\id$ est continue.
	Soit $F$ un fermé de $(E, \scr{T}_1)$ quasi-compact alors $F$ est quasi-compact.
	Son image $\id(F)=F$ est donc quasi-compact dans $(E, \scr{T}_2)$ séparé donc $F\in\complement\scr{T}_2$.
	Ainsi $\scr{T}_1\subset\scr{T}_2$.
\end{demonstration}

\part{Et après ?}

\chapter{Groupes topologiques}

\section{Définitions et premières propriétés}

\begin{definition}
	On appelle groupe topologique $(G, \star)$ un groupe muni d'une topologie pour laquelle les applications $\star:G^2\rightarrow G$ et $\inv{}:G\rightarrow G$ sont continues.
\end{definition}

\begin{proposition}
	Un groupe $(G, \star)$ munie d'une topologie est un groupe topologique si et seulement si $x, y\longmapsto x\star\inv{y}$ est continue.
\end{proposition}

\equivalence{Par double implication.}{
	Le sens direct vient de la composition d'applications continues.
}{
	En prenant $x=e$ puis $\inv{y}$ dans $x, y\longmapsto x\star\inv{y}$, on vérifie que c'est un groupe topologique.
}

%\section{Mesure de Haar}

\begin{definition}
	Sur tout groupe topologique localement compact, il existe une et une seule mesure de Borel quasi-régulière non nulle (à coefficient multiplicateur près) invariante par les translations à gauche $x\longmapsto y*x$ : la mesure de Haar.
\end{definition}

\begin{demonstration}
	Cf théorie de la mesure.
\end{demonstration}

\begin{proposition}
	Les translations $x\longmapsto x\star a$ et $x\longmapsto a\star x$ sont des homéomorphismes.
\end{proposition}

\begin{demonstration}
	Par continuité de $\star$ ces deux applications sont continues et leurs réciproques $x\longmapsto x\star\inv{a}$ et $x\longmapsto\inv{a}\star x$ sont continues.
\end{demonstration}

\begin{proposition}
	Dans un groupe topologique, les voisinages d’un point sont les translatés des voisinages de $e$.
\end{proposition}

\begin{demonstration}
	Soit $W$ un voisinage de $x$.
	Comme $y\longmapsto(x, y)$ et $(x, y)\longmapsto x\star y$ sont continues, par composition $\phi:y\longmapsto x\star y$ l'est aussi.
	Or $\phi(e)=x$ donc $V=\inv{\phi}(W)$ est un voisinage de $e$.
	Ainsi $x+V=\phi(V)=\phi(\inv{\phi}(W))=W$ par surjectivité de $\phi$.
\end{demonstration}

\begin{proposition}
	Soit $G$ et $H$ deux groupes topologiques, $f:G\rightarrow H$ un morphisme de groupe.
	$$
		f\in\cal{C}(G, H)\iff f\in\cal{C}(e_G, H)
	$$
\end{proposition}

\equivalence{Par double implication.}{
	Le sens direct est évident.
}{
	Soit $x\in G$ et $V$ un voisinage de $y=f(x)$.
	D'après la proposition précédente il existe un voisinage $W$ de $e_H$ tel que $V=y\star_HW$.
	Puisque $f$ est un morphisme de groupe, $f(e_G)=e_H$.
	Par continuité de $f$ en $e_G$, $\inv{f}(W)$ est un voisinage de $e_G$.
	Par continuité des translations, $x\star_G\inv{f}(W)$ est un voisinage de $x$.
	Or $x\star_G\inv{f}(W)\subset\inv{f}(y\star_HW)$.
	En effet si $x\star_Gz\in x\star_G\inv{f}(W)$ alors $f(x\star_Gz)=f(x)\star_Hf(z)\in y\star_HW$.
	Donc $\inv{f}(V)$ contient un voisinage de $x$.
}

\begin{proposition}
	Un groupe topologique est séparé si et seulement si le singleton ${e}$ est fermé.
\end{proposition}

\equivalence{Par double implication.}{
	Soit $y\in\widetilde{\{x\}}$.
	Alors $\exists\frak{F}\text{ filtre sur $\widetilde{\{x\}}$}, \frak{F}\rightarrow y$.
	Cette convergence a lieu pour la topologie trace c'est-à-dire pour l'ensemble des voisinages $\frak{V}'(y)=\{V\cap\ens{x\}}{V\in\frak{V}(y)}$.
	Par séparation, si $y\neq x$, il existerait un voisinage de $y$ de rencontrant pas $\{x\}$ et donc $\vide\in\frak{V}'(y)\subset\frak{F}$, ce qui est impossible pour un filtre.
	Ainsi les singletons sont fermés, notamment $\{E\}$.
}{
	Réciproquement, si $\{E\}$ est fermé alors $G\setminus\{E\}$ est ouvert.
	Par continuité de $\phi:x, y\longmapsto x\star\inv{y}$, $\inv{\phi}(G\setminus\{E\})=G^2\setminus\ens{(x, x)}{x\in G}$ est un ouvert de $G$.
	Ainsi pour $(x, y)\in G^2$ avec $x\neq y$, il existe un voisinage pour la topologie produit de $(x, y)$ ne rencontrant pas la diagonale.
	Par définition de cette topologie, il existe un voisinage $V_x$ de $x$ et $V_y$ de $y$ tel que $(V_x\times V_y)\cap\ens{(x, x)}{x\in G}=\vide$, c'est-à-dire $V_x\cap V_y=\vide$.
}

\begin{proposition}
	Soit $O$ un ouvert et $X$ une partie d'un groupe topologique.
	Alors $O\star X$ et $O\star X$ sont ouverts.
\end{proposition}

\begin{demonstration}
	$O\star X=\bigcup_{x\in X}O\star x$ est ouvert car $O\star x$ l'est car c'est l'image réciproque de l'ouvert $O$ par l'application continue $y\longmapsto y\star\inv{x}$.
	Idem pour l'autre sens.
\end{demonstration}

\begin{proposition}
	Tout groupe quotient $G/H$ d'un groupe topologique $G$ par un sous-groupe distingué $H$ est encore un groupe topologique, lorsque $G/H$ est muni de la topologie quotient.
	De plus, $G/H$ est séparé si et seulement si $H$ est fermé.
\end{proposition}

\begin{demonstration}
	Soit $OH$ un voisinage de $eH$ dans $G/H$.
	Par définition de la topologie quotient, $\inv{\pi}(OH)=O$ est un voisinage de $e$ dans $G$.
	Par continuité, $\inv{\cdot}(O)$ est un voisinage de $(e, e)$ dans $G\times G$.
	Par définition de la topologie produit, il existe deux voisinages $O_1$ et $O_2$ de $e$ tels que $\inv{\cdot}(O)=O_1\times O_2$.
	Puisque $\inv{\pi}(\pi(O_i))\subset O_i$, $\inv{\pi}(\pi(O_i))$ est un voisinage de $e$ dans $G$.
	Par définition de la topologie quotient, $\pi(O_i)$ est un voisinage de $eH$ dans $G/H$.
	Par définition de la topologie produit, $\pi(O_1)\times\pi(O_2)$ est un voisinage de $(e, e)$ dans $(G/H)^2$.

	De plus $\inv{*}(OH)\supset\pi(O_1)\times\pi(O_2)$.
	En effet, soit $(x_1H, x_2H)\in\pi(O_1)\times\pi(O_2)$, c'est-à-dire $(x_1, x_2)\in O_1\times O_2=\inv{\cdot}(O)$, alors $x_1H*x_2H=(x_1\cdot x_2)H\in OH$.
	Donc $*$ est continue en $(e, e)$ en tant que morphisme de groupe de $(G/H)^2$ dans $G/H$.
	Elle est donc continue.
	\begin{centre}
		\begin{tikzpicture}[xscale=3, yscale=1.5]
			% Noeuds (MODIFIABLES : Styles et Coefficients d'InterFeuilles)
			\node (GG) at (0, 0) {$G\times G$};
			\node (G) at (1, 0) {$G$};
			\node (QQ) at (0, 1) {$G/H\times G/H$};
			\node (Q) at (1, 1) {$G/H$};
			% Arcs (MODIFIABLES : Styles)
			\draw[black, -stealth] (GG)-- node [above, midway] {$\cdot$} (G);
			\draw[black, -stealth] (QQ)-- node [above, midway] {$*$} (Q);
			\draw[black, -stealth] (GG)-- node [left, midway] {$(\pi, \pi)$} (QQ);
			\draw[black, -stealth] (G)-- node [left, midway] {$\pi$} (Q);
		\end{tikzpicture}
	\end{centre}
\end{demonstration}

\chapter{Espace vectoriel topologique}

\part{Analyse}

\chapter{Espaces localement convexes}

\section{Définitions et premières propriétés}

\begin{definition}
	Un $\R$-espace vectoriel topologique $E$ est un $\R$-espace vectoriel dont l'addition $+:E\times E\rightarrow E$ et la multiplication externe $\cdot:\R\times E\rightarrow E$ sont continues.
\end{definition}

\begin{proposition}
	Dans un $\R$-espace vectoriel topologique, les voisinages d’un point sont les translatés des voisinages de l'origine.
\end{proposition}

\begin{demonstration}
	Soit $W$ un voisinage de $x$.
	Comme $y\longmapsto(x, y)$ et $(x, y)\longmapsto x+y$ sont continues, par composition $\phi:y\longmapsto x+y$ l'est aussi.
	Or $\phi(0)=x$ donc $V=\inv{\phi}(W)$ est un voisinage de 0.
	Ainsi $x+V=\phi(V)=\phi(\inv{\phi}(W))=W$ par surjectivité de $\phi$.
\end{demonstration}

\begin{proposition}
	Soit $E$ un $\R$-espace vectoriel topologique.
	\begin{itemise}
		\item Un voisinage $V$ de 0 est absorbant: $\forall x\in E, \exists\Lambda>0, \forall\lambda\in[-\Lambda, \Lambda], \lambda x\in V$.
		\item l'ensemble des voisinages de 0 est invariant par homothétie.
		\item Un voisinage $V$ de 0 contient un voisinage $W$ équilibré: $\forall\lambda\in[-1, 1], \lambda W\subset W$.
		\item Une base de voisinage de 0 est formée de voisinages ouverts équilibrés et absorbants.
		\item tout voisinage $V$ de 0 contient un voisinage $W$ tel que $W+W\subset V$.
	\end{itemise}
\end{proposition}

\begin{demonstration}
	Soit $V$ un voisinage de $0_E$.
	\begin{itemise}
		\item À $x$ fixé, par continuité en $0_E$ de $\lambda\longmapsto\lambda x$, il existe un ouvert $I$ de $0_\R$ vérifiant la propriété.
		$I$ peut être non borné, mais contient un voisinage de la forme $[-\Lambda, \Lambda]$.
		\item À $\lambda\neq0$ fixé, par continuité de $x\longmapsto\frac{1}{\lambda}x$, il existe un voisinage $U$ de $0_E$ tel que $\frac{1}{\lambda}U\subset V$ id est $U\subset\lambda V$, faisant de $\lambda V$ un voisinage de $0_E$.
		\item Par continuité de $\lambda, x\longmapsto\lambda x$, il existe un voisinage $[-M, M]$ de $0_\R$ et $U$ de $0_E$ tels que $\forall\mu, x\in[-M, M]\times U, \mu x\in V$.
		Il suffit de poser $W=\bigcup_{\mu\in[-m, M]}\mu U$.
		\item Il suffit de prendre l'ensemble des $W^\circ$ de la démonstration précédente.
		\item Par continuité de $x, y\longmapsto x+y$, il existe deux voisinages $U_1$ et $U_2$ de $0_E$ tels que $\forall x1, x_2\in U_1\times U_2, x_1+x_2\in V$.
		Il suffit de prendre $W=U_1\cap U_2$ qui est encore un voisinage de $0_E$ par stabilité par intersection finie.
	\end{itemise}
\end{demonstration}

\begin{proposition}
	Soit $E$ et $F$ deux $\R$-espaces vectoriels topologiques, et $f:E\rightarrow F$ linéaire.
	$$
		f\in\cal{C}(0_E, F)\implies f\in\cal{C}(E, F)
	$$
\end{proposition}

\begin{demonstration}
	Soit $x\in E$, $y=f(x)$ et $W$ un voisinage de $y$.
	Il existe alors un voisinage $V$ de $0_F$ tel que $W=y+V$
	Par translation des voisinages $\inv{f}(W)=\inv{f}(y+V)\supset x+\inv{f}(V)$.
	Par continuité en $0_E$ de $f$, $\inv{f}(V)$ est un voisinage de $0_E$.
	Ainsi $\inv{f}(W)$ est un voisinage de $x$.
\end{demonstration}

\begin{proposition}
	Soit $E$ un ensemble qui soit un espace $\R$-vectoriel et un espace topologique.
	S'il vérifie les conclusions des deux premières propositions, alors réciproquement c'est un espace vectoriel topologique.
\end{proposition}

\begin{demonstration}
	Il suffit d'utiliser le proposition précédente sur $+$ et $\times$, les hypothèses assurant leur continuité en $(0_E, 0_E)$ et $(0_\R, 0_E)$.
\end{demonstration}

\begin{definition}
	Une partie d'un $\R$-espace vectoriel topologique est bornée si et seulement si elle est absorbée par tout voisinage de 0.
\end{definition}

\begin{proposition}
	Les compacts d'un $\R$-espace vectoriel topologique sont bornés.
\end{proposition}

\begin{demonstration}
	Soient $K$ un compact et $V$ un voisinage ouvert de 0.
	Soit $x\in K$, par absorption de $K$ par $V$, $\exists n\in\N^*, \forall\lambda\in[-\frac{1}{n}, \frac{1}{n}], \lambda x\in V$, c'est-à-dire $\exists n\in\N^*, x\in nV$.
	Ainsi $K\subset\bigcup_{n\in\N^*}nV$.
	Par compacité de $K$ on peut extraire de ce recouvrement ouvert un sous recouvrement fini.
	Soit $N$ le plus grand indice de ce recouvrement fini, alors $K\subset NV$, ce qui fait de $K$ une partie bornée.
\end{demonstration}

\chapter{Espaces métriques}

\section{Définitions et premières propriétés}

\begin{definition}
	On appelle distance sur $E$ toute application $d:E\times E\rightarrow\R^+$ qui soit séparable, symétrique et Minkovskienne, c'est-à-dire:
	\begin{itemise}
		\item $\forall x, y\in E, d(x, y)=0\iff x=y$
		\item $\forall x, y\in E, d(x, y)=d(y, x)$
		\item $\forall x, y, z\in E, d(x, z)\leqslant d(x, y)+d(y, z)$
	\end{itemise}
\end{definition}

\begin{proposition}
	Tout espace métrique $(E, d)$ est séparable pour la topologie engendrée par les boules ouvertes $B_\epsilon(x)=\ens{y}{d(x, y)<\epsilon}=\inv{d(x, \cdot)}([0, \epsilon[)$.
\end{proposition}

\begin{demonstration}
	Soit $x, y\in E$ deux points différents de $E$.
	Alors $\epsilon=d(x, y)>0$ par séparabilité de la distance.
	Donc $B_{\epsilon/2}(x)$ et $B_{\epsilon/2}(y)$ sont deux voisinages disjoins de $x$ et $y$ respectivement.
\end{demonstration}

\begin{definition}
	Soit $E$ un espace topologique.
	S’il existe une distance permettant de construire une topologie, on dit que la topologie découle d’une distance et on parle d’espace métrisable.
\end{definition}

\begin{remarque}
	Différentes distances peuvent être à l’origine de la même topologie, c’est pourquoi on distingue espace métrique (où la distance est donnée) et espace métrisable (où seule la topologie est donnée).
\end{remarque}

\begin{proposition}
	Soient $(E, d)$ et $(F, \delta)$ deux espaces métriques et $f\colon E\rightarrow F$.
	$$
	f\in\cal{C}(x, F)
	\iff
	\forall\epsilon>0, \exists\eta>0, \forall y\in E, d(x, y)<\eta\implies\delta(f(x), f(y))<\epsilon
	$$
\end{proposition}

\equivalence{
	On reformule simplement la définition avec les voisinages.
}{
	Soit $\epsilon>0$.
	Puisque $B_\epsilon(f(x))$ est un voisinage de $x$, $\exists V\in\frak{V}(x), f(V)\subset B_\epsilon(f(x))$.
	Les boules ouvertes étant par construction une base de voisinages, $\exists\eta>0, B_\eta(x)\subset V$.
	On a donc bien que si $d(x, y)<\eta$, alors $y\in V$, donc $f(y)\in B_\epsilon(f(x))$, soit $\delta(f(x), f(y))<\epsilon$.
}{
	 Soit $W\in\frak{W}(f(x))$.
	Alors pour $\epsilon$ assez petit, $B_\epsilon(x)\subset W$.
	Donc $V=B_\eta(x)$ est un voisinage de $x$ tel que si $y\in V$, alors $d(x, y)<\eta$, donc $\delta(f(x), f(y))<\epsilon$, soit $f(y)\in B_\epsilon(f(x))$.
	Donc $f(V)\subset W$, assurant la continuité de $f$ en $x$.	
}

\section{Les suites suffisent}

\begin{proposition}
	Soit $E$ un espace métrique.
	$$
	(x_n)\longrightarrow x
	\iff
	\forall\epsilon>0, \exists N\in\N, \forall n\geqslant N, d(x_n, x)<\epsilon
	$$
\end{proposition}

\equivalence{
	On reformule simplement la définition avec les voisinages.
}{
	Soit $\epsilon>0$.
	Puisque $B_\epsilon(x)$ est un voisinage de $x$ il existe $N\in\N$ tel que $\forall n\geqslant N$, $x_n\in B_\epsilon(x)$, c’est-à-dire $\exists N, \forall n\geqslant N, d(x_n, x)\epsilon$.
}{
	Alors pour $\epsilon$ assez petit, $B_\epsilon(x)\subset V$.
	Or $\exists N, \forall n\geqslant N, x_n\in B_\epsilon(x)$, donc $\exists N, X_N\in V$.
	Par caractérisation du filtre engendré par une base de filtre, $(x_n)$ converge.	
}

\begin{proposition}
	Soit $X$ une partie d'un espace métrique $E$.
	Alors $\overline{X}=\overline{X}^\rm{seq}$
\end{proposition}

\begin{demonstration}
	L'inclusion réciproque à déjà été prouvée précédemment.
	Soit $x\in\overline{X}$ et $V$ un voisinage de $x$.
	Alors $V$ contient une boule ouverte centrée en $x$, c'est-à-dire $\exists\epsilon>0, B_\epsilon(x)\subset V$.
	En prenant $x_n\in B_{1/n}(x)\cap X$ qui est non vide car $x$ est dedans, avec $1/n<\epsilon$, on obtient une suite $(x_n)$ de $X$ qui converge vers $x$.
\end{demonstration}

\begin{proposition}
	Soient $E$, $F$ métriques, $f:E\rightarrow F$ une application.
	$f$ est continue en $x$ si et seulement si la convergence d'une suite $(x_n)$ vers $x$ implique la convergence de $f(x_n)$ vers $f(x)$.
\end{proposition}

\begin{demonstration}
	L'implication directe à déjà été prouvée précédemment.
	Soit $W$ un voisinage de $f(x)$.
	Alors $W$ continent une boule ouverte centrée en $f(x)$, c'est-à-dire $\exists\epsilon>0, B^F_\epsilon(f(x))\subset V$.
	Soit $(x_n)$ une suite de $\inf{f}(W)$ telle que $x_n\in B^E_{1/n}(x)$.
	Alors $x_n\rightarrow x$ donc $f(x_n)\rightarrow f(x)$, donc $\exists N, \forall n\geqslant N, f(x_n)\in B^F_\epsilon(f(x))\subset W$.
	Ainsi $f(B^E_{1/n}(x))\subset W$ et $B^E_{1/n}(x)$ est un voisinage de $x$.
\end{demonstration}

\section{Compacité}

\begin{proposition}
	Un espace topologique est compact si et seulement si de toute suite on peut en extraire une sous suite convergente.
\end{proposition}

\begin{demonstration}
	L'implication directe à déjà été prouvée précédemment.
	Soit $\bigcup_{i\in I}O_i$ un recouvrement ouvert de $E$.

	Montrons $\exists\epsilon>0, \forall x_\epsilon\in E, \exists i\in I, B_\epsilon(x_\epsilon)\subset O_i$.
	Raisonnons par l'absurde.
	En prenant $\epsilon=1/n$, on construit une suite $(x_n)$ telle que $\forall i\in I, B_{1/n}(x_n)\not\subset O_i$.
	Par hypothèse, on peut en extraire une sous suite $(y_n)$ convergente, de limite $y$.
	Puisque $\bigcup_{i\in I}O_i$ est un recouvrement de $E$ et que $y\in E$, $\exists i\in I, y\in O_i$.
	Les ouverts étant engendrés par les boules ouvertes, $\exists\eta>0, B_\eta(y)\subset O_i$.
	Par convergence de $(y_n)$ vers $y$, $\exists N, \forall n\geqslant N, d(y_n, y)<\eta/2$.
	Prenons maintenant $n\geqslant\max(N, 2/\eta)$ et $z\in B_{1/n}(y_n)$ c'est-à-dire $d(z, y_n)<1/n$.
	Alors $d(y, z)\leqslant d(y, y_n)+d(y_n, z)<\eta/2+1/n\leqslant\eta$, c'est-à-dire $B_{1/n}(y_n)\subset B_\eta(y)\subset O_i$, ce qui est impossible.

	Montrons que $\forall\eta>0$, il existe un recouvrement fini de $E$ par des boules ouvertes de rayon $\eta$.
	Raisonnons par l'absurde.
	Soit $x_0\in E$ et $R_0=\vide$.
	Définissons par récurrence $x_{n+1}=E\setminus R_n$ et $R_{n+1}=R_n\cup B_\eta(x_{n+1})$.
	Ces suites sont bien définies car $R_0$ ne recouvre pas $E$, et que si $R_n$ ne recouvre pas $E$, alors $x_{n+1}$ existe, et par hypothèse $R_{n+1}$ est une union finie de boules ouvertes donc ne peut recouvrir $E$.
	En particulier on a que $d(x_{n+1}, x_n)\geqslant\eta$.
	Par hypothèse, on peut en extraire une sous suite $(y_n)$ convergente, de limite $y$.
	Par convergence de $(y_n)$ vers $y$, $\exists N, \forall n\geqslant N, d(y_n, y)<\eta/2$.
	En particulier, $d(y_{N+1}, y_N)\leqslant d(y_{N+1}, y)+d(y, y_n)<\eta$, ce qui est impossible.

	Ainsi, on peut recouvrir $E$ par un nombre fini $n$ de boules ouvertes de rayon $\epsilon$.
	D'après la première partie, elles sont chacune incluses dans un ouvert du recouvrement de $E$.
	Ces ouverts forment donc un recouvrement fini de $E$, ce qui montre la compacité de $E$.
\end{demonstration}

\begin{definition}
	Une partie d'un espace métrique est bornée si et seulement si la distance entre chacun de ses point est majorée par une constante.
\end{definition}

\begin{proposition}
	Une partie compact d'un espace métrique est fermée et bornée.
\end{proposition}

\begin{demonstration}
	Un espace métrique est topologique séparé, et une partie compact d'un espace topologique séparé est fermé.
	De plus d'après la démonstration précédente, cette partie compact peut être recouverte par un nombre fini $n$ de boules ouvertes de rayon 1.
	Par inégalité triangulaire, on majore la distance entre n'importe quel bipoint par $1+n+1$.
\end{demonstration}

\begin{proposition}
	Une fonction numérique réelle d'un compact est bornée et atteint ses bornes.
\end{proposition}

\begin{demonstration}
	L’image d'un compact par une application continue est compacte.
	Elle est donc fermée et bornée d'après la proposition précédente.
	Par bornitude, elle admet un élément maximal $y^+$.
	Par fermeture séquentielle, il existe une suite $(y_n)$ de l'image de limite $y^+$, la convergence étant prise pour la distance $d(x, y)=|x-y|$.
	Posons $F_n=\ens{x}{y_n\leqslant f(x)\leqslant y^+}$ une suite décroissante pour l'inclusion de fermés de l'image.
	Par l'absurde, si $\bigcap_{n\in\N}F_n=\vide$, alors par compacité de l'image, $\exists N, \bigcap_{n\leqslant N}F_n=\vide$.
	C'est absurde car par décroissance des $F_n$, on aurait $\bigcap_{n\leqslant N}F_n=F_{\max(N)}=\vide$.
	Soit alors $x\in\bigcap_{n\in\N}F_n$, il vient $\forall n\in\N, y_n\leqslant f(x)\leqslant y^+$, donc $|f(x)-y^+| \leqslant |y_n-y^+|$.
	Par convergence de $(y_n)$ vers $y$, $\forall\epsilon>0, |f(x)-y^+|<\epsilon$, donc $f(x)=y^+$.
	Le maximum est donc atteint.
	On procède de même pour le minimum.
\end{demonstration}

\chapter{Espaces vectoriels normés}

\chapter{JSP}

\paragraph{Définition:} Distances équivalentes: $\exists\alpha_-\\alpha_+\quad\alpha_- d_1\le d_2\le\alpha_+ d_1 $.
C'est une relation d'équivalence.

\paragraph{Théorème:}
$
	\begin{array}{rcl}
		\text{$d_1$ est équivalente à $d_2$}
		  & \implies & \forall x, y\in E\quad\forall r > 0,
		\left\{
		\begin{array}{cc}
			y\in B_2(x, r)\implies y\in B_1(x, r/\alpha_-) \\
			y\in B_1(x, r)\implies y\in B_2(x, r/\alpha_+) \\
		\end{array}
		\right.\\
		  & \implies & \text{les voisinages issus de $d_1$ sont des voisinage pour $d_2$ et inversement} \\
		  & \implies & \text{les topologies sont égales}                                                 \\
	\end{array}
$

\paragraph{Définition:} Suite de Cauchy: $ d(x_m, x_n)\underset{m, n\rightarrow +\infty}{\rightarrow} 0$,
soit: $\quad\forall\varepsilon > 0, \exists N, \quad m, n\ge N\implies d(x_m, x_n) <\varepsilon$.

\paragraph{Définition:} Espace complet: toutes suites de Cauchy est convergente (réciproque triviale).

\section{Norme, Espace vectoriel normé}

\paragraph{Définition:}
$
	\text{$|| \cdot||:E\rightarrow\mathbb{R}^+$ est une norme sur $E$}:
	\left\ens{
	\begin{array}{ll}
		\text{$|| \cdot||$ est séparable:} & ||x||=0\iff x=0                 \\
		\text{$|| \cdot||$ est homogène:}  & || \lambda x||=}{\lambda| ||x|| \\
		\text{inégalité triangulaire:}     & ||x+z|| \le ||x+y||+||y+z||     \\
	\end{array}
	\right.
$

\paragraph{Théorème:} Un espace vectoriel normé est un espace métrique en posant $d(x, y)=||x-y||$.

\paragraph{Définition:} Normes équivalentes: $\exists\alpha_-\\alpha_+\quad\alpha_- || \cdot||_1\le || \cdot||_2\le\alpha_+ || \cdot||_1 $. C'est une relation d'équivalence.

\paragraph{Théorème:} Normes équivalentes $\iff$ Topologies issues des distances issues des normes égales.

\fbox{$\implies$} Cf théorème pour les espace métriques.

\fbox{$\Longleftarrow$}
$
	\begin{array}{rcl}
		\text{Topologies égales}
		  & \implies & \forall O_1\in\scr{T}_1, \exists O_2\in\scr{T}_2, \quad O_2\subset O_1                                                \\text{(et inversement)}\\
		  & \implies & \forall r_1, \exists r_2, \quad B_2(x, r_2)\subset B_1(x, r_1)                                                                \\text{(et inversement)}\\
		  & \implies & \forall r_1, \exists r_2, \quad ||x-y||_2<r_2\implies ||x-y||_1<r_1                                                           \\text{(et inversement)}\\
		  & \implies & \forall r_1, \exists r_2, \quad ||y||_2<r_2\implies ||y||_1<r_1                                                               \\text{(et inversement)}\\
		  & \implies & \forall r_1, \exists r_2, \quad ||x\frac{r_2}{\varepsilon+||x||_2}||_2<r_2\implies ||x\frac{r_2}{\varepsilon+||x||_2}||_1<r_1 \\text{(et inversement)}\\
		  & \implies & \forall r_1, \exists r_2, \quad ||x||_2<\varepsilon+||x||_2\implies ||x||_1<\frac{r_1}{r_2}(\varepsilon+||x||_2)              \\text{(et inversement)}\\
		  & \implies & \exists r_2, \quad ||x||_1 <\frac{1}{r_2}||x||_2                                                                              \\text{(et inversement)}\\
		  & \implies & \text{$|| \cdot||_1$ et $|| \cdot||_2$ sont équivalentes}
	\end{array}
$

\paragraph{Définition:} Espace de Banach: espace vectoriel normé et complet.
\paragraph{Définition:} Espace de Hilbert: espace vectoriel préhilbertien (avec produit scalaire) normé par celui-ci et complet.

\paragraph{Théorème:}
$
	\begin{array}{rcl}
		\text{$f$ linéaire entre evn continue en $0$}
		  & \implies & \forall r > 0, \exists\rho > 0\quad ||x||_E <\rho\implies ||f(x)||_F < r                     \\
		  & \implies & \forall r > 0, \exists\rho > 0\quad ||x-y||_E <\rho\implies ||f(x-y)||_F=||f(x)-f(y)||_F < r \\
		  & \implies & \text{$f$ est continue sur tout $E$}
	\end{array}
$

\paragraph{Théorème:}
$
	\begin{array}{rcl}
		\text{$f$ est continue sur tout $E$}
		  & \implies & \forall r > 0, \exists\rho > 0\quad ||x||_E <\rho\implies ||f(x)||_F < r                                                      \\
		  & \implies & \exists\rho > 0\quad ||x||_E <\rho\implies ||f(x)||_F < 1                                                                     \\
		  & \implies & \exists\rho > 0\quad ||y\frac{\rho}{\varepsilon + ||y||_E}||_E <\rho\implies ||f(y\frac{\rho}{\varepsilon + ||y||_E})||_F < 1 \\
		  & \implies & \exists\rho > 0\quad ||y||_E <\varepsilon + ||y||_E\implies ||f(y)||_F <\frac{1}{\rho}(\varepsilon + ||y||_E)                 \\
		  & \implies & \exists\rho > 0\quad ||f(y)||_F <\frac{1}{\rho}||y||_E                                                                        \\
		  & \implies & \text{$f$ lipschitzienne en zéro}
	\end{array}
$

\paragraph{Théorème:}
$
	\begin{array}{rcl}
		\text{$f$ lipschitzienne en zéro}
		  & \implies & \exists\rho > 0\quad ||f(y)||_F <\frac{1}{\rho}||y||_E                                            \\
		  & \implies & \exists\rho > 0, \quad\forall r > 0, ||y||_E < r/\rho\implies ||f(y)||_F <\frac{1}{\rho}||y||_E=r \\
		  & \implies & \text{$f$ est continue en $0$}
	\end{array}
$

\paragraph{Théorème:Hahn-Banach} $f$ forme linéaire partielle et majorée pas une distance Minkovskienne (application sous linéaire) $\implies$ $f$ prolongeable sur tout $E$ avec la même inégalité.

\chapter{Théorème de Hahn-Banach et dualité}

\chapter{Théorème de Baire et ses conséquences}

\chapter{Exemples d'espaces fonctionnels}

\chapter{Topologie faible}

\end{document}







\begin{proposition}
	Les quasi-uniformités forment un espace plus gros que les topologies.
\end{proposition}

\begin{demonstration}
	Posons $\Phi:\scr{T}\mapsto\sigma_\rm{quasi}(\frak{B})$ avec $\frak{B}$ la famille des intersections finies de $\frak{B}'=\ens{O^2\cup\complement O\times E}{O\in\scr{T}}$ (appelée prébase de Pervin).
	Montrons que $\frak{B}$ est une base de quasi-uniformité.
	\begin{itemise}
		\item Si $X\in\frak{B}'$ alors $\Delta_E\subset X$.
		C'est encore vrai après intersection, donc pour $\frak{B}$.
		\item Par définition $\frak{B}$ est stable par intersection finie.
		\item Soit $X'=O^2\cup\complement O\times E\in\frak{B}'$ et $(x, z)\in X'\circ X'$ alors $\exists y\in E, (x, y), (y, z)\in X'$. \\
		Si $x\in O$ alors $(x, y)\notin\complement O\times E$ donc $y\in O$ de même $z\in O$ donc $(x, z)\in O^2\subset X'$. \\
		Si $x\in\complement O$ alors $(x, y)\in\complement O\times E=X'$.
		Ainsi $\forall X'\in\frak{B}', X'\circ X'\subset X'$.
		Cette propriété passe aux intersections finies donc si $X\in\frak{B}, X\circ X\subset X$.
	\end{itemise}
	Posons $\Psi:\scr{U}\mapsto\ens{U[x]}{U\in\scr{U}}$ pour $U[x]=\ens{y\in E}{(x, y)\in U}$.
	Montrons que pour $\scr{U}$ une quasi-uniformité non vide $\frak{V}=\Psi(\scr{U})$ est une voisinagination.
	\begin{itemise}
		\item Soit $U\in\scr{U}$.
		Puisque $U\subset E\times E$ alors $E\times E\in\scr{U}$, donc $E=(E\times E)[x]\in\frak{V}(x)$.
		\item Si $U[x]\in\frak{V}(x)$, puisque $\Delta_E\subset U$ alors $(x, x)\in U$ donc $x\in U[x]$.
		\item Si $U[x], V[x]\in\frak{V}(x)$ alors $U[x]\cap V[x]=(U\cap V)[x]\in\frak{B}$ car $U\cap V\in\scr{U}$.
		\item Si $U[x]\in\frak{V}(x)\et U[x]\subset T$ en posant $V=U\cup\{x\}\times T$ on a $U\subset V$ donc $V\in\scr{U}$ et $V[x]=(U\cup\{x\}\times T)[x]=U[x]\cup T=T\in\frak{V}(x)$ car $U[x]\subset T$.
		\item Si $U[x]\in\frak{V}(x)$ alors $\exists V\in\scr{U}, V\circ V\subset U$.
		De $y\in V[x]\et z\in V[y]$ on a $(x, y), (y, z)\in V$ donc $(x, z)\in V\circ V\subset U$.
		Ainsi $z\in U[x]$ puis $V[y]\subset U[x]$.
		Or $V[y]\in\frak{V}(y)$ donc d'après le point précédent $U[x]\in\frak{V}(y)$
	\end{itemise}
	Montrons que $\Psi\circ\Phi\cong\id$, c'est-à-dire $\scr{T}\cong\frak{V}$ (l'ismorphisme canonique défini en permière partie) par double inclusion.
	\begin{itemise}
		\item[$\subset$] Si $O\in\scr{T}$ alors $O^2\cup\complement O\times E\in\frak{B}'$ donc $O=(O^2\cup\complement O\times E)[x]\in\frak{V}(x)$.
		\item[$\supset$] Si $U[x]\in\frak{V}(x)$ alors $\exists X_{1\leqslant i\leqslant n}\in\frak{B}', \bigcap_{i=1}^nX_i\subset U$ avec $X_i=O_i^2\cup\complement O_i\times E$ et $O_i\in\scr{T}$.
		Si $x\in O_i$ alors $X_i[x]=O_i$ sinon $X_i[x]=E$.
		En posant $J=\ens{j\in I}{x\in O_j}$ qui est fini on a $(\bigcap_{i=1}^nX_i)[x]=\bigcap_{i=1}^nX_i[x]=\bigcap_{j\in J}O_j\in\scr{T}$ et $(\bigcap_{i=1}^nX_i)[x]\subset U[x]$.
	\end{itemise}
\end{demonstration}




\usepackage{tikz}  % dessins
\usetikzlibrary{arrows}
\usetikzlibrary{calc}
\usetikzlibrary{decorations.pathreplacing}
\tikzset{%
	show curve controls/.style={
			postaction={
					decoration={
							show path construction,
							curveto code={
									\draw [blue]
									(\tikzinputsegmentfirst)--(\tikzinputsegmentsupporta)
									(\tikzinputsegmentlast)--(\tikzinputsegmentsupportb);
									\fill [red, opacity=0.5]
									(\tikzinputsegmentsupporta) circle [radius=.5ex]
									(\tikzinputsegmentsupportb) circle [radius=.5ex];
								}
						},
					decorate
				}}}

\begin{tikzpicture}
	\draw [show curve controls] (-3, 4) .. controls ++(135:-1) and ++(135:1) .. (0, 4);
	\draw [show curve controls] (0, 0)
	.. controls ++(0:-1) and ++(240: 1) .. (3, 2)
	.. controls ++(240:-1) and ++(165:-1) .. (2, 4)
	.. controls ++(165: 1) and ++(175:-2) .. (-1, 2)
	.. controls ++(175: 2) and ++(165: 1) .. (0, 0);
\end{tikzpicture}

\begin{centre}
	\begin{tikzpicture}[scale=0.5]
		\draw [help lines] (0, 0) grid (3, 3);
		\draw [show curve controls] (0, 0)
		.. controls ++(45:-1) and ++(45:1) .. (0, 2)
		.. controls (1, 2) .. (2, 2)
		.. controls (2, 1) .. (2, 0)
		.. controls (1, 0) .. (0, 0);
	\end{tikzpicture}
\end{centre}

\begin{tikzpicture}
	\begin{scope}[scale=0.5]
		\draw (-3, 0.6) .. controls +(1, 0) and +(-1, 0) .. (0, 1.8)
		.. controls +(1, 0) and +(0, -3) .. (5, 3.2)
		.. controls +(0, 2) and +(2, 0)  .. (0, 5.2)
		.. controls +(-1, 0) and +(0, 3) .. (-4.5, 2.2)
		.. controls +(0, -1) and +(-1, 0).. (-3, 0.6);
		\begin{scope}  % pour limiter la portée du clip
			\clip (-3, 0.6) .. controls +(1, 0) and +(-1, 0) .. (0, 1.8)
			.. controls +(1, 0) and +(0, -3) .. (5, 3.2)
			.. controls +(0, 2) and +(2, 0)  .. (0, 5.2)
			.. controls +(-1, 0) and +(0, 3) .. (-4.5, 2.2)
			.. controls +(0, -1) and +(-1, 0).. (-3, 0.6);
			\draw (-5, 0) grid[step=1](5, 6);
		\end{scope}
		\draw[<->] (-5, 2)--(-5, 3) node[midway, left]{$\varepsilon$};
		\node at (0, 1) {$\Omega$};
	\end{scope}
\end{tikzpicture}
}

\begin{proposition}
	Si $\frak{B}$ est une base $\sigma_\rm{topologie}(\frak{B})=\{U\subset E| \exists X_{i\in I}\in\frak{B}, U=\bigcup_{i\in I}X_i}$
\end{proposition}

\begin{demonstration}
	Notons l'ensemble de droite $\scr{T}=\ens{U\subset E}{\exists X_{i\in I}\in\frak{B}, U=\bigcup_{i\in I}X_i}$ et montrons $\scr{T}=\sigma_\rm{topologie}(\frak{B})$ par double inclusion.
	\begin{itemise}
		\item[$\subset$] Soit $O\in\scr{T}$ alors $\exists X_{i\in I}\in\frak{B}, O=\bigcup_{i\in I}X_i$.
		Or $\sigma_\rm{topologie}(\frak{B})$ continent $\frak{B}$ donc $X_{i\in I}\in\sigma_\rm{topologie}(\frak{B})$, et est une topologie donc $O=\bigcup_{i\in I}X_i\in\sigma_\rm{topologie}(\frak{B})$.
		\item[$\supset$] Montons que $\scr{T}$ est une topologie.
		\begin{itemise}
			\item de $E\subset\bigcup_{X\in\frak{B}}X$ on a $E\in\scr{T}$.
			Pour $I=\vide$ on obtient $\vide\in\scr{T}$
			\item Soit $U, V\in\scr{T}$ alors $\exists X_{i\in I}, X_{j\in J}\in\frak{B}, U=\bigcup_{i\in I}X_i$ et $V=\bigcup_{j\in J}X_j$.
			Puisque $\frak{B}$ est une base $\exists Z^{ij}_{k\in K}, X_i\cap X_j=\bigcup_{k\in K}Z^{ij}_k, \forall i, j\in I\times J$.
			Donc $U\cap V=\bigcup_{i\in I, j\in J}X_i\cap X_j=\bigcup_{i\in I, j\in J, k\in K}Z^{ij}_k\in\scr{T}$.
			\item Si $O_{i\in I}\in\scr{T}$ alors $\exists X^i_{j\in J}\in\frak{B}, O_i=\bigcup_{i\in I, j\in J}X^i_j$ donc $\bigcup_{i\in I}O_i=\bigcup_{i\in I, j\in J}X^i_j\in\scr{T}$.
		\end{itemise}
		De plus $\scr{T}$ contient $\frak{B}$.
		Or $\sigma_\rm{topologie}(\frak{B})$ est la plus petite topologie contenant $\frak{B}$.
		Donc $\scr{T}\supset\sigma_\rm{topologie}(\frak{B})$.
	\end{itemise}
\end{demonstration}

Soit $\frak{V}$ une voisinagination de $E$.
Vérifions que $\scr{T}_\rm{ouvert}=\Phi(\frak{V})$ est une topologie d'ouverts.
\begin{itemise}
	\item $\forall x\in\vide, \vide\in\frak{V}(x)$ est vrai donc $\vide\in\scr{T}_\rm{ouvert}$ et $E\in\scr{T}_\rm{ouvert}$ par hypothèse.
	\item Si $O, O'\in\scr{T}_\rm{ouvert}$ alors $x\in O\cap O'$ implique $x\in O\et x\in O'$ donc $O\in\frak{V}(x)\et O'\in\frak{V}(x)$, soit $O, O'\in\frak{V}(x)$.
	Par définition des voisinaginations $O\cap O'\in\frak{V}(x)$ soit $O\cap O'\in\scr{T}_\rm{ouvert}$.
	\item Si $O_{i\in I}\in\scr{T}_\rm{ouvert}$ alors $x\in\bigcup_{i\in I} O_i$ implique $\exists i\in I, x\in O_i$.
	Ainsi $O_i\in\frak{V}(x)\et O_i\subset\bigcup_{i\in I} O_i$, donc par définition des voisinaginations $\bigcup_{i\in I} O_i\in\frak{V}(x)$ soit $\bigcup_{i\in I} O_i\in\scr{T}_\rm{ouvert}$.
\end{itemise}

$$
	\scr{T}_\rm{ouvert}\underset{\Psi}{\longmapsto}\frak{V}\underset{\Phi}{\longmapsto}\scr{T}_\rm{ouvert}'
$$

Vérifions que $\Psi\circ\Phi=\id$.
Soit $\scr{T}_\rm{ouvert}$ une topologie d'ouverts de $E$.
Vérifions que $\scr{T}_\rm{ouvert}'=\scr{T}_\rm{ouvert}$.

$$
	\frak{V}\underset{\Phi}{\longmapsto}\scr{T}_\rm{ouvert}\underset{\Psi}{\longmapsto}\frak{V}'
$$
Soit $\frak{V}$ une voisinagination de $E$.
Vérifions que $\frak{V}'=\frak{V}$.
\begin{itemise}
	\item Si $V\in\frak{V}(x)$ posons $O=\ens{y\in E}{V\in\frak{V}(y)}$.
	Puisque $x\in V$ alors $x\in O$.
	Si $y\in O$ alors $V\in\frak{V}(y)$ donc $y\in V$ soit $O\subset V$, et $\exists W\in\frak{V}(y), \forall z\in W, V\in\frak{V}(z)$ soit $z\in O$  donc $W\subset O$ puis $O\in\frak{V}(y)$ par définition de voisinagination, donc $O\in\Psi(\frak{V})=\scr{T}_\rm{ouvert}$.
	Ainsi $\exists O\in\scr{T}_\rm{ouvert}, x\in O\subset V$ donc $V\in\Psi(\scr{T}_\rm{ouvert})(x)=\frak{V}'(x)$.
	\item Si $V'\in\frak{V}'(x)$ alors $\exists O\in\scr{T}_\rm{ouvert}, x\in O\subset V'$.
	Or $\scr{T}_\rm{ouvert}=\Phi(\frak{V})$ donc $O\in\frak{V}(x)$, puis $V'\in\frak{V}(x)$ par définition d'une voisinagination.
\end{itemise}


Soit $O$ un ouvert non vide et $x\in O$.
Alors $O$ est un voisinage de $x$, et $x\in\overline{X}$ donc $O\cap X\neq\vide$.

Soit $O$ un ouvert de $E$ inclus dans $\complement X$.
Par l'absurde supposons $O\neq\vide$.
Alors $O\cap X\neq\vide$, puis $\vide\neq O\cap X\subset\complement X\cap X=\vide$.
C'est absurde, donc $O$ est vide, puis $\complement\overline{X}=\vide$, soit $\overline{X}=E$.

\begin{definition}
	Soient $(E_i)_{i\in I}$ des espaces topologiques, $E=\prod_{i\in I}E_i$ leur produit cartésien.
	La topologie produit sur $E$ est celle engendrée par la base précédente des pavés formés d'ouverts différents de $E_i$ dans un ensemble de directions $J$ fini.
	%Il est alors facile de voir qu’une base de voisinages de $x=(x_i)_{i\in I}$ est formée des pavés $\prod_{i\in I} V_i$ où les $V_i\in\frak{V}(x_i)$ sont tous égaux à $E_i$ , sauf un nombre fini d’entre eux.
\end{definition}

\begin{remarque}
	Cette définition de la topologie produit peut sembler surprenante, car les voisinages ne sont susceptibles de manifester de restriction que dans un nombre fini de directions (la fameuse tranche Napolitaine).
	C’est en fait la proposition suivante qui justifie cette définition.
\end{remarque}

\begin{proposition}
	La topologie produit est la topologie initiale associée aux projections canoniques $\pi_i:E\rightarrow E_i$, c'est-à-dire la moins fine rendant les $\pi_i$ continues.
\end{proposition}

\begin{demonstration}
	Soit $O_i$ un ouvert de $E_i$ , alors $\inv{\pi_i}(O_i)=O_i\times\prod_{j\neq i} E_j$ est un ouvert de $E$, rendant $\pi_i$ continue.
	Soient $\scr{T}$ une topologie sur $E$ rendant les $p_i$ continues.
	Soit $O=\prod_{j\in J}O_j\times\prod_{i\in I\setminus J}E_i$ dans $\frak{B}$.
	Remarquons que :
	$$
		p_i\left(\bigcap_{j\in J}\inv{p_j}(O_j)\right)
		=\bigcap_{j\in J}p_i(\inv{p_j}(O_j))
		=\bigcap_{j\in J}\left\{
		\begin{array}{cc}
			O_i & \text{si $i=j$}     \\
			E_i & \text{si $i\neq j$}
		\end{array}
		\right.
		=\left\{
		\begin{array}{cc}
			O_i & \text{si $i\in J$}     \\
			E_i & \text{si $i\not\in J$}
		\end{array}
		\right.
		=O_i
	$$
	Puisque $X=\prod_{i\in I}p_i(X)$ pour tout $X\subset E$ on obtient:
	$$
		O
		=\prod_{i\in I}O_i
		=\prod_{i\in I}p_i\left(\bigcap_{j\in J}\inv{p_j}(O_j)\right)
		=\bigcap_{j\in J}\inv{p_j}(O_j)
	$$
	Par continuité de $p_i$, $\inv{p_j}(O_j)\in\scr{T}$ et l'intersection étant finie $O\in\scr{T}$.
	Par minimalité de la topologie engendrée, la topologie produit est incluse dans $\scr{T}$.
\end{demonstration}